\newpage
\section{Appendix} \label{appendix}
\subsection{Discussion on Assumptions \ref{cvx}, \ref{bddness} and \ref{feas-constr}}\label{app:assumptions}
Assumptions \ref{cvx} and \ref{bddness} are standard in the online learning literature. The feasibility assumption (Assumption \ref{feas-constr}) is analogous to the \emph{realizability} assumption in learning theory \citep{pmlr-v178-hopkins22a} and is commonly used in the COCO literature \citep{neely2017online, yu2016low,yuan2018online,yi2023distributed, georgios-cautious}. Assumption 3 requires the existence of a single admissible action $x^\star \in \mathcal{X}$ that satisfies the constraints in \emph{every} round. Consequently, all constraint functions are required to be non-positive over a non-empty common subset. This assumption is weakened in Section \ref{simul_constr}, Assumption \ref{s-feas-assump}, which only requires the existence of a fixed admissible action $x^\star$ that satisfies the constraints \emph{on average}. Specifically, Assumption \ref{s-feas-assump} requires that the sum of the constraint functions evaluated at some admissible $x^\star$ over any interval of length $S$ is non-positive. Notably, throughout the paper, we \emph{do not} assume Slater's condition as it does not hold in many problems of interest \citep{yu2016low}. As a result, unlike many previous works \citep{yu2017online}, our bounds are \emph{independent} of Slater's constant, which can be problem-dependent. Furthermore, we do not restrict the sign of either cost or constraint functions, allowing them to take both positive and negative values. 
%See the discussion on preprocessing the constraint functions in Section \ref{gen_oco}.
%Inspired by the Lyapunov method in the control theory, in the following, we propose an online meta-policy for the \textsc{OCS} problem and show that it yields optimal violation bounds. 
%Our main technical contribution is that while the classic works, such as \citet{neely2010stochastic}, use the Lyapunov theory in a stochastic setting; we adapt it to the adversarial setting by combining the Lyapunov method with the OCO framework.
%study the adversarial version of the problem through the lens of the OCO framework.
\subsection{Preliminaries on Online Convex Optimization (OCO)} \label{prelims}
 The standard OCO problem can be described as a repeated game between an online policy and an adversary \citep{hazan2022introduction}.
Let $\mathcal{X} \subseteq \mathbb{R}^d$ be a convex decision set, which we refer to as the \emph{admissible} set.
In each round $t\geq 1,$ an online policy selects an action $x_t \in \mathcal{X}.$ After the action $x_t$ is chosen, the adversary reveals a convex cost function $f_t : \mathcal{X} \mapsto \mathbb{R}$. 
%Since any convex function could be non-differentiable on a set of measures at most zero, without affecting the regret bounds, wherever convenient, we will assume the cost functions to be differentiable everywhere and replace gradients by subgradients at any point of non-differentiability \citep{hazan2007adaptive}. 
The goal of the online policy is to choose an admissible action sequence $\{x_t\}_{t\geq 1}$ so that its total cost over a horizon of length $T$ is not significantly larger than the total cost incurred by any fixed admissible action $x^\star \in \mathcal{X}$. More precisely, the objective is to 
minimize the static regret, defined as:
\begin{eqnarray} \label{regret-def}
	\textrm{Regret}_T \equiv \sup_{x^\star \in \mathcal{X}} \textrm{Regret}_T(x^\star), ~\textrm{where~}\textrm{Regret}_T(x^\star) \equiv \sum_{t=1}^T f_t(x_t) - \sum_{t=1}^T f_t(x^\star).
\end{eqnarray}


\begin{algorithm} 
\caption{Online Gradient Descent (OGD)}
\label{ogd-policy}
\begin{algorithmic}[1]
\State \algorithmicrequire{ Non-empty closed convex set $\mathcal{X} \subseteq \mathbb{R}^d$, sequence of convex cost functions $\{f_t\}_{t\geq 1},$ step sizes $\eta_1, \eta_2, \ldots, \eta_T >0,$ Euclidean projection operator $\mathcal{P}_\mathcal{X}(\cdot)$ onto the set $\mathcal{X}$}
\State{\textbf{Initialization}:} Set $x_1 \in \mathcal{X}$ arbitrarily
\ForEach {round $t\geq 1$}
\State Play $x_t$, observe $f_t$, incur a cost of $f_t(x_t)$.
\State Compute a (sub)gradient $\nabla_t \equiv \nabla f_t(x_t)$. %Define $G_t=||\nabla_t||_2.$
\State Update $x_{t+1}=\mathcal{P}_{\mathcal{X}}(x_t-\eta_t \nabla_t).$
\EndForEach 
\end{algorithmic}
\end{algorithm}


In a seminal paper, \citet{zinkevich2003online} showed that the online gradient descent policy, outlined in Algorithm \ref{ogd-policy}, run with an appropriately chosen constant step size sequence, achieves a sublinear regret bound $\textrm{Regret}_T = O(\sqrt{T})$ for Lipschitz-continuous convex cost functions. 
%In this paper, we are interested in stronger adaptive regret bounds where the bound is given in terms of the norm of the gradients and the strong-convexity parameters of the online cost functions. 
In Theorem \ref{data-dep-regret}, we recall two standard results on further refined data-dependent adaptive regret bounds achieved by the OGD policy with appropriately chosen adaptive step size sequences. 
%We will use the OGD policy with an appropriate step size sequence as a subroutine in our proposed online algorithm. 
%\edit{Take a look at \cite{yang2014regret} for variational bounds. Also, take a look at the case of exp-concave losses given in \cite{zhang2022simple}}.
\begin{theorem} \label{data-dep-regret}
Consider the generic OGD policy outlined in Algorithm \ref{ogd-policy}. 
%Depending on the step-size sequence, the OGD policy achieves the following data-dependent adaptive regret bounds. 
\begin{enumerate}
	%\item {\cite[Theorem 4.14]{orabona2019modern}} 
	\item {\citep{duchi2011adaptive}, \cite[Theorem 4.14]{orabona2019modern}} Let the cost functions $\{f_t\}_{t \geq 1}$ be convex and the step size sequence be adaptively chosen as $\eta_t= \frac{\sqrt{2}D}{2\sqrt{\sum_{\tau=1}^{t} G_\tau^2}}, t \geq 1,$ where $D$ is the Euclidean diameter of the admissible set $\mathcal{X}$ and $G_t=||\nabla f_t(x_t)||_2, t\geq 1.$ Then Algorithm \ref{ogd-policy} achieves the following regret bound: 
	\begin{eqnarray} \label{cvx-reg-bd}
			 \textrm{Regret}_T \leq \sqrt{2}D \sqrt{\sum_{t=1}^T G_t^2}.
	\end{eqnarray}
	The OGD policy with the above adaptive step-size sequence is known as (a variant of) the AdaGrad policy in the literature \citep{duchi2011adaptive}. 

	\item {\cite[Theorem 2.1]{hazan2007adaptive}} Let the cost functions $\{f_t\}_{t\geq 1}$ be strongly convex and let $H_t>0$ be the strong convexity parameter\footnote {The strong convexity of $f_t$ implies that $f_t(y)\geq f_t(x) + \langle \nabla f_t(x), y-x\rangle + \frac{H_t}{2}||x-y||^2, \forall x, y \in \mathcal{X}, \forall t.$} for the cost function $f_t$. Let the step size sequence be adaptively chosen as $\eta_t = \frac{1}{\sum_{s=1}^{t} H_s}, t \geq 1.$ Then Algorithm \ref{ogd-policy} achieves the following regret bound:
	\begin{eqnarray} \label{str-cvx-reg-bd}
		\textrm{Regret}_T \leq \frac{1}{2}\sum_{t=1}^T \frac{G_t^2}{\sum_{s=1}^t H_s}.
	\end{eqnarray} 
	\iffalse
	\item {\cite[Theorem 4.25]{orabona2019modern}, \cite[Theorem 2]{zhang2019adaptive}} Let the cost functions $\{f_t\}_{t \geq 1}$ be convex, non-negative, and $M$-smooth \footnote{$H$-smoothness means that $|| \nabla f_t(x)- \nabla f_t(y)||_2 \leq M||x-y||_2, \forall x,y\in \mathcal{X}, \forall t.$ }. Then the OGD policy with the same step-size sequence as in part 1 achieves the following regret bound:
	\begin{eqnarray*}
		\mathcal{R}_T(x^\star) \leq 4MD^2 + 4D\sqrt{M\sum_{t=1}^T f_t(x^\star)},
	\end{eqnarray*}
	where $x^\star \in \mathcal{X}$ is the static benchmark used in the definition of regret \eqref{regret-def}.
	\fi
	\end{enumerate}
\end{theorem}
%Note that the above adaptive bounds are mentioned for the simplicity of the regret expressions and the corresponding policy and for no other particular reason. 
Similar adaptive regret bounds are known for various other online learning policies as well. For structured domains, one can use other algorithms such as AdaFTRL \citep{orabona2018scale} which gives better regret bounds for high-dimensional problems. Furthermore, for problems with combinatorial structures, adaptive oracle-efficient algorithms, \emph{e.g.,} Follow-the-Perturbed-Leader (FTPL)-based policies, can be employed \citep[Theorem 11]{abernethy2014online}. 
Our proposed policies are agnostic to the specific online learning subroutine used for the surrogate OCO problem - what matters is that the subroutine provides adaptive regret bounds similar to \eqref{cvx-reg-bd} and \eqref{str-cvx-reg-bd}. This flexibility allows for an immediate extension of our algorithm to a wide range of settings, such as delayed feedback \citep{joulani2016delay} or combinatorial actions.  


%In the following sections, we propose an online learning policy for the \texttt{OCS} and \texttt{constrained OCO} problems which may use any OCO subroutine with an adaptive regret bound, such as the bounds given in Theorem \ref{data-dep-regret}. Interestingly, our analysis is oblivious to the OCO policy and only depends on the adaptive regret bound guaranteed by the particular OCO policy. This property can be exploited to immediately extend the scope of our proposed algorithm to various other non-standard settings, \emph{e.g.,} delayed feedback \citep{joulani2016delay}. 


%\appendix
\iffalse
\subsection{The \textsc{Hidden Set} Problem} \label{hidden_set}
Let $\mathcal{X}$ be an \emph{admissible} set of actions which is known to a policy. Let $\mathcal{X}^\star$, called the \emph{feasible} set, be a closed and convex subset of $\mathcal{X}$. Due to a large number of defining constraints, the feasible set $\mathcal{X}^\star$ is too complex to communicate to the policy \emph{a priori}. However, an efficient separation oracle for $\mathcal{X}^\star$ is assumed to be available. On the $t$\textsuperscript{th} round, the policy first selects an admissible action $x_t \in \mathcal{X}$ and then, the adversary reveals a convex cost function $f_t$ and \emph{some} convex constraint of the form $g_t(x) \leq 0,$ which contains the unknown feasible set $\mathcal{X}^\star$. As an example, the constraints could come from the separation oracle that, if $x_t$ is infeasible, outputs a hyperplane separating the current action $x_t \in \mathcal{X}$ and the hidden feasible set $\mathcal{X}^\star$. The objective of the policy is to perform as well as any action from the hidden feasible set $\mathcal{X}^\star$ in terms of the regret and the cumulative constraint violation metrics. 
 
 \begin{figure}
 \centering
 	\includegraphics[scale=0.4]{figures/hidden2.pdf}
 	\caption{\small{Illustrating the \textsc{Hidden Set} problem. In this figure, the sphere $\mathcal{X}^\star$ is the hidden set. On every round $t$, the adversary reveals a hyperplane supporting $\mathcal{X}^\star.$ }}
 \end{figure}
 \fi
 \iffalse
\subsection{Why is the problem non-trivial?} \label{non-trivial}
%We first argue that the \ocs ~problem is non-trivial to solve.
Let us first consider the $\ocs$.
A first attempt to solve the $\ocs$ could be to scalarize it by taking a  \emph{fixed} linear combination (\emph{e.g.,} the sum) of the constraint functions and then running a standard OCO policy on the scalarized cost functions (see \cite[Section 5.3.3]{boyd} for an offline version of the above problem, where the coefficients of the linear combination are taken to be the optimal solution to the dual problem). The above strategy immediately yields a sublinear regret guarantee on the same linear combination (\emph{i.e.,} the sum) of the constraint functions. However, since the constraint functions could take both positive and negative values, the constraint violation component of some streams could still be arbitrarily large even when the overall sum is small. Hence, this strategy does not yield individual cumulative violation bounds, where we need to control the more stringent $\ell_\infty$-norm of the cumulative violation vector. 
%In the particular case of the online constraint satisfaction (\texttt{OCS}) problem 
%if only one constraint function is revealed on each round, by simply running an OCO policy on the given constraint function yields a sublinear regret and hence, a sublinear constraint violation penalty. However, 
%with two or more constraint functions (see Section \ref{simul_constr}), 
Hence, to meet the objective with this scalarization strategy, the ``correct'' coefficients of the linear combination must be learned adaptively in an online fashion. This is exactly what our online meta-policy, described in Section \ref{meta-policy-ocs}, does.

To resolve the above issue, one may alternatively attempt to scalarize the $\ocs$ by considering a non-negative \emph{surrogate} cost function, \emph{e.g.,} the hinge loss function, defined as $\hat{g}_{t,i}(x)=\max(0, g_{t,i}(x)),$ for each constraint $i \in [k]$. However, if the original constraint function $g_{t,i}$ is strongly convex,  this transformation does not necessarily preserve the strong convexity. Furthermore, the above strategy does not work even for convex functions for $S$-feasible benchmarks with $S\geq 2.$ This is because, due to the impossibility of cancellation of positive violations by strictly feasible constraints on different rounds, an $S$-feasible benchmark for the original constraints does not remain feasible for the transformed non-negative surrogate constraints (see Section \ref{ext}). Finally, the above transformation fails in the case of stochastic constraints where the constraint is satisfied only in expectation, i.e., $\mathbb{E} g_t(x) \leq 0, \forall t\geq 1$ \citep{yu2017online}.
%Finally, since we are interested in bounding the maximum violation penalty over any sub-interval in the entire time horizon \eqref{violation-def1}, it is natural to turn to the strongly adaptive algorithms as a subroutine, which are inefficient as they need to run $O(\log T)$ number of experts algorithms on each round \citep{orabona2018scale}. 
The above discussion shows why designing an efficient and universal policy for the \texttt{OCS} problem, and consequently, for the constrained OCO problem - which generalizes \ocs, is highly non-trivial. 

\fi

%\iffalse

\subsection{Online Multi-task Learning as an Instance of the \ocs~ Problem} \label{mtl-ocs}
\begin{figure}[!h]
 \centering
 	\includegraphics[scale=0.45]{figures/multi-task-cropped.pdf}
  \put(-210,110){\small{Shared weights}}
 	\caption{A schematic for the online multi-task learning problem}
 	\label{multi-task}
 \end{figure}
%\paragraph{Example: Online Multi-task Learning:} 
Consider the problem of online multi-task learning where  a single model is trained to perform a number of related tasks \citep{ruder2017overview,dekel2006online, murugesan2016adaptive}. %The instances for each task may be chosen adversarially. 
See Figure \ref{multi-task} for a simplified schematic of the multi-task learning pipeline. In this setup, the action $x_t$ naturally corresponds to the shared weight vector that specifies the common model for all tasks. The loss function for the $j$\textsuperscript{th} task on round $t$ is given by the function $g_{t,j}(\cdot), j \in [k].$ A task is assumed to be satisfactorily completed (\emph{e.g.,} correct prediction in the case of classification problems) on any round if the corresponding loss is non-positive. As an example, using linear predictors for the binary classification problem, the requirement for the $j$\textsuperscript{th} task on round $t$ can be taken to be $g_{t,j}(x_t) \equiv \langle z_{t,j}, x_t\rangle \leq 0,$ where $z_{t,j}$ is the feature vector for the $j$\textsuperscript{th} task. The goal in multi-task learning is to sequentially update the shared weight vectors $\{x_t\}_{t=1}^T$ so that all tasks are successfully completed. Formally, we require that the maximum cumulative loss of each task over any sub-interval grows sub-linearly. Since the weight vector is shared across the tasks, the above goal would be impossible to achieve had the tasks not been related to each other \citep{ruder2017overview}. Theorem \ref{S-benchmark} and Theorem \ref{P_T-benchmark} give performance bounds for Algorithm \ref{ocs-policy} under different task-relatedness assumptions. 
%Hence, we make the feasibility assumption that there exists a fixed admissible action $x^\star$ that can successfully perform all tasks. 
%These assumption often holds in overparameterized neural network models which are known to perfectly fit the data \citep{belkin2019reconciling}. 
%In Theorem \ref{mistake-bd-thm}, we give an explicit mistake bound for the multi-task binary classification problem under the usual $\gamma$-margin assumption for the \textsc{OCS} policy proposed in this paper. This result generalizes the well-known mistake bound for the Perceptron algorithm, which assumes a single task \citep{novikoff1962convergence}. See Section \ref{mistake_bd} for details. Finally, see Section \ref{app} for an application of the tools and techniques developed for the \ocs ~problem to a queueing problem.
%\fi
%\edit{A factor of $D$ (diameter of the set) is missing from some regret bounds. Please correct!}
\iffalse
\subsection{Proof of Theorem \ref{constr-violation}} \label{constr-violation-pf}
\subsubsection{Convex constraint functions} \label{cvx-sec}
From  \eqref{surrogate-def}, we have that 
\[ \nabla \hat{f}_t(x) = 2 \sum_{i=1}^k Q_i(t) \nabla g_{t,i}(x),\]
where recall that $Q_i(t)\ge 0$
Hence, using the triangle inequality for the Euclidean norm, we have 
\[ ||\nabla \hat{f}_t(x_t)||_2 \leq 2 \sum_{i=1}^k Q_i(t) ||\nabla g_{t,i}(x)||_2. \]
 Using the Cauchy-Schwarz inequality and the gradient bounds for the constraint functions as given in Assumption \eqref{bddness}, the squared $\ell_2$-norm of the (sub)-gradients of the surrogate cost functions \eqref{surrogate-def} can be bounded as follows:
\begin{eqnarray} \label{grad-bd2}
	||\nabla \hat{f}_t(x_t)||^2_2 \leq  4 (\sum_{i=1}^k Q_i^2(t))(\sum_{i=1}^k||\nabla g_{t,i}(x_t)||_2^2) \leq kG^2\sum_{i=1}^k Q_i^2(t),
\end{eqnarray}
where we have used the fact that $||\nabla g_{t,i}(x)||_2 \leq G/2, \forall t,i.$
%In the above, we have slightly overloaded the notations by letting $\nabla h$ denote any subgradient of the convex function $h$ if it is not differentiable at the point of interest.
 
%Note that the previous bound depends on the queue lengths, which, in turn, depends on the policy. 
In the \ocs ~meta-policy given in Algorithm \ref{ocs-policy}, let us now take the base OCO policy to be the OGD policy with the adaptive step sizes given in part 1 of Theorem \ref{data-dep-regret}. 
Using \eqref{grad-bd2} in \eqref{cvx-reg-bd}, and substituting that in 
\eqref{q-regret-reln}, we obtain the following sequential inequality:
%Hence, using the OGD policy with adaptive step sizes as before, we have the following inequality for all rounds $t \geq 1:$
\begin{eqnarray} \label{q-ineq}
	\sum_{i=1}^kQ_i^2(t) \leq c \sqrt{\sum_{\tau=1}^t \big(\sum_{i=1}^kQ_i^2(\tau)\big)}, ~t\geq 1, 
\end{eqnarray}
where $c\equiv GD \sqrt{2k}$ is a time-invariant problem-specific parameter that depends on the bounds of the gradient norms, the number of constraints on a round, and the diameter of the admissible set. Note that Algorithm \ref{ocs-policy} is fully \emph{parameter-free} as it uses only available causal information on the constraint functions and does not need to know any parametric bounds (\emph{e.g.,} $G$) on the future constraint functions. 
\paragraph{Analysis:}
From \eqref{V-Q}, we know that to bound CCV, it is sufficient to bound $Q_i(t)$. Next, to get an explicit bound 
on $Q_i(t)$ from recursion \eqref{q-ineq}, we  define the auxiliary variables $Q^2(t)\equiv \sum_{i=1}^k Q_i^2(t), t\geq 1.$ From \eqref{q-ineq}, the variables $\{Q(t)\}_{t\geq 1}$ satisfy the following non-linear system of inequalities that we need to solve.
\begin{eqnarray} \label{q-ineq2}
	Q^2(t) \leq c \sqrt{\sum_{\tau=1}^t Q^2(\tau)} \leq c \sqrt{\sum_{\tau=1}^T Q^2(\tau)},~ 1\leq t \leq T.
\end{eqnarray}
Summing up the above inequality over all rounds $1\leq t \leq T$ and simplifying, we have 
\begin{eqnarray}\label{q-bd-eq-4} 
	\sqrt{\sum_{\tau=1}^T Q^2(t)} \leq cT. 
\end{eqnarray}
Substituting the above bound in inequality \eqref{q-ineq2}, we have 
\begin{eqnarray} \label{cum-viol-bd}
	\max_i \mathbb{V}_i(T) \stackrel{(a)}{\leq} \max_{i=1}^k Q_i(T) \leq Q(T) \leq c\sqrt{T},
\end{eqnarray}
where in inequality (a), we have used \eqref{V-Q}.

%For analytical purposes, we may consider the single scalar quantity $Q(t) \equiv \max(Q_1(t), Q_2(t))$ and obtain an integral inequality for this. 
\begin{observation} \label{sur-obs1}
 By replacing the original constraint function $g_t(x)$ with a surrogate constraint function $\hat{g}_t(x), \forall t\geq 1,$ where the function $\hat{g}_t(x)$ satisfies the assumptions in Section \ref{assump} and enjoys the property that  $\hat{g}_t(x)\geq g_t(x), \forall x \in \mathcal{X},$ we have 
 %the same $O(\sqrt{T})$ cumulative constraint violation guarantee can be established for surrogate constraint functions as well, \emph{i.e.,}
\begin{eqnarray*}
\sum_{t=1}^T g_t(x_t)	\leq \sum_{t=1}^T \hat{g}_t(x_t) = O(\sqrt{T}).
\end{eqnarray*}
This observation is particularly useful when the original constraint functions are non-convex, \emph{e.g.}, $0-1$ loss, which can be upper bounded with the hinge loss function. In particular, we can also define the surrogate function as 
$\hat{g}_t(x) \equiv \psi(g_t(x)),$ where $\psi: \mathbb{R}\to \mathbb{R}_+$ is any non-decreasing, non-negative, and convex function with $\psi(0)=0.$ From basic convex analysis, it follows that the surrogate function is convex.
% the same proof described above gives an $O(\sqrt{T})$ cumulative violation bound for the stronger cumulative surrogate constraint function, \emph{i.e.,}
This observation substantially generalizes our previous bound \eqref{cum-viol-bd} on the cumulative constraint violations. As an example, we can recover \citet{yuan2018online}'s result for time-invariant constraints by defining the surrogate function to be $\hat{g}_t(x)=\psi(g_t(x))\equiv (\max(0,g_t(x)))^2.$ 
%Furthermore, the above transformation can be applied even when the original constraints are not convex. For example, the non-convex $0-1$ loss can be replaced with the hinge loss function. 
\end{observation} 
%the results of \citet{yuan2018online}
\subsubsection{Strongly-convex constraint functions} \label{str-cvx-sec}
Next, we consider the case when the sequence of constraint functions $g_{t,i}$'s are uniformly $\alpha$-strongly convex. Let the base OCO policy be taken as the OGD policy with step sizes chosen in part 2 of Theorem \ref{data-dep-regret}. 
Using \eqref{grad-bd2} in \eqref{str-cvx-reg-bd} for the surrogate functions and the $\alpha$-strong-convexity of the constraint functions, and substituting that in 
\eqref{q-regret-reln} we get
\begin{eqnarray} \label{q-str-cvx-cnstr}
	\sum_{i=1}^k Q_i^2(t) \leq c \sum_{\tau=1}^t \frac{\sum_{i=1}^k Q_i^2(\tau)}{\sum_{s=1}^\tau \sum_{i=1}^kQ_i(s)},~ t\geq 1,
\end{eqnarray}
where $c\equiv \frac{kG^2}{4\alpha}$ is a problem-specific parameter. 

\paragraph{Analysis:} As before, let $Q^2(t) \equiv \sum_{i=1}^kQ_i^2(t), t\geq 1.$ Since the queue variables $Q_i(t)$ are non-negative, we have for any $s \in [T]$:
\begin{eqnarray*}
	\sum_{i=1}^k Q_i(s) \geq Q(s).
\end{eqnarray*}
Hence, from  \eqref{q-str-cvx-cnstr}, we have
\begin{eqnarray} \label{str-q-recur}
	Q^2(t) \leq c \sum_{\tau=1}^t \frac{Q^2(\tau)}{\sum_{s=1}^\tau Q(s)}.
\end{eqnarray}
The following Proposition bounds the growth of any sequence that satisfies the above system of inequalities.
\begin{proposition} \label{q-bd-prop}
Let $\{Q(t)\}_{t \geq 1}$ be any non-negative sequence with 
  $Q(1)> 0$. 
Suppose that the $t$\textsuperscript{th} term of the sequence satisfies the inequality 
\begin{eqnarray} \label{seq_bd}
	Q^2(t) \leq c\sum_{\tau=1}^t  \frac{Q^2(\tau)}{\sum_{s=1}^\tau Q(s)}, ~ \forall t\geq 1,
\end{eqnarray}
where $c >0$ is a constant.
Then $Q(t)\leq  c\ln(t) + O(\ln \ln t), \forall t \geq 1$\footnote{If the sequence is identically equal to zero, then there is nothing to prove. Otherwise, by skipping the initial zero terms, one can always assume that the first term of the sequence is non-zero.}.
\end{proposition}
%See Section \ref{q-bd-prop-pf} in the Appendix for the proof of the above result.

 %\subsubsection{Proof of Proposition \ref{q-bd-prop}} \label{q-bd-prop-pf} 
 \iffalse
\begin{proposition} \label{q-bd-prop}
Let $\{Q(t)\}_{t \geq 1}$ be a non-negative sequence with 
  $Q(1)> 0$. 
Suppose that the $t$\textsuperscript{th} term of the sequence satisfies the inequality 
\begin{eqnarray} \label{seq_bd}
	Q^2(t) \leq c\sum_{\tau=1}^t  \frac{Q^2(\tau)}{\sum_{s=1}^\tau Q(s)}, ~ \forall t\geq 1,
\end{eqnarray}
where $c >0$ is a constant.
Then $Q(t)\leq  c\ln(t) + O(\ln \ln t), \forall t \geq 1$\footnote{If the sequence is identically equal to zero, then there is nothing to prove. Otherwise, by skipping the initial zero terms, one can always assume that the first term of the sequence is non-zero.}.
\end{proposition}
\fi
%\begin{proof}
%Note that if all terms of the sequence are less than $1,$ then there is nothing to prove. Otherwise, by possibly shifting the sequence to left, we can assume that the first term of the sequence $Q(1)$ is at least $1$. 
\begin{proof} Upon dividing each term of the sequence $\{Q(t)\}_{t \geq 1}$ \eqref{seq_bd} by $c,$ without any loss of generality, we may assume that $c=1$. 
	Hence, from Eqn.\ \eqref{seq_bd}, we have the following preliminary bound on the growth of the sequence:
	\begin{eqnarray} \label{prelim_bd1}
		Q^2(t) \stackrel{(a)}{\leq} \sum_{\tau=1}^t \frac{Q^2(\tau)}{Q(\tau)} = \sum_{\tau=1}^t Q(\tau), ~ \forall t \geq 1. 
	\end{eqnarray}
	where in (a), we have used the fact that $\sum_{s=1}^\tau Q(s) \ge  Q(\tau)$ since each term of the sequence $Q(s)$ is non-negative. Substituting the bound \eqref{prelim_bd1} into the RHS of  \eqref{seq_bd}, we obtain
	\begin{eqnarray*}
		Q^2(t) \leq \sum_{\tau=1}^t \frac{\sum_{s=1}^\tau Q(s)}{\sum_{s=1}^\tau Q(s)}= t.
	\end{eqnarray*}
	This yields 
	\begin{eqnarray} \label{pre-bd1}
			Q(t) \leq \sqrt{t}, ~\forall t\geq 1.
	\end{eqnarray}
	
	 Next, for any fixed $t \geq 1,$ let $z=\arg\max_{\tau=1}^t Q(\tau)$ (ties, if any, are broken arbitrarily). Without any loss of generality, we may assume $Q(z) >0$. Then from \eqref{seq_bd}, we have  
	 \begin{eqnarray*}
	 	 Q^2(z) \stackrel{}{\leq}  \sum_{\tau=1}^z  \frac{Q^2(\tau)}{\sum_{s=1}^\tau Q(s)} \stackrel{(b)}{\leq} \sum_{\tau=1}^t\frac{Q^2(\tau)}{\sum_{s=1}^\tau Q(s)} \stackrel{(c)}{\leq} Q(z) \sum_{\tau=1}^t\frac{Q(\tau)}{\sum_{s=1}^\tau Q(s)}
	 \end{eqnarray*}
	 where  in $(b)$, we have used the non-negativity of the sequence $Q(s)$ and by definition $t\ge z$, and in $(c)$, we have used the fact that $Q(z) \geq Q(\tau), 1\leq \tau \leq t.$ Dividing both sides by $Q(z),$ the above inequality yields:
	 \begin{eqnarray*}
	 	Q(z) \leq \sum_{\tau=1}^t\frac{Q(\tau)}{\sum_{s=1}^\tau Q(s)}.
	 \end{eqnarray*}
	 Finally, for any $t\geq 1,$ we have
	 \begin{eqnarray*}
	 	Q(t) \leq Q(z) 
	 	&\leq& \sum_{\tau=1}^t\frac{Q(\tau)}{\sum_{s=1}^\tau Q(s)} \\
	 	&\leq& \sum_{\tau=1}^t \int_{\sum_{s=1}^{\tau-1} Q(s)}^{\sum_{s=1}^\tau Q(s)}\frac{dx}{x} \\
	 	&\leq& 1+\int_{Q(1)}^{\sum_{s=1}^t Q(s)}\frac{dx}{x}\\
	 	&\stackrel{(d)}{=}& \ln(\sum_{s=1}^t Q(s)) + c_1 \\
	 	&\stackrel{(e)}{\leq}& \ln(\sum_{s=1}^t \sqrt{s}) + c_1\leq \frac{3}{2}\ln(t)+c_1,   
	 \end{eqnarray*}
	  where in $(d)$ we have defined the constant $c_1\equiv 1-\ln(Q(1))$ and in $(e)$, we have used the bound given by Eqn.\ \eqref{pre-bd1}. We can tighten the previous bound further by substituting it in the inequality $(e)$ above, which results in
	  \begin{eqnarray*}
	  	Q(t) \leq \ln(\sum_{s=1}^t \frac{3}{2} \ln s + c_1t) \leq \ln(t)+ \ln(c_1+\frac{3}{2}\ln(t)). 
	  \end{eqnarray*}
\end{proof}
\fi
%\subsection{Multi-task Binary Classification Mistake bound implied by the \textsc{OCS} Meta-policy} \label{mistake_bd}
In this section, we consider the online multi-task binary classification problem. We derive an upper bound to the maximum number of mistakes for any task achieved by the \textsc{OCS} policy under a margin assumption.

\paragraph{Problem statement:} Assume that at the beginning of each time $t \geq 1,$ an online policy first chooses a weight vector for a linear classifier $x_t \in \mathbb{R}^d, ||x_t|| \leq 1.$ After that, the policy is given a set of $k$ data points, each belonging to a separate task, with feature vectors  $(z_{t,1}, z_{t,2}, \ldots, z_{t,k}),$ where $z_{t,j} \in \mathbb{R}^d, \forall j,t.$ Without any loss of generality, we scale feature vectors so that $||z_{t,j}||\leq 1, \forall j,t.$ The linear classifier classifies the data points from each task into two classes according to the sign of the inner-product of the weight vector $x_t$ and the corresponding feature vector, \emph{i.e.,} it predicts \[\hat{y}_{t,j}= \textrm{sign}(\langle z_{t,j}, x_t\rangle), ~\forall j,t.\] Finally, the true labels of the data points $(y_{t,1}, y_{t,2}, \ldots, y_{t,j}),$ $y_{t,j} \in \{\pm 1\}$ are revealed to the online policy.
% and then it determines the set of datapoints which has been misclassified, \emph{i.e.,} $M_t= \{j: \hat{y}_{j,t}= y_{j,t}\}.$ 
Let $N_{\mathcal{T},j}$ denote the total number of misclassified data points from the $j$\textsuperscript{th} class up to time $\mathcal{T}, $ \emph{i.e.,}
\[N_{\mathcal{T},j}= \sum_{t=1}^{\mathcal{T}} \mathds{1}(\hat{y}_{t,j} \neq y_{t,j}). \]
Our goal is to upper bound $N\stackrel{\textrm{(def.)}}{=}\sup_{\mathcal{T}} \max_j N_{\mathcal{T},j}.$ We now make the following standard assumption on the data distribution.

\begin{assumption}[$\gamma$-Margin assumption] \label{margin-assumption}
	We assume that each data point can be correctly classified by a weight vector $x^\star, ||x^\star|| = 1,$ with a margin of at least $\gamma,$ i.e., $y_{t,j}\langle z_{t,j}, x^\star\rangle \geq \gamma, \forall j,t.$
\end{assumption}

\subsection{An Online Multi-task Classification Policy}
In the following, we solve the above classification problem using the \textsc{OCS} Meta-policy. The admissible set $\mathcal{X}$ is given by the unit ball and the sequence constraint functions are defined next.
\paragraph{Construction of an \textsc{OCS} instance:}
On each new round, we first check whether there is a data point which has been misclassified by the current classifier. If there is none, we simply skip this round and go to the next round. Otherwise, if there is a data point which has been misclassified, we increment the round counter for the \textsc{OCS} Meta-policy by $1$ and define a set of $k$ linear constraint functions for this round $t$ as follows:
\begin{eqnarray} \label{constr-viol}
	 g_{t,j}(x) = \big(\gamma - y_{t,j} \langle z_{t,j}, x\rangle\big)^+, ~\forall j,t.
\end{eqnarray} 
Clearly, in this case, Assumption \ref{cvx} and Assumption \ref{bddness} are satisfied with $G/2=D=1.$
Furthermore, the feasibility requirement given by Assumption \ref{feas-constr} is satisfied thanks to the $\gamma$-margin Assumption \ref{margin-assumption} above. 
\paragraph{Analysis:} 
Let $N$ be the maximum number of misclassifications for any task, and let $T$ be the number of times for which at least one mistake was made. In other words, $T$ denotes the number of rounds for which the \textsc{OCS} policy was run. Since there are $k$ tasks, we have $T \leq kN.$ Using the definition of the constraint function \eqref{constr-viol}, for a misclassified data point belonging to any task, the corresponding constraint is violated by at least $\gamma.$ Hence, the CCV incurred by the \textsc{OCS} Meta-policy can be lower bounded as 
\[\mathbb{V}_T \geq \gamma N. \]
  Furthermore, using Theorem \ref{constr-violation}, we also have for \textsc{OCS} Meta-policy that
\[ \mathbb{V}_T \leq O(\sqrt{kT}) = O(k\sqrt{N}).\]
Hence, combining the above two inequalities, we conclude that the number of mistakes for any task is upper bounded as $N \leq O(\frac{k^2}{\gamma^2}).$ We formally state the above result in the following theorem. 
\begin{theorem}[Multi-task Mistake bound] \label{mistake-bd-thm}
	Under the $\gamma$-margin assumption (Assumption \ref{margin-assumption}), \textsc{OCS} Meta-policy makes at most $O(\frac{k^2}{\gamma^2})$ mistakes for any task.
\end{theorem}



%\cmt{I think, the dependence of the above mistake bound w.r.t. $k$ can be improved. Think, e.g., running the usual Perceptron algorithm on any task that has been misclassified. Does not it imply an $O(1/\gamma^2)$ mistake bound?}
\subsection{Proof of Theorem \ref{thm:lbcoco}}\label{app:lbcoco}
%\begin{proof}
    We prove the Theorem via constructing an explicit input sequence for which no online policy can have better than $\Omega(\sqrt{T})$ regret and CCV.
%\edit{We are proving only the violation bound and not the regret bound, right?}
\paragraph{Action space $\mathcal{X}$:} 
Let $d=T$. Let  $\mathcal{X}$  be the $d$-dimensional cuboid $0\le x_i \le \frac{1}{\sqrt{d}}, \ 1\le i\le d$. Clearly, the Euclidean diameter of $\mathcal{X}$ is $1$. 


\paragraph{Input:} For each round we will only consider the case when only one constraint is revealed, \emph{i.e.}, $k=1$. On round $t=1, \dots, d$, choose the constraint $g_t$ to be $x_{t} \le \frac{1}{4\sqrt{d}}$ or $x_{t} \ge \frac{3}{4\sqrt{d}}$ with equal probability of $\frac{1}{2}$ for ${\bf x} = (x_1, \dots, x_d) \in \mathcal{X}$. Thus, at round $t$, only the $t\textsuperscript{th}$ dimension has an effective constraint.
If the chosen $g_t$ is $x_{t} \le \frac{1}{4\sqrt{d}}$ then pick $f_t= |x-\frac{1}{4\sqrt{d}}| $, otherwise pick  $f_t= |x-\frac{3}{4\sqrt{d}}|$.




For any online policy ${\cal A}$, the expected constraint violation at round $t$ is at least 
$\frac{1}{8\sqrt{d}}$. Thus, the overall expected constraint violation over rounds $t=1, \dots, d$ is at least 
$\frac{\sqrt{d}}{8}$. Moreover, the expected cost ${\mathbb E}[ f_t(x_t)]$ of ${\cal A}$ is at least $\frac{1}{8\sqrt{d}}$ for each $t=1, \dots, d$, and the overall cost ${\mathbb E}\big[\sum_{t=1}^T f_t(x_t)\big]$ is at least $\frac{\sqrt{d}}{8}$. 


Recall that the choice of input has to satisfy Assumption \ref{feas-constr}, i.e., $\mathcal{X}^\star \neq \emptyset.$ We next demonstrate that for  the prescribed input $\exists \ {\bf x}^\star\in \mathcal{X}^\star$.
\paragraph{Choosing a feasible ${\bf x}^\star$:} When $g_t$ is such that the constraint is $x_{t} \le \frac{1}{4\sqrt{d}}$ choose ${\bf x}^\star \in \mathcal{X}$ such that $x^\star_t = \frac{1}{4\sqrt{d}}$ for $t=1,2,\dots, d$,  while if $g_t$ is such that $x_{t} \ge \frac{3}{4\sqrt{d}}$, then choose $x^\star_t = \frac{3}{4\sqrt{d}}$ for $t=1,2,\dots, d$. 
Thus, a single vector ${\bf x}^\star$ satisfies all the revealed constraints. Moreover, with this choice of ${\bf x}^\star$, the  overall cost of  ${\bf x}$,  $\sum_t f_t({\bf x}^\star)$, is $0$.


Since $d=T$, we get that for any online policy ${\cal A}$ its regret is at least $\Omega(\sqrt{T})$ and the cumulative constraint violation is $\Omega(\sqrt{T})$.$~~~~\blacksquare$


%\end{proof}



%\subsection{Proof of Theorem \ref{thm:lbcoco}}\label{app:lbcoco}
%%\begin{proof}
%Let $d'=\min\{d, T\}$.
%    We prove the Theorem via constructing an explicit input sequence for which no online policy can have better than $\Omega(\sqrt{d'})$ regret and CCV.
%%\edit{We are proving only the violation bound and not the regret bound, right?}
%\paragraph{Action space $\mathcal{X}$:} 
%Let us choose $\mathcal{X}$ to be the $d'$-dimensional cuboid $0\le x_i \le \frac{1}{\sqrt{d'}}, \ 1\le i\le d'$. Clearly, the Euclidean diameter of $\mathcal{X}$ is $1$. 
%
%\paragraph{Input:} For each round we will only consider the case when only one constraint is revealed, i.e. $k=1$. On round $t=1, \dots, d'$, choose the constraint $g_t$ to be $x_{t} \le \frac{1}{4\sqrt{d'}}$ or $x_{t} \ge \frac{3}{4\sqrt{d'}}$ with equal probability of $\frac{1}{2}$ for ${\bf x} = (x_1, \dots, x_{d'}) \in \mathcal{X}$. Thus, at time $t$, only the $t\textsuperscript{th}$ dimension has an effective constraint.
%If the chosen $g_t$ is $x_{t} \le \frac{1}{4\sqrt{d'}}$ then pick $f_t(x)= |x-\frac{1}{4\sqrt{d'}}| $, otherwise pick  $f_t(x)= |x-\frac{3}{4\sqrt{d'}}|$.
%For round $t=d'+1, \dots, T$, $g_t(x)=0, f_t(x)=0, \ \forall \ x $.
%
%For any online policy ${\cal A}$, the expected constraint violation at time $t\le d'$ is at least 
%$\frac{1}{8\sqrt{d'}}$. Thus, the overall expected constraint violation over time slots $t=1, \dots, d'$ is at least 
%$\frac{\sqrt{d'}}{8}$. Moreover, the expected cost ${\mathbb E}[ f_t(x_t)]$ of ${\cal A}$ is at least $\frac{1}{8\sqrt{d'}}$ for each $t=1, \dots, d'$, and the overall cost ${\mathbb E}\big[\sum_{t=1}^T f_t(x_t)\big]$ is at least $\frac{\sqrt{d'}}{8}$. 
%
%Recall that the choice of input has to satisfy Assumption \ref{feas-constr}, i.e., $\mathcal{X}^\star \neq \emptyset.$ We next demonstrate that for  the prescribed input $\exists \ {\bf x}^\star\in \mathcal{X}^\star$.
%\paragraph{Choosing a feasible ${\bf x}^\star$:} When $g_t$ is such that the constraint is $x_{t} \le \frac{1}{4\sqrt{d'}}$ choose ${\bf x}^\star \in \mathcal{X}$ such that $x^\star_t = \frac{1}{4\sqrt{d'}}$ for $t=1,2,\dots, d'$,  while if $g_t$ is such that $x_{t} \ge \frac{3}{4\sqrt{d'}}$, then choose $x^\star_t = \frac{3}{4\sqrt{d'}}$ for $t=1,2,\dots, d'$. 
%Thus, a single vector ${\bf x}^\star$ satisfies all the revealed constraints. Moreover, with this choice of ${\bf x}^\star$, the  overall cost of  ${\bf x}$,  $\sum_t f_t({\bf x}^\star)$, is $0$.
%
%%Choosing $d=T$, thus we get that for any online policy ${\cal A}$ its regret is at least $\Omega(\sqrt{T})$ and the cumulative constraint violation is $\Omega(\sqrt{T})$.$~~~~\blacksquare$
%
%%\end{proof}
\iffalse
\subsection{Comments on OCO Meta-policy}\label{app:commentsMetaOCO}

 It can be verified that the surrogate cost function sequence $\{\hat{f}_t\}_{t\geq 1}$ is a legitimate input to any base OCO policy as all the ingredients required to compute the surrogate cost function $\hat{f}_t$ (\emph{i.e.,} $f_t, Q(t),\textrm{and}~ g_t$) are known to the policy at the end of round $t$. 
 
 \textbf{Remarks:} 
2.  Although we introduce the auxiliary queue-length process $\{Q(t)\}_{t \geq 1}$ as a convenient mathematical tool to bound the constraint violation penalty $\mathbb{V}_T$, in some problems, the process defined in \eqref{q-ev} arises quite naturally as the evolution of some physical queueing process. In these problems, controlling the queue length itself is of primary interest. When the penalty function $\psi(g_(\cdot))$ can assume negative values as well (which is the case, \emph{e.g.,} when $\psi$ is the identity function and $g_t$'s are linear functions, upper bounding the queue lengths is strictly harder than upper bounding the violation penalty.
\fi
\iffalse
\textbf{Remarks:} 
%1. As in the \ocs ~problem, we are taking full advantage of the adaptive nature of the OCO sub-routine by exploiting the fact that the adversary is allowed to choose the surrogate cost function $\hat{f}_t$ \emph{after} seeing the current action $x_t$ on any round. %The policy is illustrated in the following pseudocode.
 Since $\nabla \hat{f}_t(x_t)= V \nabla f_t(x_t) + 2Q(t)\nabla g_t(x_t),$ the meta-policy, in reality, needs to pass only the (sub-)gradients $\nabla f_t(x_t), \nabla g_t(x_t)$, and the current violation $g_t(x_t)$ to an OGD sub-routine. The full description of the functions at every point in its domain is \emph{not required} by the meta-policy, and hence, the proposed policy is more efficient compared to the policy proposed by \citet{guo2022online}, which needs to solve a convex optimization problem involving the constraint function $g_t$. 
 \fi
%\hrule 

%\subsection{Boundedness Assumption} \label{bdd-assumption}
\iffalse
Using the triangle inequality for the Euclidean norms, the norm of the (sub)-gradients of the surrogate cost functions can be bounded as follows:
\begin{eqnarray} \label{grad-bd}
	||\nabla \hat{f}_t(x_t)||_2 \leq V ||\nabla f_t(x_t)||_2 + 2Q(t)||\nabla g_t(x_t)||_2 \leq (V+Q(t))G.
\end{eqnarray}
The above bound on the norm of the gradient of the surrogate costs depends on the queue length, and hence, it depends on the past actions of the online policy.
\fi
\subsection{Comparison with Previous Works}
\label{app:comparisonpolicies}
\subsubsection{ \cite{neely2017online}, \cite{yu2017online} and \cite{georgios-cautious}}
%At a high-level our proposed policy is similar to  \cite{neely2017online}, \cite{yu2017online} and \cite{georgios-cautious}.  
Policies  proposed by \cite{yu2017online} and \cite{georgios-cautious} are almost  identical to \cite{neely2017online}. The policy proposed in 
\cite{neely2017online}, however, is highly customized, does not fully exploit 
the best guarantees available for the standard OCO problem, and obtains sub-optimal performance bounds that depend inversely on Slater constant, which is assumed to be strictly positive. In a nutshell, \cite{neely2017online} choose the next action $x_{t+1}$ using the algorithm described below. For all rounds $t \geq 1, $ define the following evolution for $Q(t):$
 \begin{eqnarray} \label{q-ev}
 	Q(t) = (Q(t-1)+ g_t(x_t) + \nabla^T g_t(x_t)(x_t-x_{t-1}))^+, ~Q(0)=0. 
	\end{eqnarray}
	The next action is chosen by solving the following quadratic optimization problem: 
	$$x_{t+1} = \arg \min_{x\in {\mathcal X} } \big[\langle V \nabla^T f_{t}(x_{t}) + Q(t)\nabla g_t(x_t), x \rangle + \alpha ||x-x_{t-1}||^2\big],$$
	where $V$ and $\alpha$ are suitably chosen parameters. 
	
	In comparison, we have a different and simpler update rule:
	 \begin{eqnarray} \label{q-ev}
 	Q(t) = Q(t-1)+ (2GD)^{-1}(g_t(x_t))^+, ~Q(0)=0. \end{eqnarray}
	We then construct a convex surrogate function $\hat{f}_t(x) \equiv f_t(x) + \frac{1}{4GD\sqrt{T}}e^{\frac{Q(t)}{2\sqrt{T}}}(g_t(x))^+,$ whose gradient is then passed directly to the AdaGrad subroutine.
	\paragraph{Remarks:} We emphasize that Theorem \ref{main_result}, which shows that it is possible to simultaneously achieve $O(\sqrt{T})$ regret and $\tilde{O}(\sqrt{T})$ CCV in the convex setting without assuming Slater's condition, is highly surprising and unexpected. In fact, \citet[Section 4]{georgios-cautious} had previously commented that: 
	
	"\emph{$\ldots$ On the other hand, the point $O(\sqrt{T})$, $O(\sqrt{T})$ achieved by \citet{neely2017online} for $K = 1$ is not part of our achievable guarantees; we attribute this gap to the stricter Slater assumption studied by \citet{neely2017online}." }  \\\\
	Theorem \ref{main_result} squarely falsifies the last conjecture. 
 \subsubsection{ \cite{guo2022online}}
The policy in \cite{guo2022online} is a slightly modified form of the policy proposed in \citep{neely2017online}. In particular, it chooses the action $x_t$ by solving the following quadratic optimization problem over $\mathcal{X}:$
\begin{eqnarray*}
	x_t = \arg \min_{x\in {\mathcal X}} \big[ \langle \nabla f_{t-1}(x_{t-1}), x-x_{t-1} \rangle + Q(t-1) \gamma_{t-1} g_{t-1}^+(x) + \alpha_{t-1}|| x- x_{t-1}||^2 \big],
\end{eqnarray*} 
where the $Q$ variables are updated as follows:
\[ Q(t)= \max(Q(t-1)+ \gamma_{t-1} g_{t-1}^+(x_t), \eta_t).\]


\iffalse
based on the Lagrangian principle of solving constrained optimization problem, where the 
Lagrangian is 
$$ L (x,\lambda_t) = f_t(x) + \lambda_t g_t(x).$$
%Since $f_t$ and $g_t$ are revealed after action $x_t$ is chosen, 
The cost function $f_t$ is linearly approximated by 
$${\hat f}_t(x)  = f_{t-1}(x_{t-1}) + \langle \nabla f_{t}(x_{t-1}), x-x_{t-1}\rangle ,$$
and the constraint function $g_t$ is replaced by $\gamma_{t-1}g_{t-1}(x)^+$, $\lambda_t$ by $Q_t$ which is updated as  $Q(t) = \max(Q(t-1)+ \gamma_{t-1}g_{t-1}(x_{t-1})^+, \eta_t)$ and a quadratic regularization term $\alpha_{t-1} ||x-x_{t-1}||^2$ is added to the Lagrangian. 
\fi

Here $\alpha_t, \eta_t, \gamma_t$ are suitably chosen learning rate parameters. Essentially, this policy is trying to find the local optimum of an augmented Lagrangian under the online information model ($f_t$ and $g_t$ are revealed after action $x_t$ is chosen). Since their augmented Lagrangian involves the constraint function $g_{t-1},$ their policy needs to solve a full-fledged constrained convex optimization problem over the set $\mathcal{X}$ after having full access to the constraint function. In comparison, our policy, rather than using approximations to Lagrangian and adding regularizers, makes full use of the well-developed theory for OCO and uses first-order methods that need to compute only a gradient and perform one Euclidean projection on each round.
\subsubsection{\cite{jenatton2016adaptive}}
The policy proposed by \cite{jenatton2016adaptive} is based on the idea of primal-dual algorithm for optimizing the augmented Lagrangian 
$$ L_t (\lambda,x) = f_t(x) + \lambda g_t(x) - \frac{\theta_t}{2} \ \lambda^2,$$ where $\frac{\theta_t}{2} \lambda^2$ is the augmentation term. The primal variable $x_t$ and the dual variable $\lambda_t$ are updated by executing projected gradient descent and gradient ascent on the Lagrangian as follows:
$$x_{t+1} = \mathcal{P}_\mathcal{X}(x_t - \eta_t \nabla_x L_t (x_t, \lambda_t))$$ and 
$$\lambda_{t+1} = (\lambda_t + \mu_t \nabla_\lambda L_t (x_t, \lambda_t))^+,$$
where $\theta_t, \eta_t$, and $\mu_t$ are parameters to be chosen.





\iffalse
\subsection{Proof of Theorem \ref{gen-cvx-bd}} \label{gen-cvx-bd-pf}
\paragraph{Bounding the constraint violations:}
%Since the (sub)-gradients of the cost function is assumed to be bounded, 
Since the cost functions are assumed to be Lipschitz with Lipschitz constant $G$ (Assumption \ref{bddness}),
we have $\textrm{Regret}_t(x^\star) \geq -GDt.$ Hence, from  \eqref{main_eq}, we have 
	\begin{eqnarray} \nonumber 
	Q^2(t) &\leq&VGD t + 2GDV\sqrt{t} + 2GD \sqrt{\sum_{\tau=1}^t Q^2(\tau)} \\\label{eq12}
		 &\leq& VG\delta t + 2GD \sqrt{\sum_{\tau=1}^t Q^2(\tau)}.
	\end{eqnarray}
where we have defined $\delta \equiv D(1+\frac{2}{\sqrt{t}}) \leq 3D.$ Hence, for all $1\leq \tau \leq t,$ we have 
\begin{eqnarray*}
	Q^2(\tau) \leq 3VGD t+ 2GD \sqrt{\sum_{\tau=1}^t Q^2(\tau)}.
\end{eqnarray*}
Summing up the above inequalities for $1\leq \tau \leq t,$ we obtain
\begin{eqnarray*}
	\sum_{\tau=1}^t Q^2(\tau) \leq 3VGD t^2 + 2GDt\sqrt{\sum_{\tau=1}^t Q^2(\tau)}. 
\end{eqnarray*}
Solving the above quadratic inequality in $x$ with $x\equiv \sqrt{\sum_{\tau=1}^t Q^2(\tau)}$, we conclude
\begin{eqnarray} \label{quad-bd}
\sqrt{\sum_{\tau=1}^t Q^2(\tau)} \leq t (2GD+ \sqrt{3VGD}).
\end{eqnarray}	
Finally, combining the bound \eqref{quad-bd} with  \eqref{eq12}, we obtain 
\begin{eqnarray}\label{eq:dummy1}
	Q^2(t) \leq O(Vt)+ O(t\sqrt{V}) \implies Q(t) = O(\sqrt{Vt}) \stackrel{(\text{choosing} \ V=\sqrt{T})}{\implies} Q(t)= O(T^{3/4}).
\end{eqnarray} 
The above concludes the proof of the CCV bound as claimed in the theorem. $\blacksquare$

\paragraph{Note:} In the proof above, we have not used the fact that the constraint functions are non-negative. Hence, the above violation bound holds for constraints which are not necessarily pre-processed (see Section \ref{gen_oco}). However, the rest of the proof exploits the non-negativity of the pre-processed constraint functions.
% where we redefine $g_t(x)\gets \max(0, g_t(x)), \forall t.$ 
\paragraph{Bounding the regret:}

\iffalse
Plugging in the above bound in \eqref{main_eq}, we conclude that 
\begin{eqnarray*}
	V\textrm{Regret}_t(x^\star) \leq 2Gt(2G+ \sqrt{VG\delta}) + 2GV \sqrt{t}  \implies \textrm{Regret}_t(x^\star)= O(\frac{t}{\sqrt{V}}+ \sqrt{t}). 
\end{eqnarray*}
Setting $V=\sqrt{T}$ yields $\textrm{Regret}_t(x^\star) = O(t^{3/4}).$ 
\fi
Since the constraint functions are non-negative, the sequence $\{Q(t)\}_{t \geq 1}$ is monotone non-decreasing. Hence, from  \eqref{main_eq} using the fact that $\sum_{\tau=1}^t Q^2(\tau)\le \sqrt{t}Q(t)$, we have 
\begin{eqnarray*}
	V\textrm{Regret}_t(x^\star) \leq 2GDQ(t)\sqrt{t}-Q^2(t)+ 2GDV\sqrt{t},
\end{eqnarray*}
where, in the first term on the RHS, we have used the monotonicity of the queue-length sequence. Completing the square on the RHS, we have 
\begin{eqnarray} \label{reg-plus-q-sq-bd}
	V\textrm{Regret}_t(x^\star) &\leq& 2GD V\sqrt{t} + G^2D^2t - (Q(t) - GD\sqrt{t})^2 \\
	&\leq & 2GD V\sqrt{t} + G^2D^2t.
\end{eqnarray}
Hence, similar to \eqref{eq:dummy1}, we conclude that
\begin{eqnarray*}
	\textrm{Regret}_t(x^\star) \leq 2GD \sqrt{t} + \frac{G^2D^2t}{V} \stackrel{(\text{choosing} \ V=\sqrt{T})}{\implies}\textrm{Regret}_t(x^\star)= O(\sqrt{t}). ~\blacksquare
\end{eqnarray*}
\paragraph{Bounding the queue-length when the regret is non-negative:}
We now establish the improved queue length bound for rounds where the worst-case regret is non-negative. Consider a round $t\geq 1$ where $\sup_{x^\star \in \mathcal{X}^\star}\textrm{Regret}_t(x^\star) \geq 0.$ Then from \eqref{reg-plus-q-sq-bd}, we have 
\begin{eqnarray*}
	(Q(t) - GD\sqrt{t})^2 \leq 2GD V\sqrt{t} + G^2D^2t.
\end{eqnarray*}
%Hence, we have 
This yields
\begin{eqnarray*}
	Q(t) \leq GD\sqrt{t} + \sqrt{2GD V\sqrt{t}}+ GD \sqrt{t}. 
\end{eqnarray*}
Setting $V=\sqrt{T},$ the above bound yields
\begin{eqnarray*}
	Q(t)= O((Tt)^{1/4})+O(\sqrt{t}) = O(\sqrt{T}),~ \forall t \geq 1.~~~~\blacksquare 
\end{eqnarray*}
\fi
\iffalse
\subsection{Proof of Theorem \ref{cvx-lb}} \label{cvx-lb-pf}
%\begin{proof}
\edit{The proof has holes. What if we can conclude $x^\star$ by inverting $b_t=g_t(x^\star)$?}


	Assume that on round $t \geq 1$, the adversary chooses a convex cost function $g_t(x)$ and a constraint $g_t(x) \leq b_t,$ where the constant $b_t$ will be specified later. Let $x^\star \in \mathcal{X}^\star$ be a feasible action in hindsight, which minimizes the cumulative cost $\sum_{t=1}^T g_t(x).$ We now choose the constant $b_t$ as $b_t = g_t(x^\star), t \in [T].$ Clearly, $x^\star \in \mathcal{X}^\star$ is a feasible point that satisfies all constraints. Hence, the cumulative violation penalty incurred by any online policy $\pi$ over the entire horizon $T$ is given by: 
	\begin{eqnarray*}
	\sup_{\mathcal{I}} \mathbb{V}_{\mathcal{I}} \geq \mathbb{V}_T = \sum_{t=1}^T \big(g_t(x_t) - b_t\big) = \sum_{t=1}^T g_t(x_t) - \sum_{t=1}^T g_t(x^\star) \equiv \textrm{Regret}_T(x^\star) \geq \mathcal{X}(\sqrt{T}), 
	\end{eqnarray*} 
	where the last inequality follows from the lower bound on the achievable regret for adversarially chosen convex cost functions over a convex domain \citep[Table 3.1]{hazan2016introduction}. This establishes part 1 of the Theorem. The proof for the strongly convex costs is exactly similar, where we choose the functions $g_t$ to be $\alpha$-strongly convex instead and use the corresponding lower bound for the unconstrained OCO problem. 
%\end{proof}

\fi
\iffalse
\subsection{Proof of Theorem \ref{cvx-slater}} \label{cvx-slater-pf}
\begin{proof}
%	Since the (sub)-gradient of the cost function is bounded, we have $\textrm{Regret}_t(x^\star) \geq -GDt.$ Hence, \eqref{main_eq} gives
%	\begin{eqnarray} \label{eq1}
%		Q^2(t) + 2\eta^\star\sum_{\tau=1}^t Q(\tau) \leq VGD t + 2GV\sqrt{t} + 2G \sqrt{\sum_{\tau=1}^t Q^2(\tau)}.
%	\end{eqnarray}
%	Define $\delta \equiv \frac{\max(2,D)}{3}.$ From the above equation, we have that for each $1\leq \tau \leq t:$
%	\begin{eqnarray*}
%		Q^2(\tau) \leq VG\delta \tau + 2G\sqrt{\sum_{\tau=1}^t Q^2(\tau)}. 
%	\end{eqnarray*}
%	Summing up the above inequalities for $\tau=1$ to $\tau=t,$ and defining $S_t= \sqrt{\sum_{\tau=1}^t Q^2(\tau)},$ we obtain the following quadratic equation in $S_t:$
We start from Eqn.\ \eqref{quad-bd} which states that
	\begin{eqnarray*}
		\sqrt{\sum_{\tau=1}^t Q^2(\tau)} \leq t(2G+ \sqrt{VG\delta}).
	\end{eqnarray*} 
Substituting the above bound into \eqref{main_eq} and lower bounding the non-negative terms on the left by zero, we obtain the following upper bound on the average queue lengths
\begin{eqnarray} \label{eq3_thm2}
	 \frac{1}{t}\sum_{\tau=1}^t Q(\tau) \leq \frac{1}{2\eta^\star}\big(VG \delta + 4G^2+2G\sqrt{VG\delta}\big)= O(V).  
\end{eqnarray}
This proves the first part of the theorem.
Next, upper-bounding the $\ell_2$-norm with the $\ell_1$-norm in Eqn.\ \eqref{eq12}, and using the upper bound \eqref{eq3_thm2} for the average queue length, we conclude that 
\begin{eqnarray} \label{eq4_thm2}
	Q^2(t) \leq t(VG\delta+  \frac{2G}{t}\sum_{\tau=1}^t Q(\tau) ) ~ \implies Q^2(t) \leq c_1 Vt,
\end{eqnarray}
where $c_1$ is a constant that depends only on the parameters $G, D,$ and  $\eta^\star.$ Using Eqn.\ \eqref{main_eq} again and dropping the non-negative terms on the LHS, we conclude
\begin{eqnarray*}
	V \textrm{Regret}_t(x^\star) &\leq& 2G \sqrt{\max_{1\leq \tau \leq t}Q(\tau) \sum_{\tau=1}^t Q(\tau)}+2GV \sqrt{t}\\
	&\stackrel{(a)}{\leq}& c_2 (Vt)^{3/4} + 2GV\sqrt{t}
\end{eqnarray*}
where $c_2$ is a constant that depends on $G, D,$ and  $\eta^\star.$ In (a) above, we have used the bounds in \eqref{eq3_thm2} and \eqref{eq4_thm2} for separately bounding the maximum and the average queue lengths. The above inequality finally leads to the following regret bound:
\begin{eqnarray*}
	\textrm{Regret}_t(x^\star) \leq c_2 \frac{t^{3/4}}{V^{1/4}} + 2G\sqrt{t} \implies \textrm{Regret}_t(x^\star) \stackrel{(V=\sqrt{T})}{=} O(T^{5/8}).
\end{eqnarray*}
%where in the last inequality, we have set $V=\sqrt{T}.$
\end{proof}
\subsection{Proof of Theorem \ref{positive-regret}} \label{positive-regret-proof}
\begin{proof}
	Using the assumption in Eqn.\ \eqref{main_eq}, we have 
	\begin{eqnarray} \label{t4eq1}
		Q^2(t) \leq 2G \sqrt{\sum_{\tau=1}^t Q^2(\tau)}+ 2GV\sqrt{t}.
	\end{eqnarray}
	The above set of inequalities implies for each $t \geq 1:$
	\begin{eqnarray*}
		\sum_{\tau=1}^t Q^2(\tau) \leq 2Gt\sqrt{\sum_{\tau=1}^t Q^2(\tau)}+ 2GV t^{3/2}.
	\end{eqnarray*}
	Solving the above quadratic inequality, we obtain 
	\begin{eqnarray} \label{t4eq2}
		\sqrt{\sum_{\tau=1}^t Q^2(\tau)} \leq 2Gt + \sqrt{2GV}t^{3/4}.
	\end{eqnarray}
	Plugging in the above bound in \eqref{t4eq1}, we obtain
	\begin{eqnarray*}
		Q(t) \leq O(\sqrt{t}) + O(V^{1/4}t^{3/8}) + O(V^{1/2}t^{1/4}) \stackrel{(V=\sqrt{T})}{\implies} Q(t)=O(\sqrt{T}).
	\end{eqnarray*}
	Finally, plugging \eqref{t4eq2} into \eqref{main_eq}, we obtain for any $x^\star \in \mathcal{X}_t:$
	\begin{eqnarray*}
		\textrm{Regret}_t(x^\star) \leq O(\frac{t}{V}) + O(\frac{t^{3/4}}{\sqrt{V}}) + O(\sqrt{t}) \stackrel{(V=\sqrt{T})}{\implies} 	\textrm{Regret}_t(x^\star) = O(\sqrt{T}).
	\end{eqnarray*}
\end{proof}
\fi
 \subsection{Proof of Theorem \ref{str-cvx-bd}} \label{str-cvx-pf}
  %\subsubsection{Preliminaries} \label{prelim}
  \paragraph{Bounding the CCV:} 
  \iffalse
  We give two different proofs for the constraint violation bound. The first proof does not assume the non-negativity of the constraint functions and, hence, works for un-pre-processed constraints and is slightly more complex. The second proof exploits the non-negativity of the constraint functions and is elementary. 
  \paragraph{Proof:}
     In this proof, we will use the classical Gr\"onwall's inequality for bounding the growth of the queue length sequence $\{Q(t)\}_{t \geq 1}.$ For the ease of reference, we state the result below. 
%\begin{framed}
\begin{theorem}[(Gr\"onwall's inequality \citep{bainov1992integral})]
Let $I$ denote an interval $[a,b],$ $\alpha : I \to \mathbb{R}$ be a non-decreasing function and $\beta : I \to \mathbb{R}$ be a non-negative function. Let the continuous function $u : I \to \mathbb{R}$ satisfies the following integral inequality:
\begin{eqnarray*}
	u(t) \leq \alpha(t)+\int_{a}^t \beta(\tau) u(\tau) d\tau. 
\end{eqnarray*} 
Then we have
\begin{eqnarray} \label{gw_ineq}
	u(t) \leq \alpha(t) \exp\left( \int_{a}^t \beta(\tau) d\tau \right).
\end{eqnarray}
\end{theorem}
%\end{framed}
 Since the (sub)-gradients of the cost function are assumed to be bounded, we have $\textrm{Regret}_t(x^\star) \geq -GDt.$
%As in the previous proof, we have $\textrm{Regret}_t \geq -GDt.$
%\subsection{Analysis}
 Hence, from inequality \eqref{Gronwall-ineq}, we have that: 
\begin{eqnarray} \label{str-cvx-eqn}
	Q^2(t) \leq VG\delta t+ \frac{G^2}{\alpha V}\sum_{\tau=1}^t \frac{Q^2(\tau)}{\tau},  
\end{eqnarray}
where we have defined $\delta\equiv (\frac{G}{\alpha}+D).$
Clearly, we can extend any non-negative sequence $\{Q^2(\tau)\}_{\tau \geq 1}$ to a continuous function by linearly interpolating the sequence values between every two integers. Furthermore, since the sequence is non-negative, the integral of the continuous extension up to any integer $t$ is at least half of the sum of the first $t$ elements of the sequence. Hence, the resulting continuous extension $\big(Q^2(t), \nicefrac{1}{2} \leq t \leq T\big)$ satisfies the following integral inequality:
\begin{eqnarray*}
	Q^2(t) \leq VG\delta t + \frac{2G^2}{\alpha V} \int_{\nicefrac{1}{2}}^t \frac{Q^2(\tau)}{\tau}d\tau.
\end{eqnarray*}
Using Gr\"onwall's inequality \eqref{gw_ineq} with $u(t)\equiv Q^2(t)$, it immediately follows that 
\begin{eqnarray} \label{Q-bd-str-cvx}
	Q^2(t) \leq VG\delta t (2t)^{\frac{2G^2}{\alpha V}}, ~\forall t \geq 1.
\end{eqnarray}
\iffalse
Plugging in the above bound in \eqref{Gronwall-ineq} leads to the following regret bound $\forall x^\star \in \mathcal{X}_t$: 
\begin{eqnarray*}
	\textrm{Regret}_t(x^\star) &\leq& \frac{G^2}{\alpha} \ln(t) + \frac{2G^2}{\alpha V^2}VG\delta \int_{\nicefrac{1}{2}}^{t}(2\tau)^{\nicefrac{2G^2}{\alpha V}}d\tau \\
	&\leq& \frac{G^2}{\alpha} \ln(t) + \frac{2G^3\delta}{\alpha V}t (2t)^{\nicefrac{2G^2}{\alpha V}}.
\end{eqnarray*}
\fi
Hence, upon setting $V \geq \frac{2G^2}{\alpha} \ln(2T),$ we obtain the following bound for the queue length 
\begin{eqnarray*}
	Q(t) \leq 2 \sqrt{VG\delta t}= O(\sqrt{V t \log T}).~\blacksquare
	 %~ \textrm{Regret}_t(x^\star) \leq O\big(\ln(t)\big)+O\big(\frac{t}{V}\big). 
\end{eqnarray*}
As in the convex case, we have not used the fact that the constraints are non-negative in the above proof. Hence, the above bound holds for constraints which are not necessarily pre-processed (see Section \ref{gen_oco}). The remaining part of the proof exploits the non-negativity of the constraints where we redefine $g_t(x)\gets \max(0, g_t(x)), \forall t \geq 1.$
\fi
%Recall that with the abuse of notation we have used $g_t(x) = \max\{g_t(x),0\}$ in the COCO case. Thus, the queue length sequence $Q(t)$ is monotone non-decreasing. Hence, 
Choosing $\Phi(x)=x^2$ in Eqn.\ \eqref{Gronwall-ineq}, we have for any feasible $x^\star \in \mathcal{X}^\star:$ 
\begin{eqnarray} \label{q-reg-str-cvx-bd}
	Q^2(t) + V\textrm{Regret}_t(x^\star) \leq \frac{VG^2}{\alpha} (1+\ln(t)) + \frac{4G^2Q^2(t)\ln(Te)}{\alpha V},
\end{eqnarray}
where, on the last term in the RHS, we have used 
%the non-decreasing property of the queue lengths and 
the fact that $t\leq T$. Setting $V = \frac{8G^2 \ln(Te)}{\alpha},$ and transposing the last term on the RHS to the left, the above inequality yields
\begin{eqnarray} \label{Q-bd-str-cvx}
	Q^2(t) + 2V\textrm{Regret}_t(x^\star) \leq \frac{2VG^2}{\alpha} (1+\ln(t)).
\end{eqnarray}
%Since the (sub)-gradients of the cost function is assumed to be bounded, as before, the regret at any round is lower bounded as $\textrm{Regret}_t(x^\star) \geq -GDt.$ 
Since the cost functions are assumed to be $G$-Lipschitz (Assumption \ref{bddness}),
we trivially have $\textrm{Regret}_t(x^\star) = \sum_{t=1}^T (f_t(x_t)-f_t(x^\star)) \geq -GDt.$
Hence, from Eqn.\ \eqref{Q-bd-str-cvx}, we obtain
\begin{eqnarray*}
	Q^2(t) \leq 2VGDt + \frac{2VG^2}{\alpha} (1+\ln(t)) \implies Q(t) \stackrel{(a)}{\leq} 4G \sqrt{\frac{GD}{\alpha}t \ln(Te)} + \frac{4G^2 \ln(Te)}{\alpha}.
\end{eqnarray*}
where step (a), we have substituted $V = \frac{8G^2 \ln(Te)}{\alpha}.$ Hence, we have the following bound $\textrm{CCV}_t = O\big(\sqrt{\frac{t \log T}{\alpha}}\big).$

\paragraph{Bounding the regret:} Using the above choice for the parameter $V$ and the fact that $Q^2(t)\geq 0,$ from Eqn.\ \eqref{Q-bd-str-cvx}, we have
%in Eqn.\ \eqref{q-reg-str-cvx-bd}, we have 
\begin{eqnarray*}
	2V\textrm{Regret}_t(x^\star) \leq \frac{2VG^2}{\alpha} (1+\ln(t)). 
	%- \frac{Q^2(t)}{2} \leq \frac{VG^2}{\alpha} \ln(t).
\end{eqnarray*}
%where we have set $V = \frac{2G^2 \ln(Te)}{\alpha}.$
This leads to the following logarithmic bound for regret for any feasible $x^\star \in \mathcal{X}^\star:$ 
\[ \textrm{Regret}_t(x^\star) \leq \frac{G^2}{\alpha} (1+\ln(t)).~~~~\blacksquare \]

\paragraph{A sharper CCV bound under the non-negative regret assumption:} We now establish an improved CCV bound when the worst-case regret is non-negative on  some round $t \geq 1$. Let $\sup_{x^\star \in \mathcal{X}^\star}\textrm{Regret}_t(x^\star) \geq 0$ for some round $t \geq 1.$ Letting $V = \frac{8G^2 \ln(Te)}{\alpha}$ as above, from Eq.\ \eqref{Q-bd-str-cvx} we have 
\begin{eqnarray*}
	Q^2(t) \leq \frac{2VG^2}{\alpha} (1+\ln(t)) \implies Q(t) = O\big(\frac{\ln T}{\alpha}\big), t \in [T]. ~\blacksquare
\end{eqnarray*}
\paragraph{Comment:} From the above proof, it immediately follows that the same conclusion holds even under the weaker assumption of $-\textrm{Regret}_T= O(\frac{\log T}{\alpha}).$ 
\subsection{Adversaries Ensuring Non-negative Regret} \label{improved_rates}
%In Theorem \ref{gen-cvx-bd} and \ref{str-cvx-bd}, we showed that under the assumption of the non-negativity of the worst-case regret, the constraint violation bounds can be improved to $O(\sqrt{{T}})$ and $O(\ln T/\alpha)$ for convex and strongly-convex cost functions, respectively. In this section, we additionally show that the same improved bounds hold in the case of a time-invariant fixed constraint function and a certain class of worst-case adversaries, called \emph{convex adversary}, defined next. COCO with time-invariant constraints has been studied in the literature where the main objective is to design gradient-based first-order policies that avoid the costly projection step on the constraint set \citep{jenatton2016adaptive, yuan2018online}. 

\paragraph{Convex adversary:} An adversary is called \emph{convex} if for any sequence of  action sequence $\{x_t\}_{t=1}^T,$ the adversary chooses the cost function sequence $\{f_t\}_{t=1}^T$ such that for any $T \geq 1,$ we have
\begin{eqnarray} \label{jensenadv}
	\sum_{t=1}^T f_t(x_t) \geq \sum_{t=1}^T f_t(\bar{x}_T),
\end{eqnarray}     
where $\bar{x}_T \equiv \frac{1}{T}\sum_{t=1}^T x_t.$ Hence, by definition, a convex adversary guarantees a non-negative regret with respect to the average action $\bar{x}_T$ for all rounds. In the following, we give two examples of convex adversaries.


 \paragraph{1. Fixed adversary:} An adversary which always selects a fixed convex function $f$ on all rounds is a convex adversary. In this case, Eqn.\ \eqref{jensenadv} holds due to the Jensen's inequality. 

\paragraph{2. Minimax adversary:} Let $\mathcal{F}$ denote an arbitrary non-empty set of convex functions defined on the admissible set $\mathcal{X}$. Consider an adversary $\mathcal{M}$, which, upon seeing the selected action $x_t$, chooses the worst cost function $f_t$ from the set $\mathcal{F}$ on round $t:$ 
\[f_t \in \arg\max_{f\in \mathcal{F}} f(x_t). \]
We now show that $\mathcal{M}$ is a convex adversary. By definition, for any round $\tau \in [T],$ we have 
\[ f_\tau(x_\tau) \geq f_t(x_\tau) \implies f_\tau(x_\tau) \geq \frac{1}{T} \sum_{t=1}^T f_{t}(x_\tau). \]
Summing up the above inequalities for each $\tau \in [T],$ we have 
\begin{eqnarray}\label{conv-adv-def}
 \sum_{\tau=1}^T f_\tau(x_\tau) \geq \sum_{t=1}^T \frac{1}{T}\sum_{\tau=1}^T f_t(x_{\tau}) \stackrel{\textrm{(a)}}{\geq} \sum_{t=1}^T f_t(\bar{x}_T),
 \end{eqnarray}
where inequality (a) follows upon applying Jensen's inequality to each cost function. 
Eqn.\ \eqref{conv-adv-def} shows that $\mathcal{M}$ is a convex adversary. 

P.S. It can be easily seen that Fixed adversary is a special case of Minimax adversary where $\mathcal{F}=\{f\}.$
\iffalse
\subsection{Assumptions}
\begin{assumption}[Time-invariant constraints] \label{constr_assump1}
Assume that the constraint functions on every round are fixed and given by $g_t=g.$ Hence, the feasible set is defined as:
\begin{eqnarray*}
	\mathcal{X}^\star = \{x^\star \in \mathcal{X}: g(x^\star) \leq 0\}.  
\end{eqnarray*}
\end{assumption}
\begin{assumption}[Convex adversary] \label{adversary_assump2}
	The cost functions are chosen by a convex adversary. 
\end{assumption}
\begin{assumption}[Bounded dual optimal variable] \label{sensitivity}
	Let $\bar{f}_T \equiv \frac{1}{T} \sum_{t=1}^T f_t$ be the average of the cost functions. Consider the following optimization problem $P_T:$
	%For any $\beta \geq 0,$ let us define:
	%\begin{eqnarray*}
		%S_T(\beta)= \min_{x \in \mathcal{X}} \{\bar{f}_T (x), ~ \textrm{s.t.} ~ g(x) \leq \beta\}.
		\[ \min_{x\in \mathcal{X}} \bar{f}_T(x)~ \textrm{s.t.}~ g(x) \leq 0. \]
		We assume that for each $T \geq 1,$ strong duality holds for the problem $P_T$ and that the limsup of a sequence of optimal dual variables is strictly bounded above by a finite constant $\lambda $. Note that the value of $\lambda$ need not be known a priori.
%	\end{eqnarray*}
	%It is easy to verify that $S_T (\cdot)$ is a non-increasing, convex function of $\beta$ \citep[Section 5.6.1]{boyd}. We assume that $\limsup_{T} |S_T'(0^+)| \leq \lambda$ for some finite $\lambda.$
\end{assumption}
\textbf{Remark:} In the case of a fixed adversary, we have $\bar{f}_T = f.$ Hence, $\lambda$ can be efficiently determined by examining the dual of a single convex problem. Note that \citep{nedic2009subgradient} made a similar assumption for the standard offline convex optimization problem with a given cost and a constraint function. However, their algorithm must know an upper bound to $\lambda$ \emph{a priori}.  The following proposition gives a sufficient condition for Assumption \ref{sensitivity}. In particular, it shows that if the unclipped constraint function satisfies Slater's condition, then Assumption \ref{sensitivity} holds \footnote{Clearly, the clipped constraint function can not satisfy Slater's condition. Hence, we study the unclipped constraint function in Proposition \ref{slater-bdd}.}. 
 \begin{proposition} \label{slater-bdd}
 	Let $\tilde{g}$ be the constraint function which can be negative and that satisfies Slater's condition, \emph{i.e.,} there exists an admissible $z \in \mathcal{X}$ s.t. $\tilde{g}(z) \leq -\epsilon$ for some $\epsilon >0. $ Then Assumption \ref{sensitivity} holds for some $\lambda = O(\frac{1}{\epsilon})$.
 \end{proposition} 
 \begin{proof}
 	It is easy to verify that any dual optimal solution to the problem with the unclipped constraint function is also a dual optimal solution to $P$ with the clipped constraint function. Let $(z^\star, \lambda)$ be an optimal primal-dual solution pair for the problem with the unclipped constraint function. Since $(z^\star, \lambda)$ is a saddle point for the Lagrangian, we can write 
 	\begin{eqnarray*}
 		\bar{f}_T(z)+\lambda \tilde{g}(z) \geq \bar{f}_T(z^\star)+ \lambda \tilde{g}(z^\star)= \bar{f}_T(z^\star), 
 	\end{eqnarray*}
 	where we have used the complementary slackness property on the RHS. Now since $\tilde{g}(z)\leq -\epsilon,$ from the above, we have 
 	\begin{eqnarray*}
 		\bar{f}_T(z) -\lambda \epsilon \geq \bar{f}_T(z^\star) ~\implies \lambda = O(\frac{1}{\epsilon}),
 	\end{eqnarray*}
 	where in the last step, we have used the fact that the range of any convex function with a bounded subgradient over a bounded domain is bounded.
 \end{proof}
 

\subsection{Cumulative violation bound under Assumptions \ref{constr_assump1}, \ref{adversary_assump2}, and \ref{sensitivity}} 
Taking  \eqref{q-bd-eqn} as our starting point, for any feasible action $x^\star \in \mathcal{X}^\star$, we have:
\begin{eqnarray} \label{master_eq2}
	Q^2(T) + V \textrm{Regret}_T(x^\star) \leq \textrm{Regret}_T'(x^\star). 
\end{eqnarray} 
%We can lower bound the regret term as follows:
In Theorem \ref{gen-cvx-bd} and \ref{str-cvx-bd}, we trivially lower bounded the regret term by a negative linear term, which resulted in a sub-optimal violation bound. Using Assumptions \ref{constr_assump1}, \ref{adversary_assump2}, and \ref{sensitivity}, we next derive a tighter lower bound to the regret by directly relating it to the queue length. We have 
\begin{eqnarray} \label{reg_lb_jensen}
	\textrm{Regret}_T(x^\star) &=& \sum_{t=1}^T f_t(x_t) - \sum_{t=1}^T f_t(x^\star) \nonumber\\
	&\stackrel{(a)}{\geq} & \sum_{t=1}^T f_t(\bar{x}_T) - \sum_{t=1}^T f_t(x^\star) \nonumber\\
	&=& T (\bar{f}_T(\bar{x}_T) - \bar{f}_T(x^\star)). 
\end{eqnarray}
where in step (a), we have used the fact that the adversary is convex. Next, from the CCV bound of COCO Meta-policy, we have 
\begin{eqnarray*}
T g(\bar{x}_T)	\stackrel{(\textrm{Jensen's ineq.})}{\leq} \sum_{t=1}^T g(x_t) \leq Q(T),
\end{eqnarray*}
\emph{i.e.,} 
\begin{eqnarray*}
	g(\bar{x}_T) \leq \frac{Q(T)}{T}.
\end{eqnarray*}
Now let $y^\star \in \mathcal{X}$ be a solution to the following optimization problem:
\begin{eqnarray}\label{opt_prob2}
	y^\star \in \arg \min_{x \in \mathcal{X}} \bar{f}_T(x), ~ \textrm{s.t.}~ g(x) \leq \frac{Q(T)}{T}.
\end{eqnarray}
Since the average action $\bar{x}_T$ is a feasible solution to the above program, we have 
\begin{eqnarray*}
	\bar{f}_T(\bar{x}_T) \geq \bar{f}_T(y^\star).
\end{eqnarray*}
Finally, choose the feasible action $x^\star$ as follows:
\begin{eqnarray}\label{opt_prob3}
	x^\star \in \arg\min_{x \in \mathcal{X}} \bar{f}_T(x), ~ \textrm{s.t.} ~ g(x) \leq 0. 
\end{eqnarray}
Hence, from Eqn.\ \eqref{reg_lb_jensen}, \eqref{opt_prob2}, and \eqref{opt_prob3}, we have 
\begin{eqnarray*}
	\textrm{Regret}_T(x^\star) \geq T(\bar{f}_T(y^\star)- \bar{f}_T(x^\star)).
\end{eqnarray*}
Since $Q(T)=o(T),$ from Theorem \ref{gen-cvx-bd}, we now make the key observation that $x^\star$ and $y^\star$ are the solutions to optimization problems \eqref{opt_prob2} and \eqref{opt_prob3} respectively, where the latter problem has been obtained from the former by perturbing the inequality constraint by a small amount for a large $T$. Hence, the difference in their objective values can be obtained by studying the sensitivity of the convex programs. Hence, using Eqn.\ (5.57) from \citet[Section 5.6.2]{boyd}, we conclude that for a sufficiently large horizon length $T,$ we have:
\begin{eqnarray*}
	\textrm{Regret}_T(x^\star) \geq T \big(-\lambda \frac{Q(T)}{T}\big) = - \lambda Q(T). 
\end{eqnarray*}
Plugging in the above bound in  \eqref{master_eq2}, we have the following relaxation of our key regret decomposition result for large enough $T:$
\begin{eqnarray} \label{reg_decomp_new}
	 Q^2(T) - V\lambda Q(T) \leq \textrm{Regret}_T'(x^\star),
\end{eqnarray}
where, as before, the $\textrm{Regret}_T'$ term on the right-hand side denotes the worst-case regret (over the admissible set $\mathcal{X}$) of the OCO sub-routine. We now have the following theorem
\begin{theorem} \label{improved_violation_bd}
	Under Assumptions \ref{constr_assump1}, \ref{adversary_assump2}, and \ref{sensitivity}, the cumulative constraint violation bounds in Theorem \ref{gen-cvx-bd} and Theorem \ref{str-cvx-bd} can be improved to $Q(T)= \mathbb{V}(T)=O((1+\lambda) \sqrt{T}),$ and $Q(T)=\mathbb{V}(T)=O((1+\lambda)\frac{\log T}{\alpha}),$ respectively. 
\end{theorem} 
\begin{proof}
	The proof proceeds similarly to Theorem \ref{gen-cvx-bd} part 1 and  Theorem \ref{str-cvx-bd} part 1, respectively, where we now take into account the additional linear term. 
	\paragraph{Case I (Convex costs):}
	Similar to Eqn.\ \eqref{main_eq}, plugging in the adaptive regret bound for convex cost functions on the RHS of Eqn.\ \eqref{reg_decomp_new}, we have 
	\begin{eqnarray*}
		Q^2(T)-V\lambda Q(T) \leq 2GDQ(T)\sqrt{T}+ 2GDV\sqrt{T},
	\end{eqnarray*}	
	where, as before, we have used the non-decreasing property of the queue-length sequence. Setting $V=\sqrt{T}$ and completing the square, we obtain the following bound for the queue-length sequence for a sufficiently large horizon length $T:$
	\begin{eqnarray*}
		Q(T) = O(\lambda\sqrt{T})+O(\sqrt{T}).
	\end{eqnarray*} 
		\paragraph{Case II (Strongly-convex costs):}
		Plugging in the adaptive regret bound \eqref{adaptive_str_cvx_bd} for strongly-convex cost functions on the RHS of Eqn.\ \eqref{reg_decomp_new}, we obtain: 
		\begin{eqnarray*}
			Q^2(T) - V\lambda Q(T) \leq \frac{VG^2\ln (Te)}{\alpha} + \frac{G^2 \ln (Te)}{\alpha V} Q^2(T),
		\end{eqnarray*}
		where, as before, we have used the non-decreasing property of the queue-length sequence. Setting $V=\frac{2G^2 \ln(Te)}{\alpha}$ as before, we have
		\begin{eqnarray*}
			Q^2(T) - 2V\lambda Q(T) \leq 2\frac{VG^2}{\alpha}\ln(Te). 
		\end{eqnarray*}
		Completing the square, we conclude that 
		\begin{eqnarray*}
			Q(T) = O(\frac{\lambda \log T}{\alpha})+ O(\frac{\log T}{\alpha}).
		\end{eqnarray*}
\end{proof}
\fi



\subsection{Connection Between \textsc{OCS} and the Convex Body Chasing Problem}  \label{cbc}
A well-studied problem related to the \textsc{OCS} problem is the
{\it nested convex body chasing (NCBC)} problem \citep{bansa2018nested,argue2019nearly,bubeck2020chasing}, 
where at each round $t$, a convex set $\chi_t \subseteq \chi$ is revealed such that 
$\chi_t\subseteq \chi_{t-1}$, where  $\chi_0=\chi \subseteq {\mathbb R}^d$ is a convex, compact, and bounded set. 
The objective is to choose  $x_t \in \chi_t$ so as to minimize the total movement cost across rounds
$C =   \sum_{t=1}^T  ||x_t - x_{t-1}||_2,$
where $x_0 \in \chi$ is some fixed action.
In NCBC, action $x_t$ is chosen \emph{after} the set $\chi_t$ is revealed. This is in contrast to the \textsc{OCS} problem, where $x_t$ must be chosen \emph{before} the constraints $g_{t,i}$'s are revealed at round $t$. Moreover, note that the nested condition $\chi_t \subseteq \chi_{t-1}$ is stricter than Assumption \ref{feas-constr}, which is applicable to the \textsc{OCS} problem.
However, as we show next, a feasible algorithm for NCBC also provides an upper bound on the CCV of the \textsc{OCS} problem under Assumption \ref{feas-constr}.

In this reduction, we define $\chi_t $ as the intersection of the first $kt$ convex constraints $g_{\tau,i} \leq 0, 1\leq \tau \leq t, i\in [k],$ revealed up to round $t$ for the \textsc{OCS} problem. It is easy to see that $\chi_t$ is convex and $\chi_t \subseteq \chi_{t-1}, \forall t.$
Let $x_t$ be the action chosen by an algorithm $\cal A$ for the NCBC problem after the set $\chi_t$ is revealed. Note that $\chi_t \neq \emptyset,$ thanks to Assumption \ref{feas-constr}. We now choose $y_{t} := x_{t-1}$ as the action for the \textsc{OCS} problem on round $t$, ensuring that action $y_t$ is chosen before the set $ \chi_t$ is revealed.
The resulting $i^{th}$ constraint violation for the \textsc{OCS} problem at round $t$ is given by 
\[
	g_{t,i}(y_{t}) \stackrel{(a)}\le g_{t,i}(y_{t}) - g_{t,i}(y_{t+1}) \le G ||y_{t} - y_{t+1}||,
\]
where $(a)$ follows from the feasibility of $\cal A$ for NCBC, $y_{t+1}= x_{t} \in \chi_{t}$ and hence $g_{t,i}(y_{t+1}) \leq 0$. Summing across rounds $t=1, \dots, T$, and taking the $\max$ over  all the $k$ constraints, we get that the CCV using $\cal A$ for the \textsc{OCS} is upper bounded by $ \sum_{t=2}^T G ||y_{t} - y_{t+1}|| \le \sum_{t=2}^T G ||x_{t-1} - x_{t}|| \le G \cdot C_{\cal A},$
where $C_{\cal A}$ is the movement cost of $\cal A$ for the NCBC problem.

From prior work \cite{bansa2018nested,argue2019nearly,bubeck2020chasing}, it is known that for NCBC, a Steiner point-based algorithm that chooses $x_t$ as the Steiner point of $\chi_t$ can achieve
$C_{\cal A} = O(\sqrt{d \log d})$, where $\chi \subset {\mathbb R}^d$. Thus, the Steiner point-based algorithm (even though computationally intensive) provides an $O(\sqrt{d \log d})$ constraint violation for the 
\textsc{OCS} as well. However, this result is effective for problems where  $\sqrt{d \log d} = o(T).$ Our result efficiently overcomes this hurdle and provides a bound under weaker feasibility assumptions even beyond $\sqrt{d \log d} = o(T)$ -- a setting that is better motivated in practice for modern deep learning applications which are characteristically high-dimensional.


%\section{Generalizing the \ocs ~problem with the $S$-feasibility assumption}
%\subsection{Adversaries Ensuring Non-negative Regret} \label{improved_rates}
%In Theorem \ref{gen-cvx-bd} and \ref{str-cvx-bd}, we showed that under the assumption of the non-negativity of the worst-case regret, the constraint violation bounds can be improved to $O(\sqrt{{T}})$ and $O(\ln T/\alpha)$ for convex and strongly-convex cost functions, respectively. In this section, we additionally show that the same improved bounds hold in the case of a time-invariant fixed constraint function and a certain class of worst-case adversaries, called \emph{convex adversary}, defined next. COCO with time-invariant constraints has been studied in the literature where the main objective is to design gradient-based first-order policies that avoid the costly projection step on the constraint set \citep{jenatton2016adaptive, yuan2018online}. 

\paragraph{Convex adversary:} An adversary is called \emph{convex} if for any sequence of  action sequence $\{x_t\}_{t=1}^T,$ the adversary chooses the cost function sequence $\{f_t\}_{t=1}^T$ such that for any $T \geq 1,$ we have
\begin{eqnarray} \label{jensenadv}
	\sum_{t=1}^T f_t(x_t) \geq \sum_{t=1}^T f_t(\bar{x}_T),
\end{eqnarray}     
where $\bar{x}_T \equiv \frac{1}{T}\sum_{t=1}^T x_t.$ Hence, by definition, a convex adversary guarantees a non-negative regret with respect to the average action $\bar{x}_T$ for all rounds. In the following, we give two examples of convex adversaries.


 \paragraph{1. Fixed adversary:} An adversary which always selects a fixed convex function $f$ on all rounds is a convex adversary. In this case, Eqn.\ \eqref{jensenadv} holds due to the Jensen's inequality. 

\paragraph{2. Minimax adversary:} Let $\mathcal{F}$ denote an arbitrary non-empty set of convex functions defined on the admissible set $\mathcal{X}$. Consider an adversary $\mathcal{M}$, which, upon seeing the selected action $x_t$, chooses the worst cost function $f_t$ from the set $\mathcal{F}$ on round $t:$ 
\[f_t \in \arg\max_{f\in \mathcal{F}} f(x_t). \]
We now show that $\mathcal{M}$ is a convex adversary. By definition, for any round $\tau \in [T],$ we have 
\[ f_\tau(x_\tau) \geq f_t(x_\tau) \implies f_\tau(x_\tau) \geq \frac{1}{T} \sum_{t=1}^T f_{t}(x_\tau). \]
Summing up the above inequalities for each $\tau \in [T],$ we have 
\begin{eqnarray}\label{conv-adv-def}
 \sum_{\tau=1}^T f_\tau(x_\tau) \geq \sum_{t=1}^T \frac{1}{T}\sum_{\tau=1}^T f_t(x_{\tau}) \stackrel{\textrm{(a)}}{\geq} \sum_{t=1}^T f_t(\bar{x}_T),
 \end{eqnarray}
where inequality (a) follows upon applying Jensen's inequality to each cost function. 
Eqn.\ \eqref{conv-adv-def} shows that $\mathcal{M}$ is a convex adversary. 

P.S. It can be easily seen that Fixed adversary is a special case of Minimax adversary where $\mathcal{F}=\{f\}.$
\iffalse
\subsection{Assumptions}
\begin{assumption}[Time-invariant constraints] \label{constr_assump1}
Assume that the constraint functions on every round are fixed and given by $g_t=g.$ Hence, the feasible set is defined as:
\begin{eqnarray*}
	\mathcal{X}^\star = \{x^\star \in \mathcal{X}: g(x^\star) \leq 0\}.  
\end{eqnarray*}
\end{assumption}
\begin{assumption}[Convex adversary] \label{adversary_assump2}
	The cost functions are chosen by a convex adversary. 
\end{assumption}
\begin{assumption}[Bounded dual optimal variable] \label{sensitivity}
	Let $\bar{f}_T \equiv \frac{1}{T} \sum_{t=1}^T f_t$ be the average of the cost functions. Consider the following optimization problem $P_T:$
	%For any $\beta \geq 0,$ let us define:
	%\begin{eqnarray*}
		%S_T(\beta)= \min_{x \in \mathcal{X}} \{\bar{f}_T (x), ~ \textrm{s.t.} ~ g(x) \leq \beta\}.
		\[ \min_{x\in \mathcal{X}} \bar{f}_T(x)~ \textrm{s.t.}~ g(x) \leq 0. \]
		We assume that for each $T \geq 1,$ strong duality holds for the problem $P_T$ and that the limsup of a sequence of optimal dual variables is strictly bounded above by a finite constant $\lambda $. Note that the value of $\lambda$ need not be known a priori.
%	\end{eqnarray*}
	%It is easy to verify that $S_T (\cdot)$ is a non-increasing, convex function of $\beta$ \citep[Section 5.6.1]{boyd}. We assume that $\limsup_{T} |S_T'(0^+)| \leq \lambda$ for some finite $\lambda.$
\end{assumption}
\textbf{Remark:} In the case of a fixed adversary, we have $\bar{f}_T = f.$ Hence, $\lambda$ can be efficiently determined by examining the dual of a single convex problem. Note that \citep{nedic2009subgradient} made a similar assumption for the standard offline convex optimization problem with a given cost and a constraint function. However, their algorithm must know an upper bound to $\lambda$ \emph{a priori}.  The following proposition gives a sufficient condition for Assumption \ref{sensitivity}. In particular, it shows that if the unclipped constraint function satisfies Slater's condition, then Assumption \ref{sensitivity} holds \footnote{Clearly, the clipped constraint function can not satisfy Slater's condition. Hence, we study the unclipped constraint function in Proposition \ref{slater-bdd}.}. 
 \begin{proposition} \label{slater-bdd}
 	Let $\tilde{g}$ be the constraint function which can be negative and that satisfies Slater's condition, \emph{i.e.,} there exists an admissible $z \in \mathcal{X}$ s.t. $\tilde{g}(z) \leq -\epsilon$ for some $\epsilon >0. $ Then Assumption \ref{sensitivity} holds for some $\lambda = O(\frac{1}{\epsilon})$.
 \end{proposition} 
 \begin{proof}
 	It is easy to verify that any dual optimal solution to the problem with the unclipped constraint function is also a dual optimal solution to $P$ with the clipped constraint function. Let $(z^\star, \lambda)$ be an optimal primal-dual solution pair for the problem with the unclipped constraint function. Since $(z^\star, \lambda)$ is a saddle point for the Lagrangian, we can write 
 	\begin{eqnarray*}
 		\bar{f}_T(z)+\lambda \tilde{g}(z) \geq \bar{f}_T(z^\star)+ \lambda \tilde{g}(z^\star)= \bar{f}_T(z^\star), 
 	\end{eqnarray*}
 	where we have used the complementary slackness property on the RHS. Now since $\tilde{g}(z)\leq -\epsilon,$ from the above, we have 
 	\begin{eqnarray*}
 		\bar{f}_T(z) -\lambda \epsilon \geq \bar{f}_T(z^\star) ~\implies \lambda = O(\frac{1}{\epsilon}),
 	\end{eqnarray*}
 	where in the last step, we have used the fact that the range of any convex function with a bounded subgradient over a bounded domain is bounded.
 \end{proof}
 

\subsection{Cumulative violation bound under Assumptions \ref{constr_assump1}, \ref{adversary_assump2}, and \ref{sensitivity}} 
Taking  \eqref{q-bd-eqn} as our starting point, for any feasible action $x^\star \in \mathcal{X}^\star$, we have:
\begin{eqnarray} \label{master_eq2}
	Q^2(T) + V \textrm{Regret}_T(x^\star) \leq \textrm{Regret}_T'(x^\star). 
\end{eqnarray} 
%We can lower bound the regret term as follows:
In Theorem \ref{gen-cvx-bd} and \ref{str-cvx-bd}, we trivially lower bounded the regret term by a negative linear term, which resulted in a sub-optimal violation bound. Using Assumptions \ref{constr_assump1}, \ref{adversary_assump2}, and \ref{sensitivity}, we next derive a tighter lower bound to the regret by directly relating it to the queue length. We have 
\begin{eqnarray} \label{reg_lb_jensen}
	\textrm{Regret}_T(x^\star) &=& \sum_{t=1}^T f_t(x_t) - \sum_{t=1}^T f_t(x^\star) \nonumber\\
	&\stackrel{(a)}{\geq} & \sum_{t=1}^T f_t(\bar{x}_T) - \sum_{t=1}^T f_t(x^\star) \nonumber\\
	&=& T (\bar{f}_T(\bar{x}_T) - \bar{f}_T(x^\star)). 
\end{eqnarray}
where in step (a), we have used the fact that the adversary is convex. Next, from the CCV bound of COCO Meta-policy, we have 
\begin{eqnarray*}
T g(\bar{x}_T)	\stackrel{(\textrm{Jensen's ineq.})}{\leq} \sum_{t=1}^T g(x_t) \leq Q(T),
\end{eqnarray*}
\emph{i.e.,} 
\begin{eqnarray*}
	g(\bar{x}_T) \leq \frac{Q(T)}{T}.
\end{eqnarray*}
Now let $y^\star \in \mathcal{X}$ be a solution to the following optimization problem:
\begin{eqnarray}\label{opt_prob2}
	y^\star \in \arg \min_{x \in \mathcal{X}} \bar{f}_T(x), ~ \textrm{s.t.}~ g(x) \leq \frac{Q(T)}{T}.
\end{eqnarray}
Since the average action $\bar{x}_T$ is a feasible solution to the above program, we have 
\begin{eqnarray*}
	\bar{f}_T(\bar{x}_T) \geq \bar{f}_T(y^\star).
\end{eqnarray*}
Finally, choose the feasible action $x^\star$ as follows:
\begin{eqnarray}\label{opt_prob3}
	x^\star \in \arg\min_{x \in \mathcal{X}} \bar{f}_T(x), ~ \textrm{s.t.} ~ g(x) \leq 0. 
\end{eqnarray}
Hence, from Eqn.\ \eqref{reg_lb_jensen}, \eqref{opt_prob2}, and \eqref{opt_prob3}, we have 
\begin{eqnarray*}
	\textrm{Regret}_T(x^\star) \geq T(\bar{f}_T(y^\star)- \bar{f}_T(x^\star)).
\end{eqnarray*}
Since $Q(T)=o(T),$ from Theorem \ref{gen-cvx-bd}, we now make the key observation that $x^\star$ and $y^\star$ are the solutions to optimization problems \eqref{opt_prob2} and \eqref{opt_prob3} respectively, where the latter problem has been obtained from the former by perturbing the inequality constraint by a small amount for a large $T$. Hence, the difference in their objective values can be obtained by studying the sensitivity of the convex programs. Hence, using Eqn.\ (5.57) from \citet[Section 5.6.2]{boyd}, we conclude that for a sufficiently large horizon length $T,$ we have:
\begin{eqnarray*}
	\textrm{Regret}_T(x^\star) \geq T \big(-\lambda \frac{Q(T)}{T}\big) = - \lambda Q(T). 
\end{eqnarray*}
Plugging in the above bound in  \eqref{master_eq2}, we have the following relaxation of our key regret decomposition result for large enough $T:$
\begin{eqnarray} \label{reg_decomp_new}
	 Q^2(T) - V\lambda Q(T) \leq \textrm{Regret}_T'(x^\star),
\end{eqnarray}
where, as before, the $\textrm{Regret}_T'$ term on the right-hand side denotes the worst-case regret (over the admissible set $\mathcal{X}$) of the OCO sub-routine. We now have the following theorem
\begin{theorem} \label{improved_violation_bd}
	Under Assumptions \ref{constr_assump1}, \ref{adversary_assump2}, and \ref{sensitivity}, the cumulative constraint violation bounds in Theorem \ref{gen-cvx-bd} and Theorem \ref{str-cvx-bd} can be improved to $Q(T)= \mathbb{V}(T)=O((1+\lambda) \sqrt{T}),$ and $Q(T)=\mathbb{V}(T)=O((1+\lambda)\frac{\log T}{\alpha}),$ respectively. 
\end{theorem} 
\begin{proof}
	The proof proceeds similarly to Theorem \ref{gen-cvx-bd} part 1 and  Theorem \ref{str-cvx-bd} part 1, respectively, where we now take into account the additional linear term. 
	\paragraph{Case I (Convex costs):}
	Similar to Eqn.\ \eqref{main_eq}, plugging in the adaptive regret bound for convex cost functions on the RHS of Eqn.\ \eqref{reg_decomp_new}, we have 
	\begin{eqnarray*}
		Q^2(T)-V\lambda Q(T) \leq 2GDQ(T)\sqrt{T}+ 2GDV\sqrt{T},
	\end{eqnarray*}	
	where, as before, we have used the non-decreasing property of the queue-length sequence. Setting $V=\sqrt{T}$ and completing the square, we obtain the following bound for the queue-length sequence for a sufficiently large horizon length $T:$
	\begin{eqnarray*}
		Q(T) = O(\lambda\sqrt{T})+O(\sqrt{T}).
	\end{eqnarray*} 
		\paragraph{Case II (Strongly-convex costs):}
		Plugging in the adaptive regret bound \eqref{adaptive_str_cvx_bd} for strongly-convex cost functions on the RHS of Eqn.\ \eqref{reg_decomp_new}, we obtain: 
		\begin{eqnarray*}
			Q^2(T) - V\lambda Q(T) \leq \frac{VG^2\ln (Te)}{\alpha} + \frac{G^2 \ln (Te)}{\alpha V} Q^2(T),
		\end{eqnarray*}
		where, as before, we have used the non-decreasing property of the queue-length sequence. Setting $V=\frac{2G^2 \ln(Te)}{\alpha}$ as before, we have
		\begin{eqnarray*}
			Q^2(T) - 2V\lambda Q(T) \leq 2\frac{VG^2}{\alpha}\ln(Te). 
		\end{eqnarray*}
		Completing the square, we conclude that 
		\begin{eqnarray*}
			Q(T) = O(\frac{\lambda \log T}{\alpha})+ O(\frac{\log T}{\alpha}).
		\end{eqnarray*}
\end{proof}
\fi




\subsection{Proof of Theorem \ref{S-benchmark}} \label{S-benchmark-pf}
\label{ext}
\iffalse
In all the previous sections, we assumed the existence of a fixed action $x^\star \in \mathcal{X}$ that satisfies each of the online constraints on \emph{each round}. In particular, we assumed that $\mathcal{X}^\star \neq \emptyset.$
In this section, we revisit the \ocs ~problem by relaxing this assumption and only assuming that there is a feasible action $x^\star$ that satisfies the aggregate of the constraints over any consecutive $S$ rounds\footnote{This extension is meaningful only when the range of the constraint functions includes both positive and negative values. For non-negative constraints, clearly, $\mathcal{X}_S = \mathcal{X}^\star, \forall S\geq 1.$}, where the parameter $S \in [T]$ need not be known to the policy \emph{a priori}. Towards this end, we define the set of all $S$-feasible actions as below: 
\begin{eqnarray} \label{extended-benchmark}
\mathcal{X}_S =\{x^\star: \sum_{\tau \in |\mathcal{I}|} g_{\tau,i}(x^\star) \leq 0, \forall \textrm{sub-intervals}~ \mathcal{I} \subseteq [1,T], ~|\mathcal{I}| = S, \forall i \in [k]\}. 
\end{eqnarray}
We now replace Assumption \ref{feas-constr} with the following
\begin{assumption}[$S$-feasibility] \label{s-feas-assump}
	$\mathcal{X}_S \neq \emptyset.$
\end{assumption}
We now only assume that $\mathcal{X}_S \neq \emptyset$ for some $S: 1\leq S \leq T$. Clearly, Assumption \ref{s-feas-assump} is weaker than Assumption \ref{feas-constr} as $\mathcal{X}^\star \subseteq \mathcal{X}_S, \forall S \geq 1.$ 
This problem was considered earlier by \cite{georgios-cautious} in the COCO setting with a single constraint function per round. However, since the parameter $V$ used in their algorithm is restricted as $S \leq V \leq T,$ their algorithm naturally needs to know the value of $S.$ In reality, the value of the parameter $S$ is generally not available \emph{a priori} as it depends on the online constraint functions. Fortunately, our proposed meta-policy does not need to know the value of $S$ and hence, it can automatically adapt itself to the best feasible $S$.
\fi 

%\edit{argue why our result is better}.

%By extending the drift-plus-penalty methodology of \cite{neely2017online}, they proved a regret bound of $O(T^{1-\frac{\epsilon}{2}})$ and cumulative violation penalty of $O(T^{1-\frac{\epsilon}{4}})$ for $S=T^{1-\epsilon}$ \citep[Theorem 1]{georgios-cautious}. For the \texttt{OCS} problem, we show that our policy improves the latter bound to $O(T^{1-\frac{\epsilon}{2}}).$

\paragraph{Generalized regret decomposition:} Fix any $S$-feasible benchmark $x^\star \in \mathcal{X}_S,$ as given by Eqn.\ \eqref{extended-benchmark}. Then, from Eqn.\ \eqref{drift-bd}, we have 
\begin{eqnarray*}
	\Phi(\tau)- \Phi(\tau-1) &\leq& 2 \sum_{i=1}^k Q_i(\tau)g_{\tau, i}(x_\tau) \\
	&=& 2 \sum_{i=1}^k Q_i(\tau)\big(g_{\tau, i}(x_\tau)-g_{\tau, i}(x^\star)\big) + 2\sum_{i=1}^k Q_i(\tau)g_{\tau, i}(x^\star)\\
	&=& \hat{f}_\tau (x_\tau) - \hat{f}_\tau(x^\star)  +  2\sum_{i=1}^k Q_i(\tau)g_{\tau, i}(x^\star). 
\end{eqnarray*} 
 Summing up the above inequalities from $\tau = 1$ to $\tau=t,$ we have
 \begin{eqnarray} \label{new-reg-decomp}
 	\sum_{i=1}^k Q_i^2(t) =\Phi(t) \leq \textrm{Regret}'_t(x^\star) + 2 \sum_{i=1}^k\sum_{\tau=1}^t Q_i(\tau) g_{\tau, i}(x^\star),
 \end{eqnarray}
 where $\textrm{Regret}'(\cdot)$ refers to the regret of the surrogate costs as before. 
 We now bound the last term by making use of the $S$-feasibility of the action $x^\star$ as given by Eqn.\ \eqref{extended-benchmark}.
 Let us now divide the entire interval $[1,t]$ into disjoint and consecutive sub-intervals $\{\mathcal{I}_j\}_{j=1}^{\lceil t/S \rceil},$ each of length $S$ (except the last interval which could be of a smaller length). %Let $Q^\star_i(j) = \max_{\tau \in \mathcal{I}_j}Q_i(\tau)$ be the maximum queue
 Let $Q^\star_i(j)$ be the value of the variable $Q_i(\cdot)$ at the beginning of the $j$\textsuperscript{th} interval. We have
 \begin{eqnarray} \label{S-bd1}
 	\sum_{\tau=1}^t Q_i(\tau)g_{\tau, i}(x^\star) = \sum_{j=1}^{\lceil t/S \rceil} \sum_{\tau \in \mathcal{I}_j}\big(Q_i(\tau)-Q_i^\star(j)\big)g_{\tau, i}(x^\star) + \sum_{j=1}^{\lceil t/S \rceil}Q_i^\star(j) \sum_{\tau \in \mathcal{I}_j}g_{\tau, i}(x^\star) .   
 \end{eqnarray}
 %Let $g_{t,i}(x) \leq F, \forall x \in \mathcal{X}, t, i.$ 
% From the Lipschitzness assumption, we have $g_{t,i}(x) \leq GD \equiv F ~(\textrm{say}), \forall x \in \mathcal{X}, t, i.$ 
Using the boundedness assumption, let $g_{t,i}(x) \leq F, \forall x \in \mathcal{X}, t, i.$
 Using the Lipschitzness property of the queueing dynamics \eqref{q-ev2} with respect to time, we have 
 \begin{eqnarray*}
 	\max_{\tau \in \mathcal{I}_j} |Q_i(\tau)-Q_i^\star(j)| \leq F(S-1).
 \end{eqnarray*}
 Substituting the above bound into Eqn.\ \eqref{S-bd1}, we obtain 
 \begin{eqnarray} \label{new-Q-S}
 	\sum_{\tau=1}^t Q_i(\tau)g_{\tau, i}(x^\star)  \leq \big(1+\frac{t}{S}\big) F^2S(S-1) + F(S-1)(Q_i(t)+F(S-1)), 
 \end{eqnarray}
 where in the last term, we have used the $S$-feasibility of the action $x^\star$ in all intervals, except possibly the last interval. 
 %Clearly, when $S=1$, the RHS of the above bound becomes zero, and we recover Eqn.\ \eqref{q-regret-reln}. 
 Substituting the bound \eqref{new-Q-S} into Eqn.\ \eqref{new-reg-decomp}, we arrive at the following extended regret decomposition inequality:
 \begin{eqnarray} \label{gen-reg-decomp2}
 	\sum_{i=1}^k Q_i^2(t) &\leq& \textrm{Regret}'_t(x^\star) +  2kF^2S t + 2FS\sum_{i=1}^k Q_i(t) + 4F^2S^2k.
%&\leq & \textrm{Regret}'_t(x^\star) + 6kF^2S t + 2FS \sqrt{k} \sqrt{\sum_{i=1}^k Q_i^2(t)}.
 \end{eqnarray}
Eqn.\ \eqref{gen-reg-decomp2} leads to the following bound on the cumulative constraint violation.
% 
% \begin{theorem} \label{S-benchmark}
%Using the OGD policy with adaptive step-sizes given in part 1 of Theorem \ref{data-dep-regret} as a sub-routine, Algorithm \ref{ocs-policy} achieves the following CCV bound with the $S$-feasibility assumption (Assumption \ref{s-feas-assump}) for convex constraints: 
% 	 \[\max_{i=1}^k\mathbb{V}_i(T)= O(\max(\sqrt{ST},S )).\]
% \end{theorem}
% See below for the proof of the result.
%
%\paragraph{Remarks:} Recall that our proof of the $O(\sqrt{T})$ regret bound for the COCO problem with the $1$-benchmark in Theorem \ref{gen-cvx-bd} crucially uses the non-negativity of the pre-processed constraint functions. However, with $S$-feasible benchmarks, pre-processing by clipping the constraints does not work as then the positive violations can not be cancelled with a strictly feasible violation on a different round. We leave the problem of obtaining an optimal $O(\sqrt{T})$ regret bound for Algorithm \ref{g-oco-policy} for the COCO problem with the $S$-feasibility assumption as an open problem. 
%It is not clear how to  
%which is better than $O(T^{1-\epsilon/4})$ constraint violation bound obtained by \citet{georgios-cautious}.

  

%\textbf{Note:} This extension is meaningful only for the convex case. Any strongly-convex function 
%\vspace{5pt}
%\hrule 
%\textbf{Note:} \footnote{This extension is meaningful only when the range of the constraint functions includes both positive and negative values. For non-negative constraints, clearly $\mathcal{X}_S = \mathcal{X}_1 \forall S\geq 1.$}\\
%\hrule

\subsubsection{CCV Bound} 
%\paragraph{Constraint violation for convex constraints:}
 We now apply the generalized regret decomposition bound given in \eqref{gen-reg-decomp2} to the case of convex constraint functions. Substituting the regret bound \eqref{cvx-reg-bd} of the AdaGrad policy into Eqn.\  \eqref{gen-reg-decomp2}, we have 
\begin{eqnarray*}
		\sum_{i=1}^k Q_i^2(t) \leq c_1 \sqrt{\sum_{\tau=1}^t \big(\sum_{i=1}^kQ_i^2(\tau)\big)}+ c_2 S t +   c_3S\sum_{i=1}^k Q_i(t)+ c_4S^2
\end{eqnarray*}
where the constants $c_1 \equiv O(GD \sqrt{k}),c_2=O(kF^2), c_3=O(F), c_4=O(kF^2) $ are problem-specific parameters that depend on the bounds on the gradients and the maximum value of the constraint functions, the number of constraints,  and the diameter of the admissible set. Defining $Q^2(t) \equiv \sum_i Q_i^2(t),$ we obtain:
\begin{eqnarray*}
	Q^2(t) \leq c_1 \sqrt{\sum_{\tau=1}^t Q^2(\tau)} + c_2St + c_3 S \sum_{i=1}^k Q_i(t) + c_4S^2.
\end{eqnarray*}
Since $Q_i(t) \leq Ft, \forall i,$ the above inequality can be simplified to 
\begin{eqnarray} \label{simplified-q-bd}
	Q^2(t) \leq c_1 \sqrt{\sum_{\tau=1}^t Q^2(\tau)} +  c_2'St + c_4S^2, ~ \forall t\geq 1,
\end{eqnarray}
where we have defined $c_2'\equiv c_3kF+c_2.$
To solve the above system of inequalities, note that for each $1 \leq \tau \leq t,$ we have
\begin{eqnarray*}
	Q^2(\tau) \leq c_1 \sqrt{\sum_{\tau=1}^t Q^2(\tau)} +  c_2'St + c_4S^2.
\end{eqnarray*}
Summing up the above inequalities for $1\leq \tau \leq t$ and defining $Z_t \equiv \sqrt{\sum_{\tau=1}^t Q^2(\tau)},$ we obtain
\begin{eqnarray*}
	Z^2_t &\leq&  c_1t Z_t + c_2'St^2 + c_4S^2t \\
	\emph{i.e.,}~Z_t^2 &\leq & 3 \max(c_1t Z_t, c_2'St^2, c_4S^2t) \\
	\emph{i.e.,}~ Z_t &=& O(\max(t, t\sqrt{S}, S \sqrt{t})).  
\end{eqnarray*}
Substituting the above bound for $Z(t)$ in Eqn.\  \eqref{simplified-q-bd}, we have for any $t\geq 1$:
\begin{eqnarray*}
Q^2(t) &=& O(\max(Z_t, St, S^2)) \\
\emph{i.e.,}~ Q(t) &=& O(\max(\sqrt{Z_t}, \sqrt{St}, S))\\
\textrm{Hence,}~~ Q_i(t) \leq Q(t) &=& O(\max(\sqrt{t},  \sqrt{t}S^{1/4}, \sqrt{S} t^{1/4}, \sqrt{St}, S))\\
& =& O(\max(\sqrt{St},S )), ~\forall i\in [k]. 
\end{eqnarray*}
The final result follows upon appealing to Eqn.\ \eqref{V-Q}.$~~~~\blacksquare$
%In the special case $S=T^{1-\epsilon}, 0<\epsilon<1,$ the above bound yields $Q_i(t) = O(T^{1-\epsilon/2}), \forall t\geq 1.$ 
 %Hence, by setting $V=O(T),$ we get the optimal logarithmic regret for the strongly convex losses but it leads to a trivial linear constraint violation. However, by taking a smaller $V,$ we obtain a both sublinear regret and constraint violation penalty. In general, the following regret $R_T$ and violation penalty $\mathbb{V}_T$ profile is feasible (up to logarithmic factors in $T$): 
 %\begin{eqnarray*}
% 	(R_T, \mathbb{V}_T) = \tilde{O}(\frac{T}{V}, \sqrt{VT}).
% \end{eqnarray*}
% 

\subsection{Proof of Theorem \ref{P_T-benchmark}} \label{P_T_constrained}
%We now consider a different relaxation to the instantaneous feasibility assumption where we now assume that 
%We now drop the feasibility assumption and assume that the CCV incurred by the optimal static offline policy, as defined in Eqn.\ \eqref{violation-def1} is $P_TF,$ for some $P_T \leq T.$
\iffalse
Under Assumption \ref{pt-feas-assump}, there exists a fixed admissible action $x^\star \in \mathcal{X}$ such that the cumulative violation over any sub-interval of the horizon is upper bounded by $P_TF$, \emph{i.e.,}
\[ \sum_{\tau \in \mathcal{I}} g_{\tau, i}(x^\star) \leq P_TF, ~ \forall ~\textrm{sub-intervals}~ \mathcal{I} \subseteq [T], ~\forall i \in [k]. \]
%From the definition given in Eqn.\ \eqref{}, it follows that $P_TF$ is the cost of the offline  
In the non-trivial case, $P_T$ increases \emph{sub-linearly} with the horizon-length $T$. However, the value of $P_T$ is not necessarily known to the algorithm.  As before, our objective is to show that the proposed \ocs ~policy achieves a sublinear queue-length bound under the above assumption. 

 \begin{theorem} \label{P_T-benchmark}
 Using the OGD policy with adaptive step-sizes given in part 1 of Theorem \ref{data-dep-regret} as a sub-routine, Algorithm \ref{ocs-policy} achieves the following CCV bound for the $P_T$-constrained adversary as described above for convex cost functions: 
 	 \[\max_{i=1}^k \mathbb{V}_i(T)= O(P_T^{\nicefrac{1}{3}}T^{\nicefrac{2}{3}}).\]
 \end{theorem}
%For this purpose, similar to the treatment in \citet{liang2018minimizing}, we reduce the above problem to an $S$-feasible instance where the value of $S$ depends on the parameter $P_T$. 

%\paragraph{Analysis of CCV:} 
%Define an auxiliary sequence of reduced constraint functions $\tilde{g}_{t,i}(x)= g_{t,i}(x)-a, \forall t,i,$ where the value of the parameter $a$ will be decided later. 
%\begin{proof}
\fi
We will use a similar line of arguments used in the analysis of an $S$-constrained adversary for a suitable value of $S$ to be determined later. We start from Eqn.\ \eqref{S-bd1}, which holds for any value of the sub-interval length $S \geq 1$ and any arbitrary adversary. Furthermore, from the definition of a $P_T$-constrained adversary, we know that there exists a benchmark $x^\star \in \mathcal{X}$ such that for any interval $\mathcal{I}_j$ and any $i \in [k],$ we have:
\[ \sum_{\tau \in \mathcal{I}_j} g_{\tau, i} (x^\star) \leq P_TF, \]
where $F$ is the maximum absolute value of the constraint functions as given in Definition \ref{pt-feas-assump}.
Hence, 
\begin{eqnarray*}
	\sum_{j=1}^{\lceil t/S \rceil}Q_i^\star(j) \sum_{\tau \in \mathcal{I}_j}g_{\tau, i}(x^\star) &\leq& P_TF \sum_{j=1}^{\lceil t/S \rceil}Q_i^\star(j) \\
	&\leq & \frac{P_TF}{S} \sum_{j=1}^{\lceil t/S \rceil} \sum_{\tau \in \mathcal{I}_j}\big(Q_i^\star(j)-Q_i(\tau)\big) + \frac{P_TF}{S}\sum_{\tau=1}^t Q_i(\tau).
\end{eqnarray*}
Hence, from Eqn.\ \eqref{S-bd1}, we have that 
\begin{eqnarray*}
	 	\sum_{\tau=1}^t Q_i(\tau)g_{\tau, i}(x^\star)  &\leq& \big(1+\frac{t}{S}\big) F^2S(S-1) + (1+\frac{t}{S})P_TF^2(S-1) + \frac{P_TF}{S}\sum_{\tau=1}^t Q_i(\tau) \\
	 	&\leq & F^2(S+P_T)(S+t)+ \frac{P_TF}{S}\sum_{\tau=1}^t Q_i(\tau).
\end{eqnarray*}
Substituting the above bound into Eqn.\ \eqref{new-reg-decomp}, we have that 
\begin{eqnarray*}
	\sum_{i=1}^k Q_i^2(t) \leq \textrm{Regret}_t'(x^\star) + 2kF^2(S+P_T)(S+t)+ \frac{2P_TF}{S}\sum_{\tau=1}^t \sum_{i=1}^k Q_i(\tau).
\end{eqnarray*}
Plugging in the regret bound of the AdaGrad policy for the surrogate cost functions, the above equation yields
\begin{eqnarray} \label{p_t-bd-eq}
		\sum_{i=1}^k Q_i^2(t) \leq GD \sqrt{2k}\sqrt{\sum_{\tau=1}^t \big(\sum_{i=1}^kQ_i^2(\tau)\big)}+ 2kF^2(S+P_T)(S+t)+ \frac{2P_TF}{S}\sum_{\tau=1}^t \sum_{i=1}^k Q_i(\tau).
\end{eqnarray} 
Using Cauchy-Schwarz inequality, the last term of the above inequality can be upper bounded by \[\frac{2P_TF \sqrt{kt}}{S} \sqrt{\sum_{\tau=1}^t \big(\sum_{i=1}^kQ_i^2(\tau)\big)}.\]
Hence, we have the following inequality which holds for any $1\leq S \leq t$ and $1\leq \tau \leq t:$
\begin{eqnarray} \label{p_t-bd-eq2}
		\sum_{i=1}^k Q_i^2(\tau) \leq \bigg(GD \sqrt{2k}+ \frac{2P_TF \sqrt{kt}}{S}\bigg)\sqrt{\sum_{\tau=1}^t \big(\sum_{i=1}^kQ_i^2(\tau)\big)}+ 2kF^2(S+P_T)(S+t).
\end{eqnarray} 
Summing up the above inequalities for $1\leq \tau \leq t$ and defining $Z_t^2 \equiv \sum_{\tau=1}^t \sum_{i=1}^kQ_i^2(\tau),$ we have:
\begin{eqnarray*}
	Z_t^2 &\leq& \bigg(GD \sqrt{2k}+ \frac{2P_TF \sqrt{kt}}{S}\bigg)tZ_t + 2kF^2t(S+P_T)(S+t) \\
	&\leq& 2 \max \bigg(\big(GD \sqrt{2k}+ \frac{2P_TF \sqrt{kt}}{S}\big)tZ_t, 2kF^2t(S+P_T)(S+t)\bigg). 
\end{eqnarray*}
The above inequality implies that 
\begin{eqnarray}
	Z_t &\leq& 2 \max \bigg(\big(GD \sqrt{2k}+ \frac{2P_TF \sqrt{kt}}{S}\big)t, F\sqrt{kt(S+P_T)(S+t)}\bigg) \nonumber \\
	&\leq & 2 \max \bigg(\big(GD \sqrt{2k}+ \frac{2P_TF \sqrt{kT}}{S}\big)T, FT\sqrt{2k(S+P_T)}\bigg), \label{Z-t-bd2}
\end{eqnarray}
where in the last step, we have used the fact that $t \leq T$ and $S \leq T$. Now, let us choose $S\equiv  P_T^{\nicefrac{2}{3}}T^{\nicefrac{1}{3}}$. With the above choice of $S$, from the above inequality, we have the following bound for $Z_t:$
\begin{eqnarray*}
	Z_t \leq 2 \max \bigg(\big(GD \sqrt{2k}+ 2F\sqrt{k}P_T^{\nicefrac{1}{3}}T^{\nicefrac{1}{6}}\big)T, 2F\sqrt{k}P_T^{\nicefrac{1}{3}}T^{\nicefrac{7}{6}}\bigg) = O(P_T^{\nicefrac{1}{3}}T^{\nicefrac{7}{6}})+O(T),
\end{eqnarray*}
where we have used the fact that $P_T \leq T.$
Substituting the above bound in \eqref{p_t-bd-eq2}, we have for any $1\leq i\leq k$ and any $t \leq T:$
\begin{eqnarray*}
	\sum_{i=1}^k Q_i^2(t) \stackrel{(a)}{=} O(P_T^{\nicefrac{2}{3}}T^{\nicefrac{4}{3}})+ O(T)+ O(P_T^{\nicefrac{2}{3}}T^{\nicefrac{4}{3}}) = O(P_T^{\nicefrac{2}{3}}T^{\nicefrac{4}{3}}) +O(T),
\end{eqnarray*}
where in (a), we have used the fact that $T\geq S\geq P_T$ in bounding the last term.
Hence, we have the following upper bound on the queue lengths for any $1\leq t \leq T$
\begin{eqnarray*}
||\bm{Q}(t)||_\infty \leq ||\bm{Q}(t)||_2 = O(P_T^{\nicefrac{1}{3}}T^{\nicefrac{2}{3}})+O(\sqrt{T}).
\end{eqnarray*}
The final result follows upon appealing to the relation \eqref{V-Q}. $~~~~\blacksquare$
%where we recall that $P_T$ is proportional to the CCV achieved by the optimal static offline policy.
%\end{proof}
%\section{Application to a canonical network switching problem} \label{app}
As an interesting application of the machinery developed in this paper, we revisit the classic problem of packet scheduling in an $N \times N$ input-queued switch in an internet router with \emph{adversarial} arrival and service processes. This problem has been extensively studied in the networking literature in the stochastic setting; however, to the best of our knowledge, no provable result is known for the problem in the adversarial context. In the stochastic setting with independent arrivals and constant service rates, the celebrated Max-Weight policy is known to achieve the full capacity region \citep{mckeown1999achieving, tassiulas1990stability}. However, this result immediately breaks down when the arrival and service processes are decided by an adversary. In this section, we demonstrate that the proposed \ocs ~meta-policy, described in Algorithm \ref{ocs-policy}, can achieve a sublinear queue length in the adversarial setting as well.
\begin{figure}
\centering
	\includegraphics[scale=0.7]{./figures/input-q-sw}
	\caption{A $4 \times 4$ input-queued switch used in a router. Figure taken from \citet{hajek2015random}.}
	\label{ipq}
\end{figure}

\textbf{Problem description:} An $N\times N$ input-queued switch has $N$ input ports and $N$ output ports. Each of the $N$ input ports maintains a separate FIFO queue for each output port. The input and output ports are connected in the form of a bipartite network using a high-speed switch fabric (see Figure \ref{ipq} for a simplified schematic). At any round (also called \emph{slots}), each input port can be connected to at most one output port for transmitting the packets. The objective of a switching policy is to choose an input-output matching at each round to route the packets from the input queues to their destinations so that the input queue lengths grow sub-linearly with time (\emph{i.e.,} they remain \emph{rate-stable} \citep{neely2010stochastic}). Please refer to the standard references \emph{e.g.,} \citet{hajek2015random} and the original papers \citep{tassiulas1990stability, mckeown1999achieving} for a more detailed description of the input-queued switch architectures and constraints. 
\paragraph{Admissible actions and the queueing process:}
Let $\Omega$ be the set of all $N \times N$ doubly stochastic matrices (\emph{a.k.a.} the Birkhoff polytope). The set $\Omega$ is known to coincide with the convex hull of all incidence vectors corresponding to the perfect matchings of the $N\times N$ bipartite graph \citep{hajek2015random}. At each round, the policy chooses a feasible action $x(t)\equiv \big(x_{ij}(t), 1\leq i,j\leq N\big)$ from the admissible set $\Omega.$ It then randomly samples a matching $z(t)\equiv \big(z_{ij}(t) \in \{0,1\}, 1\leq i,j\leq N\big)$ using the Birkhoff-Von-Neumann decomposition, such that $\mathbb{E}z(t)=x(t)$. 
At the same time, the adversary reveals a packet arrival vector $\bm{b}(t)$ and a service rate vector $\bm{s}(t)$. The arrival and service processes could be binary-valued or could assume any non-negative integers from a bounded range. As a result, the $i$\textsuperscript{th} input queue length corresponding to the $j$\textsuperscript{th} output port evolves as follows:
\begin{eqnarray} \label{Q-ev-iqs}
	Q_{ij}(t)=\big(Q_{ij}(t-1)+b_{ij}(t)- s_{ij}(t) z_{ij}(t))^+, Q_{ij}(0)=0.
\end{eqnarray} 
\paragraph{$S$-Feasibility:}
We assume that the adversary is $S$-feasible, \emph{i.e.,} $\exists x^\star \in \mathcal{X}$ such that
\[\sum_{t \in \mathcal{I}}(b_{ij}(t)-s_{ij}(t)x^\star_{ij}) \leq 0, \forall i,j, \forall \textrm{sub-intervals}~ \mathcal{I}, |\mathcal{I}|=S.\] The fixed admissible action $x^\star$ is unknown to the switching policy. Since we do not assume Slater's condition, the queue-length bounds derived in \cite{neely2017online} do not apply here. 
\paragraph{Reduction to the \ocs ~problem:} The above scheduling problem can be straightforwardly reduced to the \ocs ~problem where we consider $k=N^2$ linear constraints, where the $(i,j)$\textsuperscript{th} constraint function is defined as 
\[ g_{t, (ij)}(x)\equiv b_{ij}(t)-s_{ij}x_{ij}, ~1\leq i,j \leq N. \] 
It follows that the auxiliary queueing variables in the \ocs ~meta-policy in Algorithm \ref{ocs-policy} evolve similarly to Eqn.\ \eqref{Q-ev-iqs}. Hence, taking expectation over the randomness of the policy and using arguments exactly similar to the proof of Theorem \ref{S-benchmark}, it follows that under Algorithm \ref{ocs-policy}, each of the $N^2$ input queues grows sublinearly as $\mathbb{E}Q_{ij}(t) = O(\max(\sqrt{St}, S)),$ where the expectation is taken over the randomness of the policy. Hence, as long as $S$ is a constant, the queues remain rate stable. To exploit the combinatorial structure of the problem, it is computationally advantageous to use the FTPL sub-routine \citep[Theorem 11]{abernethy2014online} as the base OCO policy in Algorithm \ref{ocs-policy}, which can be implemented efficiently by a maximum-weight matching oracle and leads to the same queue length bounds.  
%Since we are only interested in the stability of the queues, we set the cost functions $f_t=0, \forall t$. 
\iffalse
\paragraph{Derivation of an Online Policy:} Note that there are $N^2$ queueing processes $\{Q_{ij}(t)\}_{i,j}$, unlike in Section \ref{policy} where we defined only a single process. Nevertheless, using the quadratic Lyapunov function  
\[\Phi(t) \equiv \sum_i Q_{ij}^2(t), \]
and using similar steps, we now define a sequence of linear surrogate cost functions: 
\begin{eqnarray*}
	f_t'(x) \equiv -\sum_{i,j} Q_{ij}(t) g_{ij}(t)x_{ij} 
\end{eqnarray*}
which we feed to an OGD policy with adaptive step sizes \cite{orabona2019modern}.
\fi
\iffalse
\paragraph{Experimental Setup} We consider $N=4.$ In our simulations, we take $x^\star$ to be a specific convex combination of $k$ matchings $\{\mathcal{M}_i\}_{i=1}^k.$
\begin{eqnarray*}
	x^\star = \sum_{i=1}^k \alpha_i \mathcal{M}_i.
\end{eqnarray*}
The coefficients and the matchings are kept hidden from the algorithm. Next, we choose the service rates for the current round in some adversarial fashion. Upon choosing the service rates, we set the arrival vector as follows
\begin{eqnarray*}
	b_{ij}(t) = g_{ij}(t)x^\star_{ij}.
\end{eqnarray*}
The above choice ensures that the adversary is admissible however, the Slater's condition does not necessarily hold.  
\fi

\iffalse
 \subsection{Proof of Theorem \ref{str-cvx-slater}} \label{str-cvx-slater-proof}
 Assuming that Slater's condition holds for some $\eta^\star \geq 0,$ from Eqn.\ \eqref{Gronwall-ineq}, we have that 
 \begin{eqnarray*}
 	2\eta^\star \sum_{\tau=1}^t Q(\tau) \leq VG\delta t + \frac{2G^2}{\alpha V} \int_{\nicefrac{1}{2}}^t \frac{Q^2(\tau)}{\tau}d\tau,
 \end{eqnarray*}
 where, as before, we have defined $\delta\equiv (\frac{G}{\alpha}+D).$ Substituting the upper bound \eqref{Q-bd-str-cvx} on the RHS, we obtain
 \begin{eqnarray*}
 	\frac{1}{t}\sum_{\tau=1}^t Q(\tau) \leq \frac{1}{2\eta^\star}\big(VG\delta + O(\frac{1}{V})\big) = O(V). 
 \end{eqnarray*}

\subsection{Proof of Theorem \ref{positive-regret-str-cvx}} \label{positive-regret-str-cvx-proof}
Under the assumption of uniform non-negativity of the regret, Eqn.\ \eqref{Gronwall-ineq} implies that the queue variables satisfy the following integral inequality:
\begin{eqnarray*}
	 	Q^2(t)  \leq \frac{VG^2}{\alpha} \ln(t) + \frac{G^2}{\alpha V} \sum_{\tau=1}^t \frac{Q^2(\tau)}{\tau} \leq \frac{VG^2}{\alpha} \ln(t) + \frac{2G^2}{\alpha V} \int_{\nicefrac{1}{2}}^t \frac{Q^2(\tau)}{\tau}d\tau, ~ \forall t \geq 1,
\end{eqnarray*}
where in the second inequality, we have linearly interpolated the discrete $\{Q^2(t)\}_{t \geq 1}$ variables between every two integers as in the proof of Theorem \ref{str-cvx-bd}.
An application of Gr\"onwall's inequality \eqref{gw_ineq}, yields the following bound on the queue-lengths:
\begin{eqnarray*}
	Q^2(t) \leq \frac{VG^2}{\alpha} \ln(t) (2t)^{\frac{2G^2}{\alpha V}}, ~ \forall t \geq 1.
\end{eqnarray*}
Next, we set $V \geq \frac{2G^2}{\alpha} \ln(2T),$ so that the last factor is bounded by $e$. This yields the following improved bound for the queue length
\begin{eqnarray*}
	Q(t) \leq \frac{2G}{\sqrt{\alpha}}\sqrt{V \ln t}, ~ \forall t \geq 1.
\end{eqnarray*}
Plugging in the above bound in \eqref{Gronwall-ineq}, we have the following regret bound
\begin{eqnarray*}
	V \textrm{Regret}_t(x^\star) &\leq& \frac{VG^2}{\alpha} \ln(t) + \frac{6G^4}{\alpha^2} \int_{\nicefrac{1}{2}}^t \frac{\ln \tau}{\tau}d\tau \\
	&\leq& \frac{VG^2}{\alpha} \ln(t) + \frac{3G^4}{\alpha^2}(\ln t)^2.
\end{eqnarray*}
This yields the following regret bound:
\begin{eqnarray*}
	\textrm{Regret}_t(x^\star) \leq \frac{G^2}{\alpha} \ln(t) + \frac{3G^4}{\alpha^2 V}(\ln t)^2 \stackrel{(V=\mathcal{X}(\ln T))}{=} O(\frac{(\ln t)^2}{V}).
\end{eqnarray*}
\fi
 
% 
%
%\subsection{Proof of Theorem \ref{constraint-sat-cvx}} 
%\begin{proposition} \label{q-bd-prop}
%Let $\{Q(t)\}_{t \geq 1}$ be a non-negative sequence with 
%  $Q(1)> 0$. 
%Suppose that the $t$\textsuperscript{th} term of the sequence satisfies the inequality 
%\begin{eqnarray} \label{seq_bd}
%	Q^2(t) \leq c\sum_{\tau=1}^t  \frac{Q^2(\tau)}{\sum_{s=1}^\tau Q(s)}, ~ \forall t\geq 1,
%\end{eqnarray}
%where $c >0$ is a constant.
%Then $Q(t)\leq  c\ln(t) + O(\ln \ln t), \forall t \geq 1$\footnote{If the sequence is identically equal to zero then there is nothing to prove. Otherwise, by skipping the initial zero terms, one can always assume that the first term of the sequence is non-zero.}.
%\end{proposition}
%\begin{proof}
%%Note that if all terms of the sequence are less than $1,$ then there is nothing to prove. Otherwise, by possibly shifting the sequence to left, we can assume that the first term of the sequence $Q(1)$ is at least $1$. 
%By dividing each term of the sequence by $c,$ WOLOG, we can assume that $c=1$. 
%	From Eqn.\ \eqref{seq_bd}, we have the following preliminary bound on the growth of the sequence:
%	\begin{eqnarray} \label{prelim_bd1}
%		Q^2(t) \leq \sum_{\tau=1}^t \frac{Q^2(\tau)}{Q(\tau)} \stackrel{(a)}{=} \sum_{\tau=1}^t Q(\tau), ~ \forall t \geq 1. 
%	\end{eqnarray}
%	where in (a), we have used the fact that each term of the sequence is non-negative. Substituting the bound \eqref{prelim_bd1} into Eqn.\ \eqref{seq_bd}, we obtain
%	\begin{eqnarray*}
%		Q^2(t) \leq \sum_{\tau=1}^t \frac{\sum_{s=1}^\tau Q(s)}{\sum_{s=1}^\tau Q(s)}= t.
%	\end{eqnarray*}
%	Hence, we have 
%	\begin{eqnarray} \label{pre-bd1}
%			Q(t) \leq \sqrt{t}, \forall t\geq 1.
%	\end{eqnarray}
%	
%	 Next, for any fixed $t \geq 1,$ let $\arg\max_{\tau=1}^t Q(\tau)= z$ (break ties arbitrarily). We have: 
%	 \begin{eqnarray*}
%	 	 Q^2(z) \stackrel{(b)}{\leq}  \sum_{\tau=1}^z  \frac{Q^2(\tau)}{\sum_{s=1}^\tau Q(s)} \stackrel{(c)}{\leq} \sum_{\tau=1}^t\frac{Q^2(\tau)}{\sum_{s=1}^\tau Q(s)} \stackrel{(d)}{\leq} Q(z) \sum_{\tau=1}^t\frac{Q(\tau)}{\sum_{s=1}^\tau Q(s)}
%	 \end{eqnarray*}
%	 where in (b), we have used \eqref{seq_bd} for the index $z$, in (c), we have used the non-negativity of the sequence, and in (d), we have used the fact that $Q(z) \geq Q(\tau), 1\leq \tau \leq t.$ Dividing both sides by $Q(z),$ the above inequality yields:
%	 \begin{eqnarray*}
%	 	Q(z) \leq \sum_{\tau=1}^t\frac{Q(\tau)}{\sum_{s=1}^\tau Q(s)}.
%	 \end{eqnarray*}
%	 Finally, for all $t\geq 1,$ we have
%	 \begin{eqnarray*}
%	 	Q(t) \leq Q(z) 
%	 	&\leq& \sum_{\tau=1}^t\frac{Q(\tau)}{\sum_{s=1}^\tau Q(s)} \\
%	 	&\leq& \sum_{\tau=1}^t \int_{\sum_{s=1}^{\tau-1} Q(s)}^{\sum_{s=1}^\tau Q(s)}\frac{dx}{x} \\
%	 	&\leq& 1+\int_{Q(1)}^{\sum_{s=1}^t Q(s)}\frac{dx}{x}\\
%	 	&\stackrel{(e)}{=}& \ln(\sum_{s=1}^t Q(s)) + c \\
%	 	&\stackrel{(f)}{\leq}& \ln(\sum_{s=1}^t \sqrt{s}) + c\leq \frac{3}{2}\ln(t)+c,   
%	 \end{eqnarray*}
%	  where in (c), we have defined the constant $c\equiv 1-\ln(Q(1))$ and in (f), we have used the bound \eqref{pre-bd1}. We can tighten the result further by substituting the previous bound in the inequality $(e)$ above, which results in
%	  \begin{eqnarray*}
%	  	Q(t) \leq \ln(\sum_{s=1}^t \frac{3}{2} \ln s + ct) \leq \ln(t)+ \ln(c+\frac{3}{2}\ln(t)). 
%	  \end{eqnarray*}
%\end{proof}
