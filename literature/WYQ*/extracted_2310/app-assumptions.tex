\subsection{Discussion on Assumptions \ref{cvx}, \ref{bddness} and \ref{feas-constr}}\label{app:assumptions}
Assumptions \ref{cvx} and \ref{bddness} are standard in the online learning literature. The feasibility assumption (Assumption \ref{feas-constr}) is analogous to the \emph{realizability} assumption in learning theory \citep{pmlr-v178-hopkins22a} and is commonly used in the COCO literature \citep{neely2017online, yu2016low,yuan2018online,yi2023distributed, georgios-cautious}. Assumption 3 requires the existence of a single admissible action $x^\star \in \mathcal{X}$ that satisfies the constraints in \emph{every} round. Consequently, all constraint functions are required to be non-positive over a non-empty common subset. This assumption is weakened in Section \ref{simul_constr}, Assumption \ref{s-feas-assump}, which only requires the existence of a fixed admissible action $x^\star$ that satisfies the constraints \emph{on average}. Specifically, Assumption \ref{s-feas-assump} requires that the sum of the constraint functions evaluated at some admissible $x^\star$ over any interval of length $S$ is non-positive. Notably, throughout the paper, we \emph{do not} assume Slater's condition as it does not hold in many problems of interest \citep{yu2016low}. As a result, unlike many previous works \citep{yu2017online}, our bounds are \emph{independent} of Slater's constant, which can be problem-dependent. Furthermore, we do not restrict the sign of either cost or constraint functions, allowing them to take both positive and negative values. 
%See the discussion on preprocessing the constraint functions in Section \ref{gen_oco}.
%Inspired by the Lyapunov method in the control theory, in the following, we propose an online meta-policy for the \textsc{OCS} problem and show that it yields optimal violation bounds. 
%Our main technical contribution is that while the classic works, such as \citet{neely2010stochastic}, use the Lyapunov theory in a stochastic setting; we adapt it to the adversarial setting by combining the Lyapunov method with the OCO framework.
%study the adversarial version of the problem through the lens of the OCO framework.