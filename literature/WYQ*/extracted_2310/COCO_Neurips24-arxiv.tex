\documentclass{article}


% if you need to pass options to natbib, use, e.g.:
%     \PassOptionsToPackage{numbers, compress}{natbib}
% before loading neurips_2024


% ready for submission
%\usepackage{neurips_2024}


% to compile a preprint version, e.g., for submission to arXiv, add add the
% [preprint] option:
    % \usepackage[preprint]{neurips_2024}


% to compile a camera-ready version, add the [final] option, e.g.:
     \usepackage[final]{neurips_2024}


% to avoid loading the natbib package, add option nonatbib:
%    \usepackage[nonatbib]{neurips_2024}


\usepackage[utf8]{inputenc} % allow utf-8 input
\usepackage[T1]{fontenc}    % use 8-bit T1 fonts
\usepackage{hyperref}       % hyperlinks
\usepackage{url}            % simple URL typesetting
\usepackage{booktabs}       % professional-quality tables
\usepackage{amsfonts}       % blackboard math symbols
\usepackage{nicefrac}       % compact symbols for 1/2, etc.
\usepackage{microtype}      % microtypography
%\usepackage{xcolor}         % colors
% added packages 
%---------------------------------------------
\newcommand\hmmax{0}
\newcommand\bmmax{0}
%\usepackage[english]{babel}
%\usepackage{algpseudocode}
%\usepackage{amsfonts}
\usepackage{amssymb}
\usepackage{amsmath}
\usepackage{amsthm}
%\usepackage{framed}
% \usepackage{overpic}
% %\usepackage{bbm}
\usepackage{cleveref}
\usepackage{autonum}
\usepackage{MnSymbol} 
% \usepackage{bm}
\newcommand{\grad}{\nabla}
\usepackage{caption}
%\usepackage{subcaption}
%\usepackage[cmex10]{amsmath} 
\usepackage{url}
\usepackage{lettrine}
\usepackage{framed}
\usepackage{color}
%\usepackage{algorithm}
%\usepackage{algorithmic}
%\usepackage{authblk}

\usepackage{algorithm,algpseudocode}

\algnewcommand{\algorithmicforeach}{\textbf{for each}}
\algdef{SE}[FOR]{ForEach}{EndForEach}[1]
  {\algorithmicforeach\ #1\ \algorithmicdo}% \ForEach{#1}
  {\algorithmicend\ \algorithmicforeach}% \EndForEach

\usepackage{overpic}
\usepackage{dsfont}
\usepackage{booktabs} 
%\usepackage{framed}
\usepackage{bm}
\usepackage{bbm}
\usepackage{ifthen}
%\usepackage{cite}
\usepackage{graphicx}
\usepackage{subcaption}
%\usepackage{subfigure}
\usepackage{dsfont}
\usepackage{framed}
\usepackage{overpic}
\usepackage[svgnames]{xcolor}
%\usepackage{xcolor}
\newtheorem{theorem}{Theorem}
\usepackage{apxproof}
\newtheorem{proposition}{Proposition}
\newtheorem{definition}{Definition}
\newtheorem{lemma}{Lemma}
\newtheorem{assumption}{Assumption}
\newtheorem{corollary}{Corollary}[theorem]
\hypersetup{
	colorlinks=true,
	linkcolor=blue,
	urlcolor=red,
	citecolor=DarkBlue,
	linkbordercolor={0 0 1}
}

\newcommand{\alglinelabel}{%
  \addtocounter{ALC@line}{-1}% Reduce line counter by 1
  \refstepcounter{ALC@line}% Increment line counter with reference capability
  \label% Regular \label
}

\DeclareMathOperator*{\argmax}{argmax}
\DeclareMathOperator*{\argmin}{argmin}
%\newtheorem{assumption}{Assumption}
\newtheorem{observation}{Observation}
\newtheoremrep{theorem}{Theorem}
%\usepackage{algpseudocode}% http://ctan.org/pkg/algorithmicx
\newcommand{\ocs}{\textsc{OCS}}
\newcommand{\coco}{\texttt{COCO}~}
\renewcommand{\algorithmicrequire}{\textbf{Input:}}
\renewcommand{\algorithmicensure}{\textbf{Output:}}
\newcommand{\cmt}[1]{\textcolor{red}{\textbf{(#1)}}}
\newcommand{\algorithmicinit}{\textbf{Initialize:}}
\newcommand{\INIT}{\item[\algorithmicinit]}
\newcommand{\abs}[1]{\left|#1\right|}
\newcommand{\indicator}[1]{\mathbbm{1}\left\{#1\right\}}
\newcommand{\real}{\mathbb{R}}
\newcommand{\prob}[1]{\mathbb{P}\left(#1\right)}
\newcommand{\expect}[1]{\mathbb{E}\left[#1\right]}
\newcommand{\proj}[1]{\bm{P}_{#1}}
\newcommand{\dualproj}[1]{\bm{P}_{#1}^\perp}
\newcommand{\spn}[1]{\texttt{span}\left(#1\right)}
\newcommand{\bvec}[1]{\bm{#1}}
\newcommand{\norm}[1]{\left\|#1\right\|_2}
\newcommand{\opnorm}[2]{\left\|#1\right\|_{#2}}
\newcommand{\diag}[1]{\texttt{diag}\left(#1\right)}
%\newcommand{\bhat}[1]{\pmb{\hat{#1}}}
\newcommand{\reg}{\textrm{Reg}(T)}
%\newcommand{\edit}[1]{{\color{red} (AS: #1)}}
\newcommand{\edit}[1]{{}}
\theoremstyle{definition}
\newtheorem{example}{Example}[section]


%\title{Playing in the Dark: No-regret Learning with Adversarial Constraints}
\title{Optimal Algorithms for Online Convex Optimization with Adversarial Constraints}

% The \author macro works with any number of authors. There are two commands
% used to separate the names and addresses of multiple authors: \And and \AND.
%
% Using \And between authors leaves it to LaTeX to determine where to break the
% lines. Using \AND forces a line break at that point. So, if LaTeX puts 3 of 4
% authors names on the first line, and the last on the second line, try using
% \AND instead of \And before the third author name.


\author{%
  Abhishek Sinha, Rahul Vaze \\
 School of Technology and Computer Science \\
  Tata Institute of Fundamental Research \\
  Mumbai 400005, India \\
  \texttt{abhishek.sinha@tifr.res.in}, 
  \texttt{rahul.vaze@gmail.com}
}


\begin{document}


\maketitle

%\section{Abstract}
\begin{abstract}
%Constrained online convex optimization problem (COCO) is considered where after an online policy chooses an action $x_t$ on round $t$, the adversary reveals a convex cost function $f_t$, and a set of $k$ convex constraints of the form $g_{t,i}(x) \leq 0, i \in [k]$. The cost function $f_t$ and the constraint functions $g_{t,i}$'s could change arbitrarily with time, and no information about the future functions is assumed to be available. 
A well-studied generalization of the standard online convex optimization (OCO) framework is constrained online convex optimization (COCO). In COCO, on every round, a convex cost function and a convex constraint function are revealed to the learner after it chooses the action for that round. The objective is to design an online learning policy that simultaneously achieves a small regret while ensuring a small cumulative constraint violation (CCV) against an adaptive adversary interacting over a horizon of length $T$. A long-standing open question in COCO is whether an online policy can simultaneously achieve $O(\sqrt{T})$ regret and $\tilde{O}(\sqrt{T})$ CCV without any restrictive assumptions. For the first time, we answer this in the affirmative and show that a simple first-order policy can simultaneously achieve these bounds. Furthermore, in the case of strongly convex cost and convex constraint functions, the regret guarantee can be improved to $O(\log T)$ while keeping the CCV bound the same as above.
We establish these results by effectively combining adaptive OCO policies as a blackbox with Lyapunov optimization - a classic tool from control theory. Surprisingly, the analysis is short and elegant. 
%A potential function-based proof technique is presented that reveals a connection between regret and certain sequential inequalities through a novel decomposition result. The resulting guarantees either match the best-known results or provide new results in certain cases.
%We show that optimal performance bounds can be achieved by solving the surrogate problem using \emph{any} adaptive OCO policy that enjoys a standard data-dependent regret bound.  We conclude the paper by highlighting the application of the above framework to online multi-task learning and network control problems.
\end{abstract}
%\begin{keywords}
% Online learning, Constrained optimization, Regret bounds
%\end{keywords}

\iffalse
We consider a generalization of the classic Online Convex Optimization (OCO) problem with long-term budget constraints. Specifically, we assume that on round $t$ the policy takes action $x_t$ from an admissible set $\Omega,$ and the adversary reveals a convex cost function $f_t: \Omega \mapsto \mathbb{R}$ and a collection of $k$ constraints of the form $g_{t,i}(x)\leq 0, 1\leq i\leq k $ where each $g_{t,i}: \Omega \mapsto \mathbb{R}$ is a convex function. The cost and the constraint functions could be arbitrarily varying with time, and no information about the future cost and constraint functions (including bounds to the norm of the sub-gradients and/or strong-convexity parameters) is available.
%The cost and the constraint functions could be decided arbitrarily by an adversary and \emph{no} information about the future cost and constraint functions (including upper-bounds to the norm of the sub-gradients and/or strong convexity parameters) are available. 
We design a meta-policy that achieves a sublinear constraint violation penalty and sublinear regret against any feasible set of fixed actions. This is achieved by constructing a black box reduction of the problem to an OCO problem with a carefully constructed sequence of surrogate cost functions. Optimal performance bounds are achieved by solving the surrogate problem using any adaptive online policy with a standard data-dependent regret bound. A new potential-based proof technique is presented that reveals a new connection between deriving regret bounds and solving certain sequential inequalities through a novel regret decomposition result.
\fi
%Our analysis introduces a new general technique for deriving performance bounds by bounding the growth of a sequence of real numbers satisfying certain recursive inequalities.
% Our proof technique is flexible and can be used to extend a wide array of unconstrained online problems to their constrained version in almost a routine fashion. To be more specific, our reduction enables one to translate the problem of bounding the regret and constraint violation penalties into mere calculus exercise of bounding the order of growth of real sequences satisfying certain recursive inequalities.  
 %Compared to previous works on this problem which use a constant \citep{yuan2018online}, or a fixed sequence of step sizes \citep{jenatton2016adaptive} for the primal problem, our proposed policy uses a sequence of \emph{adaptive} step-sizes which explicitly depend on the magnitude of the past gradients.. 
\section{Introduction} \label{intro}
Online convex optimization (OCO) is a standard framework for modelling and analyzing a broad family of online decision problems under uncertainty. In the OCO problem, on every round $t$, an online policy first selects an action $x_t$ from a closed and convex admissible set (\emph{a.k.a.} decision set) $\mathcal{X}.$ Then the adversary reveals a convex cost function $f_t$, resulting in a cost of $f_t(x_t)$. 
The goal of an online policy is to choose an admissible action sequence $\{x_t\}_{t=1}^T$ so that its cumulative cost is not much larger than that of any fixed admissible action chosen in hindsight. In particular, the objective is to minimize the static regret defined below
%\vspace{-0.1in}
\begin{eqnarray} \label{intro-regret-def}
	\textrm{Regret}_T \equiv \sup_{\{f_t\}_{t=1}^T} \sup_{x^\star \in \mathcal{X}} \textrm{Regret}_T(x^\star), ~\textrm{where~}\textrm{Regret}_T(x^\star) \equiv \sum_{t=1}^T f_t(x_t) - \sum_{t=1}^T f_t(x^\star).
\end{eqnarray}
 The term \emph{static} refers to using a fixed benchmark, specifically only one action $x^\star$ throughout the horizon of length $T$. 
 
 In this paper, we consider a generalization of the standard OCO framework. In this problem, on every round $t,$ the online policy first chooses an admissible action $x_t \in \mathcal{X},$ 
 and then the adversary chooses a convex cost function $f_t: \mathcal{X} \to \mathbb{R}$ and $k$ constraints of the form $g_{t,i}(x) \leq 0, \ i \in [k],$ where $g_{t,i}: \mathcal{X} \to \mathbb{R}$ is a convex function for each $i \in [k]$\footnote{Notations: For any natural number $n$, we define $[n] \equiv \{1,2,\ldots, n\}.$ For any real number $z$, we define $(z)^+ \equiv \max(0,z).$}. Since $g_{t, i}$'s are revealed after the action $x_t$ is chosen, an online policy need not necessarily take feasible actions on each round, and the obvious metric of interest in addition to \eqref{intro-regret-def} is the total cumulative constraint violation (CCV) $\mathbb{V}(T)$ defined as 
 %\vspace{-0.1in}
  \begin{eqnarray} \label{intro-gen-oco-goal}
 	\textrm{CCV}_T \equiv \mathbb{V}(T) = \max_{i=1}^k \mathbb{V}_i(T)\quad  \text{where} \quad \mathbb{V}_i(T)  = \sum_{t=1}^T (g_{t,i}(x_t))^+. 
	\end{eqnarray}
Let $\mathcal{X}^\star$ be the feasible set consisting of all admissible actions that satisfy all constraints $g_{t,i}(x) \leq 0, \ i \in [k], t\in [T]$. Under the standard assumption that $\mathcal{X}^\star$ is not empty, the goal is to design an online policy to simultaneously achieve a small regret \eqref{intro-regret-def} with respect to any admissible benchmark $x^\star \in \mathcal{X}^\star$ and a small CCV \eqref{intro-gen-oco-goal}. We refer to this problem as the constrained OCO (COCO). The assumption $\mathcal{X}^\star \neq \emptyset $ will be relaxed in Section \ref{simul_constr} for the  Online Constraint Satisfaction (OCS) problem where the cost functions are set to zero, and the objective is to minimize just the CCV.
%Compared to the stringent condition that 
%$\mathcal{X}^\star \neq \emptyset$, a relaxed condition is called the $S$-feasible benchmark, where the fixed action $x^\star$ in an $S$-feasible benchmark enjoys the property that the sum of the constraint functions evaluated at $x^\star$ over any consecutive sequence of $S \geq 1$ rounds is non-positive. Clearly, $S=1$ corresponds to  $\mathcal{X}^\star \neq \emptyset$. We consider both $S=1$ and general $S$-feasible benchmarks in this paper.


%control both the regret and the cumulative violation penalty optimally. This problem departs from the celebrated OCO framework because of the instantaneous constraints. This begets a natural question - Is it possible to efficiently reduce the constrained problem to a standard OCO problem while obtaining the optimal regret and cumulative violation bounds? In this paper, we answer this question affirmatively in a constructive fashion. 
%In this paper, our objective is to design a class of efficient online learning policies to simultaneously minimize the regret and the cumulative constraint violations.
%In addition to having a small regret, we require the cumulative constraint violations to be sublinear in time. 

%We consider the problem of Online Convex Optimization with time-varying instantaneous online constraints where the cost and constraint functions could be chosen adversarially. 
%\paragraph{Applications:} 
COCO arises in many applications, including online portfolio optimization with risk constraints, resource allocation in cloud computing with time-varying demands, pay-per-click online ad markets with budget constraints \citep{georgios-cautious}, online recommendation systems, dynamic pricing, revenue management, robotics and path planning problems, and multi-armed bandits with fairness constraints \citep{sinha2023banditq}. 
%Control policies for emerging applications such as autonomous vehicles face multi-dimensional constraints at each time, resulting from the requirement to stay on the course, speed constraints, collision avoidance constraints, and traffic regulatory constraints \citep{feng2023dense}. Offline optimization problems with a huge set of constraints can also be conveniently formulated in this framework. 
The necessity for revealing the constraints sequentially may also arise, \emph{e.g.,} in communication-limited settings, where it might be infeasible to reveal all constraints defining the feasible set at a time (\emph{e.g.,} combinatorial auctions). See Section \ref{expts} for an application of the COCO framework in fraud detection which involves binary classification with a highly-imbalanced dataset.  
%For further motivation, see Section \ref{hidden_set} in the Appendix, where we introduce a natural convex optimization over a hidden constraint set, called the \textsc{Hidden Set} problem.
 %To further motivate the problem, we now introduce a convex optimization over a hidden constraint set, which we call the \textsc{Hidden Set} problem.
 % The above set up can be motivated by imagining that there is a 
 \iffalse
 \paragraph{The \textsc{Hidden Set} Problem:}
Let $\mathcal{X}$ be an \emph{admissible} set of actions which is known to the policy. Let $\mathcal{X}^\star$, called the \emph{feasible} set, be a closed and convex subset of $\mathcal{X}$. Due to a large number of defining constraints, the feasible set $\mathcal{X}^\star$ is too complex to communicate to the policy \emph{a priori}. However, an efficient separation oracle for $\mathcal{X}^\star$ is assumed to be available. On the $t$\textsuperscript{th} round, the policy first selects an admissible action $x_t \in \mathcal{X}$ and then, the adversary reveals a convex cost function $f_t$ and \emph{some} convex constraint of the form $g_t(x) \leq 0,$ which contains the unknown feasible set $\mathcal{X}^\star$. As an example, the constraints could come from the separation oracle that, if $x_t$ is infeasible, outputs a hyperplane separating the current action $x_t \in \mathcal{X}$ and the hidden feasible set $\mathcal{X}^\star$. The objective of the policy is to perform as well as any action from the hidden feasible set $\mathcal{X}^\star$ in terms of the regret and the cumulative constraint violation metrics. 
 
 \begin{figure}
 \centering
 	\includegraphics[scale=0.4]{figures/hidden2.pdf}
 	\caption{\small{Illustrating the \textsc{Hidden Set} problem. In this figure, the sphere $\mathcal{X}^\star$ is the hidden set. On every round $t$, the adversary reveals a hyperplane supporting $\mathcal{X}^\star.$ }}
 \end{figure}
 \fi
 
 %As an interesting byproduct, our proposed online learning policy also yields throughput-optimal control policies for controlling queueing systems under adversarial arrival and service processes. This aspect has been illustrated in Section \ref{app}.
\vspace{-0.1in}
\subsection{Related Work} \label{related}
%The general problem has remained open for a long time. It was conjectured that designing a policy with both sublinear regret and sublinear constraint violations is impossible without additional assumptions \cite{neely2017online}. 
%In fact, the authors in  \cite{mannor2009online} even commented that ``\ldots it is unlikely that such a reduction is possible.''
\paragraph{Unconstrained OCO:}
In a seminal paper, \citet{zinkevich2003online} showed that for solving \eqref{intro-regret-def}, the ubiquitous projected online gradient descent (OGD) policy achieves an $O(\sqrt{T})$ regret for convex cost functions with uniformly bounded sub-gradients. A number of follow-up papers proposed adaptive and parameter-free versions of OGD 
%that do not need to know any non-causal information 
\citep{hazan2007adaptive, orabona2018scale}. See \citet{orabona2019modern, hazan2022introduction} for textbook treatments of the OCO framework and associated algorithms.

%However, these lines of work do not consider additional constraints - a problem which has been systematically explored only recently (see Table \ref{gen-oco-review-table} for a brief summary). The constraint functions could either be known \emph{a priori} or revealed sequentially along with the cost functions. 
%\paragraph{Constrained OCO (COCO):} 
 {\bf Constrained OCO (COCO): (A) Time-invariant constraints:} A number of papers considered COCO with time-invariant constraints, \emph{i.e.,} $g_{t,i} = g_i, \forall \ t$ \citep{yuan2018online, jenatton2016adaptive, mahdavi2012trading, yi2021regret}.  These works assume that the functions $g_i$'s are known to the policy \emph{a priori}. However, they allowed the policy to remain infeasible on any round to avoid the costly projection step of the vanilla projected OGD  policy. Their main objective was to design an \emph{efficient} policy (avoiding the explicit projection step) with a small regret and CCV. 

{\bf (B) Time-varying constraints:} Solving the COCO problem when the constraint functions, \emph{i.e.}, $g_{t,i}$'s, change arbitrarily with time $t$ is more challenging. In this case,  except for 
 \cite{neely2017online} and \cite{georgios-cautious}, most of the prior works 
 %assumes that the policy has access to some non-causal information, \emph{e.g.,} a uniform upper bound to the norm of the future gradients of $f_t,g_{t,i}$. This non-causal information is used by their proposed policies to 
 construct some Lagrangian function and then update the primal and dual variables \citep{yu2017online, pmlr-v70-sun17a, yi2023distributed}. However, the performance bounds obtained with this approach remain suboptimal.
   Both \citet{neely2017online} and \cite{georgios-cautious} use the drift-plus-penalty (DPP) framework introduced by \citet{neely2010stochastic} to solve the constrained problem under various assumptions. In particular,
  \citet{neely2017online} proposed a DPP-based policy for COCO  upon assuming the Slater's condition, \emph{i.e.,} $g_{t,i}(x^\star) < -\eta$, for some $\eta>0$ $\forall i,t$. Clearly, this condition precludes the important case of non-negative constraint functions (\emph{e.g.,} constraint functions of the form $\max(0, g_t(x))$). Furthermore, the bounds obtained upon assuming Slater's condition depend inversely with the Slater's constant $\eta$ (usually hidden under the big-Oh notation). Since $\eta$ could be arbitrarily small, these bounds could be arbitrarily loose. 
  %Furthermore, a sublinear violation bound obtained upon assuming Slater's condition is loose by a quantity that increases \emph{linearly} with the horizon-length $T$ compared to a sublinear violation bound obtained without this assumption.    
%(which arises, e.g., upon a $\max(0,\cdot)$ operation with a given constraint). 
%Moreover, the regret bound presented in \cite{neely2017online} diverges to infinity as $\eta \searrow 0.$ 
 \cite{georgios-cautious} extended \cite{neely2017online}'s result by considering a weaker form of the feasibility assumption without assuming Slater's condition. 
% they show that a DPP-based policy achieves a regret $\mathcal{R}_T = O(ST/V + \sqrt{T})$ and CCV ${\mathbb V}(T) = O(\sqrt{VT})$. Here, $V$ is an adjustable parameter that can take any value in $[S, T).$ Hence, \emph{a priori}, their algorithm needs to know the value of the parameter $S,$ which, unfortunately, depends on the online constraints.
%It can be seen that the violation penalty bound achieved by their policy is at least $O(T^{3/4}),$ which is suboptimal. 
Furthermore, although these DPP-based results are interesting, they have not been able to provide improved regret or CVV bounds 
when the cost functions $f_t$'s are strongly convex because of the linearization step inherent in this approach. 


In a recent paper, \citet{guo2022online} considered COCO and obtained the best-known prior results without assuming Slater's condition. However, in addition to yielding sub-optimal bounds, their policy is quite computationally intensive since it requires solving a convex optimization problem on each round. Compared to this, all policies proposed in this paper take only a single gradient-descent step and perform only one Euclidean projection on each round. 
%Moreover, it is unclear how to extend \citet{guo2022online}'s policy to the more general $S$-feasible benchmark, where it is necessary to compensate for constraint violations at some rounds with strictly satisfying constraints on some other rounds. 
Please refer to Table \ref{gen-oco-review-table} for a brief summary of the results and Section \ref{app:comparisonpolicies} in the Appendix for a qualitative comparison.
%inefficient as, instead of performing a single gradient-descent step per round (as in our and most of the previous algorithms), their algorithm needs to solve a general convex optimization problem at every round. Moreover, their algorithm needs access to the full description of the constraint function $g_t(\cdot)$ for the optimization step, whereas ours and most of the previous algorithms need to know only the gradient and the value of the constraint function for the current action $x_t$. 
%In a recent paper, \cite{yi2023distributed} consider the same problem in a distributed setup and derive tighter bounds upon assuming Slater's condition.
The COCO problem has been considered in the {\it dynamic} setting as well  \citep{chen2018bandit, cao2018online, vazecocowiopt2022, liu2022simultaneously} where the benchmark $x^\star$ in \eqref{intro-regret-def} is replaced by $x_t^\star$ that is also allowed to change its actions over time. However, we focus our attention on achieving the optimal performance bounds for the static version.
%\paragraph{The Online Constraint Satisfaction (\textsc{OCS}) Problem:} %In addition to studying the above problem, we also introduce a new but related problem, 
A special case of COCO is the 
\textsc{Online Constraint Satisfaction} (\ocs) problem that does not involve any cost function, \emph{i.e.,} $f_t=0, \ \forall t,$ and the only object of interest is the CCV. The \ocs ~problem becomes especially interesting in the setting where the feasible set may be empty.
%To the best of our knowledge, the \ocs ~problem has not been previously studied in the setting where the feasible set $\mathcal{X}^\star$ might be empty. 
%However, results derived for COCO when specialized to the case of $f_t=0$ are summarized in Table \ref{gen-oco-review-table}.

%In this problem, on every round $t$, $k$ constraints of the form $g_{t,i}(x) \leq 0$ are revealed to the policy in an online fashion, where the function $g_{t,i}$ comes from the $i$\textsuperscript{th} stream. The constraint functions could be unrestricted in sign - they may potentially take both positive and negative values within their admissible domain. The objective is to control the cumulative violation of each separate stream by choosing a common action sequence $\{x_t\}_{t \geq 1}$.  
%%Although the \ocs ~problem can be reduced to the previous problem with dynamic constraints upon setting the cost function $f_t \equiv 0, \forall t$, this reduction turns out to be provably sub-optimal. 
%Without making any assumption, the best-known bound for the cumulative violation for a single convex constraint is known to be $O(T^{3/4})$ and no violation penalty bound is known for strongly convex constraints (see Table \ref{review-table}). For the first time, we show that it is possible to achieve a cumulative violation bound of $O(\sqrt{ST})$ for convex constraints with the general $S$-benchmark and $O(\log T)$ for strongly convex constraints with the usual $1$-benchmark. Section \ref{non-trivial} in the Appendix briefly outlines the difficulty of the problem and discusses why simple approaches fail. 
%We begin our discourse with the \ocs ~problem, which clearly illustrates the new ideas and analytical techniques. These are then suitably extended in Section \ref{gen_oco}, which considers the OCO problem with dynamic constraints.     
 
%To the best of our knowledge, the problem of designing a parameter-free, fully adaptive online policy for this problem has been open for a long time. 


\iffalse
In this paper, we propose an online learning policy without making any extraneous assumptions. We begin with considering the non-trivial special case of the \textsc{Online Constraint Satisfaction} (\texttt{OCS}) problem in Section \ref{simul_constr} where all cost functions are zero and a set of $k$ constraints are presented on every round. This section clearly illustrates the main ideas and analytical techniques, which are then suitably extended in Section \ref{gen_oco}, which considers the general problem. 
\fi
%\edit{Include a table comparing the results.}

%Our analysis technique is new, which reduces the problem of bounding regret and violation penalty to solving difference inequalities for which a wealth of analytical techniques are known, \emph{e.g.,} Gr\"onwall's inequality.   


%\hspace{-30pt}

%\begin{table*}[t]
%\hspace{-30pt}
  %\caption{Summary of the results for the Online Constraint Satisfaction (\texttt{OCS}) problem}
  %\centering
  \begin{table*}[t]
%\hspace{-30pt}
  \title{Summary of the results for the constrained OCO problem}
  %\centering
  \hspace{-40pt}
  \begin{tabular}{llllll}
    \toprule
    %\multicolumn{2}{c}{Part}                   \\
   % \cmidrule(r){1-2}
   \small { Reference}  & \small {Regret} & \small {CCV} & \small {Complexity per round}& \small {Assumptions} \\
    \midrule
  %  a & b& c& d & e  \\
      \small {\citet{mahdavi2012trading}}  & \small {$O(\sqrt{T})$} & \small {$O(T^{\nicefrac{3}{4}})$} & \small {Projection} & \small{Time-invariant constraints} \\
    \small {\citet{jenatton2016adaptive}}  & \small {$O(T^{\max(\beta, 1-\beta)})$} & \small {$O(T^{1-\beta/2})$} & \small {Projection} & \small{Time-invariant constraints} \\
    \small {\citet{pmlr-v70-sun17a}}  & \small {$O(\sqrt{T})$} & \small {$O(T^{\nicefrac{3}{4}})$}& \small {Bregman Projection} & \small -  \\
    \small {\citet{neely2017online}}  & \small {$O(\sqrt{T})$} & \small {$O(\sqrt{T})$} & \small {Conv-OPT} & \small {Slater condition} \\
    %    \small {\citet{yu2017online}}  & \small {$O(\sqrt{T})$} & \small {$O(\sqrt{T})$} & \small {Conv-OPT} & \small {Slater condition} \\
  \small {\citet{yuan2018online}} & \small {$O(T^{\max(\beta, 1-\beta)})$} & \small {$O(T^{1-\beta/2})$ }  & \small {Projection} & \small{Time-invariant constraints} \\
      \small {\citet{yu2020low}}  & \small {$O(\sqrt{T})$} & \small {$O(1)$} & \small {Conv-OPT} & \small {Slater \& Time-invariant constraints} \\
 % \citet{yu2017online} & Stochastic & $O(\sqrt{T})$ & $O(\sqrt{T})$& OGD+drift+penalty & Slater condition \\
  \small {\citet{yi2021regret}} & \small {$O(T^{\max(\beta, 1-\beta)})$} & \small {$O(T^{(1-\beta)/2})$} & \small {Conv-OPT} & \small{Time-invariant constraints} \\ 
  \small {\citet{yi2022regret}}  & \small {$O(T^{\beta})$} & \small {$O(T^{1-\beta/2})$} & \small {Projection} & \small {Strongly convex cost} \\
    \small {\citet{guo2022online}}  & \small {$O(\sqrt{T})$} & \small {$O(T^{\nicefrac{3}{4}})$} & \small {Conv-OPT} & - \\
        \small {\citet{guo2022online}}  & \small {$O(\log T)$} & \small {$O(\sqrt{T \log T})$} & \small {Conv-OPT} & \small {Strongly convex cost} \\ 
  \small {\citet{yi2023distributed}}  & \small {$O(T^{\max(\beta, 1-\beta)})$} & \small {$O(T^{1-\beta/2})$} & \small {Conv-OPT} & - \\
    \small {\citet{yi2023distributed}} & \small {$O(\log(T))$} & \small {$O(\sqrt{T \log T})$} & \small {Conv-OPT} & \small {Strongly convex cost} \\
    %\citet{georgios-cautious} & Adversarial, convex & $S$ & $O(\sqrt{ST})$ & OGD & Known $S$ \\
               %\citet{guo2022online} & Adversarial, strongly-convex & $1$ & $O(\sqrt{T \log T})$ & Convex opt. each round & -do-, Known $\alpha$ \\
   %      \small {\citet{georgios-cautious}}  & &\ \textcolor{red}{($S$,$O(\sqrt{ST})$)}& \small {OGD} & \small {Known $S$} \\
  %\small {\textbf{This paper}}  & \small {$O(\sqrt{T})$} & \small {$O(T^{3/4})$} & \small {Ad-OCO} & - \\
     \small {\textbf{This paper}} & \small {$O(\sqrt{T})$} & \small {$O(\sqrt{T}\log T)$} & \small {Projection} & -\\
 \small {\textbf{This paper}} &   \small {$O(\log T)$} & \small {$O(\sqrt{T\log T})$} &\small {Projection} & \small{Strongly convex cost} \\
 \small {\textbf{This paper}} & \small {$O(\log T)$} & \small {$O(\frac{\log T}{\alpha})$} &\small {Projection} & \small {Strongly convex cost, $\textrm{Regret}_T \geq 0,$} \\
       \bottomrule
  \end{tabular}
  \vspace{5pt}
  \caption{\small{Summary of the results on COCO. Unless stated otherwise, we assume arbitrary time-varying convex constraints and convex cost functions. In the above table, $0\leq \beta \leq 1$ is an adjustable parameter, $\alpha$ is the strong convexity parameter of the strongly convex cost functions. Conv-OPT refers to solving a constrained convex optimization problem on each round. Projection refers to the Euclidean projection operation on the convex set $\mathcal{X}$. For typical convex sets (\emph{e.g.,} Euclidean box, probability simplex), projection operations are substantially more efficient than solving a constrained convex optimization problem.}}
    \label{gen-oco-review-table}
\end{table*}
%\footnotetext[1]{Bounds on the cumulative violation over the entire horizon, which is weaker than \eqref{violation-def1}.}
%\footnotetext[1]{Cumulative violation defined as $\sum_{t}([g(x_t)]_+)^2.$}
%  \footnotetext[1]{Upon optimally setting $V=S.$}
%\footnotetext{Third footnote}

%\subsection{Why is the problem non-trivial?}
%%We first argue that the \ocs ~problem is non-trivial to solve.
%Let us consider the \ocs ~problem.
%A first attempt to solve the \ocs ~problem could be to scalarize it by taking a  \emph{fixed} linear combination (\emph{e.g.,} the sum) of the constraint functions and then running a standard OCO policy on the scalarized cost functions \editr{(see \cite[Section 5.3.3]{boyd} for an offline version of the above problem, where the coefficients of the linear combination are taken to be the optimal solution to the dual problem}). The above strategy immediately yields a sublinear regret guarantee on the same linear combination (\emph{i.e.,} the sum) of the constraint functions. However, since the constraint functions could take both positive and negative values, the constraint violation component of some streams could still be arbitrarily large even when the overall sum is small. Hence, this strategy does not yield individual cumulative violation bounds, where we need to control the more stringent $\ell_\infty$-norm of the cumulative violation vector. 
%%In the particular case of the online constraint satisfaction (\texttt{OCS}) problem 
%%if only one constraint function is revealed on each round, by simply running an OCO policy on the given constraint function yields a sublinear regret and hence, a sublinear constraint violation penalty. However, 
%%with two or more constraint functions (see Section \ref{simul_constr}), 
%\editr{Hence, to meet the objective with this scalarization strategy, the ``correct'' coefficients of the linear combination} must be learned adaptively in an online fashion. This is exactly what our online meta-policy, which we \editr{describe} in Section \ref{meta-policy-ocs}, does.
%
%To get around the above issue, one may alternatively attempt to scalarize the \ocs ~problem by considering a non-negative \emph{surrogate} cost function, \emph{e.g.,} the hinge loss function, defined as $\hat{g}_{t,i}(x)=\max(0, g_{t,i}(x)),$ for each constraint $i \in [k]$. However, it can be easily seen that this transformation does not preserve the strong convexity as the function $\hat{g}_{t,i}$ is \emph{not} strongly convex even when the original constraint function $g_{t,i}$ is strongly convex. Furthermore, the above strategy does not work even for convex functions for $S$-feasible benchmarks with $S\geq 2.$ This is because, due to the impossibility of cancellation of positive violations by strictly feasible constraints on different rounds, an $S$-feasible benchmark for the original constraints does not remain feasible for the transformed non-negative surrogate constraints (see Section \ref{ext}). Finally, the above transformation fails in the case of stochastic constraints where the constraint is satisfied only in expectation, i.e., $\mathbb{E} g_t(x) \leq 0, \forall t\geq 1$ \citep{yu2017online}.
%%Finally, since we are interested in bounding the maximum violation penalty over any sub-interval in the entire time horizon \eqref{violation-def1}, it is natural to turn to the strongly adaptive algorithms as a subroutine, which are inefficient as they need to run $O(\log T)$ number of experts algorithms on each round \citep{orabona2018scale}. 
%The above discussion shows why designing an efficient and universal policy for the \texttt{OCS} problem, and consequently, for the constrained OCO problem - which generalizes \ocs, is highly non-trivial. 
%
%
%
%





 








%\vspace{-0.1in}
\subsection{Our Contributions} \label{contribution}
In this paper, we consider both COCO and $\ocs$ problems and make the following contributions. 
\begin{enumerate}
%\vspace{-.1in}
	\item 
	We propose an efficient first-order policy that simultaneously achieves $O(\sqrt{T})$ regret and $O(\sqrt{T}\log T)$ CCV for the COCO problem. Our result breaks the long-standing $O(T^{\nicefrac{3}{4}})$ barrier for the CCV and matches the lower bound (derived in Theorem \ref{thm:lbcoco}, previously missing from the literature) up to a logarithmic term. For strongly convex cost functions, the regret guarantee is improved to $O(\log T)$ while  keeping the CCV bound the same as above. Under an additional assumption that the regret is non-negative, we obtain a further improved logarithmic CCV bound in the strongly convex setting (see Table \ref{gen-oco-review-table}). 
%We propose a general framework for designing efficient learning policies for both OCS  and COCO. The performance 
%of our proposed policies either matches the best-known bound or we get new guarantees for specific cases
  
%that achieve simultaneous sublinear regret and a sublinear constraint violation penalty for any sequence of convex cost and constraint functions. 
%In contrast with the prior works that propose and analyze problem-specific policies, we give a \emph{policy-agnostic} black box reduction of both problems to the unconstrained OCO using \emph{any} adaptive learning policy with a standard data-dependent regret bound. With convex cost and constraint functions, our performance bounds match the best-known result \citep{guo2022online}, however, our policy requires less information and is more efficient as it uses only the gradient information and just takes a single gradient step per round compared to \citep{guo2022online} that requires full access to the cost and constraint functions and solves a convex problem per round.  
%Consequently, we obtain efficient policies by using standard unconstrained OCO sub-routines that take only one gradient step per round. 
%In practice, where the cost and constraint functions are given by the output of a neural network and the actions correspond to its weights, this yields an especially convenient training policy with the standard optimization algorithms (such as AdaGrad).
%or using subroutines that exploit the structure (\emph{e.g.,} combinatorial) of the action set. 
%In Section \ref{non-trivial} in the Appendix, we discuss the difficulty of the $\ocs$ problem and explain why simple approaches fail. 
%By exploiting the generality of our technique, we give the first logarithmic regret and constraint violation bounds for strongly convex constraints.
%\vspace{-.2in}
\item We additionally consider a special case of the COCO problem, called Online Constraint Satisfaction (OCS), under relaxed feasibility assumptions and obtain sub-linear CCV bounds. 

\item On the algorithmic side, our policy simply runs an adaptive first-order OCO algorithm as a blackbox on a specially constructed convex surrogate cost function sequence. On every round, the policy needs to compute only two gradients and an Euclidean projection. This is way more efficient compared to the policies proposed in the previous works \citep{guo2022online, neely2017online}, which need to solve expensive convex optimization problems on each round while yielding sub-optimal bounds. Furthermore, in the special case of time-invariant constraints, our results yield an efficient first-order OCO policy with competitive regret and CCV bounds \citep{mahdavi2012trading, jenatton2016adaptive, yi2021regret}. 

\item Our results are obtained by introducing a crisp and elegant potential function-based algorithmic technique for simultaneously controlling the regret and the CCV. In brief, the regret and CCV bounds are derived from a single inequality that arises from plugging in off-the-shelf adaptive regret bounds in a new regret decomposition result (Eqn.\ \eqref{gen-reg-decomp}). This new analytical technique might also be of independent interest. 
%\item Finally, we experiment with our algorithm in the credit card fraud detection setting with a highly imbalanced dataset and obtain 

%\item Our proposed meta-policy for the \ocs ~problem is \emph{parameter-free} and does not need any non-causal information, including uniform upper bounds to the gradients or the strong-convexity parameters of the future constraint functions. Yet, it achieves the optimal violation bounds without making any extraneous assumptions, such as Slater's condition (see Table \ref{gen-oco-review-table}).
%Since the meta-policy is oblivious to the details of the OCO sub-routine, efficient policies with better regret bounds can be obtained by using known OCO sub-routines that exploit the special structure of the feasible action set (\emph{e.g.,} FTPL for combinatorial problems).  

%\item In the special case of the Online Constraint Satisfaction (\textsc{OCS}) problem, we prove optimal constraint violation penalty bounds for convex and strongly convex constraints with no extraneous assumptions (\emph{e.g.,} Slater's condition). 
%\item When the worst-case regret on a round is non-negative for our proposed algorithm, we derive 
%tighter bounds on regret and CCV that match the best bounds derived in \citet{neely2017online} under the Slater's condition.  
%for COCO are obtained by using a new criterion - the non-negativity of the worst-case regret on a round (see Table \ref{gen-oco-review-table}). 
%This provides an alternative to the stronger and often infeasible assumption of Slater's condition considered in the literature. 
%Surprisingly, this new condition leads to an $O(\log T)$ CCV bound for convex (not necessarily strongly convex) constraints and strongly convex cost functions.  
%Furthermore, we prove the minimax optimal bounds for the constrained OCO problem by introducing a mild assumption on the sign of regret. 
%\item As a by-product of our algorithm for the \ocs ~problem, we obtain a new class of stabilizing control policies for the classic input-queued switching problem with adversarial arrival and service processes (see Section \ref{app}). 
%\item we obtain a strongly stable queue control policy for adversarial environments.  
%\item Different from previous works on this problem, which mostly proceed by constructing and bounding an augmented Lagrangian with a quadratic penalty function \citep{yuan2018online}, 
%\item We reduce the problem to a standard OCO problem via a new regret decomposition inequality. As a result, unlike the previous works, our algorithm is \emph{parameter-free} - it does not need any non-causal information, \emph{e.g.,} uniform upper bounds on the gradients or strong-convexity parameters of the future cost/constraint functions. 
%To the best of our knowledge, this is the first such separability result for this problem and might be of independent interest. 
%\item Thanks to the general regret decomposition result, our proof arguments are crisp, requiring only a few lines of algebra. This should be contrasted with often long and intricate arguments in many of the previous papers.

%\item As an application of the algorithm developed for the \textsc{OCS} problem, we obtain a class of stabilizing control policies for input-queued switches with adversarial arrival and/or service processes. 
\item Finally, in Section \ref{expts}, we evaluate the practical performance of our algorithm in the online credit card fraud detection problem with a highly imbalanced dataset.
\end{enumerate}
%The practical performance of our algorithm in the online credit card fraud detection problem with a highly imbalanced dataset has been discussed in Section \ref{expts}.
%\begin{table*}[t]
%%\begin{table*}[t]
%%\hspace{-70pt}
%  \begin{tabular}{llllll}
%    \toprule
%    %\multicolumn{2}{c}{Part}                   \\
%   % \cmidrule(r){1-2}
%    \small {Reference}   & \hspace{-20pt}\small {Benchmark} & \small {Violation} & \small {Algorithm}& \small {Assumptions} \\
%    \midrule
%  %  a & b& c& d & e  \\
%   \small {\citet{jenatton2016adaptive}}  & \small {$1$} & \small {$O(T^{3/4})$} & \small {Primal-Dual GD} & \small {Fixed constraints, Known $G$} \\
%  \small {\citet{yuan2018online}}  & \small {$1$} & \small {$O(1)$} & \small {Primal-Dual MD} & \small {Fixed constraints} \\
%  \small {\citet{yu2017online}} & \small {$T$} & \small {$O(\sqrt{T})$}&\small {OGD+drift+penalty} & \small {Stochastic constraints, Slater} \\
%  \small {\citet{yi2023distributed}}  & \small {$1$} & \small {$O(\sqrt{T})$} & \small {Primal-Dual MD} & \small {Known $G$} \\
%    \small {\citet{pmlr-v70-sun17a}} & \small {$1$} & \small {$O(T^{3/4})$}& \small {OMD} & \small {Known $G$}  \\
%    \small {\citet{neely2017online}} & \small {$1$} & \small {$O(\sqrt{T})$} & \small {OGD} & \small {Slater} \\
%    \small {\citet{georgios-cautious}}  & \small {$S$} & \small {$O(\sqrt{ST})$} & \small {OGD} & \small {Known $S$} \\
%       \small{\citet{guo2022online}}& \small {$1$} & \small {$O(T^{3/4})$} & \small {Convex opt. each round} & \small {Full access to $\{g_t\}_{t\geq 1}$} \\
%      %  \citet{guo2022online} & Adversarial, convex & $1$ & $O(T^{3/4})$ & Convex opt. each round & Full access to $\{g_t\}_{t\geq 1}$ \\
%         %\citet{guo2022online} & Adversarial, strongly-convex & $1$ & $O(\sqrt{T \log T})$ & Convex opt. each round & -do-, Known $\alpha$ \\
%    \small {\textbf{This paper}}  & \small {$S$} & \small {$O(\sqrt{ST})$} & \small {Any adaptive} & - \\
% \small {\textbf{This paper}} & \small {$1$} & \small {$O(\log T)$} &\small {Any adaptive} & \small{Strongly-convex constraints} \\
%       \bottomrule
%  \end{tabular}
%  \vspace{5pt}
%  \caption{\small{Summary of the results for the \textsc{OCS} problem. Unless indicated otherwise, the constraints are assumed to be convex and adversarially chosen. Results from other papers, which consider only the constrained OCO problem, have been appropriately adapted for the \textsc{OCS} problem by taking the cost function to be identically equal to zero and quoting the best violation bound. Excepting our paper, all other papers referenced above bound the cumulative violation over the entire horizon of length $T$, which is weaker than \eqref{violation-def1} for constraints assuming both positive and negative signs. $G$ is an upper bound to the norm of the gradients, $\alpha$ is the strong-convexity parameter; Abbreviations: MD= Mirror Descent, OMD = Online Mirror Descent, OGD = Online Gradient Descent.} }
%    \label{review-table}
%\end{table*}
% \begin{table*}[t]
% %\hspace{-30pt}
%   \title{Summary of the results for the constrained OCO problem}
%   %\centering
%   \hspace{-80pt}
%   \begin{tabular}{llllll}
%     \toprule
%     %\multicolumn{2}{c}{Part}                   \\
%    % \cmidrule(r){1-2}
%    \small { Reference}  &  \small {Constraints} & \hspace{-20pt}\small {Regret} & \small {Violation} & \small {Algorithm}& \small {Assumption} \\
%     \midrule
%   %  a & b& c& d & e  \\
%     \small {\citet{jenatton2016adaptive}} & \small {Fixed} & \small {$O(T^{\max(\beta, 1-\beta)})$} & \small {$O(T^{1-\beta/2})$} & \small {Primal-Dual GD} & \small {Fixed constraints} \\
%   \small {\citet{yuan2018online}} & \small {Fixed} & \small {$O(T^{\max(\beta, 1-\beta)})$} & \small {$O(T^{1-\beta/2})$}  & \small {Primal-Dual MD} & \small {Fixed constraints} \\
%  % \citet{yu2017online} & Stochastic & $O(\sqrt{T})$ & $O(\sqrt{T})$& OGD+drift+penalty & Slater condition \\
%   \small {\citet{yi2021regret}} & \small {Fixed} & \small {$O(T^{\max(\beta, 1-\beta)})$} & \small {$O(T^{(1-\beta)/2})$} & \small {OGD+drift+penalty} & \small {Fixed constraints} \\ 
%   \small {\citet{yi2022regret}} & \small {Adversarial, str-convex cost} & \small {$O(T^{\beta})$} & \small {$O(T^{1-\beta/2})$} & \small {Primal-Dual} & \small {Known $G, \alpha$} \\ 
%   \small {\citet{yi2023distributed}} & \small {Adversarial} & \small {$O(T^{\max(\beta, 1-\beta)})$} & \small {$O(T^{1-\beta/2})$} & \small {Primal-Dual MD} & \small {Known $G$} \\
%     \small {\citet{yi2023distributed}} & \small {Adversarial, str-convex cost} & \small {$O(\log(T))$} & \small {$O(\sqrt{T \log T})$} & \small {Primal-Dual MD} & \small {Known $G, \alpha$} \\
%     \small {\citet{pmlr-v70-sun17a}} & \small {Adversarial} & \small {$O(\sqrt{T})$} & \small {$O(T^{3/4})$}& \small {OMD} & \small {Known $G$}  \\
%     \small {\citet{neely2017online}} & \small {Adversarial} & \small {$O(\sqrt{T})$} & \small {$O(\sqrt{T})$} & \small {OGD+drift+penalty} & \small {Slater condition} \\
%     %\citet{georgios-cautious} & Adversarial, convex & $S$ & $O(\sqrt{ST})$ & OGD & Known $S$ \\
%         \small {\citet{guo2022online}} & \small {Adversarial} & \small {$O(\sqrt{T})$} & \small {$O(T^{3/4})$} & \small {Convex opt. each round} & \small {Full access to $\{g_t\}_{1}^T$} \\
%         \small {\citet{guo2022online}} & \small {Adversarial, str-convex cost} & \small {$O(\log T)$} & \small {$O(\sqrt{T \log T})$} & \small {Convex opt. each round} & \small {Full access to $\{g_t\}_{1}^T$} \\
%          %\citet{guo2022online} & Adversarial, strongly-convex & $1$ & $O(\sqrt{T \log T})$ & Convex opt. each round & -do-, Known $\alpha$ \\
%   \small {\textbf{This paper}} &  \small {Adversarial} & \small {$O(\sqrt{T})$} & \small {$O(T^{3/4})$} & \small {Any adaptive OCO} & - \\
%      \small {\textbf{This paper}} &  \small {Adversarial} & \small {$O(\sqrt{T})$} & \small {$O(\sqrt{T})$} & \small {Any adaptive OCO} & \small {$\mathcal{R}_T \geq 0$}\\
%  \small {\textbf{This paper}} &  \small {Adversarial, str-convex cost}& \small {$O(\log T)$} & \small {$O(\sqrt{T\log T})$} &\small {Any adaptive OCO} & \small{Known $G, \alpha$} \\
%  \small {\textbf{This paper}} &  \small {Adversarial, str-convex cost}& \small {$O(\log T)$} & \small {$O(\frac{\log T}{\alpha})$} &\small {Any adaptive OCO} & \small {$\mathcal{R}_T \geq 0, \textrm{known} ~ G, \alpha$} \\
%        \bottomrule
%   \end{tabular}
%   \vspace{5pt}
%   \caption{\small{Summary of the results for the constrained OCO problem. Unless mentioned otherwise, we assume $ 1$ feasibility, arbitrary convex constraints, and convex cost functions while stating the bounds. In the above, $0\leq \beta \leq 1$ is an adjustable parameter. The parameter $\alpha$ denotes the strong convexity parameter of the cost functions. $G$ denotes a uniform upper bound to the gradient of cost and constraint functions. $\mathcal{R}_T$ denotes the worst-case regret against admissible actions $\mathcal{X}^\star$.}}
%     \label{gen-oco-review-table}
% \end{table*}

%\paragraph{On the analytical contribution:}
%Lyapunov or the potential function method is a flexible technique which has been extensively used in the literature for designing and analyzing control policies for linear and non-linear systems. The stochastic variant of the Lyapunov method, and especially the Foster-Lyapuov theorem, has played a pivotal role in designing stabilizing control policies for stochastic queueing networks \citep{meyn2008control, neely2010stochastic}. However, this classical technique has found quite limited application to systems with adversarial dynamics (see \citet{xu2023drift} for a recent survey of the application of the drift method to networking and learning problems). In this paper, we show how the Lyapunov and the drift-plus-penalty technique \citep{neely2010stochastic} can be effectively combined with the OCO framework to design new online algorithms with tight performance bounds. We expect that the analytical methods developed in this paper can be generalized to more complex problems with adversarial inputs. 
%\paragraph{\bf Note:} If a convex function $f$ is not differentiable at the point $x$ then by $\nabla f(x)$ we denote any sub-gradient of the function at $x$. Recall that any convex function could be non-differentiable only on a set of measures at most zero \citep[Theorem 25.5]{convex_rockafeller72}. Hence, by smoothly perturbing at all points of non-differentiability by an infinitesimal amount, we can alternatively and without any loss of generality assume all cost and constraint functions to be differentiable without altering the regret/constraint violation bounds presented in this paper \citep{hazan2007adaptive}.  
% 






























%\input{lower_bounds2}
%\vspace{-0.1in}
\section{The Constrained OCO (COCO) Problem} \label{gen_oco}
%\vspace{-0.1in}
%\edit{Remove all $\psi$ and mention at the end that we can also consider surrogate costs.}
\iffalse
In this section, we generalize the $\ocs$  and the usual unconstrained OCO  and consider the COCO where
 on each round $t, t=1, \dots, T$ an online policy first chooses an admissible action $x_t \in \mathcal{X}$ 
% from a feasible closed and bounded convex set $\mathcal{X}.$ On the same round, 
 and then the adversary chooses a convex cost function $f_t: \mathcal{X} \to \mathbb{R}$ and a constraint of the form $g_t(x) \leq 0,$ where $g_t: \mathcal{X} \to \mathbb{R}$ is a convex function.\footnote{For notational simplicity, in this section, we assume that only one constraint function is revealed on each round (\emph{i.e.,} $k=1$). The general case of $k>1$ can be handled similarly as in Section \ref{simul_constr}.} Let $\mathcal{X}^\star$ be the feasible set satisfying all constraints as defined in Assumption \ref{feas-constr}. Then, the objective is to simultaneously minimize the regret and the CCV as defined in \eqref{intro-regret-def} and \eqref{intro-gen-oco-goal} with $k=1$, respectively. 
 
 %Our objective is to design an online policy that achieves a sublinear cumulative violation and a sublinear worst-case regret over the feasible set $\mathcal{X}^\star$. Specifically, we require the following conditions to be satisfied
% \begin{eqnarray} \label{gen-oco-goal}
% 	\mathbb{V}(T)\equiv \sum_{t=1}^T (g_t(x_t))^+ = o(T),~\textrm{and}~
% 	\sup_{x^\star \in \mathcal{X}^\star} \mathcal{R}_T(x^\star)= o(T),~ \forall T \geq 1,
% \end{eqnarray}
% where the regret $\mathcal{R}_T(x^\star)$ w.r.t. the fixed action $x^\star\in \mathcal{X}^\star$ has been defined earlier in Eqn.\ \eqref{regret-def}. 
 \iffalse The COCO  can be motivated by the following offline convex optimization problem where the functions $\{f_t, g_t\}_{t=1}^T$ are known \emph{a priori}: 
 %and the objective is to solve the following problem:
 \begin{eqnarray*}
 	\min \sum_{t=1}^T f_t(x),
 \end{eqnarray*}
subject to the constraints
 \begin{eqnarray*}
 	g_t(x) \leq 0, ~ \forall t \in [T],~ 
 	x \in \mathcal{X}.
 \end{eqnarray*}
 \fi
 Since $(g_t(x_t))^+ \geq 0,$  \eqref{intro-gen-oco-goal} uses a stronger definition of CCV compared to the $\ocs$  (c.f.  \eqref{violation-def1}), where the strict feasibility at some round may compensate for infeasibility in other rounds  \citep{guo2022online}. With abuse of notation, from here onwards, we redefine the convex constraint functions as $g_t(x) \gets (g_t(x))^+, \forall x \in \mathcal{X}, t\geq 1$. In other words, the constraints are pre-processed by passing them through the standard ReLU unit, ensuring that $g_t(x) \geq 0, \forall x \in \mathcal{X}, \forall t \geq 1.$
 \fi
 
 %Having $g_t(x_t) \geq 0,$ allows some technical advantages for establishing tighter regret and CCV by ensuring the monotonicity of the queue lengths as defined in  \eqref{q-ev} below.  %As we will see, this can be attributed to the fact that, unlike Section \ref{simul_constr}, here, we no longer consider strongly convex constraint functions (however, we do consider strongly convex cost functions). 
 %An upper bound to the above violation metric $\mathbb{V}(T)$ ensures that the constraint violation in one round can not be compensated by a strictly feasible constraint in a different round \citep{guo2022online}.
%where the function $g_t: \mathcal{X} \to \mathbb{R}$ is also $\alpha$-strongly convex. Both the functions $f_t$ and $g_t$ are assumed to be $L$-Lipschitz. 
%The policy chooses its action \emph{before} the adversary reveals its choices for $f_t$ and $g_t$ for round $t$. 
%No non-causal information, including upper bounds to the norm of the gradients, and/or strong-convexity parameters of either the cost function or the constraints is known to the policy.
%Compared to $\ocs$ that did not require the knowledge of horizon length $T$, with COCO, we assume that $T$ is known and will be used for setting a parameter. 
%However, this assumption can be dropped using the standard doubling trick \citep{hazan2016introduction}.  
% We begin the discussion on COCO with a lower bound on CCV and regret, which does not follow from the regret lower bound for the unconstrained OCO. 
\subsection{Assumptions}  \label{assump}
%In this section, we list the general assumptions which apply to both the \ocs ~problem and the COCO, described later in Section \ref{gen_oco}. Since the \ocs ~problem does not contain any cost function, the cost functions mentioned below necessarily refer to COCO only.
We now state the assumptions considered in this paper. These assumptions are standard in literature on the COCO problem \citep{guo2022online, yi2021regret, neely2017online}.
\begin{assumption}[Convexity] \label{cvx}
	The cost function $f_t: \mathcal{X} \mapsto \mathbb{R}$ and the constraint function $g_{t,i}: \mathcal{X} \mapsto \mathbb{R}$ are convex for all $t\geq 1, i\in [k]$. The admissible set (\emph{a.k.a.} the decision set or the action set) $\mathcal{X} \subseteq \mathbb{R}^d$ is closed and convex and has a finite Euclidean diameter $D$. 
 %Moreover, $D$ is known ahead of time.
\end{assumption}
%\vspace{-0.18in}
\begin{assumption}[Lipschitzness] \label{bddness}
 %We have $\textrm{diam}(\mathcal{X}) \leq D, ||\nabla f_t(x)||_2 \leq G/2, \textrm{and}~ ||\nabla g_t(x))||_2 \leq G/2,~\forall t, \forall x\in \mathcal{X}$ for some finite constants $D$ and $G.$ If the functions are not necessarily differentiable, we require that the maximum magnitude of the subgradients be bounded accordingly.  Each
All cost functions $\{f_t\}_{t\geq 1}$ and the constraint functions $\{g_{t,i}\}_{i\in [k], t\geq 1}$'s are $G$-Lipschitz. In other words, for any $x, y \in \mathcal{X},$ we have 
 \begin{eqnarray*}
 	|f_t(x)-f_t(y)| \leq G||x-y||,~
 	|g_{t,i}(x)-g_{t,i}(y)| \leq G||x-y||, ~\forall t\geq 1, i\in [k].
 \end{eqnarray*}
	\end{assumption}

	Unless specified otherwise, the norm $||\cdot||$ will refer to the standard Euclidean norm and $\nabla f$ will refer to an arbitrary subgradient of a convex function $f$. Assumption \ref{bddness} implies that the $\ell_2$-norm of the (sub)gradients of the cost and constraint functions are uniformly upper-bounded by $G$ over the admissible set $\mathcal{X}.$ Finally, we make the following feasibility assumption about the constraint functions.
\begin{assumption}[Feasibility] \label{feas-constr}
	There exists a feasible action $x^\star \in \mathcal{X} $ s.t. $g_{t,i}(x^\star) \leq 0, \forall t, i.$ The feasible set $\mathcal{X}^\star$ is defined to be the set of all feasible actions. The feasibility assumption implies that $\mathcal{X}^\star \neq \emptyset.$
\end{assumption}
The feasibility assumption distinguishes the cost functions from the constraint functions and is commonly assumed in the literature \citep{guo2022online, neely2017online, yu2016low,yuan2018online,yi2023distributed, georgios-cautious}.  In Section \ref{simul_constr}, we will consider a constraint-only variant of the problem where the feasibility assumption (Assumption \ref{feas-constr}) will be relaxed. 
See Appendix \ref{app:assumptions} for a brief discussion on the assumptions.

\paragraph{Remarks:} On each round, multiple constraints of the form $g_{t,i}(x) \leq 0, i\in [k]$ can be replaced by a single new constraint $g_t(x) \leq 0$
%Multiple constraints per round can be reduced to a single constraint by simply clipping each of the constraints and 
%by replacing them with a new constraint 
where the constraint function $g_t$ is defined to be the pointwise maximum of the given constraints, \emph{i.e.,} 
 $g_t(x) \equiv \max_{i=1}^k g_{t,i}(x), x \in \mathcal{X}.$ It is easy to verify that if each of the constraint functions $\{g_{t,i}\}_{i=1}^k$ satisfies the above assumptions, then the constraint function $g_t$ defined above also satisfies the assumptions. Hence,  throughout this section and without loss of generality, we will assume that only one constraint function is revealed on each round. That being said, under the relaxed feasibility assumption in Section \ref{simul_constr}, this trick does not work and there we will need to consider the full set of $k$ constraint functions.  
 %(\emph{i.e.,} $k=1.$)
 %\edit{include a table summarizing the known results for this problem}.
%\begin{framed}

%\paragraph{Note:} 
%\begin{enumerate}
	%The regret bounds derived below hold in the general case when, instead of a linear cost function, the adversary chooses any Lipschitz continuous convex cost function $f_t$ and we set $c_t=\nabla f_t(x_t).$ Due to the Lipschitzness assumption, $||c_t||$ is bounded for all $t \geq 1$.
 
 %In the case of multiple constraints, we construct a single constraint by summing up the coefficient vectors and pass this single aggregate constraint to our algorithm. Alternatively, we can also associate each constraint with a separate queueing process (see Section \ref{app} for an example of the latter construction). 
%\end{enumerate}    
%\end{framed}

\iffalse
For any round $t\geq 1,$ let $\mathcal{X}_t$ be the set of all feasible actions satisfying \emph{all} constraints up to and including time $t,$ \emph{i.e.,}
\[\mathcal{X}_t = \{x \in \mathcal{X} : g_\tau(x) \leq 0,  1\leq \tau\leq T\}.\] 
The set $\mathcal{X} \equiv \mathcal{X}_T$ is assumed to be non-empty. To maintain consistency in the notations, we set $\mathcal{X}_0=\mathcal{X}.$ The regret of the online policy with respect to a fixed action $x_T^\star \in \mathcal{X}_T$ and its cumulative violation penalty $\mathbb{V}_\mathcal{I}$ over an interval $\mathcal{I}\equiv [a,b] \subseteq [T]$ are respectively defined as 
\begin{eqnarray}
\textrm{Regret}_T(x_T^\star) &=& \sum_{t=1}^T \big(f_t(x_t)-f_t(x_T^\star)\big) \label{reg-def} \\
\mathbb{V}_\mathcal{I} &=& \sum_{t \in \mathcal{I}} g_t(x_t). \label{violation-def}
\end{eqnarray} 
\fi
%\paragraph{Pre-processing the constraint functions by clipping:} 
%%In this section, we pass pre-processed constraint functions to the meta-learning algorithm, where the pre-processing step simply 
%In the COCO , we first pre-process each of the constraint functions by clipping them below zero, \emph{i.e.,} we redefine the $t$\textsuperscript{th} constraint function as \footnote{Since, in this section, we work exclusively with the pre-processed constraint functions, by a slight abuse of notations, we use the same symbol $g_t$ for the pre-processed functions.} $g_t(x) \gets (g_t(x))^+, x \in \mathcal{X},~~ \forall t\geq 1. $
%It immediately follows that the pre-processed constraint functions are also convex and keep the feasible set $\mathcal{X}^\star$ unchanged.
%%represent the same constraint set as the original constraint $g_t(x)\leq 0.$ 
%%With this pre-processed constraint functions, the queue-length sequence, defined recursively in Eqn.\ \eqref{q-ev2}, becomes non-decreasing. 
%The pre-processing step allows us to bound the hard constraint violation $\mathbb{V}(T)$ as defined in \eqref{intro-gen-oco-goal}.  
%%Hence,
%%an upper bound on the cumulative violation penalty on the pre-processed constraint functions \eqref{violation-def}, implies the same upper bound on the cumulative violation penalty of the original constraint functions. 
%Furthermore, the pre-processing step also offers some technical advantages for establishing tighter regret and constraint violation bounds by ensuring the monotonicity of the queue lengths as defined in  \eqref{q-ev} below.
\iffalse
\edit{This needs to be rewritten}
Note that one can redefine $g_t(x) \gets \psi(g_t(x))$,
where $\psi : \mathbb{R} \to \mathbb{R}$ is a non-decreasing convex \emph{penalty} function with the property that $\psi(z) \leq 0, \forall z\leq 0.$ Examples for the penalty function include $\psi(z)=z, (z)^+, [(z)^+]^2.$ Under the above assumptions, it immediately follows that the pre-processed constraint functions are also convex and encode the same constraint set as the original constraint $g_t(x)\leq 0.$ In the special case when $\psi(z)=(z)^+,$ we have $\psi(g_t(x)) \geq g_t(x).$ In this case, referred to as the \emph{hard constraints} in \citet{guo2022online}, constraint violation in one round can not be compensated by a strictly feasible constraint in some other round. Hence,
an upper bound on the cumulative violation penalty on the pre-processed constraint functions \eqref{violation-def} implies the same upper bound on the cumulative violation penalty of the original constraint functions. 
\fi
%However, the bounds on the gradients and the numerical value of constraint violation may differ depending on the function $\psi(\cdot)$.

%Our objective is to design an online learning policy that achieves a sublinear cumulative violation penalty uniformly over \emph{any} interval   
%We aim to design an online learning policy $\{x_t\}_{t\geq 1}$ that admits a sublinear maximum violation penalty 
%$\sup_{\mathcal{I}} \mathbb{V}_{\mathcal{I}}$ and a sublinear regret with respect to any $x^\star \in \mathcal{X}.$
\iffalse 
 \subsection{Assumptions} \label{assumps}
 We now define a set of assumptions of interest. Each of our results will be valid under some subset of the following assumptions. 
%As standard in the literature, we will assume the feasible set is bounded and that the norm of the gradients of the functions is uniformly bounded. This is made precise in the following assumption.
\begin{assumption}[(Convexity and Boundedness)] \label{bddness}
1. 	Each of the cost functions $f_t : \mathcal{X} \mapsto \mathbb{R}$ and the penalty function $\psi \circ g_t : \mathcal{X} \mapsto \mathbb{R}$ are convex for all $t\geq 1$. The feasible set $\mathcal{X} \subset \mathbb{R}^d$ is closed and convex. 

2.  We have $\textrm{diam}(\mathcal{X}) \leq D, ||\nabla f_t(x)||_2 \leq G, \textrm{and}~ 2\psi'(g_t(x))||\nabla \psi(g_t(x))||_2 \leq G,~\forall t, \forall x\in \mathcal{X}$ for some finite constants $D$ and $G.$ If the functions are not necessarily differentiable, we require that the maximum magnitude of the subgradients be bounded accordingly. We emphasize that the values of the parameters $G$ and $D$ are not necessarily known to the policy.
	\end{assumption}
%Next we define the notion of the feasibility of an admissible action $x^\star \in \mathcal{X}.$
%\begin{framed}
\begin{assumption}[Feasibility] \label{feas}
	%Let $\eta^\star \geq 0$ be a non-negative constant.
	 An admissible action $x^\star \in \mathcal{X}$ is called  feasible if $\psi(g_t(x^\star)) \leq 0, \forall t.$ 
\end{assumption}
The following condition strengthens Assumption \ref{feas}.

\begin{assumption}[Slater's Condition] \label{slater}
	Let $\eta^\star > 0$ be a strictly positive constant. An admissible action $x^\star \in \mathcal{X}$ is said to satisfy Slater's condition if  $\psi(g_t(x^\star)) \leq -\eta^\star, \forall t.$ 
\end{assumption}
The following regularity assumption pertains to a certain class of worst-case adversaries.
\begin{assumption}[Non-negativity of Regret] \label{non-neg-regret}
	The adversary is said to satisfy the uniform non-negativity of regret condition under a policy if for each $t\geq 1,$ we have $\textrm{Regret}_t(x_t^\star) \geq 0$ for some $x_t^\star \in \mathcal{X}_t.$
	\end{assumption}
	\fi
%	We emphasize that our proposed adaptive learning policy does not assume the knowledge of any of the parameters $G,D$ or $x^\star.$
	
	
%\hline
\iffalse
\textbf{Remarks} 1. The assumption of the existence of actions satisfying Slater's condition is a strong one which severely limits the practical applicability of the theory. In particular, if the penalty function $\psi(\cdot)$ is non-negative, then clearly, no feasible action satisfying Slater's condition exists. In this paper, we will obtain sublinear regret and constraint violation penalty bounds simultaneously by considering Assumption \ref{bddness} and Assumption \ref{feas} only. 
 However, we will see that Assumption \ref{slater} or Assumption \ref{non-neg-regret} is sometimes useful for proving improved performance bounds. %Furthermore, in the final section, we will show that neither Assumption \ref{slater} or Assumption \ref{non-neg-regret} is an absolute must - there exist online policies that lead to the same improved regret bound using Assumption \ref{bddness} and Assumption \ref{feas} only.   

2. In Section \ref{uncond}, we will show that Assumption \ref{non-neg-regret} can be dropped from the results by slightly modifying the proposed online policy.
%\end{framed}
%\end{framed}
%\hline
%\begin{framed}
%  \textbf{Notes:} 
\fi
%5\end{framed}
%\hline
%Our main results are summarized in the table below:
%\textcolor{blue}{Draw a table}


\iffalse
\subsection{Lower bound on regret and CCV for COCO} \label{lower_bound_section}
In this section, we prove that under Assumptions \ref{cvx}, \ref{bddness}, and \ref{feas-constr}, the regret and the CCV achieved by any online policy for the COCO over any horizon of length $T$ are both lower bounded by $\Omega(\sqrt{T}).$ Note that the CCV lower bound does not follow from the corresponding regret lower bound for the OCO problem due to the additional Assumption \ref{feas-constr}, which puts an implicit constraint on the admissible constraint functions.
\begin{proof}
Using the standard lower bound proof strategy, we define an ensemble of \textsc{OCS}  problem instances with a single linear constraint function on every round. Using a Coupon collector's argument, we then argue that any online policy must incur at least $\Omega(\log T)$ cumulative constraint violation on at least one instance of the ensemble. The detailed construction is given below:
	\paragraph{Action space $\Omega$:} The standard probability simplex $\Delta_d$ with $d$ coordinates. The time horizon $T$ is taken to be $T=\frac{1}{2}d \log d.$
	\paragraph{Constraint functions:} 
	On each round $t \in [T]$ let random variable $I_t$ be distributed uniformly on $\{1,2,\ldots,d\}$ independently of everything else. The constraint function on round $t$ is taken to be a random linear function with coefficients $g_{t,i}:=\mathds{1}(i=I_t), i \in [d].$ In other words, the $t$\textsuperscript{th} constraint is defined as $x_{I_t}\leq 0.$ Clearly, the diameter of the admissible set is bounded as $D \leq \sqrt{2}$ and the constraint functions are $1$-Lipschitz\footnote{To see the diameter bound, let $x,y \in \Delta_d.$ We have $||x-y||^2 = \sum_i (x_i-y_i)^2 \leq \sum_i |x_i-y_i| \leq \sum_i x_i + \sum_i y_i = 2.$}.
	 
%	Let $\{\epsilon_{t,i}\}_{t=1}^T, i=1,2,$ be a sequence of iid random variables each distributed uniformly on the interval $[-1,1]$. We consider an ensemble of constraint functions where the $t$\textsuperscript{th} constraint is randomly chosen to be 
%	\begin{eqnarray} \label{constr_def_lb}
%	 g_{t}(x_t) \equiv \epsilon_{t,1}x_{t,1} + \epsilon_{t,2}x_{t,2} \leq 0, ~ 1 \leq t \leq T.
%	 \end{eqnarray}
	\paragraph{Checking feasibility:} We now argue that, for a large enough horizon $T$, the above sequence of constraint functions is feasible with high probability. Using a standard result on the Coupon collector's problem \citep[Theorem 3.8]{motwani1995randomized}, we conclude that w.h.p. there exists a coordinate $I^\star \in [d]$ which does not appear in any of the constraints. We can now find a feasible action $\bm{x}^\star$ for the above $T$ constraints by choosing $x_i^\star=\mathds{1}(i=I^\star), i\in [d].$
	\paragraph{Bounding cumulative violations:} Consider any online policy which takes action $x(t) \in \Delta_d$ on round $t$ and let $\{\mathcal{F}_t\}_{t=1}^T$ denote the natural filtration. The conditional expectation and variance of the constraint violation incurred by the policy on round $t$ is given by \[\mathbb{E}[g_t(x_t)|\mathcal{F}_{t-1}]= \frac{1}{d}\sum_{i=1}^d x_{t,i}= \frac{1}{d},~~ \textrm{Var}(g_t(x_t)) < \frac{1}{d}\sum_{i}x^2_{t,i} \leq \frac{1}{d}.  \]
	%~ \mathbb{E}g_t^2(x_t)=\frac{1}{d}\sum_{i} x_{t,i}^2.\]
	Hence, for any online policy, the expected cumulative violation is given by $\mathbb{E}V_T = \mathbb{E}\sum_{t=1}^T g_t(x_t)=\frac{T}{d}$ with its variance upper bounded as 
	$\textrm{Var}(V_T) = \sum_{t=1}^T \textrm{Var}(g_t(x_t)) \leq \frac{T}{d},$ where in the last step we have used the Pythagorean formula for the zero-mean martingale sequence $\{\sum_{\tau=1}^tg_\tau(x_\tau) - \frac{t}{d}\}_{t \geq 1}$ \citep[Section 12.1, Eq. (b)]{williams1991probability}.
	%the fact that the constraints are independently selected. 
Hence, we have  
	\[ \mathbb{E}V_T^2 = \textrm{Var}(V_T)+ (\mathbb{E}V_T)^2 \leq \frac{T^2}{d^2}+\frac{T}{d}.\]
	Hence, using Paley-Zygmund inequality for the non-negative cumulative violation random variable $V_T$ \citep[Theorem 1.4.3(b)]{chandra2012borel}, we have 
	\[ \mathbb{P}\big(V_T \geq \frac{T}{2d}\big) \geq \frac{1}{4} \frac{(\mathbb{E}V_T)^2}{\mathbb{E}V_T^2}\geq \frac{1}{4} \frac{T^2/d^2}{T^2/d^2+T/d}\geq \frac{1}{8},\]
	where, in the last step, we have used the fact that $T=\frac{1}{2}d\log d$ and assumed $\log d \geq 2.$ Hence, from the previous steps, it follows that for any online algorithm, there exists a sequence of feasible constraint functions $\{g_t\}_{t=1}^T$ for which the algorithm incurs a cumulative violation of \[V_T \geq \frac{T}{2d}= \frac{1}{4}\log d = \frac{1}{4}(\log T - \log \log d + \log 2)=\Omega(\log T - \log \log T)=\Omega(\log T).\]	

\end{proof}

%\subsection*{Proof for $\Omega(\sqrt{T})$ 
%Regret and $\Omega( \sqrt{T})$ Constraint Violation bound}
\begin{theorem}\label{thm:lbcoco}
Under Assumptions \ref{cvx}, \ref{bddness}, and \ref{feas-constr}, the regret and CCV as defined in \eqref{intro-regret-def} and \eqref{intro-gen-oco-goal} with $k=1$ are both $\Omega(\sqrt{T})$.
\end{theorem}
Proof of Theorem \ref{thm:lbcoco} can be found in Appendix \ref{app:lbcoco}.

\fi


	 
 
	%	
%	In other words, we show that almost surely for a given random sequence of $T$ constraint functions, there exists a probability vector $x$ such that we have $g_t(x) \leq 0.$ We can write 
%	\begin{eqnarray*}
%		g_t(x) = \epsilon_{t,1}x_{1}+ \epsilon_{t,2} (1-x_1) = \epsilon_{t,2} + (\epsilon_{t,1}-\epsilon_{t,2}) x_1. 
%	\end{eqnarray*}
%	Now let us choose $x_1 \sim \textsf{U}(0,1).$ Hence, we have 
%	\begin{eqnarray*}
%		\mathbb{P}(\cap_{t=1}^T g_t(x) \leq 0) = \int_{0}^1 \mathbb{P}(\cap_{t=1}^T g_t(x) \leq 0|x) dx = \frac{1}{2^T}>0,
%	\end{eqnarray*}
%	where the last line follows from the symmetry of the distribution around zero.
	%Clearly, the above sequence is feasible by taking $x_1=x_2=1/2.$


\iffalse
\subsubsection{Lower Bound to the Achievable CCV}
%We now establish a lower bound to the achievable violation penalty for 
The following result shows that any online policy must incur a CCV of $\mathcal{X}(\sqrt{T}).$
%The following result is straightforward.
%\begin{framed}
\begin{theorem}[(Lower bound on the Constraint Violations)]
	Consider the constrained OCO problem with linear cost functions and affine constraints. Then any online policy will incur CCV $\mathbb{V}_T$ of at least $\mathcal{X}(\sqrt{T})$ over a horizon of length $T.$ 
\end{theorem}
%\end{framed}
\begin{proof}
	Assume that on round $t$, the adversary chooses a linear cost function $g_t(x)$ and an affine constraint $g_t(x) \leq b_t,$ where the constant $b_t$ will be specified later. Let $x^\star \in \mathcal{X}$ be the best-fixed action in hindsight, which minimizes the cumulative cost $\sum_{t=1}^T g_t(x).$ We now set the constant $b_t$ to be $b_t = g_t(x^\star), t \in [T].$ Clearly, $x^\star \in \mathcal{X}_T$ is a feasible point that satisfies all constraints up to time $T$. Hence, the CCV incurred by any online policy $\pi$ over the entire horizon $T$ is given by: 
	\begin{eqnarray*}
	\sup_{\mathcal{I}} \mathbb{V}_{\mathcal{I}} \geq \mathbb{V}_T = \sum_{t=1}^T \big(g_t(x_t) - b_t\big) = \sum_{t=1}^T g_t(x_t) - \sum_{t=1}^T g_t(x^\star) \equiv \textrm{Regret}_T(x^\star) \geq \mathcal{X}(\sqrt{T}), 
	\end{eqnarray*} 
	where the last inequality follows from the lower bound on the achievable regret for adversarially chosen linear cost functions over a convex domain. 
\end{proof}

\fi 
\vspace{-0.1in}
\subsection{Online Policy for COCO} \label{policy}
\iffalse
In this section, we propose a generalization of the \textsc{OCS} Meta-policy that is applicable for COCO and derive an upper bound on its regret and CCV.
As in the $\ocs$, we keep track of the CCV through the same queueing process $\{Q(t)\}_{t \geq 1}$ that evolves as follows.
% \footnote{Since the pre-processed constraint functions are non-negative, the $\max(0,\cdot)$ operation in Eqn.\ \eqref{q-ev} is superfluous.}:
 \begin{eqnarray} \label{q-ev}
 	Q(t) = (Q(t-1)+ g_t(x_t))^+, ~Q(0)=0. \end{eqnarray}
 	%Since the pre-processed constraint functions are non-negative, the $\max(0,\cdot)$ operation in  \eqref{q-ev} is superfluous. 
	Moreover, since $g_t(x_t)\geq 0$, we have $\mathbb{V}(t)= Q(t), \forall t.$
 	 As before, let us define the potential function $\Phi(t)\equiv Q^2(t).$ From  \eqref{q-bd2},
 	 \iffalse 
 	 Observe that for any real number $x$, we have $((x)^+)^2=xx^+.$ Hence, 
 \begin{eqnarray*}
 	Q^2(t) &=& \big(Q(t-1)+ g_t(x_t)\big)Q(t) \\
 	&=& Q(t-1)Q(t)+ Q(t) g_t(x_t)\\
 	&\stackrel{(a)}{\leq}& \frac{1}{2}Q^2(t)+ \frac{1}{2}Q^2(t-1) + Q(t) g_t(x_t). 
 \end{eqnarray*}
 where in (a), we have used the AM-GM inequality. Rearranging the above inequality,
% where $\max_{x \in \mathcal{X}} g_t^2(x)+ b_t^2 + 2|g_t(x)b_t| = \max_{x \in \mathcal{X}} (|g_t(x)|+|b_t|)^2\leq B.$ 
\fi
 the \emph{one-step drift} of the potential $\Phi(t)$ can be upper bounded as 
 \begin{eqnarray} \label{dr-bd}
 	\Phi(t)-\Phi(t-1) = Q^2(t)-Q^2(t-1)\leq 2Q(t)g_t(x_t).
 \end{eqnarray}
 Now, motivated by the drift-plus-penalty framework in the stochastic network control theory \citep{neely2010stochastic}, we \emph{define} a surrogate cost function $\hat{f}_t:\mathcal{X} \to \mathbb{R}$ as follows:
 \begin{eqnarray} \label{surrogate-def2}
 	\hat{f}_t(x) \equiv Vf_t(x) + 2Q(t) g_t(x), x \in \mathcal{X}, ~\forall t\geq 1,
 \end{eqnarray}
 where the first term corresponds to the cost and the second term is an upper bound to the drift of the quadratic potential function given by \eqref{dr-bd}. In the above,
  $V>0$ is an adjustable parameter that will be fixed later. It can be immediately verified that the surrogate function $\hat{f}_t(\cdot)$ is convex for all $t \geq 1$. 
 
  Next, we propose the \textbf{COCO Meta-Policy:} (Algorithm \ref{g-oco-policy}) to solve the COCO, that passes a sequence of surrogate cost functions $\{\hat{f}_t\}_{t\geq 1}$ to a base online convex optimization (OCO) subroutine with a data-dependent \emph{adaptive} regret-bound as given in Theorem \ref{data-dep-regret}. 

 \begin{algorithm}
\caption{The COCO Meta-policy}
\label{g-oco-policy}
\begin{algorithmic}[1]
\State \algorithmicrequire{Sequence of cost functions $\{f_t\}_{t \geq 1}$ and constraint functions $\{g_{t}\}_{t\geq 1},$ a base OCO sub-routine $\Pi$ with an adaptive regret bound (see Theorem \ref{data-dep-regret}), admissible set $\mathcal{X}$, parameter $V>0$}
\State \algorithmicensure{Sequence of admissible actions $\{x_t\}_{t\geq 1}$}
\INIT $x_1=\bm{0}, Q(0)=0, ~\forall i \in [k].$ 

\ForEach {each round $t \geq 1$}
\State The adversary reveals the cost function $f_t$ and the constraint function $g_{t}.$

\textbf{[Preprocessing]:} $g_t \gets \max(0, g_t).$ 
\State [\textbf{Queue updates}] $Q(t) = Q(t-1)+ g_{t}(x_t)$.
\State [\textbf{Surrogate cost}] Construct the surrogate cost function $\hat{f}_t(x) \equiv Vf_t(x) + 2Q(t) g_t(x)$.
\State [\textbf{OCO step}] Pass $\hat{f}_t$ to the base OCO sub-routine $\Pi$, that outputs an action $x_{t+1} \in \mathcal{X}.$ \label{oco-step}
\EndForEach
\end{algorithmic}
\end{algorithm}

 %\begin{framed}
 %\end{framed}
  %The proposed policy is formally described in Algorithm \ref{g-oco-policy}. 
 \iffalse
 To clarify this point further, let us explicitly illustrate the order of events on round $t$. 
\begin{enumerate}
	\item The online policy chooses a feasible point $x_t \in \mathcal{X}.$ 
	\item The adversary reveals the cost function $f_t$ and the constraint function $g_t.$ 
	\item  We compute the cumulative violation $Q(t)$ (which is a non-negative number) using the recursion \eqref{q-ev}. Note that $Q(t)$ depends on the action $x_t$ on the current round $t$.
	\item  The surrogate cost function $f_t'$ defined in  \eqref{surrogate-def} is passed to the policy.
\end{enumerate}
\fi
 

\iffalse
\hrule
\fi



%We will also assume that the cost functions are normalized so that $||f_t||_\infty \leq 1/2, \forall t.$
%\begin{framed}
%\textbf{Variants of the Constraint violation metric:} Our framework is flexible. Then we can consider a modified evolution for $Q(t):$
%\begin{eqnarray*}
%	Q(t)=\big(Q(t-1)+\psi(g_t(x)\big).~~, Q(0)=0\footnote{If $h>0,$ it is unnecessary to take $\max(0,\cdot)$ of $Q(t)$.}.
%\end{eqnarray*}
%Using the same analysis as before, we have 
%\begin{eqnarray*}
% 	\Phi(t)-\Phi(t-1) = Q^2(t)-Q^2(t-1)\leq 2Q(t)\psi(g_t(x_t)).
% \end{eqnarray*}
%  Motivated by the above calculations, we now define a surrogate cost function $f_t':\mathcal{X} \to \mathbb{R}$ as follows:
% \begin{eqnarray} \label{surrogate-def}
% 	f'_t(x) \equiv Vf_t(x) + 2Q(t) \psi(g_t(x)),
% \end{eqnarray}
%\end{framed}
\vspace{-0.1in}
Similar to the OCS Meta-Policy, the main tool for analyzing the COCO Meta-Policy is a regret decomposition inequality that we describe next.
\subsection{Analysis: Regret Decomposition:}
%We now derive the regret decomposition inequality for the \textsc{constrained OCO} Meta-policy, which will be central to our analysis. 
Fix any feasible action $x^\star \in \mathcal{X}^\star.$ 
 %Note that when Assumption \ref{slater} does not hold but Assumption \ref{feas} holds then we have $\eta^\star=0.$ 
 Recalling $\Phi(t)= Q^2(t)$, for any round $\tau \in [T]$, we have  
 %$\Phi(\tau)-\Phi(\tau-1) + V\big(f_\tau(x_\tau)-f_\tau(x^\star)\big)$
 \begin{eqnarray} \nonumber
 &&\Phi(\tau)-\Phi(\tau-1) + V\big(f_\tau(x_\tau)-f_\tau(x^\star)\big) \nonumber \\
 	 & \stackrel{\eqref{dr-bd}}{\leq} & \big(V f_\tau(x_\tau) + 2Q(\tau)g_\tau(x_\tau)\big) - Vf_\tau(x^\star),   \nonumber \\ \nonumber
 	&\stackrel{(a)}{\leq}  & \big(V f_\tau(x_\tau) + 2Q(\tau)g_\tau(x_\tau)\big) - \big(Vf_\tau(x^\star) + 2Q(\tau)g_\tau(x^\star)\big),\\  \label{drift_ineq}
 	&= & \hat{f}_{\tau}(x_\tau) - \hat{f}_\tau(x^\star), 
 \end{eqnarray}
 where in (a), we have used the feasibility of the action $x^\star \in \mathcal{X}^\star,$ which yields $g_\tau(x^\star)\leq 0$.\footnote{We actually have an equality in this step for the redefined non-negative clipped constraint functions.} 
 %(\emph{i.e.,} $g_\tau(x^\star) \leq 0$ and that $\psi(z) \leq 0, \forall z\leq 0$). 
 Summing up the inequalities \eqref{drift_ineq} from $\tau=t_1+1$ to $\tau=t_2$,  we relate the regret for learning  $\{f_t\}_{t\geq 1}$'s  with the regret  for learning  $\{\hat{f}_t\}_{t\geq 1}$'s:
 \begin{eqnarray} \label{coco:coco:master_eqn}
 	\Phi(t_2)-\Phi(t_1) + V \textrm{Regret}_{t_1+1:t_2}(x^\star)\leq \textrm{Regret}_{t_1+1:t_2}'(x^\star), ~ \forall x^\star \in \mathcal{X}^\star. 
 \end{eqnarray}
%We emphasize that the regret on the RHS implicitly depends on the auxiliary variables $\{Q(t)\}_{t\geq 1},$ which are controlled by the online policy. In particular, b
By setting $t_1=0$ and $t_2=t,$ and recalling that $\Phi(0)= Q(0)=0,$ \eqref{coco:coco:master_eqn} yields the key \emph{regret decomposition inequality}:
\begin{eqnarray} \label{q-bd-eqn}
	Q^2(t) + V\textrm{Regret}_t(x^\star) \leq \textrm{Regret}'_t(x^\star), ~ \forall x^\star \in \mathcal{X}^\star. 
\end{eqnarray}
Inequality \eqref{q-bd-eqn} is a general result and does not rely on the convexity of the cost/constraint functions.
\iffalse
\paragraph{Regret Decomposition:} 
 Fix any $\eta^\star$-feasible admissible action $x^\star \in \mathcal{X}_\tau.$ Note that when Assumption \ref{slater} does not hold but Assumption \ref{feas} holds then we have $\eta^\star=0.$ Then, for any $\tau \in [T],$ we have 
 \begin{eqnarray} \label{drift_ineq}
 	&&\Phi(\tau)-\Phi(\tau-1) + V\big(f_\tau(x_\tau)-f_\tau(x^\star)\big) \nonumber  \\
 	&\leq&  \big(V f_\tau(x_\tau) + 2Q(\tau)\psi(g_\tau(x_\tau))\big) - Vf_\tau(x^\star) \nonumber  \\
 	&\stackrel{(a)}{\leq} &  \big(V f_\tau(x_\tau) + 2Q(\tau)\psi(g_\tau(x_\tau))\big) - \big(Vf_\tau(x^\star) + 2Q(\tau)\psi(g_\tau(x^\star))\big)\nonumber -2\eta^\star Q(\tau) \\
 	&= & f_{\tau}'(x_\tau) - f_\tau'(x^\star) -2\eta^\star Q(\tau), 
 \end{eqnarray}
 where in (a), we have used the $\eta^\star$-feasibility of the fixed action $x^\star.$ 
 %(\emph{i.e.,} $g_\tau(x^\star) \leq 0$ and that $\psi(z) \leq 0, \forall z\leq 0$). 
 Summing up the inequalities \eqref{drift_ineq} above, we obtain the following bound, which relates the regret of learning the original cost functions to the regret of learning the surrogate cost functions:
 \begin{eqnarray} \label{coco:master_eqn}
 	\Phi(t_2)-\Phi(t_1) + V \textrm{Regret}_{t_1:t_2}(x^\star)\leq \textrm{Regret}_{t_1:t_2}'(x^\star) - 2\eta^\star \sum_{\tau=t_1}^{t_2} Q(\tau), ~ \forall x^\star \in \mathcal{X}_{t_2}. 
 \end{eqnarray}
We emphasize that the regret on the RHS depends on the auxiliary variables $\{Q(t)\}_{t\geq 1},$ which are implicitly controlled by the learning policy. In particular, by setting $t_1=0, t_2=t$ and recalling that $\Phi(0)= Q(0)=0,$ we have that 
\begin{eqnarray} \label{q-bd-eqn}
	Q^2(t) + V\textrm{Regret}_t(x^\star) \leq \textrm{Regret}'_t(x^\star) - 2\eta^\star \sum_{\tau=1}^t Q(\tau), ~ \forall x^\star \in \mathcal{X}_t. 
\end{eqnarray}
\fi
%\begin{framed}

%We consider an instance of the constrained Online Linear Optimization (OLO) problem where on every round $t,$ the learner chooses an action $x_t$ from a feasible closed and bounded convex set $\mathcal{X}.$ An adversary then reveals a cost vector $c_t$ and a constraint vector and budget tuple $(a_t, b_t),$ which encodes a linear constraint of the form
%\[\langle a_t, x \rangle \geq b_t.\]
%\begin{framed}
%\textbf{Note:} 
%\begin{enumerate}
%	\item 
% %The regret bounds derived below holds in the general case when, instead of a linear cost function, the adversary chooses any Lipschitz continuous convex cost function $f_t$ and we set $c_t=\nabla f_t(x_t).$ Due to the Lipschitzness assumption, $||c_t||$ is bounded for all $t \geq 1$.
%
%\item Without any loss of generality, we assume that only one constraint is revealed at each round. In the case of multiple constraints, we construct a single constraint by summing up the coefficient vectors and pass this single aggregate constraint to our algorithm.
%\end{enumerate}    
%\end{framed}
%For any round $t\geq 1,$ let $\mathcal{X}_t$ be the set of all feasible actions satisfying \emph{all} constraints up to and including time $t,$ \emph{i.e.,}
%\[\mathcal{X}_t = \{x \in \mathcal{X} : \langle a_\tau, x \rangle \geq b_\tau,  1\leq \tau\leq T\}.\] 
%The set $\mathcal{X} \subseteq \mathcal{X}$ is assumed to be non-empty. The regret $R_T(x^\star)$ of the online policy with respect to a fixed action $x^\star \in \mathcal{X}$ and its CCV $\mathbb{V}_\mathcal{I}$ over an interval $\mathcal{I}\equiv [a,b] \subseteq [T]$ are respectively defined as 
%\begin{eqnarray*}
%\textrm{Regret}_T(x^\star) &=& \sum_{t=1}^T \langle c_t, x_t-x^\star\rangle \\
%\mathbb{V}_\mathcal{I} &=& \sum_{t \in \mathcal{I}} \left(b_t- \langle a_t, x_t\rangle\right).
%\end{eqnarray*}
%Our objective is to design an online learning policy that achieves a sublinear CCV uniformly over \emph{any} interval   
%%We aim to design an online learning policy $\{x_t\}_{t\geq 1}$ that admits a sublinear maximum violation penalty 
%$\sup_{\mathcal{I}} \mathbb{V}_{\mathcal{I}}$ and a sublinear regret with respect to any $x^\star \in \mathcal{X}.$


%\section{An Online Learning Policy}
%Our proposed policy transforms the constrained problem to a standard unconstrained OCO problem in a \emph{blackbox} fashion. We show that by making use of the state-of-the regret bounds for the unconstrained problem, we get a sharp performance bound for the constrained problem.
%
%
%We will keep track of the constraint violations through a  variable $Q(t)$, which evolves according to a \emph{queueing} recursion as follows:
%\begin{eqnarray} \label{q-ev}
%    Q(t+1) = \left(Q(t)+ b_t-\langle a_t, x_t \rangle\right)^+, ~ Q(0)=0.
%\end{eqnarray}
%Note that, by unfolding the queueing recursion (also known as the Lindley recursion), we have $\sup_{\mathcal{I}}\mathbb{V}_{\mathcal{I}} \leq \sup_t Q(t).$ Hence, an upper bound on $\sup_t Q(t)$ gives an upper bound to the constraint violation penalty. 
%
%Let $V>0$ be a constant, which will be fixed later. Define a quadratic potential function
%\begin{eqnarray*}
%    \Phi(t) = Q^2(t).
%\end{eqnarray*}
%To simplify the notations, assume that $||a_t||^2+||c_t||^2+b_t^2 \leq 1, \forall t$ (this is without any loss of generality as the inequality constraint can be normalized suitably) and $||x||^2 \leq 1, \forall x \in \mathcal{X}.$
%Hence, the change of the potential on round $t$ can be upper-bounded as 
%\begin{eqnarray} \label{driftbound}
%    \Phi(t+1)-\Phi(t) &\leq&  \left(Q(t)+b_t -\langle a_t, x_t \rangle\right)^2 - Q^2(t) \nonumber \\
%    &\leq & 3 + 2Q(t)(b_t -\langle a_t, x_t \rangle),
%\end{eqnarray}
%%where $B$ comes from a priori bounds on the constraints.
% Using the principle of drift-plus-penalty, we are now motivated to consider a standard online linear optimization problem whose cost vector on round $t$ is given by 
%\begin{eqnarray} \label{reduced_cost}
%    c_t'= Vc_t-2Q(t) a_t.
%\end{eqnarray}
%To solve the resulting OLO problem, we can use a projected Online Gradient descent policy with an adaptive step size. This would give a second-order regret bound, which depends on the queue variables $\{Q(t)\}_{t=1}^T.$ The final step is to control the magnitude of the queue variables using an analysis similar to \cite{sinha2023banditq}. 
%
\vspace{-0.1in}
\subsubsection{Convex Cost and Convex Constraint functions}
% Let $x^\star \in \mathcal{X}$ be any feasible action. Then from Eq.\ \ref{driftbound} we have 
%\begin{eqnarray*}
%	&& \Phi(\tau)-\Phi(\tau-1) + V\langle c_\tau, x_\tau - x^\star \rangle \\
%	&\leq& 3 + 2Q(\tau)(b_\tau- \langle a_\tau, x_\tau \rangle) + V \langle c_\tau, x_\tau \rangle - V \langle c_\tau, x^\star\rangle \\
%	&\stackrel{(a)}{=}& 3 + 2Q(\tau) b_\tau + \langle c_\tau', x_\tau \rangle - V \langle c_\tau, x^\star\rangle \\
%	&\stackrel{(b)}{\leq} & 3 + 2Q(\tau) \langle a_\tau, x^\star \rangle  + \langle c_\tau', x_\tau \rangle - V \langle c_\tau, x^\star\rangle \\
%	&=& 3 + \langle c_\tau', x_\tau - x^\star \rangle. 
%\end{eqnarray*}
%where in (a), we have used the definition of $c_\tau'$ from \ref{reduced_cost} and in (b), we have used the feasibility of the action $x^\star.$ Summing up the above inequalities from $\tau=1$ to $\tau=t$ and recalling that $\Phi(t)= Q^2(t),$ we have 
%We first consider the case when the both the cost function $(f_t)$ and the constraint function $(g_t)$ are convex. 
We now apply \eqref{q-bd-eqn} to the case of convex cost $f_t$ and convex constraint functions $g_t$.  
%without making any additional assumption beyond Assumption \ref{bddness} and Assumption \ref{slater}. 
 Let the base OCO subroutine $\Pi$ be taken to be the OGD policy with adaptive step sizes as given in part 1 of Theorem \ref{data-dep-regret}. In this case, using \eqref{q-bd-eqn}, we obtain the following bound for any feasible action $x^\star \in \mathcal{X}^\star:$
%\textcolor{blue}{Note that OGD also offers an adaptive regret bound (starting from any point in time). It might be possible to exploit this further. }
\begin{eqnarray}\nonumber
	\hspace{-.15in}Q^2(t) + V \textrm{Regret}_t(x^\star) 
	&\leq& \textrm{Regret}_t'(x^\star) 
	\stackrel{(c)}{\leq}   D\sqrt{2\sum_{\tau=1}^t ||\nabla \hat{f}_\tau (x_\tau)||^2} 
	 \nonumber \\  \label{main_eq}
	&\stackrel{(d)}{\leq}&  GD \sqrt{2\sum_{\tau=1}^t (Q(\tau)+V)^2} \stackrel{(e)}{\leq} 2GD \sqrt{\sum_{\tau=1}^t Q^2(\tau)}+ 2GDV\sqrt{t},
\end{eqnarray}
%where $\textrm{Regret}_t'$ denotes the regret for the OLO algorithm with the cost vector sequence $\{f'_\tau\}_{\tau}.$ 
where $(c)$ follows from the regret guarantee of the adaptive OGD policy \eqref{cvx-reg-bd}, (d) follows from Eqn.\ \eqref{grad-bd} in the Appendix, and $(e)$ follows from  the facts that $(a+b)^2 \leq 2(a^2+b^2)$ and $\sqrt{x+y} \leq \sqrt{x}+\sqrt{y}.$ 
%The above functional inequality can be thought of as a discrete version of Gronwall's inequality.  
%\subsection{Performance guarantees}
%\textbf{The Slater's condition does not necessarily hold}
%\begin{framed}
The following theorem gives the performance bounds for the \textsc{COCO} Meta-policy. 
%\vspace{-0.15in}
 \begin{theorem} \label{gen-cvx-bd}
 	%Without making any assumption on the Slater's condition (\emph{i.e.,} $\eta^\star=0$), 
 %	The online meta-policy described above achieves the following bounds for any convex cost and convex constraint functions 
%Suppose Assumptions \ref{bddness} and \ref{feas} hold.
For COCO with convex costs and convex constraint functions, choosing $V=\sqrt{T},$ the COCO Meta-policy, described in Algorithm \ref{g-oco-policy}, achieves the following regret and CCV:
 %We have the following upper bounds to the CCV and regret achieved by the proposed meta-policy upon setting $V=\sqrt{T}$: 
		\[\quad \textrm{Regret}_t(x^\star) = O(\sqrt{t}), ~\forall x^\star \in \mathcal{X}^\star, ~\textrm{and}~ \mathbb{V}(t) = O(T^{3/4}) , ~\forall t \in [T].\]
		Furthermore, for any round $t \geq 1$ where the worst-case regret is non-negative \newline \emph{i.e.,} $\sup_{x^\star \in \mathcal{X}^\star}\textrm{Regret}_t(x^\star) \geq 0$,\footnote{ This is a non-trivial condition since the offline benchmark $x^\star$ is constrained to belong to the feasible set $\mathcal{X}^\star$ whereas the actions of the policy $\{x_t\}_{t\geq 1}$ belong to the larger admissible set $\mathcal{X} \supseteq \mathcal{X}^\star.$ See Section \ref{improved_rates} in the Appendix.}  we have $\mathbb{V}(t)= O(\sqrt{T}).$
%(b) For a natural class of adversaries, called convex adversaries defined in Eqn.\ \eqref{jensenadv} in Appendix \ref{improved_rates}, we also have $\mathbb{V}(t)= O(\sqrt{T})$ under some mild assumptions. See Theorem \ref{improved_violation_bd} for a precise statement.
	 \end{theorem}
 %\end{framed}
 \vspace{-0.01in}
 See Section \ref{gen-cvx-bd-pf} for the proof of Theorem \ref{gen-cvx-bd}. The first part of the theorem proves the same best-known bound from \citet{pmlr-v70-sun17a, guo2022online}. However, our policy is simple with an elegant analysis and does not need to know $G$ or the full functions $f_t$ and $g_t$ after $x_t$ has been chosen. %and does not assume Slater's condition. 
 The second part of the theorem states that the regret and CCV cannot be large simultaneously at any round - if the worst-case regret at a round $t$ is non-negative, then the CCV up to that round is at most $O(\sqrt{T}).$ In comparison \citet{neely2017online} get this improved result upon assuming Slater's condition.
 For a natural class of adversaries, called convex adversaries defined in \eqref{jensenadv} in Appendix \ref{improved_rates}, we also have $\mathbb{V}(t)= O(\sqrt{T})$ under some mild assumptions. See Theorem \ref{improved_violation_bd} for a precise statement.
 \fi
 
 
%\subsection{Overview of the technique}
Recall that compared to the standard OCO problem where the only objective is to minimize the Regret \citep{hazan2022introduction}, in COCO, our objective is twofold: to \emph{simultaneously} control the Regret \emph{and} the CCV. See Section \ref{prelims} in the Appendix for preliminaries on the OCO problem and some standard results which will be useful in our analysis. In the following, we 
propose a Lyapunov function-based policy 
%recently proposed by \citet{sinha2023playing}. While they were able to prove an $O(\sqrt{T})$ regret and $O(T^{\nicefrac{3}{4}})$ CCV with a quadratic Lyapunov function, in this paper, we show that this technique can be generalized with a power-law Lyapunov function to 
that yields the optimal Regret and CCV bounds for the COCO problem. Although for simplicity, we assume that the horizon length $T$ is known, we can use the standard doubling trick for an unknown $T.$
\iffalse
\subsection{Preliminaries}
We now briefly recall the first-order methods (\emph{a.k.a.} Projected Online Gradient Descent (OGD)) for the standard OCO problem. These methods differ among each other in the way the step sizes are chosen. For a sequence of convex cost functions $\{\hat{f}_t\}_{t \geq 1},$ a projected OGD algorithm selects the successive actions as \citep[Algorithm 2.1]{orabona2019modern}:
\begin{eqnarray}\label{ogd-policy} 
	x_{t+1} = \mathcal{P}_\mathcal{X}(x_t - \eta_t \nabla_t), ~~ \forall t\geq 1,
\end{eqnarray}
where $\nabla_t \equiv \nabla \hat{f}_t(x_t)$ is a subgradient of the function $\hat{f}_t$ at $x_t$, $\mathcal{P}_\mathcal{X}(\cdot)$ is the Euclidean projection operator on the set $\mathcal{X}$ and $\{\eta_t\}_{t \geq 1}$ is a specified step size sequence. 
The (diagonal version of the) AdaGrad policy adaptively chooses the step size sequence as a function of the previous subgradients as  $\eta_t= \frac{\sqrt{2}D}{2\sqrt{\sum_{\tau=1}^{t} G_\tau^2}},$ where $G_t=||\nabla_t||_2, t \geq 1$ \citep{duchi2011adaptive}. \footnote{We set $\eta_t=0$ if $G_t=0.$} It enjoys the following adaptive regret bound.
\begin{theorem}{\citep[Theorem 4.14]{orabona2019modern}}  The AdaGrad policy, with the above step size sequence, achieves the following regret bound for the standard OCO problem: 
	\begin{eqnarray} \label{cvx-reg-bd}
			 \textrm{Regret}_T \leq \sqrt{2}D \sqrt{\sum_{t=1}^T G_t^2}.
	\end{eqnarray}
	\end{theorem}
	\fi
%	The OGD policy with the above adaptive step-size schedule is known as (a version of) the AdaGrad policy in the literature \citep{duchi2011adaptive}. 


\subsection{Design and Analysis of the Algorithm}
 
To simplify the analysis, we pre-process the cost and constraint functions on each round as follows.

\vspace{5pt}

%the queue-lengths evolve as in Eqn.\ \eqref{q-ev}, where 
\hrule
\textbf{Pre-processing:}
On every round, we first clip the negative part of the constraint function to zero by passing it through the standard ReLU unit. Then, we scale both the cost and constraint functions by a positive factor $\beta,$ which will be determined later. In other words, 
 we work with the pre-processed inputs $\tilde{f}_t \gets \beta f_t, \tilde{g}_t \gets \beta (g_t)^+.$ Hence, the pre-processed functions are $\beta G$-Lipschitz and $\tilde{g}_t \geq 0, \forall t.$  
 \vspace{5pt}

\hrule 
In the following, we derive the Regret and CCV bounds for the pre-processed functions. The bounds for the original problem are obtained upon scaling the results back by $\beta^{-1}$ in the final step.
\subsubsection{Defining the Surrogate Cost Functions} 
%we can obtain the optimal regret and violation bound for the COCO problem with convex cost and convex constraint functions.

Let $Q(t)$ denote the CCV for the pre-processed constraints up to round $t.$ Clearly, $Q(t)$ satisfies the simple recursion $Q(t)=Q(t-1)+\tilde{g}_t(x_t), t\geq 1, $ with $Q(0)=0$. Recall that one of our objectives is to make $Q(t)$ small.
Towards this, let $\Phi: \mathbb{R}_+ \mapsto \mathbb{R}_+$ be any non-decreasing differentiable convex potential (Lyapunov) function such that $\Phi(0)=0.$ Using the convexity of $\Phi(\cdot),$ we have
%which generalizes Eqn.\ \eqref{dr-bd}:
%Also, for the sake of simplicity, we assume that the maximum magnitude of the constraint violation is upper bounded by $F=1$. 
%Since the function $ h: x \to x^n$ is convex, we have 
%\begin{eqnarray*}
%	\Phi(t)= Q^n(t) = \big(Q(t-1)+g_t(x_t)\big)^n \leq Q^{n}(t-1) + n Q^{n-1}(t) g_t(x_t). 
%	\end{eqnarray*}
%
\begin{eqnarray} \label{dr-bd-gen}
	\Phi(Q(t))  &\leq& \Phi(Q(t-1)) + \Phi'(Q(t))(Q(t)-Q(t-1)) \nonumber \\
&=& \Phi(Q(t-1)) + \Phi'(Q(t)) \tilde{g}_t(x_t). 
\end{eqnarray}
Hence, the change (\emph{drift}) of the potential function $\Phi(Q(t))$ on round $t$ can be upper bounded as 
\begin{eqnarray} \label{drift_ineq_new}
	\Phi(Q(t))-\Phi(Q(t-1)) \leq \Phi'(Q(t)) \tilde{g}_t(x_t). 
\end{eqnarray}
Recall that, in addition to controlling the CCV, we also want to minimize the cumulative cost $\sum_{t=1}^T f_t(x_t)$ (which is equivalent to the regret minimization). Inspired by the stochastic \emph{drift-plus-penalty} framework of \citet{neely2010stochastic}, we combine these two objectives to a single objective of minimizing a sequence of surrogate cost functions $\{\hat{f}_t\}_{t=1}^T$ which are obtained by taking a positive linear combination of the drift upper bound \eqref{drift_ineq_new} and the cost function. More precisely, we define
%Inspired by the stochastic \emph{drift-plus-penalty} framework of \citet{neely2010stochastic}, on round $t$, we attempt to minimize a surrogate cost function $\hat{f}_t : \mathcal{X} \to \mathbb{R},$ obtained by adding the scaled cost function $f_t$ to a functional form of the drift upper bound \eqref{drift_ineq_new} as defined below:
\begin{eqnarray} \label{surrogate_new}
	\hat{f}_t(x):= V\tilde{f}_t(x)+ \Phi'(Q(t)) \tilde{g}_t(x), ~~ t \geq 1. 
\end{eqnarray}
In the above, $V$ is a suitably chosen non-negative parameter to be determined later. In brief, the proposed policy for COCO, described in Algorithm \ref{coco_alg}, simply runs an adaptive OCO policy on the surrogate cost function sequence $\{\hat{f}_t\}_{t\geq 1}$, with a specific choice of the potential function $\Phi(\cdot)$, the parameter $V$, and step-size sequence $\{\eta_t\}_{t\geq 1}$, as dictated by the following analysis.
\begin{algorithm}[tb]
   \caption{Online Policy for COCO}
   \label{coco_alg}
\begin{algorithmic}[1]
   \State {\bfseries Input:} Sequence of convex cost functions $\{f_t\}_{t=1}^T$ and constraint functions $\{g_t\}_{t=1}^T,$ $G=$ a common Lipschitz constant, $T=$ Horizon length,
   %an upper bound $G$ to the Euclidean norm of their (sub)gradients, 
    $D=$ Euclidean diameter of the admissible set $\mathcal{X},$ $\mathcal{P}_\mathcal{X}(\cdot)=$ Euclidean projection operator on the set $\mathcal{X}$ 
     \State {\bfseries Parameter settings:} 
     \begin{enumerate}
     	\item \textbf{Convex cost functions:} $\beta = (2GD)^{-1}, V=1, \lambda = \frac{1}{2\sqrt{T}}, \Phi(x)= \exp(\lambda x)-1.$
     
    \item \textbf{$\alpha$-strongly convex cost functions:} $\beta =1, V=\frac{8G^2 \ln(Te)}{\alpha}, \Phi(x)= x^2.$
    \end{enumerate}
     %$ \alpha=\frac{1}{2GD}, n=\max(2, \lceil \ln T \rceil), V=(n-1)^{n-1}T^{\frac{n-1}{2}}, \Phi(x)=x^n.$ 
%   \REPEAT
  \State {\bfseries Initialization:} Set $ x_1 \in \mathcal{X}$ arbitrarily, $Q(0)=0$.
   \ForEach{$t=1:T$}
   \State Play $x_t,$ observe $f_t, g_t,$ incur a cost of $f_t(x_t)$ and constraint violation of $(g_t(x_t))^+$
   \State $\tilde{f}_t \gets \beta f_t, \tilde{g}_t \gets \beta \max(0,g_t).$
   \State $Q(t)=Q(t-1)+\tilde{g}_t(x_t).$
   \State Compute (sub)gradient $\nabla_t = \nabla \hat{f}_t(x_t),$ where the surrogate function $\hat{f}_t$ is defined in Eqn.\ \eqref{surrogate_new}
   \State $x_{t+1} = \mathcal{P}_\mathcal{X}(x_t - \eta_t \nabla_t)$, where 
   \begin{eqnarray*}
   \eta_t =\begin{cases}
   	\frac{\sqrt{2}D}{2\sqrt{\sum_{\tau=1}^{t} ||\nabla_\tau||_2^2}}, ~&~\textrm{for convex costs (AdaGrad stepsizes)} \\
   	\frac{1}{\sum_{s=1}^t H_s}, ~ &~ \textrm{for strongly convex costs } (H_s \textrm{= strong convexity parameter of } f_s, s\geq 1) 
   	\end{cases}
   	\end{eqnarray*}
   	
%   \IF{$x_i > x_{i+1}$}
%   \STATE Swap $x_i$ and $x_{i+1}$
%   \STATE $noChange = false$
%   \ENDIF
   \EndForEach
%   \UNTIL{$noChange$ is $true$}
\end{algorithmic}
\end{algorithm}
%\begin{framed}
%\paragraph{Policy for COCO:} Run the AdaGrad algorithm \ref{ogd-policy} on the sequence of surrogate cost functions $\{\hat{f}_t\}_{t\geq 1}$ given by Eqn.\ \eqref{surrogate_new}.
%\end{framed}

\subsubsection{The Regret Decomposition Inequality}
Let $x^\star \in \mathcal{X}^\star$ be any feasible action guaranteed by Assumption \eqref{feas-constr}. Plugging in the definition of surrogate costs \eqref{surrogate_new} into the drift inequality \eqref{drift_ineq_new}, and using the fact that $g_\tau(x^\star)\leq 0, \forall \tau \geq 1,$ we have
%Working similarly as before, we have the following inequality
\begin{eqnarray*}
	\Phi(Q(\tau))-\Phi(Q(\tau-1)) + V(\tilde{f}_\tau(x_\tau)-\tilde{f}_\tau(x^\star)) 
	\leq \hat{f}_\tau(x_\tau) - \hat{f}_\tau(x^\star), ~ \forall \tau \geq 1.
\end{eqnarray*}
Summing the above inequalities for rounds $1\leq \tau \leq t$, and using the fact that $\Phi(0)=0,$ we obtain  
\begin{eqnarray} \label{gen-reg-decomp}
	\Phi(Q(t)) +V \textrm{Regret}_t(x^\star) \leq \textrm{Regret}_t'(x^\star), ~ \forall x^\star \in \mathcal{X}^\star,
\end{eqnarray}
where $\textrm{Regret}_t$ on the LHS and $\textrm{Regret}'_t$ on the RHS of \eqref{gen-reg-decomp} refer to the regret for learning the pre-processed cost functions $\{\tilde{f}_t\}_{t\geq 1}$ and the surrogate cost functions $\{\hat{f}_t\}_{t \geq 1}$ respectively. 
%First, 
We will use the following upper bound on the $\ell_2$-norm of the (sub)gradients $G_t$ of the surrogate cost function $\hat{f}_t$ defined in Eqn.\ \eqref{surrogate_new}:
\begin{eqnarray} \label{grad_bd_new}
	%||\nabla \hat{f}_t(x_t)||_2
	G_t \equiv ||\nabla \hat{f}_t(x_t)||
	\stackrel{(a)}{\leq} V||\nabla \tilde{f}_t(x_t)||+ \Phi'(Q(t))||\nabla \tilde{g}_t(x_t)|| 
	\stackrel{(b)}{\leq} \beta G\big(V+\Phi'(Q(t)\big),
\end{eqnarray}
where in $(a)$, we have used the triangle inequality for $\ell_2$ norms and in $(b)$, we have used the fact that all pre-processed functions are $\beta G$-Lipschitz. 
\subsubsection{Convex Cost and Convex Constraint Functions}
We now apply the regret decomposition inequality \eqref{gen-reg-decomp} to the case of convex cost and convex constraint functions.  
%without making any additional assumption beyond Assumption \ref{bddness} and Assumption \ref{slater}. 
%Since Algorithm \ref{coco_alg} uses the AdaGrad algorithm for learning the surrogate cost functions, from \eqref{cvx-reg-bd}, we need to upper bound the gradients of the surrogate functions to derive the regret bound. Towards this, 
Let us choose the regret-minimizing OCO subroutine for the surrogate cost functions to be the OGD policy with adaptive step sizes (a.k.a. \emph{AdaGrad}) described in part 1 of Theorem \ref{data-dep-regret} in the Appendix (see Algorithm \ref{coco_alg}). 
Plugging in the adaptive regret bound \eqref{cvx-reg-bd} on the RHS of \eqref{gen-reg-decomp}, setting $\beta=(2GD)^{-1},$ and using Eqn.\ \eqref{grad_bd_new}, we arrive at the following inequality valid for any $t \geq 1: $
\begin{eqnarray} \label{gen-fn-ineq}
		\Phi(Q(t)) +V \textrm{Regret}_t(x^\star) \leq \sqrt{\sum_{\tau=1}^t \big(\Phi'(Q(\tau))\big)^2} + V\sqrt{t}.
\end{eqnarray}
%The above inequality is obtained by first upper-bounding 
In deriving the above result, we have utilized simple algebraic inequalities $(x+y)^2 \leq 2(x^2+y^2)$ and $\sqrt{a+b} \leq \sqrt{a} + \sqrt{b}, a, b\geq 0.$ Now recall that the sequence $\{Q(t)\}_{t\geq 1}$ is non-negative and non-decreasing as $\tilde{g}_t\geq 0.$ Furthermore, the derivative $\Phi'(\cdot)$ is non-decreasing as the function $\Phi(\cdot)$ is assumed to be convex. Hence, bounding all terms in the summation on the RHS of \eqref{gen-fn-ineq} from above by the last term, we arrive at the following inequality for any feasible $x^\star \in \mathcal{X}^\star:$ 
\begin{eqnarray} \label{gen-fn-ineq2}
	\Phi(Q(t)) +V \textrm{Regret}_t(x^\star) \leq \Phi'\big(Q(t)\big)\sqrt{t} + V\sqrt{t}.
\end{eqnarray}
The simplified regret decomposition inequality \eqref{gen-fn-ineq2} constitutes the key step for the subsequent analysis.
%Where, to ease typing, we have assumed that the functions are scaled such that $2GD \leq 1$.
\paragraph{$\blacksquare$ Performance Analysis}
%~Analysis of Algorithm \ref{coco_alg}}
\paragraph{An exponential Lyapunov function:} We now derive the Regret and CCV bounds for the proposed  policy (Algorithm \ref{coco_alg}) by choosing $\Phi(\cdot)$ to be the exponential Lyapunov function: $\Phi(x)\equiv \exp(\lambda x)-1,$ where the parameter $\lambda \geq 0$ will be fixed later. 
%An analysis with a power-law potential function is given in Appendix \ref{power-law}, which also yields similar bounds. 
Clearly, the function $\Phi(\cdot)$ satisfies the required conditions for a Lyapunov function - it is a non-decreasing and convex function with $\Phi(0)=0.$ 
\paragraph{Bounding the Regret:}
%Since the  sequence $\{Q(t)\}_{t\geq 1}$ is non-negative and non-decreasing as the pre-processed constraints are non-negative, upper-bounding all terms in the summation of the RHS of \eqref{gen-fn-ineq} by the last term, we have the following regret bound for 
With the above choice for the Lyapunov function $\Phi(\cdot)$, Eqn.\ \eqref{gen-fn-ineq2} implies that for any feasible $x^\star \in \mathcal{X}^\star$ and for any $t \in [T],$ we have
\begin{eqnarray*}
	\exp(\lambda Q(t)) -1 + V \textrm{Regret}_t(x^\star) \leq \lambda \exp(\lambda Q(t)) \sqrt{t} + V\sqrt{t}.
\end{eqnarray*}
Transposing the first term on the above inequality to the RHS and dividing throughout by $V$, we have: 
\begin{eqnarray} \label{reg-bd-exp}
	\textrm{Regret}_t(x^\star ) \leq \sqrt{t}+\frac{1}{V}+ \frac{\exp(\lambda Q(t))}{V}(\lambda \sqrt{t}-1).
\end{eqnarray}
Choosing any $\lambda \leq  \frac{1}{\sqrt{T}},$ the last term in the above inequality becomes non-positive for any $t \in [T].$ Hence, for any $x^\star \in \mathcal{X}^\star$, we have the following regret bound
\begin{eqnarray} \label{regret-bd1}
	\textrm{Regret}_t(x^\star ) \leq \sqrt{t}+\frac{1}{V}. ~~ \forall t \in [T].
\end{eqnarray}
\paragraph{Bounding the CCV:} 
Since all pre-processed cost functions are $\beta G=(2D)^{-1}$-Lipschitz,
%with Lipschitz constant $\alpha G= (2D)^{-1}$, 
we trivially have $\textrm{Regret}_t(x^\star) = \sum_{s=1}^t (\tilde{f}_s(x_s) - \tilde{f}_s(x^\star)) \geq -\frac{Dt}{2D} \geq -\frac{t}{2}.$ Hence, from Eqn.\ \eqref{reg-bd-exp}, we have that for any $\lambda < \frac{1}{\sqrt{T}}$ and any $t \in [T]:$
\begin{eqnarray} \label{q-len-exp-bd}
	\frac{\exp(\lambda Q(t))}{V}(1-\lambda \sqrt{t}) \leq  2t + \frac{1}{V} 
	 \implies  Q(t) \leq  \frac{1}{\lambda}\ln\frac{1+2Vt}{1-\lambda \sqrt{t}}.
\end{eqnarray}
Choosing $\lambda=\frac{1}{2\sqrt{T}}, V=1,$ and scaling the bounds back by $\beta^{-1}\equiv 2GD,$ we arrive at our main result.
%\begin{eqnarray*}
%	\textrm{Regret}_t \leq 2GD(\sqrt{t}+1), \forall t\in [T], ~~\textrm{CCV}_T \leq 4GD\sqrt{T}\ln(2\big(1+2T)\big).
%\end{eqnarray*}
%\paragraph{A Power-law Lyapunov Function:}
%We now specialize inequality \eqref{gen-fn-ineq} by considering the power-law Lyapunov function $\Phi(x)\equiv x^n$ for some integer $n \geq 2,$ to be fixed later. A similar analysis with an exponential potential function is given in Appendix \ref{exp-analysis}. Since the sequence $\{Q(t)\}_{t\geq 1}$ is non-negative and non-decreasing as the pre-processed constraints are non-negative, from \eqref{gen-fn-ineq}, we obtain 
%\begin{eqnarray} \label{fn-ineq2}
%	Q^n(t)+V\textrm{Regret}_t(x^\star) \leq nQ^{n-1}(t) \sqrt{t} + V\sqrt{t}.
%\end{eqnarray} 
%
%\paragraph{Bounding the Regret:}
%From Eqn.\ \eqref{fn-ineq2}, we have the following regret bound for any feasible $x^\star \in \mathcal{X}^\star:$
%\begin{eqnarray*}
%	\textrm{Regret}_t(x^\star) &\leq& \sqrt{t} + \frac{Q^{n-1}(t)}{V}(n\sqrt{t}-Q(t)) \\
%	&\leq& \sqrt{t} + \frac{(n-1)^{n-1} t^{n/2}}{V},
%\end{eqnarray*}
%where the last step follows from the AM-GM inequality. Finally, upon taking $V=(n-1)^{n-1}T^{\frac{n-1}{2}}$ as in Algorithm \ref{coco_alg}, we obtain $\textrm{Regret}_t(x^\star) \leq 2\sqrt{t}, ~\forall 1\leq t \leq T$.
% 
%\paragraph{Bounding the CCV:} Since all pre-processed cost functions are $(2D)^{-1}$-Lipschitz,
%%with Lipschitz constant $\alpha G= (2D)^{-1}$, 
%we trivially have $\textrm{Regret}_t(x^\star)\geq -\nicefrac{Dt}{2D} \geq -\frac{t}{2}.$ Hence, from Eqn.\ \eqref{fn-ineq2}, we obtain
%\begin{eqnarray} \label{qn-bd}
%	Q^n(t) \leq 2Vt + nQ^{n-1}(t) \sqrt{t}.
%\end{eqnarray}
%Finally, to bound $Q(t)$ from the above, consider the case when $Q(t) \geq 2n\sqrt{t}$. In this case, from \eqref{qn-bd}, we have 
%\begin{eqnarray*}
%	Q^n(t) \leq 2Vt + \frac{1}{2}Q^n(t) \implies Q(t) \leq (4Vt)^{\frac{1}{n}}.
%\end{eqnarray*}
%Thus, in general, the following bound holds 
%%the following violation bound 
%%\begin{eqnarray*} 
%	\[ Q(t) \leq \max(2n\sqrt{t}, (4Vt)^{\frac{1}{n}}).\]
%%\end{eqnarray*}
%Substituting the chosen value for the parameter $V$ and using the fact that $n\geq 2$, from the above, we have 
%\begin{eqnarray}\label{q-bd-new2}
%Q(t)\leq \max (2n\sqrt{T}, 2nT^{\frac{1}{2}+ \frac{1}{2n}}) = 2nT^{\frac{1}{2}+ \frac{1}{2n}}, ~ \forall t \in [T].
%\end{eqnarray} 
%Finally, setting $n= \max(2,\lceil\ln T \rceil)$ as in Algorithm \ref{coco_alg}, we obtain the following CCV bound:
%%\begin{framed}
%\begin{eqnarray*}
%	%\textrm{Regret}_T \leq 2\sqrt{T} , \textrm{and} ~ 
%	\textrm{CCV}_T=Q(T) \leq 3.3 (2+\ln T)\sqrt{T}, ~ \forall T \geq 1.
%\end{eqnarray*}
%%\end{framed}
%%Hence, we have the following main result for the original (unscaled) cost and constraint functions:
%%\begin{framed}
%Recall that the above results hold for the pre-processed functions. Hence, scaling the bounds back by $\alpha^{-1} \equiv 2GD,$ we arrive at the main result of this paper:
%We summarize the above results in the following theorem.
%\begin{framed}
\begin{theorem} \label{main_result}
For the COCO problem with adversarially chosen $G$-Lipschitz cost and constraint functions, Algorithm \ref{coco_alg}, with $\beta=(2GD)^{-1}, V=1, \Phi(x)= \exp(\frac{x}{2\sqrt{T}})-1,$ yields the following Regret and CCV bounds for any horizon length $T \geq 1:$
%for any $T \geq 1$
\begin{eqnarray*}
 \textrm{Regret}_t \leq 2GD(\sqrt{t}+1), ~~\forall t \in [T], ~~\textrm{CCV}_T \leq 4GD\ln(2\big(1+2T)\big)\sqrt{T}.
 \end{eqnarray*}
 In the above, $D$ denotes the Euclidean diameter of the closed and convex admissible set $\mathcal{X}$.
\end{theorem}
%\end{framed}

%In the final section, we show that in addition to a small regret and CCV bounds, the proposed Algorithm \ref{coco_alg} also has a small movement cost measured in the (squared) Euclidean metric. These features make it attractive for practical applications.
 
 % \vspace{-0.1in}
% \paragraph{Remark:} The non-negative regret assumption 
 %in the latter parts of Theorem \ref{gen-cvx-bd} and \ref{str-cvx-bd} 

 %The class of convex adversaries includes the standard offline convex optimization problem with a fixed cost and a fixed constraint function.
 
 %Such assumptions are necessary as otherwise, it would violate the constraint violation lower bounds given in Section \ref{lower_bound_section}.  
 %Nevertheless, in Section \ref{improved_rates} of the Appendix, we show that the tighter constraint violation bounds can be guaranteed for certain classes of worst-case adversaries, called \emph{convex adversaries}. 
 %This includes the standard offline convex optimization problem with a fixed cost and a fixed constraint function. See Theorem \ref{improved_violation_bd} in the Appendix for a concrete result.
% 
% The regret lower bound $\mathcal{X}(\sqrt{T})$ for convex cost functions without constraints is well-known \citep[Table 3.1]{hazan2016introduction}. The following result shows that any online policy must incur a CCV of $\mathcal{X}(\sqrt{T}).$ Thus the performance bound given in Theorem \ref{gen-cvx-bd} is almost optimal. 
%	
% 
%%The following result is straightforward.
%%\begin{framed}
%\begin{theorem}[(Lower bound to the CCV)] \label{cvx-lb}
%	Consider the constrained OCO problem with linear cost functions and affine constraints. Then any online policy will incur at least $\mathcal{X}(\sqrt{T})$ violation penalty $\mathbb{V}(T)$ over a horizon of length $T.$ 
%\end{theorem}
%See Section \ref{cvx-lb-pf} for the proof.
%\end{framed}
\iffalse
\begin{proof}
	Assume that on round $t$, the adversary chooses a linear cost function $g_t(x)$ and an affine constraint $g_t(x) \leq b_t,$ where the constant $b_t$ will be specified later. Let $x^\star \in \mathcal{X}$ be the best-fixed action in hindsight, which minimizes the cumulative cost $\sum_{t=1}^T g_t(x).$ We now set the constant $b_t$ to be $b_t = g_t(x^\star), t \in [T].$ Clearly, $x^\star \in \mathcal{X}_T$ is a feasible point that satisfies all constraints up to time $T$. Hence, the CCV incurred by any online policy $\pi$ over the entire horizon $T$ is given by: 
	\begin{eqnarray*}
	\sup_{\mathcal{I}} \mathbb{V}_{\mathcal{I}} \geq \mathbb{V}_T = \sum_{t=1}^T \big(g_t(x_t) - b_t\big) = \sum_{t=1}^T g_t(x_t) - \sum_{t=1}^T g_t(x^\star) \equiv \textrm{Regret}_T(x^\star) \geq \mathcal{X}(\sqrt{T}), 
	\end{eqnarray*} 
	where the last inequality follows from the lower bound on the achievable regret for adversarially chosen linear cost functions over a convex domain. 
\end{proof}
\fi

 %In this case, we still have $\textrm{Regret}_t(x^\star) = O(t). $ Using a similar analysis as in  \cite{sinha2023banditq}, we can show that $\textrm{Regret}_t(x^\star)= \mathbb{V}_T=O(T^{3/4}).$
 %In this case, there is nothing to prove on the regret. We only need to bound the queue length, which gives a bound on the violation penalty. 
\iffalse 
The following theorem shows that the additional assumption on the existence of a Slater's point implies improved bounds for both regret and CCVs. 
\begin{framed}
\begin{theorem} \label{cvx-slater}
Suppose Assumptions \ref{bddness} and \ref{slater} hold. Then, upon setting $V=\sqrt{T},$ we have 
	\begin{eqnarray*}
		\frac{1}{t}\sum_{\tau=1}^t Q(\tau) = O(\sqrt{T}), ~ \textrm{Regret}_t(x^\star) = O(T^{5/8}), \forall x^\star \in \mathcal{X}_t, ~\forall t \in [T].
	\end{eqnarray*}
\end{theorem}
\end{framed}
See Appendix \ref{cvx-slater-pf} for the proof of the above result. 

 %\begin{framed}
%\textbf{Observation:} Note that, by choosing the same cost and constraint vectors (upon setting $b_t=0$), we can immediately see that the regret is upper bounded by the CCV, \emph{i.e.,} $\mathbb{V}_T(x^\star) \geq \textrm{Regret}_T(x^\star).$ Since it is impossible to achieve a regret better than $O(\sqrt{T}),$ the above implies that $\mathbb{V}_T = \mathcal{X}(\sqrt{T}).$ (\textbf{\textcolor{blue}{Check that the above argument works. Note that we compute the regret with respect to the feasible points only}}).
%\end{framed}
%Upon setting $V=\sqrt{T},$ we show that the above online policy achieves a sublinear regret and constraint violation penalty. We will consider the following two cases. 
\iffalse 
\textbf{Case I: $\forall t, \exists x_t^\star \in \mathcal{X}:$ $\textrm{Regret}_t(x_t^\star) \geq 0.$}

In this case, from Eq.\ \ref{main_eq} we have for all $t \geq 1:$
\begin{eqnarray} \label{q-bd}
	Q^2(t) \leq 3t+ c_1 \sqrt{\sum_{\tau=1}^t Q^2(\tau)}+ c_2V\sqrt{t}. 
\end{eqnarray} 

From Eq. \ref{q-ev}, due to the boundedness assumption on the constraint vectors, we trivially have
\begin{eqnarray*}
	Q(t)= O(t), ~ \forall t\geq 1.
\end{eqnarray*}
Plugging the above bounds into Eq. \ref{q-bd}, we have 
 \begin{eqnarray*}
 	Q^2(t) \leq O(t) + O(t^{3/2}) + O(T) = O(T^{3/2}), ~ \emph{i.e.,} ~Q(t) = O(T^{3/4}).
 \end{eqnarray*}
 Plugging in the improved bound on the queue length into \ref{q-bd} again, we now have that for all $t \geq 1:$
 \begin{eqnarray*}
 	Q^2(t) \leq O(t)+ O(T^{5/4}) + O(T)= O(T^{5/4}). 
 \end{eqnarray*}
 Continuing this successive refinement process (see the proof of Proposition 5 in \cite{sinha2023banditq}), we conclude 
 \begin{eqnarray*}
 	Q(t) = O(\sqrt{T}), ~\forall t \in [T]. 
 \end{eqnarray*}
 The above bound implies that the proposed online policy incurs a sublinear ($O(\sqrt{T})$) CCV over any interval in the entire horizon of length $T.$ 
 Finally, plugging in the above bound in Eq. \ref{main_eq}, we have that 
 \begin{eqnarray*}
 	\textrm{Regret}_t(x^\star) \leq O(\sqrt{t}), ~\forall t\geq 1. 
 \end{eqnarray*}
 \fi
The following theorem shows that under a new set of assumptions, which roughly corresponds to a certain type of worst-case adversary, one can prove the optimal $O(\sqrt{T})$ regret and CCV.  
\begin{framed}
\begin{theorem} \label{positive-regret}
	%Assume that there exists $x_t^\star \in \mathcal{X}_t$ such that $\textrm{Regret}_t(x_t^\star) \geq 0, \forall t \in [T].$ 
	Suppose Assumptions \ref{bddness} and \ref{non-neg-regret} hold.
	%However, we do not make any assumption on the validity of Slater's condition (\emph{i.e.,} we set $\eta^\star=0$). 
	Then, upon setting $V=\sqrt{T},$ we have
	% the following bounds for the regret and constraint violations.
	\begin{eqnarray*}
		Q(t)=O(\sqrt{T}), ~\textrm{Regret}_t(x^\star)= O(\sqrt{T}), ~\forall x^\star \in \mathcal{X}_t, \forall t \in [T].
	\end{eqnarray*}
	%This can be ensured, \emph{e.g.,} if the adversary chooses the constraints such that the current action remains feasible. 
	% we obtain $Q(t) = O(\sqrt{t}), \forall t \geq 1$ under Assumption \ref{bddness} when. 
\end{theorem}
\end{framed}
See Appendix \ref{positive-regret-proof} for the proof. 

\begin{corollary}
	Let Assumption \ref{bddness} holds and set the cost functions $f_t=0~, \forall t.$ Then $Q(t) = O(\sqrt{t}), \forall t \geq 1$.
\end{corollary} 	
 \fi
 %\vspace{-0.1in}
 \subsubsection{Strongly Convex Cost and Convex Constraint Functions} \label{str-cvx-lin-cnst}
We now consider the setting where each of the cost functions $f_t, t\geq 1,$ is $\alpha$-strongly convex for some $\alpha>0$. The constraint functions $g_t$'s are assumed to be convex as before and not necessarily strongly convex. 
%For this case, however, we assume that the values of the parameters $\alpha$ and $G$ are known to the policy, which will be used for setting the parameter $V.$
%Note that the value of the strong-convexity parameter $\alpha$ needs not be known to the policy. 
%Also we make the standard assumption that $\textrm{diam}(\mathcal{X}) \leq D.$ Our objective is to derive better regret and constraint violation penalty bound for this case by making use of the $L^\star$ regret bound given in \cite[Theorem 4.25]{orabona2019modern}. Generalizing the policy described above, we now consider a sequence of surrogate penalty functions.
% \begin{eqnarray} \label{transformed_cost}
% 	 c_t'(x) = Vc_t(x) - 2Q(t)\langle a_t, x\rangle,  
% \end{eqnarray}
% for some constant $V>0$ that will be fixed later.
  %In Algorithm \ref{coco_alg}, 
  In this case, we choose the regret-minimizing OCO subroutine for the surrogate cost functions to be the OGD algorithm with the step-size sequence as given in part 2 of Theorem \ref{data-dep-regret} in the Appendix (see Algorithm \ref{coco_alg}).
  %, whose regret bound is given by Eqn.\  \eqref{str-cvx-reg-bd}.
  \iffalse
  following adaptive regret bound \cite[Theorem 4.1]{hazan2007adaptive}:
 \begin{eqnarray} \label{strong-cvx-regret}
 	\textrm{Regret}_T \leq \frac{1}{2}\sum_{t=1}^T \frac{G_t^2}{H_{1:t}},
 \end{eqnarray} 
 where $H_{1:t}$ is the sum of the strong convexity parameters of the first $t$ cost functions.
 \fi
% Now proceeding similarly as above, we have 
% \begin{eqnarray} \label{regret-bd}
% 	Q^2(t) + V \textrm{Regret}_t \leq \textrm{Regret}_t'(x^\star). 
% \end{eqnarray}
% Note that for the transformed sequence of functions, we have 
% \begin{eqnarray*}
% 	||\nabla c_t'(x_t)||^2 \leq 2V^2L^2 + 8Q(t)^2.
% \end{eqnarray*} 
 Since the cost functions are known to be $\alpha$-strongly convex, each of the surrogate cost functions \eqref{surrogate_new} is $V\alpha$-strongly convex. Hence, using the bound from Eqn.\ \eqref{grad_bd_new}, 
 %plugging in the upper bound of the norm of the gradient of the surrogate costs from  \eqref{grad_bd_new} into  \eqref{str-cvx-reg-bd}, 
 choosing the scaling parameter to be $\beta=1$, and simplifying the generic regret bound given by Eqn.\ \eqref{str-cvx-reg-bd}, we obtain the following regret bound for learning the surrogate cost functions $\{\hat{f}_s\}_{s\geq 1}$:
 %\vspace{-0.1in}
 \begin{eqnarray}\label{adaptive_str_cvx_bd}
 	\textrm{Regret}'_t(x^\star) \leq \frac{VG^2}{\alpha} (1+\ln(t)) + \frac{G^2}{\alpha V} \sum_{\tau=1}^t \frac{(\Phi'(Q(\tau)))^2}{\tau}, ~ x^\star \in \mathcal{X}. 
 \end{eqnarray}
 %\vspace{-0.05in}
In the above, we have used the standard bound for the Harmonic sum: $\sum_{\tau=1}^t \frac{1}{\tau} \leq 1+ \ln(t)$, as well as the fact that $(a+b)^2 \leq 2(a^2+b^2). $
 Substituting the bound \eqref{adaptive_str_cvx_bd} into the regret decomposition inequality \eqref{gen-reg-decomp}, and using the non-decreasing property of the sequence $\{Q(\tau)\}_{\tau \geq 1}$ and the derivative $\Phi'(\cdot)$, we obtain 
 %\vspace{-0.1in}
 \begin{eqnarray} \label{Gronwall-ineq}
 	\Phi(Q(t)) + V \textrm{Regret}_t(x^\star)  \leq \frac{VG^2}{\alpha} (1+\ln(t)) + \frac{G^2}{\alpha V}(1+\ln(t)) \big(\Phi'(Q(t))\big)^2, ~ \forall x^\star \in \mathcal{X}^\star, \forall t.
 \end{eqnarray}
 Finally, choosing $\Phi(\cdot)$ as the quadratic Lyapunov function, \emph{i.e.,} $\Phi(x) \equiv x^2,$ we arrive at the following result for strongly convex cost and convex constraint functions. 
 %the following theorem bounds the regret and the CCV for Algorithm \ref{coco_alg}.

 \begin{theorem} \label{str-cvx-bd}
 	% For COCO with $\alpha$-strongly convex cost and convex constraint functions, 
 	 For the COCO problem with adversarially chosen $\alpha$-strongly convex, $G$-Lipschitz cost functions and $G$-Lipschitz convex constraint functions,
 	 Algorithm \ref{coco_alg}, with $\beta=1, V=\frac{8G^2 \ln(Te)}{\alpha}, \Phi(x)= x^2,$ yields the following Regret and CCV bounds for any horizon length $T \geq 1:$
	%\begin{eqnarray*}
		\[ \textrm{Regret}_t(x^\star) \leq  \frac{G^2}{\alpha}\big(1+\ln(t)\big), ~ \textrm{CCV}_t= O\big(\sqrt{\frac{t \log T}{\alpha}}\big), \forall x^\star \in \mathcal{X}^\star, ~\forall t \in [T].\]
	%\end{eqnarray*}
Furthermore, if the worst-case regret is non-negative in some round $t$ (\emph{i.e.,} $\sup_{x^\star \in \mathcal{X}^\star}\textrm{Regret}_t(x^\star) \geq 0$), then the CCV can be further improved to $\textrm{CCV}_T= O(\frac{\log T}{\alpha})$ while keeping the regret bound the same. 
%(b) For a natural class of adversaries, called convex adversaries defined in Eqn.\ \eqref{jensenadv} in Appendix \ref{improved_rates}, we have $\mathbb{V}(t)= O(\frac{\log T}{\alpha})$ under some mild assumptions. See Theorem \ref{improved_violation_bd} for a precise statement.
	%In the above, the notation $\tilde{O}(\cdot)$ hides factors logarithmic in $T$.
 \end{theorem}
 %\end{framed}
 %\vspace{-0.1in}
 Please refer to Appendix \ref{str-cvx-pf} for the proof of Theorem \ref{str-cvx-bd}. 
 
 \textbf{Remarks:} The second part of the theorem is surprising because it says that when the regret is non-negative, a stronger logarithmic CCV bound holds for not necessarily strongly convex constraints. In Appendix \ref{improved_rates}, we give example of an interesting class of adversaries, called \emph{convex adversary}, for which the non-negative regret assumption holds true in the OCO setting.
 %Hence, the strong convexity of the cost functions leads to a sharper bound for CCV when the regret is non-negative. 
 
 \subsection{Lower Bounds} \label{lower_bound_section}
We now show that under Assumptions \ref{cvx}, \ref{bddness}, and \ref{feas-constr}, the regret and the CCV of any online policy for the COCO problem for $T$ rounds are both lower bounded by $\Omega(\sqrt{T})$ provided the problem is high-dimensional.  
Recall that if the constraint function $g_t= 0, \forall t$, then the COCO problem reduces to the standard OCO problem, and  $\Omega(\sqrt{T})$ is a well-known regret lower bound for OCO  \citep[Theorem 10]{hazan2022introduction}. In this case, we trivially have {$\textrm{CCV}= 0.$}  
The main challenge in proving a lower bound for COCO is \emph{simultaneously} bounding both the regret and CCV. Prior work does not give any simultaneous lower bounds since the standard adversarial inputs used to derive the lower bound of \citet{hazan2022introduction} do not satisfy the feasibility assumption (Assumption 3). We derive the lower bound by constructing a sequence of cost and constraint functions that satisfy Assumption \ref{feas-constr} in a $d$-dimensional Euclidean box of unit diameter.



%Note that the CCV lower bound does not follow from the corresponding regret lower bound for the OCO problem due to the additional Assumption \ref{feas-constr}, which puts an implicit constraint on the admissible constraint functions.

\begin{theorem}\label{thm:lbcoco}
Under Assumptions \ref{cvx}, \ref{bddness}, and \ref{feas-constr}, for any choice of the horizon length $T$ and online policy, there exists a problem instance with dimension $d \geq T$ where $\min (\textrm{Regret}_T, \textrm{CCV}_T) = \Omega(\sqrt{T}).$  
%the regret and CCV as defined in \eqref{intro-regret-def} and \eqref{intro-gen-oco-goal} with $k=1$ are both $\Omega(\sqrt{T})$.
\end{theorem}
In high-dimensional problems where $d \gg T,$ the above lower bound matches with the upper bound given in Theorem \ref{main_result}. The proof of Theorem \ref{thm:lbcoco} can be found in Appendix \ref{app:lbcoco}.


 %It can be verified that the regret bounds given in Theorem \ref{gen-cvx-bd} and \ref{str-cvx-bd} are optimal as they match with the corresponding lower bounds for the unconstrained OCO problem \citep[Table 3.1]{hazan2016introduction}.
%Similar to Theorem \ref{gen-cvx-bd}, the assumption of non-negativity of regret can be relaxed for \emph{convex} adversaries. See Theorem \ref{improved_violation_bd} in the Appendix for the concrete result.
% The non-negative regret assumption in the latter parts of Theorem \ref{gen-cvx-bd} and \ref{str-cvx-bd} is non-trivial as the offline benchmark $x^\star$ is constrained to belong to the feasible set $\mathcal{X}^\star$ whereas the actions of the policy $\{x_t\}_{t\geq 1}$ belong to the larger admissible set $\mathcal{X} \supseteq \mathcal{X}^\star.$ Such assumptions are necessary as otherwise, it would violate the constraint violation lower bounds given in Section \ref{lower_bound_section}.  Nevertheless, in Section \ref{improved_rates} of the Appendix, we show that the tighter constraint violation bounds can be guaranteed for certain classes of worst-case adversaries, called \emph{convex adversaries}. This includes the standard offline convex optimization problem with a fixed cost and a fixed constraint function. See Theorem \ref{improved_violation_bd} in the Appendix for a concrete result.

 
 \iffalse 
 \subsubsection{Lower bounds} 
 The optimality of the regret bounds, given in Theorems \ref{gen-cvx-bd} and \ref{str-cvx-bd}, follows from the corresponding unconstrained regret bounds \citep[Table 3.1]{hazan2016introduction}.
 We conclude this section by establishing lower bounds to the CCV with convex constraints. 
 %The regret lower bound $\mathcal{X}(\sqrt{T})$ for convex cost functions without constraints is well-known \citep[Table 3.1]{hazan2016introduction}. 
 The following result shows that, under the assumption of non-negativity of the regret, the cumulative violation bounds given in Theorems \ref{gen-cvx-bd} and \ref{str-cvx-bd} are optimal. 
	
 
%The following result is straightforward.
%\begin{framed}
\begin{theorem}[(Lower bound to the CCV)] \label{cvx-lb}
1.  Consider the constrained OCO problem with convex cost functions and convex constraint functions. Then for any online policy, we have $\mathbb{V}(T) \geq \mathcal{X}(\sqrt{T})$. 

2.If the cost functions are $\alpha$-strongly convex, then for any online policy, we have $\mathbb{V}(T) \geq \mathcal{X}(\frac{\log{T}}{\alpha})$. 

\end{theorem}
See Section \ref{cvx-lb-pf} in the Appendix for the proof of the above result.
\fi
\iffalse
As a corollary of the above result, by setting $V=\Theta(T^{1/3}),$ we obtain $O(T^{2/3})$ bounds simultaneously for regret and CCV. Clearly, this bound improves upon the bound obtained by assuming only the convexity of the cost functions as given in Theorem \ref{gen-cvx-bd}.
\fi
\iffalse
Our next result shows that by assuming that Slater's condition is true, one can obtain an \emph{exponential improvement} of the previous regret and CCV guarantee.

 \begin{framed}
\begin{theorem} \label{str-cvx-slater}
	In addition to the convexity assumptions, assume that Slater's condition, as given in Definition \ref{slater}, holds for some $\eta^\star >0$. Then, for any strongly convex cost and convex constraint functions and for any constant $V = \mathcal{X}(\log T)$, we can obtain the following bound for the average queue length:	\begin{eqnarray*}
		\frac{1}{t}\sum_{\tau=1}^t Q(\tau) = O(V), ~\forall t \in [T].
	\end{eqnarray*}
	%In particular, upon setting $V=\log T,$ we obtain logarithmic regret and average violation penalty simultaneously.
\end{theorem}
\end{framed}
\fi

\iffalse

  \begin{algorithm}
\caption{Online Learning with Constraints}
\label{learning-with-constraints}
\begin{algorithmic}[1]
%\State \algorithmicrequire{ Target reward rate vector $\bm{\vec{\lambda}}$,} Euclidean projection oracle $\Pi_{\Delta_N}(\cdot)$ onto the simplex $\Delta_N.$
\State $\bm{Q} \gets \bm{0}, \bm{x} \gets [1/N, 1/N, \ldots, 1/N], V\gets \sqrt{T}, \texttt{epoch}\gets 1$ \algorithmiccomment{\emph{Initialization}} \label{init} 
\ForEach {round:}
\State Run an appropriate OCO algorithm for the cost function $c_t'$
\If {the cumulative regret from the last epoch is non-positive}
\State Start a new \texttt{epoch} at round $t$ and run OCO with cost function given by the constraints
\State Goto line \ref{init}
\EndIf
\EndForEach
\end{algorithmic}
\end{algorithm}


Our next result shows that under the assumption that worst-case regret remains positive throughout the horizon, both the queue length and the regret can increase only logarithmically with the horizon length. 
\begin{framed}
\begin{theorem} \label{positive-regret-str-cvx}
	%Assume that there exists $x_t^\star \in \mathcal{X}_t$ such that $\textrm{Regret}_t(x_t^\star) \geq 0, \forall t \in [T].$ 
	%However, we do not make any assumption on the validity of Slater's condition (\emph{i.e.,} we set $\eta^\star=0$). 
	Suppose Assumptions \ref{bddness} and \ref{non-neg-regret} hold. In addition, assume that each of the cost functions $f_t$'s are $\alpha$-strongly convex for some $\alpha >0.$ 
	Then, upon setting $V \geq \mathcal{X}(\ln(T)),$, we have: %the following bounds for the regret and constraint violations.
	\begin{eqnarray*}
		\textrm{Regret}_t(x^\star)= O(\frac{(\ln{t})^2}{V}),~ Q(t)=O(\sqrt{V\ln{t}}), ~\forall x^\star \in \mathcal{X}_t, \forall t \in [T].
	\end{eqnarray*}
	%This can be ensured, \emph{e.g.,} if the adversary chooses the constraints such that the current action remains feasible.  
\end{theorem}	
\end{framed}
See Appendix \ref{positive-regret-str-cvx-proof} for the proof of the above result. The above result implies that by choosing $V=\Theta(\log T)$, we can obtain both logarithmic regret and logarithmic CCV with respect to the length of the time horizon.
\fi
% \subsection{Improved Bounds for a Special case}
%Consider the special case when $\forall t, \exists x_t^\star: \textrm{Regret}_t(x_t^\star) >0.$ 

 \iffalse
 \section{Strongly Convex Constraints} \label{str-cvx-cnstr}
% We consider an online learning problem where on every round $t,$ an online policy chooses an action $x_t$ from a feasible closed and bounded convex set $\mathcal{X}.$ On the same round, the adversary chooses a cost function $f_t: \mathcal{X} \to \mathbb{R}$ and a constraint of the form $g_t(x) \leq b_t.$ 
%%where the function $g_t: \mathcal{X} \to \mathbb{R}$ is also $\alpha$-strongly convex. Both the functions $f_t$ and $g_t$ are assumed to be $L$-Lipschitz. 
%The policy chooses its action \emph{before} the adversary reveals its choices for round $t$. 
%
%
%Define a queueing recursion:
% \begin{eqnarray*}
% 	Q(t) = (Q(t-1)+ g_t(x_t)-b_t)^+, ~Q(0)=0.
% \end{eqnarray*}
% Define the potential function $\Phi(t)\equiv Q^2(t).$ Observe that for any real number $x$, we have $((x)^+)^2=xx^+.$ Hence, 
% \begin{eqnarray*}
% 	Q^2(t) &=& \left(Q(t-1)+ g_t(x_t)-b_t\right)Q(t) \\
% 	&=& Q(t-1)Q(t)+ Q(t) \left(g_t(x_t)-b_t\right)\\
% 	&\stackrel{(a)}{\leq}& \frac{1}{2}Q^2(t)+ \frac{1}{2}Q^2(t-1) + Q(t) \left(g_t(x_t)-b_t\right). 
% \end{eqnarray*}
% where in (a), we have used the AM-GM inequality. Rearranging the above inequality,
%% where $\max_{x \in \mathcal{X}} g_t^2(x)+ b_t^2 + 2|g_t(x)b_t| = \max_{x \in \mathcal{X}} (|g_t(x)|+|b_t|)^2\leq B.$ 
% the \emph{one-step drift} of the potential function $\Phi(t)$ may be upper bounded as 
% \begin{eqnarray*}
% 	\Phi(t)-\Phi(t-1) = Q^2(t)-Q^2(t-1)\leq 2Q(t)\left(g_t(x_t)-b_t\right).
% \end{eqnarray*}
% Motivated by the above calculations, we now define a surrogate cost function $f_t':\mathcal{X} \to \mathbb{R}$ as follows:
% \begin{eqnarray*}
% 	f'_t(x) \equiv Vf_t(x) + 2Q(t) g_t(x),
% \end{eqnarray*}
% where $V>0$ is a tunable parameter to be fixed later.  Now fix any feasible $x^\star \in \mathcal{X},$ which satisfies all $T$ constraints, \emph{i.e.,} $g_t(x^\star)\leq b_t, \forall t \in [T].$ For any $\tau \in [T],$ we have 
% \begin{eqnarray} \label{drift_ineq}
% 	&&\Phi(\tau)-\Phi(\tau-1) + V\big(f_\tau(x_\tau)-f_\tau(x^\star)\big) \nonumber  \\
% 	&\leq&  \left(V f_\tau(x_\tau) + 2Q(\tau)g_\tau(x_\tau)\right) - \left(Vf_\tau(x^\star) + 2Q(\tau)b_\tau\right)\nonumber  \\
% 	&\stackrel{(a)}{\leq} &  \left(V f_\tau(x_\tau) + 2Q(\tau)g_\tau(x_\tau)\right) - \left(Vf_\tau(x^\star) + 2Q(\tau)g_\tau(x^\star)\right)\nonumber \\
% 	&\leq & f_{\tau}'(x_\tau) - f_\tau'(x^\star), 
% \end{eqnarray}
% where in (a), we have used the feasibility of the static action $x^\star$ (\emph{i.e.,} $b_\tau \geq g_\tau(x^\star)$). Summing up the first $t$ inequalities \eqref{drift_ineq} above, we obtain 
% \begin{eqnarray*}
% 	\Phi(t)-\Phi(0) + V \textrm{Regret}_t(x^\star)\leq \textrm{Regret}_t'(x^\star). 
% \end{eqnarray*}
In this section, we investigate the case when each constraint function $2\psi(g_t(\cdot))$ are $\alpha$-strongly convex. To establish strong results without any extraneous assumptions, we first consider the case when the cost functions are identically equal to zero, and the only objective is to derive strong bounds for the queue length variables. 
 %\subsection{Queue Stabilization and Constraint Satisfaction}
%To test the limit of the proposed policy, consider the case when one is only interested in satisfying the constraints. 
By setting $f_t=0, \forall t,$ we immediately have $\textrm{Regret}_t(x^\star)=0.$ Hence, \eqref{strong-cvx-regret} yields the following recursive bounds on the queue length process;
\begin{eqnarray} \label{q-bd-str-cvx}
	Q^2(t) \leq \frac{G^2}{4\alpha} \sum_{\tau=1}^t \frac{Q^2(\tau)}{\sum_{s=1}^\tau Q(s)}.
\end{eqnarray}
%Using the result in Proposition \ref{q-bd-prop}, we immediately obtain that $Q(t) = O(\frac{2L^2}{\alpha}\log (t)).$ 
The following result gives a strong bound on the queue lengths as implied by the above bound. 
\begin{framed}
\begin{theorem}[(Upper bound on the queue length)]\label{constraint-sat-cvx}
For $\alpha$-strongly convex constraint functions $g$'s, the queue length process is bounded as \[Q(t)=O(\frac{G^2}{4\alpha}\log (t)),~\forall t \geq 1.\]
\end{theorem}
\end{framed}
See Proposition \ref{q-bd-prop} in the Appendix for the proof of the above result.

\section{Strongly Convex Constraints}
Clearly, in this case each surrogate cost function $f_t'$, defined in \eqref{surrogate-def}, is $(V+Q(t))\alpha$-strongly convex %\footnote{\textcolor{blue}{Actually, the surrogate loss function $f_t'$ is $(V\alpha + 2Q(t-1)\alpha)$-strongly convex}.} 
 with the norm of the (sub)gradients bounded by $G_t\leq (V+Q(t))\frac{G}{2}$ \eqref{grad-bd}. Hence, using the regret bound \eqref{strong-cvx-regret}, there exists an OGD policy that achieves the following bound:
 %we have the following bound under the action of an adaptive online gradient descent  
 \begin{eqnarray} \label{d-str-cvx-bd}
 	Q^2(t) +  V \textrm{Regret}_t(x^\star) \leq \frac{G^2}{4\alpha}\sum_{\tau=1}^t \frac{V^2+ Q^2(\tau)}{V\tau+ \sum_{s=1}^\tau Q(s)}.
 \end{eqnarray}
 \fi
% Lower bounding the sum in the denominator by zero, we get the same regret and violation penalty bounds as in Section \ref{str-cvx-lin-cnst}. However, potentially, one can do better using a more sophisticated analysis (\emph{e.g.,} Bihari Laselle's inequality).  


%Although, in this special case, one can also run a no-regret policy to minimize the sum of sequence of strongly-convex functions $\{g_t\}_{t\geq 1},$ it is interesting to see that our proposed policy also yields a logarithmic violation penalty. Note that, it is by no means obvious, as our proposed policy minimizes the sum of functions $\{Q(t)g_t\}_{t\geq 1},$ where $Q(t)$ varies according to the action of the policy as in \eqref{q-ev}.


\section{The Online Constraint Satisfaction Problem (\textsc{OCS})} \label{simul_constr}
%\vspace{-0.2in}
%We begin our discourse with $\ocs,$ 
In this section, we study
a special case of the COCO problem, which involves only constraint functions and no cost functions. The OCS problem arises in many practical settings, including the multi-task learning problem (see Section \ref{mtl-ocs} in the Appendix for a brief discussion). In Section \ref{cbc} in the Appendix, we also establish a connection between the OCS problem and the well-studied Convex Body Chasing problem \citep{argue2019nearly}. The setup is similar to the COCO setting -- on every round $t\geq 1$, an online policy selects an action $x_t$ from a closed, bounded, and convex admissible set $\mathcal{X} \subseteq \mathbb{R}^d$. After observing the current action $x_t$, the adversary chooses $k$ constraints of the form $g_{t,i}(x) \leq 0, i\in [k],$ where each $g_{t,i}: \mathcal{X} \mapsto \mathbb{R} $ is a convex function. 
 Let $\mathcal{I}$ be any sub-interval of the horizon $[1,T].$ The cumulative constraint violation (CCV) $\mathbb{V}(T)$ for the \textsc{OCS} problem is defined as the maximum \emph{signed} cumulative constraint violation in any sub-interval:
 \begin{eqnarray} \label{violation-def1}
		\mathbb{V}(T) = \max_{i=1}^k \mathbb{V}_i(T), ~ \textrm{where} ~ \mathbb{V}_i(T) = \max_{\mathcal{I} \subseteq [1,T]}\sum_{t \in \mathcal{I}} g_{t,i}(x_t), ~ 1\leq i \leq k.
\end{eqnarray}
The objective is to design an online learning policy so that $\mathbb{V}(T)$ is as small as possible.
It is worth noting that in the $\ocs$ problem, we consider a soft constraint violation metric $\max_{\mathcal{I}}\sum_{t \in \mathcal{I}} g_{t,i}(x_t)$ instead of the hard violation metric $\sum_{t=1}^T (g_{t,i}(x_t))^+$ as in COCO. This allows for compensating the infeasibility on one round with strict feasibility on other rounds.
%, under the relaxed assumptions discussed below. %Essentially, this helps get 
%stronger CCV bounds than COCO. If a model requires $\sum_{t}\max\{g_{t,i}(x_t), 0\}$ to be the constraint violation cost, then results derived in Theorem \ref{gen-cvx-bd} and \ref{str-cvx-bd} for COCO will apply.
%For constraint functions that can take both positive and negative values, controlling the cumulative violations over all sub-intervals is stronger than controlling the cumulative violations over the full horizon. 
In contrast with the COCO setting, without Assumption \ref{feas-constr}, running a no-regret policy on the pointwise maximum of the constraint functions no longer works as the CCV of any fixed benchmark could grow linearly with 
%the horizon length 
$T$. In the OCS problem, we relax the feasibility assumption (Assumption \ref{feas-constr}), and consider the following two distinct alternatives instead. 
 \paragraph{\bf 1. $S$-feasibility:} Here, we assume that there is an admissible action $x^\star \in \mathcal{X}$ that satisfies the aggregate constraints over any interval of $S$ rounds. However, unlike \citet{georgios-cautious}, which also considers the same assumption, the value of the parameter $S$ is not necessarily known to the policy \emph{a priori}. Towards this end, we define the set of all $S$-feasible actions $\mathcal{X}_S$ as below: 
\begin{eqnarray} \label{extended-benchmark}
\mathcal{X}_S =\{x^\star \in \mathcal{X}: \sum_{\tau \in \mathcal{I}} g_{\tau,i}(x^\star) \leq 0, \forall \textrm{sub-intervals}~ \mathcal{I} \subseteq [1,T], ~|\mathcal{I}| = S, \forall i \in [k]\}. 
\end{eqnarray}
We now replace Assumption \ref{feas-constr} with the following weaker version:
\begin{assumption}[$S$-feasibility] \label{s-feas-assump}
	$\mathcal{X}_S \neq \emptyset$ for some $1\leq S \leq T.$
\end{assumption}
%In our analysis, we assume that $\mathcal{X}_S \neq \emptyset$ for some $S \in [T],$ which need not be known to the policy. 
Clearly, Assumption \ref{s-feas-assump} is weaker than Assumption \ref{feas-constr} as $\mathcal{X}^\star \subseteq \mathcal{X}_S, \forall S \geq 1.$ Note that even when the individual constraint functions satisfy $S$-feasibility, their pointwise maximum need not satisfy $S$-feasibility. Hence, unlike COCO under Assumption \ref{feas-constr}, this problem cannot be solved by simply running a no-regret policy on the pointwise maximum of the constraints.  

\paragraph{\bf 2. $P_T$-constrained adversary}
In this case, we drop any feasibility assumption altogether. As a consequence, any static admissible benchmark $x^\star \in \mathcal{X}$ also incurs a CCV. 
\begin{definition} \label{pt-feas-assump}
%We now consider a different relaxation to the instantaneous feasibility assumption where we now assume that 
An adversary is called $P_T$-constrained if its minimum static CCV is $P_TF$, \emph{i.e.,} 
$ \frac{1}{F} \min_{x^\star \in \mathcal{X}} \max_{\mathcal{I} \subseteq [T],i} \sum_{t \in \mathcal{I}} g_{t,i}(x^\star) = P_T$,  where $F$ is a normalizing factor denoting the maximum absolute value of the constraint functions within the compact admissible set $\mathcal{X}$. 
\end{definition}
As before, the value of $P_T$ is not necessarily known to the policy \emph{a priori}.
%\end{itemize}
%there exists a fixed admissible action $x^\star \in \mathcal{X}$ such that the cumulative violation over any sub-interval of the horizon is upper bounded by $P_TF$, \emph{i.e.,}


%Our motivation to study this special case is to .
%On a high-level, the problem is to design an online policy with an optimal CCV bounds corresponding to each constraints. 
%Although a variant of this problem can be solved by running a no-regret policy on the sequence of functions $g_t \equiv \max_i g_{t,i}, t\geq 1,$ 
  %As an example, in the problem of multi-task learning, each of the $k$ constraint sequences can be thought to belong to a specific task. 
 %The constraint functions are accessible to the policy via a first-order oracle, which returns only the values and the gradients (sub-gradients in the case of non-smooth functions) of the constraint functions at the chosen action points. 

%This problem was considered earlier by \cite{georgios-cautious} in the COCO setting with a single constraint function per round. However, since the parameter $V$ used in their algorithm is restricted as $S \leq V \leq T,$ their algorithm naturally needs to know the value of $S.$ In reality, the value of the parameter $S$ is generally not available \emph{a priori} as it depends on the online constraint functions. Fortunately, our proposed meta-policy does not need to know the value of $S$ and hence, it can automatically adapt itself to the best feasible $S$.
%As an aside, the definition of maximum cumulative regret is similar to the definition of interval regret or adaptive regret \citep{jun2017improved, hazan2007adaptive}. However, the best-known strongly adaptive algorithms are inefficient as they need to run $O(\log T)$ experts algorithm on every round \citep{jun2017improved}. Interestingly, our proposed meta-policy is efficient as it performs only one gradient step per round. 
 %Although the \textsc{OCS} problem can be seen to be a special case of the COCO, \cmt{actually not true, with COCO we have $(g_{t,i}(x_t))^+$} it is important as a standalone problem as well. 
  %In Section \ref{non-trivial}, we discuss the difficulty of the $\ocs$ problem and explain why simple approaches fail. 
 %Section \ref{mtl-ocs} in the Appendix shows that the online multi-task learning problem can be naturally formulated as an instance of the \textsc{OCS} problem. 
%So far, we have not made any assumption on the constraint functions which could be selected arbitrarily by the adversary, and no constraint satisfaction bounds can be established for any online policy. 
%\vspace{-0.1in}
\iffalse
\subsection{Assumptions}  \label{assump}
In this section, we list the general assumptions which apply to both the \ocs ~problem and the COCO, described later in Section \ref{gen_oco}. Since the \ocs ~problem does not contain any cost function, the cost functions mentioned below necessarily refer to COCO only.

\begin{assumption}[Convexity] \label{cvx}
	The cost functions $f_t: \mathcal{X} \mapsto \mathbb{R}$ and the constraint functions $g_{t,i}: \mathcal{X} \mapsto \mathbb{R}$ are convex for all $t\geq 1, i\in [k]$. The admissible set $\mathcal{X} \subseteq \mathbb{R}^d$ is closed and convex and has a finite diameter of $D$. 
 %Moreover, $D$ is known ahead of time.
\end{assumption}
%\vspace{-0.18in}
\begin{assumption}[Lipschitzness] \label{bddness}
 %We have $\textrm{diam}(\mathcal{X}) \leq D, ||\nabla f_t(x)||_2 \leq G/2, \textrm{and}~ ||\nabla g_t(x))||_2 \leq G/2,~\forall t, \forall x\in \mathcal{X}$ for some finite constants $D$ and $G.$ If the functions are not necessarily differentiable, we require that the maximum magnitude of the sub-gradients be bounded accordingly.  Each
All cost functions $f_t$ and the constraint functions $g_{t,i}$'s are Lipschitz continuous with Lipschitz constant $G$, \emph{i.e.,} for all $x, y \in \mathcal{X},$ we have 
 \begin{eqnarray*}
 	|f_t(x)-f_t(y)| \leq G||x-y||,~
 	|g_{t,i}(x)-g_{t,i}(y)| \leq G||x-y||, ~\forall t\geq 1.
 \end{eqnarray*}
	\end{assumption}
	\vspace{-0.1in}
	Unless specified otherwise, the norm $||\cdot||$ will refer to the standard Euclidean norm throughout the paper. Hence, Assumption \ref{bddness} implies that the $\ell_2$-norm of the (sub)-gradients of the cost and constraint functions are uniformly upper-bounded by $G$ over the admissible set $\mathcal{X}.$ Finally, we make the following feasibility assumption on the online constraint functions.
\begin{assumption}[Feasibility] \label{feas-constr}
	There exists some feasible action $x^\star \in \mathcal{X} $ s.t. $g_{t,i}(x^\star) \leq 0, \forall t \in T, \forall i\in [k].$ The set of all feasible actions, denoted by $\mathcal{X}^\star,$ is called the feasible set. The feasibility assumption implies that $\mathcal{X}^\star \neq \emptyset.$
\end{assumption}
\fi
%For convex constraints, Assumption \ref{feas-constr} can be considerably relaxed by requiring the cumulative constraints to be satisfied only over intervals of length $S$. See Section \ref{ext} in the Appendix for details. 

 
%\begin{framed}
%\textbf{Terminologies:}
%On every round, the online policy takes action from the set $\mathcal{X}$, which we refer to as the \emph{admissible} set. The set of all admissible actions satisfying Assumption \eqref{feas-constr} is known as the \emph{feasible} set, which we denote by $\mathcal{X}^\star \subseteq \mathcal{X}$. Note that the feasible set depends on the online constraints and could be a strict subset of the admissible set.  
%\end{framed}
%The above assumption was also made by a number of related papers  \citep{yu2016low,yuan2018online,yi2023distributed, georgios-cautious}.
 %Next, we discuss the main idea of the paper that is useful to derive an upper bound for CCV for the $\ocs$ as well as the COCO.
%\vspace{-0.1in}
\subsection{Designing an \textsc{OCS} Policy with a Quadratic Lyapunov function} \label{meta-policy-ocs}
We define a process $\bm{Q}(t)=\big(Q_i(t), i \in [k]\big), t \geq 1$, which tracks the CCV:
\begin{eqnarray}\label{q-ev2}
	Q_i(t) = \big(Q_i(t-1)+ g_{t,i}(x_t)\big)^+, ~ Q_i(0)=0, ~t \geq 1, ~\forall i \in [k].
\end{eqnarray}
Notably, in contrast to COCO, we \emph{do not} clip the constraint functions in the above recursion. 
%where we have used the notation $y^+ \equiv \max(0,y).$
Expanding Eqn.\ \eqref{q-ev2}, which is also known as the queueing recursion or the \emph{Lindley process} \citep[pp. 92]{asmussen2003applied}, and using the definition in Eqn.\ \eqref{violation-def1}, we have the following relation for all $i \in [k]:$
\begin{eqnarray} \label{V-Q}
	\mathbb{V}_i(T) \equiv \max_{t=1}^T\max (0,\max_{\tau=0}^{t-1} \sum_{s=t-\tau}^t g_{s,i}(x_l))=\max_{t=1}^TQ_i(t). 
\end{eqnarray}
Equation \eqref{V-Q} indicates that to control the CCV \eqref{violation-def1}, it is sufficient to control the $\bm{Q}(t)$ process. Similar to the COCO problem, we combine the classic Lyapunov method with adaptive no-regret OCO policies to control the $\bm{Q}(t)$ process. 

%\section{An Online Meta-Policy} \label{policy}
%We will keep track of the cumulative constraint violation through the deterministic scalar process referred to as \emph{queue lengths} $\{Q(t)\}_{t \geq 1}$ that evolves as follows:
% \footnote{If the penalty function $\psi$ is also given to be non-negative, then the $(\cdot)^+$ operation in the definition of the recursion becomes superfluous.}:
% \begin{eqnarray} \label{q-ev}
% 	Q(t) = \big(Q(t-1)+ \psi(g_t(x_t))\big)^+, ~Q(0)=0. \end{eqnarray}
%\vspace{-0.1in}
\paragraph{A Quadratic Lyapunov function:}
 	 We consider the quadratic potential function %\begin{eqnarray*}\label{eq:potfunc}
$\Phi(\bm{Q}(t))\equiv \sum_{i=1}^k Q_i^2(t), t\geq 1.$
%\vspace{-0.1in}
%\end{eqnarray*} 
Since $((x)^+)^2=xx^+, \forall x\in \mathbb{R},$ from Eqn.\ \eqref{q-ev2}, we have 
%that for each $1\leq i\leq k:$
	 %\vspace{-0.1in}
 \begin{eqnarray} \label{q-bd2}
 	Q_i^2(t) &=& \big(Q_i(t-1)+ g_{t,i}(x_t)\big)Q_i(t) = Q_i(t-1)Q_i(t)+ Q_i(t) g_{t,i}(x_t), \nonumber \\
 	&\stackrel{(a)}{\leq}& \frac{1}{2}Q_i^2(t)+ \frac{1}{2}Q_i^2(t-1) + Q_i(t) g_{t,i}(x_t), ~ \forall i \in [k]. 
 \end{eqnarray}
 where in $(a)$, we have used the AM-GM inequality. Rearranging Eqn.\ \eqref{q-bd2},
% where $\max_{x \in \mathcal{X}} g_t^2(x)+ b_t^2 + 2|g_t(x)b_t| = \max_{x \in \mathcal{X}} (|g_t(x)|+|b_t|)^2\leq B.$ 
 %the \emph{one-step drift} 
 the change of the potential function $\Phi(\bm{Q}(t))$ on round $t$ can be upper bounded as follows
 	 %\vspace{-0.125in}
 \begin{eqnarray} \label{drift-bd}
 	\Phi(\bm{Q}(t))-\Phi(\bm{Q}(t-1)) = \sum_{i=1}^k \big(Q_i^2(t)-Q_i^2(t-1)\big)\leq 2\sum_{i=1}^k Q_i(t)g_{t,i}(x_t).
 \end{eqnarray}
 Similar to \eqref{surrogate_new}, we now define a surrogate cost function $\hat{f}_t:\mathcal{X} \mapsto \mathbb{R}$ as a linear combination of the constraint functions with the coefficients given by the vector $\bm{Q}(t)$, \emph{i.e.,}
 %\vspace{-0.175in}
 \begin{eqnarray} \label{surrogate-def}
 	\hat{f}_t(x) \equiv 2\sum_{i=1}^kQ_i(t) g_{t,i}(x).
 \end{eqnarray}
 %\vspace{-0.11in}
Clearly, the surrogate cost function $\hat{f}_t(\cdot)$ is convex since the coefficients $Q_i(t)$'s are non-negative and the constraint functions are convex. Our \textsc{OCS} policy, described below, simply runs a regret-minimizing adaptive OCO subroutine on the surrogate cost function sequence \eqref{surrogate-def}.

 %\begin{framed}
\textbf{The \textsc{OCS} policy (Algorithm \ref{ocs-policy}):} Pass the surrogate cost functions $\{\hat{f}_t\}_{t\geq 1}$ to the AdaGrad algorithm which enjoys a data-dependent regret as given in part 1 of Theorem \ref{data-dep-regret} in the Appendix (Eqn.\ \eqref{cvx-reg-bd}).
 %\end{framed}
% \textcolor{blue}{Explain what data-dependent bound means}.
 %Note that the convex cost function $\hat{f}_t: \mathcal{X} \mapsto \mathbb{R}$ defined in Eqn.\ \eqref{surrogate-def} is a legitimate input cost function to any OCO subroutine as all information required to compute the surrogate cost $\hat{f}_t$ (\emph{i.e.,} $Q(t)$ and $\{g_{t,i}\}_{i=1}^k$ is causally known at the end of round $t$. 
 %A complete pseudocode of the proposed meta-policy is given in.
 %The proposed meta-policy is summarized in Algorithm \ref{ocs-policy}. 
 %To clarify this point further, let us explicitly illustrate the order of events on round $t$. 

 
\begin{algorithm} 
\caption{Online Policy for \textsc{OCS}}
\label{ocs-policy}
\begin{algorithmic}[1]
\State \algorithmicrequire{ Sequence of convex constraint functions $\{g_{t,i}\}_{i\in [k], t\geq 1}$, a closed and convex admissible set $\mathcal{X}$ with a finite Euclidean diameter $D,$ $\mathcal{P}_\mathcal{X}(\cdot)=$ Euclidean projection operator on the set $\mathcal{X}$ }
\State \algorithmicensure{  Sequence of admissible actions $\{x_t\}_{t\geq 1}$}
\State{\bf{Initialization}:} Set $ x_1 \in \mathcal{X}$ arbitrarily, $Q_i(0)=0, ~\forall i \in [k].$ 
\ForEach {each round $t \geq 1$}
\State Play $x_t,$ observe the constraint functions $\{g_{t,i}\}_{i \in [k]}$ revealed by the adversary. 
\State [\textbf{Update $\bm{Q}(t)$}] $Q_i(t) = (Q_i(t-1)+ g_{t,i}(x_t))^+, i \in [k]$.
\State [\textbf{Compute a subgradient}] %Compute the gradient of the surrogate cost function 
$\nabla_t \equiv \nabla \hat{f}_t(x_t) = 2\sum_{i=1}^kQ_i(t) \nabla g_{t,i}(x_t).$
\State [\textbf{AdaGrad step}]  Compute the next action $x_{t+1} = \mathcal{P}_\mathcal{X}(x_t - \eta_t \nabla_t)$, where 
  % \begin{eqnarray*}
   $\eta_t =
   	\frac{\sqrt{2}D}{2\sqrt{\sum_{\tau=1}^{t} ||\nabla_\tau||_2^2}}.$
   %	\end{eqnarray*}
%\State [\textbf{OCO step}] Feed $\hat{f}_t(x)$ to the base OCO sub-routine $\Pi$, which outputs an action $x_{t+1} \in \mathcal{X}.$ \label{oco-step}
\EndForEach
\end{algorithmic}
\end{algorithm}
%Note that, unlike some of the previous work based on the Lyapunov drift approach \citep{neely2017online, yu2016low}, Algorithm \ref{ocs-policy} takes full advantage of the adaptive nature of the base OCO sub-routine by exploiting the fact that the adversary is allowed to choose the surrogate cost function $\hat{f}_t$ \emph{after} seeing the current action of the policy $x_t,$ which determines the coefficients $Q_i(t)$'s.

\subsection{Performance Bounds} 
\label{ext}
%Under Assumption \ref{feas-constr}, there exists a fixed action $x^\star \in \mathcal{X}$ that satisfies \emph{each} of the online constraints on \emph{each round}, \emph{i.e.,} $\mathcal{X}^\star \neq \emptyset.$
%Since this is a reasonably strong assumption, in this section, we replace Assumption \ref{feas-constr} with  a weaker one (Assumption \ref{s-feas-assump}) or get rid of it altogether (Theorem \ref{P_T-benchmark}). 
%\subsubsection{$S$-feasibility}
%Here we assume that there is a feasible action $x^\star$ that satisfies the aggregate of the constraints over any interval of $S$ rounds, where the parameter $S \in [T]$ need not be known to the policy \emph{a priori}. Towards this end, we define the set of all $S$-feasible actions as below: 
%\begin{eqnarray} \label{extended-benchmark}
%\mathcal{X}_S =\{x^\star: \sum_{\tau \in |\mathcal{I}|} g_{\tau,i}(x^\star) \leq 0, \forall \textrm{sub-intervals}~ \mathcal{I} \subseteq [1,T], ~|\mathcal{I}| = S, \forall i \in [k]\}. 
%\end{eqnarray}
%We now replace Assumption \ref{feas-constr} with the following weaker version.
%\begin{assumption}[$S$-feasibility] \label{s-feas-assump}
%	$\mathcal{X}_S \neq \emptyset.$
%\end{assumption}
%Here we assume that $\mathcal{X}_S \neq \emptyset$ for some $S \in [T],$ which need not be known to the policy. Clearly, Assumption \ref{s-feas-assump} is weaker than Assumption \ref{feas-constr} as $\mathcal{X}^\star \subseteq \mathcal{X}_S, \forall S \geq 1.$ Note that even when the individual constraint functions satisfy $S$-feasibility, their point wise max need not satisfy $S$ feasibility. We emphasize that, unlike COCO with Assumption \ref{feas-constr}, this problem can not be solved by simply running a no-regret policy on the point wise max of the constraints.  

 \begin{theorem} \label{S-benchmark}
Under Assumptions \ref{cvx}, \ref{bddness}, and \ref{s-feas-assump}, 
%the OGD policy with adaptive step-sizes given in part 1 of Theorem \ref{data-dep-regret} as a sub-routine, 
Algorithm \ref{ocs-policy} achieves the following CCV bound for the OCS problem: 
 	 $\mathbb{V}(T)= O(\max(\sqrt{ST},S )).$
 \end{theorem}




 \begin{theorem} \label{P_T-benchmark}
 Under Assumptions \ref{cvx} and \ref{bddness}, 
% the OGD policy with adaptive step-sizes given in part 1 of Theorem \ref{data-dep-regret} as a sub-routine achieves the following CCV bound 
Algorithm \ref{ocs-policy} achieves the following CCV bound for the OCS problem
 for any $P_T$-constrained adversary as given in Definition \ref{pt-feas-assump}: 
 	 \[\mathbb{V}(T)= O(P_T^{\nicefrac{1}{3}}T^{\nicefrac{2}{3}})+O(\sqrt{T}).\]
 \end{theorem}
 Trivially, we have $S\leq T$ and  $P_T \leq T.$ In the non-trivial case where either $S$ or $P_T$ increases \emph{sub-linearly} with the horizon length $T$, the above theorems yield sublinear CCV bounds.
%In our case, the surrogate cost function $\hat{f}_t(\cdot)$ depends on $x_t$ via the coefficient vector $\bm{Q}(t).$ 
%\vspace{-0.3in}
\iffalse
 \subsection{Analysis}
%\begin{enumerate}
%	
%	%\item  We compute the cumulative violation $Q(t)$ (which is a non-negative number) using the recursion \eqref{q-ev}. Note that $Q(t)$ depends on the action $x_t$ on the current round $t$.
%	\item  The surrogate cost function $\hat{f}_t$ defined in  \eqref{surrogate-def} is passed to the policy.
%\end{enumerate}
%This should be compared with \cite{neely2017online}\\  
%\hrule
%\textbf{Remarks:} 
%2.  Although we introduce the auxiliary queue-length process $\{Q(t)\}_{t \geq 1}$ as a convenient mathematical tool to bound the constraint violation penalty $\mathbb{V}_T$, in some problems, the process defined in \eqref{q-ev} arises quite naturally as the evolution of some physical queueing process. In these problems, controlling the queue length itself is of primary interest (see Section \ref{app} for an example). When the penalty function $\psi(g_(\cdot))$ can assume negative values as well (which is the case, \emph{e.g.,} when $\psi$ is the identity function and $g_t$'s are linear functions, upper bounding the queue lengths is strictly harder than upper bounding the violation penalty.
%
%3. Since $\nabla f_t'(x_t)= V \nabla f_t(x_t) + 2Q(t) \psi'(g_t(x_t))\nabla g_t(x_t),$ in reality, we will only need to pass the (sub-)gradients $\nabla f_t(x_t), \nabla g_t(x_t)$, and the value of $g_t(x_t)$ to the OCO sub-routine. \\
%\hrule 
%
%\subsection{Boundedness Assumption} \label{bdd-assumption}


%We will also assume that the cost functions are normalized so that $||f_t||_\infty \leq 1/2, \forall t.$
%\begin{framed}
%\textbf{Variants of the Constraint violation metric:} Our framework is flexible. Then we can consider a modified evolution for $Q(t):$
%\begin{eqnarray*}
%	Q(t)=\big(Q(t-1)+\psi(g_t(x)\big).~~, Q(0)=0\footnote{If $h>0,$ it is unnecessary to take $\max(0,\cdot)$ of $Q(t)$.}.
%\end{eqnarray*}
%Using the same analysis as before, we have 
%\begin{eqnarray*}
% 	\Phi(t)-\Phi(t-1) = Q^2(t)-Q^2(t-1)\leq 2Q(t)\psi(g_t(x_t)).
% \end{eqnarray*}
%  Motivated by the above calculations, we now define a surrogate cost function $f_t':\mathcal{X} \to \mathbb{R}$ as follows:
% \begin{eqnarray} \label{surrogate-def}
% 	f'_t(x) \equiv Vf_t(x) + 2Q(t) \psi(g_t(x)),
% \end{eqnarray}
%\end{framed}


\paragraph{Regret decomposition:} 
 Fix any feasible action $x^\star \in \mathcal{X}^\star$ (Assumption \ref{feas-constr}). Using  \eqref{drift-bd}, for any round $\tau \in [T]$ we have: 
 \begin{eqnarray} \nonumber
 	\Phi(\tau)-\Phi(\tau-1)   
 	\leq  2\sum_{i=1}^k Q_i(\tau)g_{\tau,i}(x_\tau) &\stackrel{(a)}{\leq}&  2\sum_{i=1}^k Q_i(\tau)g_{\tau,i}(x_\tau) - 2\sum_{i=1}^k Q_i(\tau)g_{\tau,i}(x^\star),  \ \\\label{drift_ineq2}
 	&= & \hat{f}_{\tau}(x_\tau) - \hat{f}_\tau(x^\star), 
 \end{eqnarray}
 where in $(a)$ we have used the assumption that the action $x^\star$ is feasible and hence, $g_{\tau, i}(x^\star) \leq 0, \forall i, \tau$. 
 %(\emph{i.e.,} $g_\tau(x^\star) \leq 0$ and that $\psi(z) \leq 0, \forall z\leq 0$). 
 Summing up the inequalities \eqref{drift_ineq2} above from $\tau=t_1+1$ to $\tau=t_2$, we obtain \eqref{master_eqn} that relates the change in the potential function in an interval  with the regret \eqref{intro-regret-def} for %for learning the original cost functions to the regret for 
 learning the surrogate cost functions $\hat{f}_t$'s:
 \begin{eqnarray} \label{master_eqn}
 	\Phi(t_2)-\Phi(t_1) \leq \sum_{\tau=t_1+1}^{t_2}\hat{f}_{\tau}(x_\tau) - \sum_{\tau=t_1+1}^{t_2}\hat{f}_\tau(x^\star) \stackrel{\textrm{(def.)}}{=} \textrm{Regret}_{t_1+1:t_2}'(x^\star), ~ \forall x^\star \in \mathcal{X}^\star. 
 \end{eqnarray}
%We emphasize that the regret on the RHS depends on the queue variables $\{Q(t)\}_{t\geq 1},$ which are implicitly controlled by the online policy through the evolution \eqref{q-ev2}. 
By setting $t_1=0, t_2=t$ and recalling that $\Phi(0)= \sum_i Q_i^2(0)=0$, from \eqref{master_eqn} we have that 
$	\sum_{i=1}^kQ_i^2(t)  \leq \textrm{Regret}'_t(x^\star), ~ \forall x^\star \in \mathcal{X}^\star, t\geq 1,$ where $\textrm{Regret}'_t(x^\star)$ is the regret \eqref{intro-regret-def} with functions ${\hat f}_t$'s.
Note that \eqref{master_eqn} is valid for any feasible action $x^\star \in \mathcal{X}^\star$, thus taking the supremum of the RHS of \eqref{master_eqn} over the set of all admissible actions in $  \mathcal{X}^\star \subseteq\mathcal{X}$, we obtain
%\vspace{-0.1in}
%Using the same drift analysis (with $V=0$), we obtain the following inequality similar to \eqref{q-bd-eqn}
\begin{eqnarray} \label{q-regret-reln}
	\sum_{i=1}^kQ_i^2(t)  \leq \textrm{Regret}_t',
\end{eqnarray}
where $\textrm{Regret}_t' \equiv \sup_{x^\star \in \mathcal{X}} \textrm{Regret}'_t(x^\star)$, defined in \eqref{intro-regret-def}, denotes the worst-case regret over the entire admissible set $\mathcal{X}$ for the sequence of surrogate cost functions $\{\hat{f}_\tau(x)\}_{\tau=1}^t$. 
%Note that the surrogate cost functions explicitly depend on the queueing processes. Hence, the regret bound in \eqref{q-regret-reln} depends on the online policy employed in step \ref{oco-step} of Algorithm \ref{ocs-policy}. 
Inequality \eqref{q-regret-reln} is the key step for bounding the CCV for $\ocs$ in Theorem \ref{constr-violation}. The rest of the proof uses  \eqref{q-regret-reln} in combination with the off-the-shelf data-dependent adaptive regret bounds achieved by the base OCO subroutine (see Theorem \ref{data-dep-regret}).  To derive Theorem \ref{constr-violation}, we use the OGD with adaptive step sizes, but we emphasize that the bound \eqref{q-regret-reln} is general and can be used to obtain performance bound for any base OCO policy with similar regret bounds. The following is our main result for the $\ocs$ problem. 
%In the following two sections, we consider the case of convex and strongly convex constraint functions, respectively.
%\begin{eqnarray*}
%	\hat{f}_t(x)= 2 \sum_{i=1}^k Q_i(t) g_{t,i}(x).
%	\end{eqnarray*} 

%Hence, we have the following result.
%In Proposition \ref{q-bd-prop}, we  show that any sequence of non-negative numbers $\{Q(t)\}_{t\geq 1}$ satisfying the inequality \eqref{str-q-recur} can grow at most logarithmically. Hence, we conclude that $\max_{i=1}^k (Q_i(t)) = O(\log t).$ 
\iffalse
\subsection{Case III: Convex and Smooth Constraints} \label{smooth-constraints}
We now consider the case when each constraint function is convex, non-negative, $M$-smooth, and uniformly bounded above by $B$. This implies that the  resulting surrogate cost functions are also convex, non-negative, $\hat{M}$-smooth and uniformly bounded above by $\hat{B}$ where
\begin{eqnarray} \label{fn-bd}
	\hat{B} \leq 2B \sum_{i=1}^k Q_i(t).
\end{eqnarray}
Furthermore, the surrogate function $\hat{f}_t$ is also smooth with the smoothness parameter given by 
\[ \hat{M} \leq 2M \sum_{i=1}^k Q_i(t) \]
Let the base OCO policy be taken as the OGD policy with step sizes chosen in part 3 of Theorem \ref{data-dep-regret}. Using the adaptive $L^\star$ bound quoted in the theorem, inequality \eqref{q-regret-reln} yields the following bound on the queue length process:
\begin{eqnarray*}
	\sum_{i=1}^k Q_i^2(t) \leq c_0 + c_1 \sqrt{\sum_{\tau=1}^t \sum_{i=1}^k Q_i(\tau)},  
\end{eqnarray*}
where $c_0 \equiv$ 
\fi
%The results in Sections \ref{cvx-sec} and \ref{str-cvx-sec} lead to our main result for the \ocs ~problem.
%\begin{framed}
\begin{theorem}[Bounds on the CCV for the \textsc{OCS}] \label{constr-violation}
The \textsc{OCS} Meta-policy (Algorithm \ref{ocs-policy}) achieves a CCV of $O(\sqrt{kT})$ and $O(\frac{k}{\alpha}\log T)$  for convex and strongly-convex constraint functions respectively, by using the standard OGD subroutine with an adaptive step size schedule as described in part 1 and part 2 of Theorem \ref{data-dep-regret}.
%in step \ref{oco-step} of Algorithm \ref{ocs-policy}.	
\end{theorem}
\vspace{-.15in}
See Section \ref{constr-violation-pf} for the proof of the above result. The extension of Theorem \ref{constr-violation} to $S$-feasible benchmarks is given in Section \ref{ext}.
The dependence of $k$ (the number of constraints revealed in each round) on the CCV derived in Theorem \ref{constr-violation} for the two cases is interesting. When $g_{t,i}$'s are convex, it grows as $\sqrt{k}$, while when $g_{t,i}$'s are strongly convex, it grows linearly with $k$. \emph{A priori}, the right scaling of CCV w.r.t. $k$ is unclear as increasing $k$ increases the CCV \eqref{violation-def1} since it is the maximum over a larger number of constraints. However, increasing $k$ also reveals more information about the feasible set ${\mathcal X}^\star$ in each round, which can potentially help a policy in reducing its CCV. A lower bound of $\Omega(\sqrt{T})$ on the CCV for any policy in the convex case is presented in Theorem \ref{thm:lbcoco}.
\fi 
%We note that for strongly convex constraint functions, the finiteness of the diameter $D$ is not required.
\iffalse
\paragraph{Applications:} In Section \ref{mistake_bd}, we use Theorem \ref{constr-violation} to design a multi-task binary classification policy with a bounded number of mistakes (See Theorem \ref{mistake-bd-thm} for the precise statement). This generalizes the classical Perceptron mistake bound of \citet{novikoff1962convergence} to the multi-task setting. Furthermore,
Section \ref{app}  outlines an application of the \ocs~ algorithm to a canonical network switching problem.
\fi
%\vspace{-.175in}

%
%\subsection{Connection between \textsc{OCS} and the Convex Body Chasing Problem} 
%A related and well-studied problem to \textsc{OCS} is the
%{\it nested convex body chasing (NCBC)} \citep{bansa2018nested,argue2019nearly,bubeck2020chasing} 
%where at each time $t$, a convex set $\chi_t \subseteq \chi$ is revealed such that 
%$\chi_t\subseteq \chi_{t-1}$, and  $\chi \subseteq {\mathbb R}^d$ is a convex, compact and bounded set. 
%The objective is to choose  $x_t \in \chi_t$ so as to minimize the total movement cost across time
%$C =   \sum_{t=1}^T  ||x_t - x_{t-1}||,$
%where $x_0$ is some fixed action.
%In NCBC, action $x_t$ is chosen \emph{after} the set $\chi_t$ is revealed contrary to the \textsc{OCS} problem, where $x_t$ has to be chosen \emph{before} the constraints $g_{t,i}$'s are revealed at time $t$. Moreover, note that the nested condition $\chi_t \subseteq \chi_{t-1}$ is stricter than Assumption \ref{feas-constr} that is applicable for the \textsc{OCS}.
%However, as we show next, a feasible algorithm for NCBC also provides a bound on the CCV of the \textsc{OCS} as follows.
%
%Let $\chi_t $ be the intersection of the $k$ constraints $g_{t,i}, i=1,\dots, k$ revealed  at time $t$ for the \textsc{OCS}.
%Let $x_t$ be the action chosen by an algorithm $\cal A$ for the NCBC after $\chi_t$ is revealed. Then consider choosing $y_{t+1} = x_t$ as the action for the \textsc{OCS}  that ensures that action $y_t$ is chosen before $g_{t,i}$ or $ \chi_t$ is revealed.
%The resulting $i^{th}$ constraint violation for the \textsc{OCS}  at time $t$ is $ g_{t,i}(y_{t}) \stackrel{(a)}\le g_{t,i}(y_{t}) - g_{t,i}(y_{t+1}) \le G ||y_{t} - y_{t+1}||$
%where $(a)$ follows from the feasibility of $\cal A$ for NCBC, $y_{t+1}= x_{t} \in \chi_{t}$ and hence $g_{t,i}(y_{t+1}) \leq 0$. Summing across time $t=1, \dots, T$, and taking the $\max$ over  all the $k$ constraints, we get that the CCV using $\cal A$ for the \textsc{OCS} is upper bounded by $ \sum_{t=2}^T G ||y_{t} - y_{t+1}|| \le \sum_{t=2}^T G ||x_{t-1} - x_{t}|| \le G \cdot C_{\cal A},$
%where $C_{\cal A}$ is the cost of $\cal A$ for the NCBC.
%
%From prior work \cite{bansa2018nested,argue2019nearly,bubeck2020chasing}, it is known that for NCBC, a Steiner point-based algorithm that chooses $x_t$ as the Steiner point of $\chi_t$ can achieve
%$C_{\cal A} = O(\sqrt{d \log d})$, where $\chi \subset {\mathbb R}^d$. Thus, the Steiner point-based algorithm (even though computationally intensive) provides a $O(\sqrt{d \log d})$ constraint violation for the 
%\textsc{OCS} as well. However, this result is applicable or effective for problems where  $\sqrt{d \log d} = o(T).$ Our result (Theorem \ref{constr-violation}) efficiently overcomes this hurdle and provides a bound even beyond $\sqrt{d \log d} = o(T)$, a setting that is practically better motivated for modern machine learning applications which are characteristically high-dimensional.
%
%
%
%
%
%
%
%












%\input{unconditional_alg}
%%\section{Generalizing the \ocs ~problem with the $S$-feasibility assumption}
%\subsection{Adversaries Ensuring Non-negative Regret} \label{improved_rates}
%In Theorem \ref{gen-cvx-bd} and \ref{str-cvx-bd}, we showed that under the assumption of the non-negativity of the worst-case regret, the constraint violation bounds can be improved to $O(\sqrt{{T}})$ and $O(\ln T/\alpha)$ for convex and strongly-convex cost functions, respectively. In this section, we additionally show that the same improved bounds hold in the case of a time-invariant fixed constraint function and a certain class of worst-case adversaries, called \emph{convex adversary}, defined next. COCO with time-invariant constraints has been studied in the literature where the main objective is to design gradient-based first-order policies that avoid the costly projection step on the constraint set \citep{jenatton2016adaptive, yuan2018online}. 

\paragraph{Convex adversary:} An adversary is called \emph{convex} if for any sequence of  action sequence $\{x_t\}_{t=1}^T,$ the adversary chooses the cost function sequence $\{f_t\}_{t=1}^T$ such that for any $T \geq 1,$ we have
\begin{eqnarray} \label{jensenadv}
	\sum_{t=1}^T f_t(x_t) \geq \sum_{t=1}^T f_t(\bar{x}_T),
\end{eqnarray}     
where $\bar{x}_T \equiv \frac{1}{T}\sum_{t=1}^T x_t.$ Hence, by definition, a convex adversary guarantees a non-negative regret with respect to the average action $\bar{x}_T$ for all rounds. In the following, we give two examples of convex adversaries.


 \paragraph{1. Fixed adversary:} An adversary which always selects a fixed convex function $f$ on all rounds is a convex adversary. In this case, Eqn.\ \eqref{jensenadv} holds due to the Jensen's inequality. 

\paragraph{2. Minimax adversary:} Let $\mathcal{F}$ denote an arbitrary non-empty set of convex functions defined on the admissible set $\mathcal{X}$. Consider an adversary $\mathcal{M}$, which, upon seeing the selected action $x_t$, chooses the worst cost function $f_t$ from the set $\mathcal{F}$ on round $t:$ 
\[f_t \in \arg\max_{f\in \mathcal{F}} f(x_t). \]
We now show that $\mathcal{M}$ is a convex adversary. By definition, for any round $\tau \in [T],$ we have 
\[ f_\tau(x_\tau) \geq f_t(x_\tau) \implies f_\tau(x_\tau) \geq \frac{1}{T} \sum_{t=1}^T f_{t}(x_\tau). \]
Summing up the above inequalities for each $\tau \in [T],$ we have 
\begin{eqnarray}\label{conv-adv-def}
 \sum_{\tau=1}^T f_\tau(x_\tau) \geq \sum_{t=1}^T \frac{1}{T}\sum_{\tau=1}^T f_t(x_{\tau}) \stackrel{\textrm{(a)}}{\geq} \sum_{t=1}^T f_t(\bar{x}_T),
 \end{eqnarray}
where inequality (a) follows upon applying Jensen's inequality to each cost function. 
Eqn.\ \eqref{conv-adv-def} shows that $\mathcal{M}$ is a convex adversary. 

P.S. It can be easily seen that Fixed adversary is a special case of Minimax adversary where $\mathcal{F}=\{f\}.$
\iffalse
\subsection{Assumptions}
\begin{assumption}[Time-invariant constraints] \label{constr_assump1}
Assume that the constraint functions on every round are fixed and given by $g_t=g.$ Hence, the feasible set is defined as:
\begin{eqnarray*}
	\mathcal{X}^\star = \{x^\star \in \mathcal{X}: g(x^\star) \leq 0\}.  
\end{eqnarray*}
\end{assumption}
\begin{assumption}[Convex adversary] \label{adversary_assump2}
	The cost functions are chosen by a convex adversary. 
\end{assumption}
\begin{assumption}[Bounded dual optimal variable] \label{sensitivity}
	Let $\bar{f}_T \equiv \frac{1}{T} \sum_{t=1}^T f_t$ be the average of the cost functions. Consider the following optimization problem $P_T:$
	%For any $\beta \geq 0,$ let us define:
	%\begin{eqnarray*}
		%S_T(\beta)= \min_{x \in \mathcal{X}} \{\bar{f}_T (x), ~ \textrm{s.t.} ~ g(x) \leq \beta\}.
		\[ \min_{x\in \mathcal{X}} \bar{f}_T(x)~ \textrm{s.t.}~ g(x) \leq 0. \]
		We assume that for each $T \geq 1,$ strong duality holds for the problem $P_T$ and that the limsup of a sequence of optimal dual variables is strictly bounded above by a finite constant $\lambda $. Note that the value of $\lambda$ need not be known a priori.
%	\end{eqnarray*}
	%It is easy to verify that $S_T (\cdot)$ is a non-increasing, convex function of $\beta$ \citep[Section 5.6.1]{boyd}. We assume that $\limsup_{T} |S_T'(0^+)| \leq \lambda$ for some finite $\lambda.$
\end{assumption}
\textbf{Remark:} In the case of a fixed adversary, we have $\bar{f}_T = f.$ Hence, $\lambda$ can be efficiently determined by examining the dual of a single convex problem. Note that \citep{nedic2009subgradient} made a similar assumption for the standard offline convex optimization problem with a given cost and a constraint function. However, their algorithm must know an upper bound to $\lambda$ \emph{a priori}.  The following proposition gives a sufficient condition for Assumption \ref{sensitivity}. In particular, it shows that if the unclipped constraint function satisfies Slater's condition, then Assumption \ref{sensitivity} holds \footnote{Clearly, the clipped constraint function can not satisfy Slater's condition. Hence, we study the unclipped constraint function in Proposition \ref{slater-bdd}.}. 
 \begin{proposition} \label{slater-bdd}
 	Let $\tilde{g}$ be the constraint function which can be negative and that satisfies Slater's condition, \emph{i.e.,} there exists an admissible $z \in \mathcal{X}$ s.t. $\tilde{g}(z) \leq -\epsilon$ for some $\epsilon >0. $ Then Assumption \ref{sensitivity} holds for some $\lambda = O(\frac{1}{\epsilon})$.
 \end{proposition} 
 \begin{proof}
 	It is easy to verify that any dual optimal solution to the problem with the unclipped constraint function is also a dual optimal solution to $P$ with the clipped constraint function. Let $(z^\star, \lambda)$ be an optimal primal-dual solution pair for the problem with the unclipped constraint function. Since $(z^\star, \lambda)$ is a saddle point for the Lagrangian, we can write 
 	\begin{eqnarray*}
 		\bar{f}_T(z)+\lambda \tilde{g}(z) \geq \bar{f}_T(z^\star)+ \lambda \tilde{g}(z^\star)= \bar{f}_T(z^\star), 
 	\end{eqnarray*}
 	where we have used the complementary slackness property on the RHS. Now since $\tilde{g}(z)\leq -\epsilon,$ from the above, we have 
 	\begin{eqnarray*}
 		\bar{f}_T(z) -\lambda \epsilon \geq \bar{f}_T(z^\star) ~\implies \lambda = O(\frac{1}{\epsilon}),
 	\end{eqnarray*}
 	where in the last step, we have used the fact that the range of any convex function with a bounded subgradient over a bounded domain is bounded.
 \end{proof}
 

\subsection{Cumulative violation bound under Assumptions \ref{constr_assump1}, \ref{adversary_assump2}, and \ref{sensitivity}} 
Taking  \eqref{q-bd-eqn} as our starting point, for any feasible action $x^\star \in \mathcal{X}^\star$, we have:
\begin{eqnarray} \label{master_eq2}
	Q^2(T) + V \textrm{Regret}_T(x^\star) \leq \textrm{Regret}_T'(x^\star). 
\end{eqnarray} 
%We can lower bound the regret term as follows:
In Theorem \ref{gen-cvx-bd} and \ref{str-cvx-bd}, we trivially lower bounded the regret term by a negative linear term, which resulted in a sub-optimal violation bound. Using Assumptions \ref{constr_assump1}, \ref{adversary_assump2}, and \ref{sensitivity}, we next derive a tighter lower bound to the regret by directly relating it to the queue length. We have 
\begin{eqnarray} \label{reg_lb_jensen}
	\textrm{Regret}_T(x^\star) &=& \sum_{t=1}^T f_t(x_t) - \sum_{t=1}^T f_t(x^\star) \nonumber\\
	&\stackrel{(a)}{\geq} & \sum_{t=1}^T f_t(\bar{x}_T) - \sum_{t=1}^T f_t(x^\star) \nonumber\\
	&=& T (\bar{f}_T(\bar{x}_T) - \bar{f}_T(x^\star)). 
\end{eqnarray}
where in step (a), we have used the fact that the adversary is convex. Next, from the CCV bound of COCO Meta-policy, we have 
\begin{eqnarray*}
T g(\bar{x}_T)	\stackrel{(\textrm{Jensen's ineq.})}{\leq} \sum_{t=1}^T g(x_t) \leq Q(T),
\end{eqnarray*}
\emph{i.e.,} 
\begin{eqnarray*}
	g(\bar{x}_T) \leq \frac{Q(T)}{T}.
\end{eqnarray*}
Now let $y^\star \in \mathcal{X}$ be a solution to the following optimization problem:
\begin{eqnarray}\label{opt_prob2}
	y^\star \in \arg \min_{x \in \mathcal{X}} \bar{f}_T(x), ~ \textrm{s.t.}~ g(x) \leq \frac{Q(T)}{T}.
\end{eqnarray}
Since the average action $\bar{x}_T$ is a feasible solution to the above program, we have 
\begin{eqnarray*}
	\bar{f}_T(\bar{x}_T) \geq \bar{f}_T(y^\star).
\end{eqnarray*}
Finally, choose the feasible action $x^\star$ as follows:
\begin{eqnarray}\label{opt_prob3}
	x^\star \in \arg\min_{x \in \mathcal{X}} \bar{f}_T(x), ~ \textrm{s.t.} ~ g(x) \leq 0. 
\end{eqnarray}
Hence, from Eqn.\ \eqref{reg_lb_jensen}, \eqref{opt_prob2}, and \eqref{opt_prob3}, we have 
\begin{eqnarray*}
	\textrm{Regret}_T(x^\star) \geq T(\bar{f}_T(y^\star)- \bar{f}_T(x^\star)).
\end{eqnarray*}
Since $Q(T)=o(T),$ from Theorem \ref{gen-cvx-bd}, we now make the key observation that $x^\star$ and $y^\star$ are the solutions to optimization problems \eqref{opt_prob2} and \eqref{opt_prob3} respectively, where the latter problem has been obtained from the former by perturbing the inequality constraint by a small amount for a large $T$. Hence, the difference in their objective values can be obtained by studying the sensitivity of the convex programs. Hence, using Eqn.\ (5.57) from \citet[Section 5.6.2]{boyd}, we conclude that for a sufficiently large horizon length $T,$ we have:
\begin{eqnarray*}
	\textrm{Regret}_T(x^\star) \geq T \big(-\lambda \frac{Q(T)}{T}\big) = - \lambda Q(T). 
\end{eqnarray*}
Plugging in the above bound in  \eqref{master_eq2}, we have the following relaxation of our key regret decomposition result for large enough $T:$
\begin{eqnarray} \label{reg_decomp_new}
	 Q^2(T) - V\lambda Q(T) \leq \textrm{Regret}_T'(x^\star),
\end{eqnarray}
where, as before, the $\textrm{Regret}_T'$ term on the right-hand side denotes the worst-case regret (over the admissible set $\mathcal{X}$) of the OCO sub-routine. We now have the following theorem
\begin{theorem} \label{improved_violation_bd}
	Under Assumptions \ref{constr_assump1}, \ref{adversary_assump2}, and \ref{sensitivity}, the cumulative constraint violation bounds in Theorem \ref{gen-cvx-bd} and Theorem \ref{str-cvx-bd} can be improved to $Q(T)= \mathbb{V}(T)=O((1+\lambda) \sqrt{T}),$ and $Q(T)=\mathbb{V}(T)=O((1+\lambda)\frac{\log T}{\alpha}),$ respectively. 
\end{theorem} 
\begin{proof}
	The proof proceeds similarly to Theorem \ref{gen-cvx-bd} part 1 and  Theorem \ref{str-cvx-bd} part 1, respectively, where we now take into account the additional linear term. 
	\paragraph{Case I (Convex costs):}
	Similar to Eqn.\ \eqref{main_eq}, plugging in the adaptive regret bound for convex cost functions on the RHS of Eqn.\ \eqref{reg_decomp_new}, we have 
	\begin{eqnarray*}
		Q^2(T)-V\lambda Q(T) \leq 2GDQ(T)\sqrt{T}+ 2GDV\sqrt{T},
	\end{eqnarray*}	
	where, as before, we have used the non-decreasing property of the queue-length sequence. Setting $V=\sqrt{T}$ and completing the square, we obtain the following bound for the queue-length sequence for a sufficiently large horizon length $T:$
	\begin{eqnarray*}
		Q(T) = O(\lambda\sqrt{T})+O(\sqrt{T}).
	\end{eqnarray*} 
		\paragraph{Case II (Strongly-convex costs):}
		Plugging in the adaptive regret bound \eqref{adaptive_str_cvx_bd} for strongly-convex cost functions on the RHS of Eqn.\ \eqref{reg_decomp_new}, we obtain: 
		\begin{eqnarray*}
			Q^2(T) - V\lambda Q(T) \leq \frac{VG^2\ln (Te)}{\alpha} + \frac{G^2 \ln (Te)}{\alpha V} Q^2(T),
		\end{eqnarray*}
		where, as before, we have used the non-decreasing property of the queue-length sequence. Setting $V=\frac{2G^2 \ln(Te)}{\alpha}$ as before, we have
		\begin{eqnarray*}
			Q^2(T) - 2V\lambda Q(T) \leq 2\frac{VG^2}{\alpha}\ln(Te). 
		\end{eqnarray*}
		Completing the square, we conclude that 
		\begin{eqnarray*}
			Q(T) = O(\frac{\lambda \log T}{\alpha})+ O(\frac{\log T}{\alpha}).
		\end{eqnarray*}
\end{proof}
\fi




\subsection{Proof of Theorem \ref{S-benchmark}} \label{S-benchmark-pf}
\label{ext}
\iffalse
In all the previous sections, we assumed the existence of a fixed action $x^\star \in \mathcal{X}$ that satisfies each of the online constraints on \emph{each round}. In particular, we assumed that $\mathcal{X}^\star \neq \emptyset.$
In this section, we revisit the \ocs ~problem by relaxing this assumption and only assuming that there is a feasible action $x^\star$ that satisfies the aggregate of the constraints over any consecutive $S$ rounds\footnote{This extension is meaningful only when the range of the constraint functions includes both positive and negative values. For non-negative constraints, clearly, $\mathcal{X}_S = \mathcal{X}^\star, \forall S\geq 1.$}, where the parameter $S \in [T]$ need not be known to the policy \emph{a priori}. Towards this end, we define the set of all $S$-feasible actions as below: 
\begin{eqnarray} \label{extended-benchmark}
\mathcal{X}_S =\{x^\star: \sum_{\tau \in |\mathcal{I}|} g_{\tau,i}(x^\star) \leq 0, \forall \textrm{sub-intervals}~ \mathcal{I} \subseteq [1,T], ~|\mathcal{I}| = S, \forall i \in [k]\}. 
\end{eqnarray}
We now replace Assumption \ref{feas-constr} with the following
\begin{assumption}[$S$-feasibility] \label{s-feas-assump}
	$\mathcal{X}_S \neq \emptyset.$
\end{assumption}
We now only assume that $\mathcal{X}_S \neq \emptyset$ for some $S: 1\leq S \leq T$. Clearly, Assumption \ref{s-feas-assump} is weaker than Assumption \ref{feas-constr} as $\mathcal{X}^\star \subseteq \mathcal{X}_S, \forall S \geq 1.$ 
This problem was considered earlier by \cite{georgios-cautious} in the COCO setting with a single constraint function per round. However, since the parameter $V$ used in their algorithm is restricted as $S \leq V \leq T,$ their algorithm naturally needs to know the value of $S.$ In reality, the value of the parameter $S$ is generally not available \emph{a priori} as it depends on the online constraint functions. Fortunately, our proposed meta-policy does not need to know the value of $S$ and hence, it can automatically adapt itself to the best feasible $S$.
\fi 

%\edit{argue why our result is better}.

%By extending the drift-plus-penalty methodology of \cite{neely2017online}, they proved a regret bound of $O(T^{1-\frac{\epsilon}{2}})$ and cumulative violation penalty of $O(T^{1-\frac{\epsilon}{4}})$ for $S=T^{1-\epsilon}$ \citep[Theorem 1]{georgios-cautious}. For the \texttt{OCS} problem, we show that our policy improves the latter bound to $O(T^{1-\frac{\epsilon}{2}}).$

\paragraph{Generalized regret decomposition:} Fix any $S$-feasible benchmark $x^\star \in \mathcal{X}_S,$ as given by Eqn.\ \eqref{extended-benchmark}. Then, from Eqn.\ \eqref{drift-bd}, we have 
\begin{eqnarray*}
	\Phi(\tau)- \Phi(\tau-1) &\leq& 2 \sum_{i=1}^k Q_i(\tau)g_{\tau, i}(x_\tau) \\
	&=& 2 \sum_{i=1}^k Q_i(\tau)\big(g_{\tau, i}(x_\tau)-g_{\tau, i}(x^\star)\big) + 2\sum_{i=1}^k Q_i(\tau)g_{\tau, i}(x^\star)\\
	&=& \hat{f}_\tau (x_\tau) - \hat{f}_\tau(x^\star)  +  2\sum_{i=1}^k Q_i(\tau)g_{\tau, i}(x^\star). 
\end{eqnarray*} 
 Summing up the above inequalities from $\tau = 1$ to $\tau=t,$ we have
 \begin{eqnarray} \label{new-reg-decomp}
 	\sum_{i=1}^k Q_i^2(t) =\Phi(t) \leq \textrm{Regret}'_t(x^\star) + 2 \sum_{i=1}^k\sum_{\tau=1}^t Q_i(\tau) g_{\tau, i}(x^\star),
 \end{eqnarray}
 where $\textrm{Regret}'(\cdot)$ refers to the regret of the surrogate costs as before. 
 We now bound the last term by making use of the $S$-feasibility of the action $x^\star$ as given by Eqn.\ \eqref{extended-benchmark}.
 Let us now divide the entire interval $[1,t]$ into disjoint and consecutive sub-intervals $\{\mathcal{I}_j\}_{j=1}^{\lceil t/S \rceil},$ each of length $S$ (except the last interval which could be of a smaller length). %Let $Q^\star_i(j) = \max_{\tau \in \mathcal{I}_j}Q_i(\tau)$ be the maximum queue
 Let $Q^\star_i(j)$ be the value of the variable $Q_i(\cdot)$ at the beginning of the $j$\textsuperscript{th} interval. We have
 \begin{eqnarray} \label{S-bd1}
 	\sum_{\tau=1}^t Q_i(\tau)g_{\tau, i}(x^\star) = \sum_{j=1}^{\lceil t/S \rceil} \sum_{\tau \in \mathcal{I}_j}\big(Q_i(\tau)-Q_i^\star(j)\big)g_{\tau, i}(x^\star) + \sum_{j=1}^{\lceil t/S \rceil}Q_i^\star(j) \sum_{\tau \in \mathcal{I}_j}g_{\tau, i}(x^\star) .   
 \end{eqnarray}
 %Let $g_{t,i}(x) \leq F, \forall x \in \mathcal{X}, t, i.$ 
% From the Lipschitzness assumption, we have $g_{t,i}(x) \leq GD \equiv F ~(\textrm{say}), \forall x \in \mathcal{X}, t, i.$ 
Using the boundedness assumption, let $g_{t,i}(x) \leq F, \forall x \in \mathcal{X}, t, i.$
 Using the Lipschitzness property of the queueing dynamics \eqref{q-ev2} with respect to time, we have 
 \begin{eqnarray*}
 	\max_{\tau \in \mathcal{I}_j} |Q_i(\tau)-Q_i^\star(j)| \leq F(S-1).
 \end{eqnarray*}
 Substituting the above bound into Eqn.\ \eqref{S-bd1}, we obtain 
 \begin{eqnarray} \label{new-Q-S}
 	\sum_{\tau=1}^t Q_i(\tau)g_{\tau, i}(x^\star)  \leq \big(1+\frac{t}{S}\big) F^2S(S-1) + F(S-1)(Q_i(t)+F(S-1)), 
 \end{eqnarray}
 where in the last term, we have used the $S$-feasibility of the action $x^\star$ in all intervals, except possibly the last interval. 
 %Clearly, when $S=1$, the RHS of the above bound becomes zero, and we recover Eqn.\ \eqref{q-regret-reln}. 
 Substituting the bound \eqref{new-Q-S} into Eqn.\ \eqref{new-reg-decomp}, we arrive at the following extended regret decomposition inequality:
 \begin{eqnarray} \label{gen-reg-decomp2}
 	\sum_{i=1}^k Q_i^2(t) &\leq& \textrm{Regret}'_t(x^\star) +  2kF^2S t + 2FS\sum_{i=1}^k Q_i(t) + 4F^2S^2k.
%&\leq & \textrm{Regret}'_t(x^\star) + 6kF^2S t + 2FS \sqrt{k} \sqrt{\sum_{i=1}^k Q_i^2(t)}.
 \end{eqnarray}
Eqn.\ \eqref{gen-reg-decomp2} leads to the following bound on the cumulative constraint violation.
% 
% \begin{theorem} \label{S-benchmark}
%Using the OGD policy with adaptive step-sizes given in part 1 of Theorem \ref{data-dep-regret} as a sub-routine, Algorithm \ref{ocs-policy} achieves the following CCV bound with the $S$-feasibility assumption (Assumption \ref{s-feas-assump}) for convex constraints: 
% 	 \[\max_{i=1}^k\mathbb{V}_i(T)= O(\max(\sqrt{ST},S )).\]
% \end{theorem}
% See below for the proof of the result.
%
%\paragraph{Remarks:} Recall that our proof of the $O(\sqrt{T})$ regret bound for the COCO problem with the $1$-benchmark in Theorem \ref{gen-cvx-bd} crucially uses the non-negativity of the pre-processed constraint functions. However, with $S$-feasible benchmarks, pre-processing by clipping the constraints does not work as then the positive violations can not be cancelled with a strictly feasible violation on a different round. We leave the problem of obtaining an optimal $O(\sqrt{T})$ regret bound for Algorithm \ref{g-oco-policy} for the COCO problem with the $S$-feasibility assumption as an open problem. 
%It is not clear how to  
%which is better than $O(T^{1-\epsilon/4})$ constraint violation bound obtained by \citet{georgios-cautious}.

  

%\textbf{Note:} This extension is meaningful only for the convex case. Any strongly-convex function 
%\vspace{5pt}
%\hrule 
%\textbf{Note:} \footnote{This extension is meaningful only when the range of the constraint functions includes both positive and negative values. For non-negative constraints, clearly $\mathcal{X}_S = \mathcal{X}_1 \forall S\geq 1.$}\\
%\hrule
%\input{adv_alg}
%\input{modified_analysis}
%\section{Application to a canonical network switching problem} \label{app}
As an interesting application of the machinery developed in this paper, we revisit the classic problem of packet scheduling in an $N \times N$ input-queued switch in an internet router with \emph{adversarial} arrival and service processes. This problem has been extensively studied in the networking literature in the stochastic setting; however, to the best of our knowledge, no provable result is known for the problem in the adversarial context. In the stochastic setting with independent arrivals and constant service rates, the celebrated Max-Weight policy is known to achieve the full capacity region \citep{mckeown1999achieving, tassiulas1990stability}. However, this result immediately breaks down when the arrival and service processes are decided by an adversary. In this section, we demonstrate that the proposed \ocs ~meta-policy, described in Algorithm \ref{ocs-policy}, can achieve a sublinear queue length in the adversarial setting as well.
\begin{figure}
\centering
	\includegraphics[scale=0.7]{./figures/input-q-sw}
	\caption{A $4 \times 4$ input-queued switch used in a router. Figure taken from \citet{hajek2015random}.}
	\label{ipq}
\end{figure}

\textbf{Problem description:} An $N\times N$ input-queued switch has $N$ input ports and $N$ output ports. Each of the $N$ input ports maintains a separate FIFO queue for each output port. The input and output ports are connected in the form of a bipartite network using a high-speed switch fabric (see Figure \ref{ipq} for a simplified schematic). At any round (also called \emph{slots}), each input port can be connected to at most one output port for transmitting the packets. The objective of a switching policy is to choose an input-output matching at each round to route the packets from the input queues to their destinations so that the input queue lengths grow sub-linearly with time (\emph{i.e.,} they remain \emph{rate-stable} \citep{neely2010stochastic}). Please refer to the standard references \emph{e.g.,} \citet{hajek2015random} and the original papers \citep{tassiulas1990stability, mckeown1999achieving} for a more detailed description of the input-queued switch architectures and constraints. 
\paragraph{Admissible actions and the queueing process:}
Let $\Omega$ be the set of all $N \times N$ doubly stochastic matrices (\emph{a.k.a.} the Birkhoff polytope). The set $\Omega$ is known to coincide with the convex hull of all incidence vectors corresponding to the perfect matchings of the $N\times N$ bipartite graph \citep{hajek2015random}. At each round, the policy chooses a feasible action $x(t)\equiv \big(x_{ij}(t), 1\leq i,j\leq N\big)$ from the admissible set $\Omega.$ It then randomly samples a matching $z(t)\equiv \big(z_{ij}(t) \in \{0,1\}, 1\leq i,j\leq N\big)$ using the Birkhoff-Von-Neumann decomposition, such that $\mathbb{E}z(t)=x(t)$. 
At the same time, the adversary reveals a packet arrival vector $\bm{b}(t)$ and a service rate vector $\bm{s}(t)$. The arrival and service processes could be binary-valued or could assume any non-negative integers from a bounded range. As a result, the $i$\textsuperscript{th} input queue length corresponding to the $j$\textsuperscript{th} output port evolves as follows:
\begin{eqnarray} \label{Q-ev-iqs}
	Q_{ij}(t)=\big(Q_{ij}(t-1)+b_{ij}(t)- s_{ij}(t) z_{ij}(t))^+, Q_{ij}(0)=0.
\end{eqnarray} 
\paragraph{$S$-Feasibility:}
We assume that the adversary is $S$-feasible, \emph{i.e.,} $\exists x^\star \in \mathcal{X}$ such that
\[\sum_{t \in \mathcal{I}}(b_{ij}(t)-s_{ij}(t)x^\star_{ij}) \leq 0, \forall i,j, \forall \textrm{sub-intervals}~ \mathcal{I}, |\mathcal{I}|=S.\] The fixed admissible action $x^\star$ is unknown to the switching policy. Since we do not assume Slater's condition, the queue-length bounds derived in \cite{neely2017online} do not apply here. 
\paragraph{Reduction to the \ocs ~problem:} The above scheduling problem can be straightforwardly reduced to the \ocs ~problem where we consider $k=N^2$ linear constraints, where the $(i,j)$\textsuperscript{th} constraint function is defined as 
\[ g_{t, (ij)}(x)\equiv b_{ij}(t)-s_{ij}x_{ij}, ~1\leq i,j \leq N. \] 
It follows that the auxiliary queueing variables in the \ocs ~meta-policy in Algorithm \ref{ocs-policy} evolve similarly to Eqn.\ \eqref{Q-ev-iqs}. Hence, taking expectation over the randomness of the policy and using arguments exactly similar to the proof of Theorem \ref{S-benchmark}, it follows that under Algorithm \ref{ocs-policy}, each of the $N^2$ input queues grows sublinearly as $\mathbb{E}Q_{ij}(t) = O(\max(\sqrt{St}, S)),$ where the expectation is taken over the randomness of the policy. Hence, as long as $S$ is a constant, the queues remain rate stable. To exploit the combinatorial structure of the problem, it is computationally advantageous to use the FTPL sub-routine \citep[Theorem 11]{abernethy2014online} as the base OCO policy in Algorithm \ref{ocs-policy}, which can be implemented efficiently by a maximum-weight matching oracle and leads to the same queue length bounds.  
%Since we are only interested in the stability of the queues, we set the cost functions $f_t=0, \forall t$. 
\iffalse
\paragraph{Derivation of an Online Policy:} Note that there are $N^2$ queueing processes $\{Q_{ij}(t)\}_{i,j}$, unlike in Section \ref{policy} where we defined only a single process. Nevertheless, using the quadratic Lyapunov function  
\[\Phi(t) \equiv \sum_i Q_{ij}^2(t), \]
and using similar steps, we now define a sequence of linear surrogate cost functions: 
\begin{eqnarray*}
	f_t'(x) \equiv -\sum_{i,j} Q_{ij}(t) g_{ij}(t)x_{ij} 
\end{eqnarray*}
which we feed to an OGD policy with adaptive step sizes \cite{orabona2019modern}.
\fi
\iffalse
\paragraph{Experimental Setup} We consider $N=4.$ In our simulations, we take $x^\star$ to be a specific convex combination of $k$ matchings $\{\mathcal{M}_i\}_{i=1}^k.$
\begin{eqnarray*}
	x^\star = \sum_{i=1}^k \alpha_i \mathcal{M}_i.
\end{eqnarray*}
The coefficients and the matchings are kept hidden from the algorithm. Next, we choose the service rates for the current round in some adversarial fashion. Upon choosing the service rates, we set the arrival vector as follows
\begin{eqnarray*}
	b_{ij}(t) = g_{ij}(t)x^\star_{ij}.
\end{eqnarray*}
The above choice ensures that the adversary is admissible however, the Slater's condition does not necessarily hold.  
\fi

%\iffalse
\subsection{Lower bound on regret and CCV for COCO} \label{lower_bound_section}
In this section, we prove that under Assumptions \ref{cvx}, \ref{bddness}, and \ref{feas-constr}, the regret and the CCV achieved by any online policy for the COCO over any horizon of length $T$ are both lower bounded by $\Omega(\sqrt{T}).$ Note that the CCV lower bound does not follow from the corresponding regret lower bound for the OCO problem due to the additional Assumption \ref{feas-constr}, which puts an implicit constraint on the admissible constraint functions.
\begin{proof}
Using the standard lower bound proof strategy, we define an ensemble of \textsc{OCS}  problem instances with a single linear constraint function on every round. Using a Coupon collector's argument, we then argue that any online policy must incur at least $\Omega(\log T)$ cumulative constraint violation on at least one instance of the ensemble. The detailed construction is given below:
	\paragraph{Action space $\Omega$:} The standard probability simplex $\Delta_d$ with $d$ coordinates. The time horizon $T$ is taken to be $T=\frac{1}{2}d \log d.$
	\paragraph{Constraint functions:} 
	On each round $t \in [T]$ let random variable $I_t$ be distributed uniformly on $\{1,2,\ldots,d\}$ independently of everything else. The constraint function on round $t$ is taken to be a random linear function with coefficients $g_{t,i}:=\mathds{1}(i=I_t), i \in [d].$ In other words, the $t$\textsuperscript{th} constraint is defined as $x_{I_t}\leq 0.$ Clearly, the diameter of the admissible set is bounded as $D \leq \sqrt{2}$ and the constraint functions are $1$-Lipschitz\footnote{To see the diameter bound, let $x,y \in \Delta_d.$ We have $||x-y||^2 = \sum_i (x_i-y_i)^2 \leq \sum_i |x_i-y_i| \leq \sum_i x_i + \sum_i y_i = 2.$}.
	 
%	Let $\{\epsilon_{t,i}\}_{t=1}^T, i=1,2,$ be a sequence of iid random variables each distributed uniformly on the interval $[-1,1]$. We consider an ensemble of constraint functions where the $t$\textsuperscript{th} constraint is randomly chosen to be 
%	\begin{eqnarray} \label{constr_def_lb}
%	 g_{t}(x_t) \equiv \epsilon_{t,1}x_{t,1} + \epsilon_{t,2}x_{t,2} \leq 0, ~ 1 \leq t \leq T.
%	 \end{eqnarray}
	\paragraph{Checking feasibility:} We now argue that, for a large enough horizon $T$, the above sequence of constraint functions is feasible with high probability. Using a standard result on the Coupon collector's problem \citep[Theorem 3.8]{motwani1995randomized}, we conclude that w.h.p. there exists a coordinate $I^\star \in [d]$ which does not appear in any of the constraints. We can now find a feasible action $\bm{x}^\star$ for the above $T$ constraints by choosing $x_i^\star=\mathds{1}(i=I^\star), i\in [d].$
	\paragraph{Bounding cumulative violations:} Consider any online policy which takes action $x(t) \in \Delta_d$ on round $t$ and let $\{\mathcal{F}_t\}_{t=1}^T$ denote the natural filtration. The conditional expectation and variance of the constraint violation incurred by the policy on round $t$ is given by \[\mathbb{E}[g_t(x_t)|\mathcal{F}_{t-1}]= \frac{1}{d}\sum_{i=1}^d x_{t,i}= \frac{1}{d},~~ \textrm{Var}(g_t(x_t)) < \frac{1}{d}\sum_{i}x^2_{t,i} \leq \frac{1}{d}.  \]
	%~ \mathbb{E}g_t^2(x_t)=\frac{1}{d}\sum_{i} x_{t,i}^2.\]
	Hence, for any online policy, the expected cumulative violation is given by $\mathbb{E}V_T = \mathbb{E}\sum_{t=1}^T g_t(x_t)=\frac{T}{d}$ with its variance upper bounded as 
	$\textrm{Var}(V_T) = \sum_{t=1}^T \textrm{Var}(g_t(x_t)) \leq \frac{T}{d},$ where in the last step we have used the Pythagorean formula for the zero-mean martingale sequence $\{\sum_{\tau=1}^tg_\tau(x_\tau) - \frac{t}{d}\}_{t \geq 1}$ \citep[Section 12.1, Eq. (b)]{williams1991probability}.
	%the fact that the constraints are independently selected. 
Hence, we have  
	\[ \mathbb{E}V_T^2 = \textrm{Var}(V_T)+ (\mathbb{E}V_T)^2 \leq \frac{T^2}{d^2}+\frac{T}{d}.\]
	Hence, using Paley-Zygmund inequality for the non-negative cumulative violation random variable $V_T$ \citep[Theorem 1.4.3(b)]{chandra2012borel}, we have 
	\[ \mathbb{P}\big(V_T \geq \frac{T}{2d}\big) \geq \frac{1}{4} \frac{(\mathbb{E}V_T)^2}{\mathbb{E}V_T^2}\geq \frac{1}{4} \frac{T^2/d^2}{T^2/d^2+T/d}\geq \frac{1}{8},\]
	where, in the last step, we have used the fact that $T=\frac{1}{2}d\log d$ and assumed $\log d \geq 2.$ Hence, from the previous steps, it follows that for any online algorithm, there exists a sequence of feasible constraint functions $\{g_t\}_{t=1}^T$ for which the algorithm incurs a cumulative violation of \[V_T \geq \frac{T}{2d}= \frac{1}{4}\log d = \frac{1}{4}(\log T - \log \log d + \log 2)=\Omega(\log T - \log \log T)=\Omega(\log T).\]	

\end{proof}

%\subsection*{Proof for $\Omega(\sqrt{T})$ 
%Regret and $\Omega( \sqrt{T})$ Constraint Violation bound}
\begin{theorem}\label{thm:lbcoco}
Under Assumptions \ref{cvx}, \ref{bddness}, and \ref{feas-constr}, the regret and CCV as defined in \eqref{intro-regret-def} and \eqref{intro-gen-oco-goal} with $k=1$ are both $\Omega(\sqrt{T})$.
\end{theorem}
Proof of Theorem \ref{thm:lbcoco} can be found in Appendix \ref{app:lbcoco}.

\fi


	 
 
	%	
%	In other words, we show that almost surely for a given random sequence of $T$ constraint functions, there exists a probability vector $x$ such that we have $g_t(x) \leq 0.$ We can write 
%	\begin{eqnarray*}
%		g_t(x) = \epsilon_{t,1}x_{1}+ \epsilon_{t,2} (1-x_1) = \epsilon_{t,2} + (\epsilon_{t,1}-\epsilon_{t,2}) x_1. 
%	\end{eqnarray*}
%	Now let us choose $x_1 \sim \textsf{U}(0,1).$ Hence, we have 
%	\begin{eqnarray*}
%		\mathbb{P}(\cap_{t=1}^T g_t(x) \leq 0) = \int_{0}^1 \mathbb{P}(\cap_{t=1}^T g_t(x) \leq 0|x) dx = \frac{1}{2^T}>0,
%	\end{eqnarray*}
%	where the last line follows from the symmetry of the distribution around zero.
	%Clearly, the above sequence is feasible by taking $x_1=x_2=1/2.$


\section{Experiments: Credit Card Fraud Detection} \label{expts}
%\paragraph{Online anomaly detection:}
\begin{figure}[ht]
    \centering
    \begin{minipage}{0.49\textwidth}
        \centering
        \includegraphics[scale=0.4]{./figures/ROC_plt.pdf}
        \caption{\small{ROC curve obtained by varying $\lambda$}}
        \label{fig:ROC}
    \end{minipage}
    \hfill
    \begin{minipage}{0.49\textwidth}
        \centering
        %\includegraphics[scale=0.4]{./figures/CCV_variation_Fraud_detection.png}
                \includegraphics[scale=0.4]{./figures/CCV_variation_plt.pdf}
        \caption{\small{Typical variation of the CCV with time}}
        \label{fig:ccv}
    \end{minipage}
\end{figure}
\paragraph{Classification with a highly imbalanced dataset:}
We first formulate the credit card fraud detection problem in the COCO framework. %We use Algorithm \ref{coco_alg} to train a neural network for detecting credit card fraud using a publicly available dataset. 
%We emphasize that in contrast with the standard resampling-based strategies for imbalanced classification, our policy is online. %The online operation is indispensable in the credit card fraud detection setting, where the algorithm needs to continuously learn in a dynamic environment and make real-time predictions.
Assume that we receive a sequence of $d$-dimensional feature vectors $\{z_t\}_{t \geq 1}$ and the corresponding binary labels $\{y_t\}_{t \geq 1}$ for a sequence of credit card transactions, where each transaction can either be legitimate (\texttt{label} $=0$) or fraudulent (\texttt{label} $=1$). The problem is to predict the label $\hat{y}_t$ for each transaction $z_t$ before its true label $y_t \in \{0,1\}$ is revealed. Typically, legitimate transactions outnumber fraudulent transactions by orders of magnitude. Since the goal is to detect any fraudulent transactions (even at the cost of a few false alarms), maximizing the classification accuracy alone is insufficient due to the significant class imbalance. We propose the following reformulation for this problem within the COCO framework. 
\vspace{-8pt}
\paragraph{Formulation:} Let $\hat{y}_t(z_t,x)$ be the likelihood of class $1$ for the feature $z_t,$ given by a parameterized model with parameter $x$. Hence, the log-likelihood $\mathcal{L}(t)$ of the data on round $t$ can be expressed as: 
\begin{align}
    \mathcal{L}(t) = y_t\log(\hat{y}_t(z_t, x))+ (1-y_t)\log(1 - \hat{y}_t(z_t, x)).
%    \begin{cases} 
%    \log(\hat{y}_t) & \text{if } y_t = 1 \\ 
%    \log(1 - \hat{y}_t) & \text{if } y_t = 0
%    \end{cases}
\end{align}
We train the model by maximizing the sum of log-likelihoods for legitimate transactions, subject to the constraint that all fraudulent transactions have a likelihood value close to $1$ (\emph{i.e.,} the sum of the log-likelihoods of the fraudulent transactions remains close to zero):
\begin{eqnarray} \label{prob1}
    \max_x \sum_{t=1}^T (1-y_t) \log(1-\hat{y}_t(z_t, x)),~~ \textrm{s.t.}~~\sum_{t=1}^T y_t \log(\hat{y}_t(z_t, x)) \geq 0.
\end{eqnarray}
%s.t. 
%\begin{eqnarray} \label{constr1}
%    \sum_{t=1}^T y_t \log(\hat{y}_t) \geq 0,
%\end{eqnarray}
%where the maximization is done over the parameters of the model. In other words,  %Problem \ref{prob1} can be rewritten as
%\begin{eqnarray} \label{main_prob}
%    \min \sum_{t=1}^T -(1-y_t) \log(1-\hat{y}_t) 
%\end{eqnarray}
%s.t. 
%\begin{eqnarray} \label{new-constr}
%    \sum_{t=1}^T -y_t (\log(\hat{y}_t)) \leq 0
%\end{eqnarray}
The above problem \eqref{prob1} can be immediately recognized to be an instance of COCO with the following cost and constraint functions:
\[ f_t(x) \equiv -(1-y_t) \log(1-\hat{y}_t(z_t,x)), ~~g_t(x)\equiv -y_t \log(\hat{y}_t(z_t,x)), ~t\geq 1.\]
In our experiments, we consider the common scenario in which the likelihoods are modeled by the output of a feedforward neural network. Note that the feasibility assumption (Assumption 3) is naturally satisfied as the overparameterized neural network models are known to perfectly fit the data \citep{belkin2019reconciling}. However, in this case, the functions $f_t$ and $g_t$ are generally non-convex. 
%Nevertheless, we find that Algorithm 1 performs excellently even when the convexity assumption does not hold. 
\vspace{-8pt}
\paragraph{Experiments:}

We experiment with a publicly available credit card transaction dataset \citep{dal2014learned}. This highly imbalanced dataset contains only $492$ frauds ($\sim 0.17\%$) out of $284,807$ reported transactions. 
%\paragraph{Dataset and Network Architecture:} 
Each data point has $D_{\textrm{in}}=30$ features and binary labels. We choose a simple network architecture with a single hidden layer containing $H=10$ hidden nodes and sigmoid non-linearities. Unlike previous algorithms, our algorithm is especially suitable for training neural network models as it only needs to compute the gradients (via backward pass) and evaluate the functions (via forward pass). Initially, all weights are independently sampled from a standard  normal distribution. The network is then trained  using Algorithm \ref{coco_alg} on a quad-core CPU with 8 GB RAM. The projection operation corresponds to $L_2$-normalization. The code has been  publicly released \citep{coco-code}. 


%
%\begin{figure}
%    \centering
%    \includegraphics[width=0.7\linewidth]{./figures/ROC_plt.png}
%    \caption{Plot of the ROC curve}
%    \label{fig:ROC}
%\end{figure}
%
%\begin{figure}
%    \centering
%    \includegraphics[width=0.7\linewidth]{./figures/CCV_variation_Fraud_detection.png}
%    \caption{A plot of typical variation of CCV with time}
%    \label{fig:ccv}
%\end{figure}
\vspace{-8pt}
\paragraph{Results:}

Given the severe class imbalance, the area under the ROC curve, which plots the True Positive Rate (TPR) against the False Positive Rate (FPR), is an appropriate metric to evaluate any prediction algorithm for this problem. By varying the hyperparameter $\lambda$, we obtain the ROC curve shown in Figure \ref{fig:ROC}. The area under the ROC curve is computed to be $\approx 0.92$, which is an excellent score (cf. ideal score $=1.0$), notwithstanding the fact that, unlike the standard resampling-based techniques, the algorithm learns in an entirely online fashion starting from random initialization. Figure \ref{fig:ccv} illustrates the expected sublinear variation of CCV during one of the algorithm runs.
%\paragraph{References}
%
%[1] Dal Pozzolo, Andrea, Olivier Caelen, Yann-Ael Le Borgne, Serge Waterschoot, and Gianluca Bontempi. ``Learned lessons in credit card fraud detection from a practitioner perspective." Expert systems with applications 41, no. 10 (2014): 4915-4928.
\vspace{-0.15in}
\section{Conclusion} \label{conclusion}
\vspace{-0.15in}
\iffalse
An interesting open question is whether the assumption of the feasibility of the constraints on every slot can be suitably relaxed and still ensure sublinear regret and constraint violation penalties. Specifically, it would be interesting to extend our results to the $K$-benchmark case \citep{georgios-cautious} where the cumulative constraints hold for any consecutive $K$ intervals where $K$ is a sublinear function of the time-horizon $T$, \emph{i.e.,}
\begin{eqnarray*}
	\sum_{\tau \in \mathcal{I}}g_\tau (x^\star) \leq 0
\end{eqnarray*}
for all intervals $|\mathcal{I}| =K \leq \alpha(T)$ where $\alpha(\cdot)$ is a sublinear function. Secondly, proving joint lower bounds for static regret and constraint violation penalty would be interesting. Throughout this paper, we use the quadratic potential function to design our online policy. However, our regret decomposition result is general and could be used with any reasonable potential function. It would be interesting to see if improved performance guarantees can be established by choosing a different potential function.
\fi
In this paper, we proposed efficient online policies for the COCO problem with optimal performance bounds. We also derived sublinear CCV bounds for the OCS problem under a set of relaxed assumptions. Our analysis is streamlined, leveraging Lyapunov theory and adaptive regret bounds for the standard OCO problem. 
In the future, exploring dynamic regret bounds and a bandit extension of the COCO problem would be interesting.
%\iffalse
\section{Acknowledgement}
This work was supported by the Department of Atomic Energy, Government of India, under project no. RTI4001 and by a Google India faculty research award. The first author was also partially supported by a US-India NSF-DST collaborative grant coordinated by IDEAS-Technology Innovation Hub (TIH) at the Indian Statistical Institute, Kolkata. The authors gratefully acknowledge comments from the anonymous reviewers, which substantially improved the quality of the presentation. 
%
%\fi









%\bibliographystyle{plain}
%\clearpage
\bibliography{OCO.bib} 
\bibliographystyle{unsrtnat}
%\clearpage


%%%%%%%%%%%%%%%%%%%%%%%%%%%%%%%%%%%%%%%%%%%%%%%%%%%%%%%%%%%%

\appendix
\label{Appendix}
    
\section{Experimental Setup Details}
    \paragraph{Computing infrastructure}
    All experiments were performed using PyTorch 2.1.0 with NVIDIA GeForce RTX 3090 24GB GPU. The ROUGE scores are calculated by a third-party Python library called rouge, using version 1.0.1.
    
    \paragraph{Description of datasets}
        To have an impartial result, we experiment on four widely used benchmark datasets, Reddit, CNN/DailyMail, WikiHow, and PubMed.
        Table \ref{tb:dataset} presents an overview of these datasets.
        Specifically,
        % Reddit
        The Reddit dataset contains 120K posts of informal stories from the online discussion forum Reddit, more specifically the TIFU sub-reddit from 2013-Jan to 2018-Mar.
        We use the TIFU-long subset (using TLDR as summaries) in this work.
        % CNN/DM
        The CNN/DailyMail dataset consists of news articles and their human-written abstracts collected from CNN and Daily Mail websites. 
        Both publishers supplement their articles with bullet point summaries, which are considered as ground truth summaries.
        % WikiHow
        The WikiHow dataset is a comprehensive collection of instructional articles obtained from the WikiHow website. 
        The articles are not written by professional journalists but by ordinary individuals, detailing the steps required to perform a specific task. 
        % PubMed
        The PubMed dataset is a well-known dataset for long documents that focus on biomedical research. 
        The dataset includes large scientific articles and abstracts that have been labeled by humans. \\
        
        
        \begin{table*}[htbp]
        % 放附录
            \centering
            \begin{tabular}{ccccc}
            \hline
            Dataset & Domain & \# Docs (train/val/test) & Doc./Sum. Avg. Length & \# Ext\\
            \hline
            Reddit & Social Media &(41,675/645/645) & 482.2/28.0 & 2\\
            CNN/DM & News &(287,227/13,368/11,490) & 679.3/48.3 & 3\\
            WikiHow & Knowledge instruction &(157,252/5,599/5,577) & 502.6/45.6  &  4\\
            PubMed & Medicine &(83,233/4,946/5,025) & 3,049.0/202.4 & 6\\
            \hline
            \end{tabular}
            
            \caption{Comparison of summarization datasets in different domains: the size of dataset, the average length of source and target. \# Ext denotes the number of sentences that should be extracted from source data.}
            \label{tb:dataset}
        \end{table*}


    \paragraph{Baselines}
        In this paper, we adopt the following state-of-the-art algorithms as our baselines, which are designed for the problems of heterogeneity and data scarcity:
    \begin{itemize}
        \item Separate: To observe the effect of server aggregation, we also adopt the separate training setting, where each client updates their model by only local training.
        
        \item FedAvg \cite{mcmahan2017communication}: the original FL algorithm which does not consider performance degeneration in data scarcity.
        
        
        \item FedProx \cite{FedProx}: the representative FL algorithm for tackling the statistical heterogeneous problem by solving an optimization object with regularization constrain. 
        We compare the performance of FedProx to verify the effectiveness of FedSum in tackling heterogeneous problems.  
        
        \item Scaffold \cite{SCAFFOLD}: the FL algorithm for the statistical heterogeneous problem by constructing regularization term with aggregated variate.
        We set Scaffold in our comparative group to observe the impact of statistical heterogeneity.

        % 消除由于数据异质性导致的目标不一致性
        \item FedNova \cite{FedNova}: the FL algorithm that introduces a normalizing weight to eliminate the objective inconsistency. We compare FedNova to verify our effectiveness in dealing with data heterogeneity issues.

        \item FedProto \cite{FedProto}: the typical FL algorithm that introduces the concept of prototype learning. We analyze FedProto to evaluate the effectiveness of our approach in introducing prototypes and learning knowledge from multi-source.

        \item FedDC \cite{FedDC}: the FL algorithm that interleaving model aggregation and permutation steps. We included FedDC in our comparative group to demonstrate the effectiveness of our approach in data scarcity scenarios.


        To better observe the difference between centralized and federated paradigms, we also simulate centralized and separated methods.
        The centralized can access all training samples, and the separated takes the average of performance over FL clients, which perform supervised training as shown in Fig.\ref{fig:ExtSum}.
        \begin{figure}[phtb]
            \centering
            \includegraphics[width=0.45\columnwidth]{latex/fig/ExtSum.png}
            \caption{The standard training process of ExtSum task.}
            \label{fig:ExtSum}
        \end{figure}
    \end{itemize}
        
    \noindent {\bf Measurement.}
    To measure the quality of the output summary, we apply the commonly used overlap metric ROUGE \cite{lin-2004-rouge}. ROUGE evaluates the quality of exacted summaries by comparing the overlap between the reference and generated text at word level. In this paper, we report the ROUGE-N and ROUGE-L for a more comprehensive evaluation. ROUGE-N evaluates the quality through n-gram overlap, while ROUGE-L focuses on the longest common subsequence. To further capture lexical similarity, we report the ROUGE-Recall and ROUGE-F1 of ROUGE-1, ROUGE-2 and ROUGE-L.
    \begin{itemize}    
        \item ROUGE-Recall measures the proportion of matching n-grams in the generated text to those in the reference text. It focuses on evaluating whether the exacted summaries cover the significant information in the reference text. The formula is shown below:
        % \begin{equation}
        %     ROUGE\text{-}R = \frac{\sum_{s \in S}\sum_{gram_n \in s} \min(Count(gram_n, s), Count(gram_n, C))}{\sum_{s \in S}\sum_{gram_n \in s} Count(gram_n, s)}
        % \end{equation}
        \begin{equation}
            \text{ROUGE-Recall}=\frac{|\text{n-gram}_{\text{match}}|}{|\text{n-gram}_{\text{ref}}|}
            \label{eq: ROUGE-R}
        \end{equation}
        where $|\text{n-gram}_{\text{match}}|$ means the frequency of co-occurring n-grams between generated and reference text, $|\text{n-gram}_{\text{ref}}|$ means the number of total \text{n-grams} in the reference text.
        \item ROUGE-F1 combines ROUGE-Recall and ROUGE-Precision, comprehensively considers both coverage and accuracy of the generated text, as shown below:
        \begin{equation}
            \text{ROUGE-F1}=2\times\frac{\text{ROUGE-Recall}\times\text{ROUGE-Precision}}{\text{ROUGE-Recall}+\text{ROUGE-Precision}}
        \end{equation}
        where ROUGE-Precision is formulated as:
        \begin{equation}
            \text{ROUGE-Precision}=\frac{|\text{n-gram}_{\text{match}}|}{|\text{n-gram}_{\text{gen}}|}
        \end{equation}
        the meaning of $|\text{n-gram}_{\text{match}}|$ and $|\text{n-gram}_{\text{gen}}|$ is similar to Equation~\ref{eq: ROUGE-R}.
        
    \end{itemize}
    

    \noindent {\bf Hyperparameters.}
    We follow the hyperparameter setting proposed in several works \cite{sun-etal-2019-utilizing, Cohan_2019, zhong-etal-2020-extractive,liu2021simcls,lin2022fednlp}.
    We set communication round $T$ as 5, local epoch $E$ as 2, client number $K$ as 20, active ratio $\alpha$ as 0.2, and local batch size $B$ as 32.
    The hyperparameter of regularization term $\mu$ in FedProx \cite{FedProx} and Scaffold \cite{SCAFFOLD} are tuned in \{0.01, 0.1, 0.5, 1, 2\}, and we set $\mu=1$ according to the optimal result.
    Specifically, we set $\lambda$ as $0.5$. 
    If need to increase the attention to leading bias, then $\lambda$ can be smaller, and vice versa.
    The dropout rate of Bernoulli perturbation $\gamma$ is 0.3.
    Besides, we study the effect of different learning rates on performance.
    The results are presented in the following content and we set $\eta$ as 5e-3 for all methods according to the result.
    
\newpage

\section{Additional Numerical Experiments}
    \paragraph{Result Summary}
        To observe the effects of federated learning, we introduced a centralized setting, where all data is centralized storage, and a single global model is trained on a central server. In this setting, we temporarily set aside data privacy concerns and focus on performance evaluation. As shown in Table~\ref{Table: Centralized Method}, the global model trained by the central server faces a risk of overfitting with the number of epochs increases, leading to a decline in performance. We can observe that when the number of epochs increases from 20 to 30, the Mean ROUGE-R-1 score for CNNDM decreases 5.63.
        \begin{table*}[htbp]
  \centering
  \resizebox{1\textwidth}{!}{   
  \begin{tabular}{|c|cc|cc|cc|}
\hline
\textbf{Centralized} &
  \multicolumn{2}{c|}{\textbf{Epoch=10}} &
  \multicolumn{2}{c|}{\textbf{Epoch=20}} &
  \multicolumn{2}{c|}{\textbf{Epoch=30}} \\ \hline
\textbf{CNNDM} &
  \multicolumn{1}{c|}{\textbf{Mean}} &
  \textbf{Var.} &
  \multicolumn{1}{c|}{\textbf{Mean}} &
  \textbf{Var.} &
  \multicolumn{1}{c|}{\textbf{Mean}} &
  \textbf{Var.} \\ \hline
R-F(1/2/L) &
  \multicolumn{1}{c|}{33.92/12.79/26.77} &
  25.74/15.02/21.13 &
  \multicolumn{1}{c|}{33.94/12.56/26.52} &
  10.32/7.57/8.17 &
  \multicolumn{1}{c|}{29.4/8.86/22.54} &
  0.05/0.05/0.05 \\ \hline
R-R(1/2/L) &
  \multicolumn{1}{c|}{41.74/16.59/33.04} &
  54.04/30.39/42.68 &
  \multicolumn{1}{c|}{41.37/16.05/32.44} &
  18.61/13.67/14.56 &
  \multicolumn{1}{c|}{35.74/11.27/27.51} &
  0.13/0.09/0.11 \\ \hline
\textbf{PubMed} &
  \multicolumn{1}{c|}{\textbf{Mean}} &
  \textbf{Var.} &
  \multicolumn{1}{c|}{\textbf{Mean}} &
  \textbf{Var.} &
  \multicolumn{1}{c|}{\textbf{Mean}} &
  \textbf{Var.} \\ \hline
R-F(1/2/L) &
  \multicolumn{1}{c|}{33.35/12.91/30.11} &
  0.0/0.0/0.0 &
  \multicolumn{1}{c|}{31.16/10.85/27.99} &
  0.01/0.01/0.01 &
  \multicolumn{1}{c|}{31.67/11.28/28.46} &
  0.84/0.87/0.82 \\ \hline
R-R(1/2/L) &
  \multicolumn{1}{c|}{35.29/13.43/31.84} &
  0.0/0.0/0.0 &
  \multicolumn{1}{c|}{32.86/11.2/29.48} &
  0.01/0.02/0.01 &
  \multicolumn{1}{c|}{33.41/11.65/29.99} &
  1.08/1.06/1.04 \\ \hline
\textbf{WikiHow} &
  \multicolumn{1}{c|}{\textbf{Mean}} &
  \textbf{Var.} &
  \multicolumn{1}{c|}{\textbf{Mean}} &
  \textbf{Var.} &
  \multicolumn{1}{c|}{\textbf{Mean}} &
  \textbf{Var.} \\ \hline
R-F(1/2/L) &
  \multicolumn{1}{c|}{23.09/5.24/21.44} &
  0.39/0.11/0.35 &
  \multicolumn{1}{c|}{23.03/5.16/21.37} &
  0.03/0.01/0.03 &
  \multicolumn{1}{c|}{22.8/5.09/21.17} &
  0.32/0.07/0.28 \\ \hline
R-R(1/2/L) &
  \multicolumn{1}{c|}{34.2/8.62/31.9} &
  0.68/0.29/0.6 &
  \multicolumn{1}{c|}{33.87/8.44/31.59} &
  0.04/0.01/0.04 &
  \multicolumn{1}{c|}{33.56/8.29/31.32} &
  0.38/0.15/0.33 \\ \hline
\textbf{Reddit} &
  \multicolumn{1}{c|}{\textbf{Mean}} &
  \textbf{Var.} &
  \multicolumn{1}{c|}{\textbf{Mean}} &
  \textbf{Var.} &
  \multicolumn{1}{c|}{\textbf{Mean}} &
  \textbf{Var.} \\ \hline
R-F(1/2/L) &
  \multicolumn{1}{c|}{17.9/3.2/15.16} &
  0.0/0.0/0.0 &
  \multicolumn{1}{c|}{17.9/3.21/15.18} &
  0.0/0.0/0.01 &
  \multicolumn{1}{c|}{17.97/3.24/15.22} &
  0.02/0.0/0.02 \\ \hline
R-R(1/2/L) &
  \multicolumn{1}{c|}{41.68/9.12/35.92} &
  0.02/0.0/0.02 &
  \multicolumn{1}{c|}{41.72/9.18/35.99} &
  0.03/0.02/0.06 &
  \multicolumn{1}{c|}{41.91/9.26/36.1} &
  0.2/0.06/0.17 \\ \hline
\end{tabular}
  }
  \caption{Experimental results with centralized setting on four datasets. The mean and variance of three optimal results for each method are presented.}
  \label{Table: Centralized Method}
\end{table*}
        

    
        To evaluate the performance of each algorithm on four different datasets, we construct several sets of experiments with different levels of heterogeneity.           
        Since the performance fluctuation caused by various random sampling in FL is inevitable, we report the mean and variance of the ROUGE-Recall scores about different algorithms over 3 repeated to avoid the unfair comparison caused by random factors in practice and hyperparameter settings.
        Specifically, for a fair comparison, we pick the most optimal performance (with better ROUGE-1 values) from testing records in each communication round for each algorithm.
        As shown in experimental results (see Table \ref{Table: CNNDM full result summary}-\ref{Table: Reddit full result summary}), FedSum outperforms other baseline methods in most cases and shows a satisfactory level of ROUGE-Recall in most cases, under different heterogeneity scenarios. 
        These results support the effectiveness of our proposed method.

        
        \begin{table*}[htpb]
            \centering
            \resizebox{1\textwidth}{!}{   
            \begin{tabular}{|c|c|cc|cc|cc|cc|cc|}
            \hline
            \textbf{CNNDM} &
              \multirow{2}{*}{\textbf{Method}} &
              \multicolumn{2}{c|}{\textbf{$Dir=0.1$}} &
              \multicolumn{2}{c|}{\textbf{$Dir=0.5$}} &
              \multicolumn{2}{c|}{\textbf{$Dir=1$}} &
              \multicolumn{2}{c|}{\textbf{$Dir=8$}} &
              \multicolumn{2}{c|}{\textbf{$Dir=+\infty$}} \\ \cline{1-1} \cline{3-12} 
            \textbf{Metric} &
               &
              \multicolumn{1}{c|}{\textbf{Mean}} &
              \textbf{Var.} &
              \multicolumn{1}{c|}{\textbf{Mean}} &
              \textbf{Var.} &
              \multicolumn{1}{c|}{\textbf{Mean}} &
              \textbf{Var.} &
              \multicolumn{1}{c|}{\textbf{Mean}} &
              \textbf{Var.} &
              \multicolumn{1}{c|}{\textbf{Mean}} &
              \textbf{Var.} \\ \hline
            \multirow{8}{*}{\textbf{R-F(1/2/L)}} &
              Separate &
              \multicolumn{1}{c|}{32.21/11.52/25.38} &
              0.88/0.74/0.87 &
              \multicolumn{1}{c|}{33.38/12.33/26.47} &
              1.03/0.98/1.16 &
              \multicolumn{1}{c|}{31.88/11.21/25.07} &
              1.06/0.89/1.02 &
              \multicolumn{1}{c|}{30.72/10.17/23.85} &
              2.13/1.72/1.98 &
              \multicolumn{1}{c|}{31.59/10.87/24.72} &
              0.45/0.37/0.43 \\ \cline{2-12} 
             &
              FedAvg &
              \multicolumn{1}{c|}{34.11/12.87/27.17} &
              0.88/0.62/0.84 &
              \multicolumn{1}{c|}{33.08/12.22/26.19} &
              2.27/1.8/2.22 &
              \multicolumn{1}{c|}{33.39/12.39/26.48} &
              0.53/0.42/0.48 &
              \multicolumn{1}{c|}{34.08/13.04/27.26} &
              0.18/0.13/0.18 &
              \multicolumn{1}{c|}{\textbf{33.3/12.32/26.44}} &
              1.22/1.0/1.24 \\ \cline{2-12} 
             &
              FedProx &
              \multicolumn{1}{c|}{35.32/13.85/28.44} &
              0.55/0.47/0.59 &
              \multicolumn{1}{c|}{31.27/10.37/24.34} &
              2.28/2.05/2.27 &
              \multicolumn{1}{c|}{33.77/12.56/26.85} &
              1.48/1.26/1.52 &
              \multicolumn{1}{c|}{34.25/13.05/27.34} &
              0.35/0.21/0.31 &
              \multicolumn{1}{c|}{30.71/10.3/23.9} &
              2.2/1.71/2.0 \\ \cline{2-12} 
             &
              Scaffold &
              \multicolumn{1}{c|}{30.96/10.33/24.13} &
              1.14/0.66/0.89 &
              \multicolumn{1}{c|}{32.29/11.57/25.43} &
              1.01/0.81/1.01 &
              \multicolumn{1}{c|}{32.02/11.37/25.25} &
              2.15/1.67/2.07 &
              \multicolumn{1}{c|}{33.58/12.65/26.79} &
              0.45/0.34/0.45 &
              \multicolumn{1}{c|}{32.97/12.06/26.1} &
              1.17/0.86/1.11 \\ \cline{2-12} 
             &
              FedDC &
              \multicolumn{1}{c|}{32.91/11.87/25.98} &
              2.43/2.13/2.39 &
              \multicolumn{1}{c|}{31.05/10.32/24.18} &
              1.96/1.74/1.92 &
              \multicolumn{1}{c|}{34.13/12.98/27.17} &
              0.22/0.22/0.2 &
              \multicolumn{1}{c|}{\textbf{36.01/14.38/29.05}} &
              0.32/0.24/0.33 &
              \multicolumn{1}{c|}{33.1/12.14/26.21} &
              1.52/1.3/1.6 \\ \cline{2-12} 
             &
              FedNova &
              \multicolumn{1}{c|}{33.39/12.47/26.48} &
              0.57/0.45/0.49 &
              \multicolumn{1}{c|}{31.53/10.44/24.44} &
              1.5/1.26/1.39 &
              \multicolumn{1}{c|}{32.91/12.07/26.01} &
              0.13/0.13/0.13 &
              \multicolumn{1}{c|}{33.48/12.53/26.57} &
              \textbf{0.11/0.09/0.09} &
              \multicolumn{1}{c|}{33.36/12.42/26.45} &
              \textbf{0.03/0.02/0.03} \\ \cline{2-12} 
             &
              FedProto &
              \multicolumn{1}{c|}{31.04/10.54/24.18} &
              \textbf{0.09/0.07/0.08} &
              \multicolumn{1}{c|}{31.09/10.57/24.21} &
              \textbf{0.1/0.1/0.1} &
              \multicolumn{1}{c|}{30.96/10.45/24.09} &
              \textbf{0.13/0.11/0.14} &
              \multicolumn{1}{c|}{31.08/10.58/24.24} &
              0.07/0.04/0.07 &
              \multicolumn{1}{c|}{31.23/10.68/24.35} &
              0.11/0.08/0.11 \\ \cline{2-12} 
             &
              \textbf{FedSum} &
              \multicolumn{1}{c|}{\textbf{35.71/14.18/28.69}} &
              0.65/0.5/0.5 &
              \multicolumn{1}{c|}{\textbf{33.8/12.65/26.89}} &
              1.44/1.07/1.38 &
              \multicolumn{1}{c|}{\textbf{35.33/13.86/28.4}} &
              0.38/0.18/0.3 &
              \multicolumn{1}{c|}{35.13/13.56/28.13} &
              0.57/0.53/0.6 &
              \multicolumn{1}{c|}{32.73/11.59/25.77} &
              2.0/1.82/2.04 \\ \hline
            \multirow{8}{*}{\textbf{R-R(1/2/L)}} &
              Separate &
              \multicolumn{1}{c|}{39.7/14.8/31.35} &
              1.36/1.06/1.29 &
              \multicolumn{1}{c|}{41.56/16.05/33.02} &
              1.63/1.41/1.71 &
              \multicolumn{1}{c|}{39.34/14.43/31.0} &
              1.57/1.26/1.46 &
              \multicolumn{1}{c|}{37.53/12.99/29.23} &
              3.14/2.4/2.84 &
              \multicolumn{1}{c|}{38.77/13.92/30.41} &
              0.57/0.47/0.52 \\ \cline{2-12} 
             &
              FedAvg &
              \multicolumn{1}{c|}{42.8/16.92/34.15} &
              1.42/0.99/1.3 &
              \multicolumn{1}{c|}{40.81/15.75/32.39} &
              3.41/2.56/3.21 &
              \multicolumn{1}{c|}{41.62/16.15/33.07} &
              0.76/0.6/0.67 &
              \multicolumn{1}{c|}{42.66/17.07/34.17} &
              0.27/0.19/0.25 &
              \multicolumn{1}{c|}{\textbf{41.46/16.05/32.99}} &
              1.9/1.47/1.83 \\ \cline{2-12} 
             &
              FedProx &
              \multicolumn{1}{c|}{44.64/18.38/35.99} &
              0.84/0.69/0.84 &
              \multicolumn{1}{c|}{38.36/13.33/29.96} &
              3.47/2.9/3.3 &
              \multicolumn{1}{c|}{42.3/16.5/33.7} &
              2.35/1.87/2.27 &
              \multicolumn{1}{c|}{42.95/17.13/34.35} &
              0.62/0.34/0.5 &
              \multicolumn{1}{c|}{37.26/13.0/29.09} &
              3.24/2.4/2.88 \\ \cline{2-12} 
             &
              Scaffold &
              \multicolumn{1}{c|}{37.76/13.13/29.5} &
              1.56/0.93/1.23 &
              \multicolumn{1}{c|}{39.76/14.85/31.4} &
              1.69/1.22/1.59 &
              \multicolumn{1}{c|}{39.47/14.64/31.21} &
              3.42/2.46/3.13 &
              \multicolumn{1}{c|}{41.93/16.5/33.51} &
              0.72/0.52/0.69 &
              \multicolumn{1}{c|}{40.96/15.65/32.49} &
              1.89/1.33/1.72 \\ \cline{2-12} 
             &
              FedDC &
              \multicolumn{1}{c|}{40.82/15.43/32.29} &
              3.6/2.99/3.4 &
              \multicolumn{1}{c|}{38.17/13.3/29.82} &
              2.83/2.37/2.67 &
              \multicolumn{1}{c|}{42.61/16.94/33.99} &
              0.16/0.23/0.15 &
              \multicolumn{1}{c|}{\textbf{45.63/19.17/36.86}} &
              0.59/0.42/0.57 &
              \multicolumn{1}{c|}{40.98/15.71/32.52} &
              2.53/1.96/2.47 \\ \cline{2-12} 
             &
              FedNova &
              \multicolumn{1}{c|}{41.23/16.07/32.77} &
              0.68/0.58/0.6 &
              \multicolumn{1}{c|}{38.48/13.34/29.92} &
              2.05/1.71/1.86 &
              \multicolumn{1}{c|}{40.57/15.53/32.13} &
              \textbf{0.2/0.18/0.2} &
              \multicolumn{1}{c|}{41.41/16.19/32.93} &
              \textbf{0.12/0.11/0.1} &
              \multicolumn{1}{c|}{41.2/16.03/32.74} &
              \textbf{0.05/0.03/0.04} \\ \cline{2-12} 
             &
              FedProto &
              \multicolumn{1}{c|}{37.72/13.34/29.47} &
              \textbf{0.13/0.1/0.12} &
              \multicolumn{1}{c|}{37.8/13.38/29.53} &
              \textbf{0.15/0.14/0.15} &
              \multicolumn{1}{c|}{37.62/13.22/29.37} &
              0.22/0.17/0.22 &
              \multicolumn{1}{c|}{37.84/13.42/29.59} &
              0.13/0.08/0.13 &
              \multicolumn{1}{c|}{38.01/13.53/29.72} &
              0.18/0.13/0.17 \\ \cline{2-12} 
             &
              \textbf{FedSum} &
              \multicolumn{1}{c|}{\textbf{44.98/18.74/36.2}} &
              0.74/0.61/0.56 &
              \multicolumn{1}{c|}{\textbf{42.3/16.59/33.71}} &
              2.34/1.66/2.14 &
              \multicolumn{1}{c|}{\textbf{44.68/18.4/35.96}} &
              0.76/0.4/0.61 &
              \multicolumn{1}{c|}{44.42/18.01/35.62} &
              0.96/0.78/0.93 &
              \multicolumn{1}{c|}{40.49/15.02/31.96} &
              3.23/2.66/3.08 \\ \hline
            \end{tabular}
            }
            \caption{Experimental results with different heterogeneous settings on CNNDM.
            The mean and variance of three optimal results for each method are presented.
            }
            \label{Table: CNNDM full result summary}
            \end{table*}

        \begin{table*}[htpb]
            \centering
            \resizebox{1\textwidth}{!}{   
            \begin{tabular}{|c|c|cc|cc|cc|cc|cc|}
            \hline
            \textbf{PubMed} &
              \multirow{2}{*}{\textbf{Method}} &
              \multicolumn{2}{c|}{\textbf{$Dir=0.1$}} &
              \multicolumn{2}{c|}{\textbf{$Dir=0.5$}} &
              \multicolumn{2}{c|}{\textbf{$Dir=1$}} &
              \multicolumn{2}{c|}{\textbf{$Dir=8$}} &
              \multicolumn{2}{c|}{\textbf{$Dir=+\infty$}} \\ \cline{1-1} \cline{3-12} 
            \textbf{Metric} &
               &
              \multicolumn{1}{c|}{\textbf{Mean}} &
              \textbf{Var.} &
              \multicolumn{1}{c|}{\textbf{Mean}} &
              \textbf{Var.} &
              \multicolumn{1}{c|}{\textbf{Mean}} &
              \textbf{Var.} &
              \multicolumn{1}{c|}{\textbf{Mean}} &
              \textbf{Var.} &
              \multicolumn{1}{c|}{\textbf{Mean}} &
              \textbf{Var.} \\ \hline
            \multirow{8}{*}{\textbf{R-F(1/2/L)}} &
              Separate &
              \multicolumn{1}{c|}{31.01/10.76/27.8} &
              0.26/0.31/0.27 &
              \multicolumn{1}{c|}{30.67/10.28/27.45} &
              0.04/0.07/0.04 &
              \multicolumn{1}{c|}{30.67/10.38/27.47} &
              0.04/0.04/0.04 &
              \multicolumn{1}{c|}{31.25/10.91/28.04} &
              0.07/0.07/0.06 &
              \multicolumn{1}{c|}{31.16/10.9/27.96} &
              0.28/0.25/0.27 \\ \cline{2-12} 
             &
              FedAvg &
              \multicolumn{1}{c|}{31.1/10.81/27.9} &
              0.15/0.13/0.13 &
              \multicolumn{1}{c|}{31.18/10.92/27.97} &
              0.17/0.16/0.16 &
              \multicolumn{1}{c|}{31.22/10.96/28.01} &
              0.21/0.2/0.19 &
              \multicolumn{1}{c|}{31.1/10.81/27.87} &
              0.21/0.19/0.19 &
              \multicolumn{1}{c|}{30.95/10.68/27.75} &
              0.18/0.16/0.17 \\ \cline{2-12} 
             &
              FedProx &
              \multicolumn{1}{c|}{31.29/11.0/28.07} &
              0.09/0.09/0.08 &
              \multicolumn{1}{c|}{31.0/10.77/27.81} &
              0.13/0.11/0.12 &
              \multicolumn{1}{c|}{31.13/10.87/27.94} &
              0.16/0.13/0.16 &
              \multicolumn{1}{c|}{31.18/10.9/27.98} &
              0.05/0.03/0.05 &
              \multicolumn{1}{c|}{31.0/10.74/27.8} &
              0.23/0.22/0.22 \\ \cline{2-12} 
             &
              Scaffold &
              \multicolumn{1}{c|}{31.08/10.79/27.87} &
              0.07/0.09/0.08 &
              \multicolumn{1}{c|}{30.91/10.64/27.71} &
              0.13/0.15/0.13 &
              \multicolumn{1}{c|}{31.04/10.76/27.83} &
              0.21/0.19/0.21 &
              \multicolumn{1}{c|}{30.92/10.63/27.72} &
              0.27/0.27/0.27 &
              \multicolumn{1}{c|}{30.99/10.75/27.79} &
              0.24/0.23/0.24 \\ \cline{2-12} 
             &
              FedDC &
              \multicolumn{1}{c|}{31.43/11.13/28.22} &
              0.0/0.01/0.0 &
              \multicolumn{1}{c|}{31.26/11.02/28.06} &
              0.18/0.16/0.18 &
              \multicolumn{1}{c|}{31.2/10.96/28.01} &
              0.23/0.19/0.22 &
              \multicolumn{1}{c|}{\textbf{31.4/11.1/28.19}} &
              0.04/0.06/0.05 &
              \multicolumn{1}{c|}{31.29/11.06/28.09} &
              0.09/0.02/0.07 \\ \cline{2-12} 
             &
              FedNova &
              \multicolumn{1}{c|}{30.76/10.51/27.55} &
              0.01/0.01/0.01 &
              \multicolumn{1}{c|}{30.95/10.62/27.74} &
              0.26/0.34/0.27 &
              \multicolumn{1}{c|}{30.8/10.51/27.59} &
              \textbf{0.02/0.0/0.02} &
              \multicolumn{1}{c|}{30.77/10.55/27.58} &
              \textbf{0.01/0.01/0.02} &
              \multicolumn{1}{c|}{30.78/10.55/27.59} &
              \textbf{0.01/0.01/0.01} \\ \cline{2-12} 
             &
              FedProto &
              \multicolumn{1}{c|}{30.72/10.41/27.51} &
              \textbf{0.02/0.02/0.02} &
              \multicolumn{1}{c|}{30.73/10.42/27.53} &
              \textbf{0.03/0.05/0.02} &
              \multicolumn{1}{c|}{30.7/10.39/27.5} &
              0.02/0.02/0.01 &
              \multicolumn{1}{c|}{30.73/10.42/27.53} &
              0.03/0.02/0.02 &
              \multicolumn{1}{c|}{30.74/10.44/27.54} &
              0.06/0.06/0.06 \\ \cline{2-12} 
             &
              \textbf{FedSum} &
              \multicolumn{1}{c|}{\textbf{31.48/11.2/28.26}} &
              0.02/0.02/0.03 &
              \multicolumn{1}{c|}{\textbf{31.53/11.25/28.32}} &
              0.4/0.33/0.37 &
              \multicolumn{1}{c|}{\textbf{31.42/11.14/28.22}} &
              0.38/0.31/0.37 &
              \multicolumn{1}{c|}{31.23/10.98/28.04} &
              0.02/0.06/0.04 &
              \multicolumn{1}{c|}{\textbf{31.12/10.85/27.91}} &
              0.33/0.3/0.32 \\ \hline
            \multirow{8}{*}{\textbf{R-R(1/2/L)}} &
              Separate &
              \multicolumn{1}{c|}{32.73/11.13/29.32} &
              0.32/0.36/0.33 &
              \multicolumn{1}{c|}{32.33/10.58/28.89} &
              0.03/0.08/0.05 &
              \multicolumn{1}{c|}{32.39/10.73/28.98} &
              0.05/0.04/0.05 &
              \multicolumn{1}{c|}{32.94/11.24/29.53} &
              0.07/0.08/0.07 &
              \multicolumn{1}{c|}{32.89/11.25/29.48} &
              0.28/0.24/0.27 \\ \cline{2-12} 
             &
              FedAvg &
              \multicolumn{1}{c|}{32.85/11.18/29.44} &
              0.14/0.12/0.13 &
              \multicolumn{1}{c|}{32.96/11.32/29.54} &
              0.17/0.15/0.16 &
              \multicolumn{1}{c|}{33.0/11.36/29.58} &
              0.23/0.21/0.21 &
              \multicolumn{1}{c|}{32.87/11.21/29.44} &
              0.21/0.19/0.2 &
              \multicolumn{1}{c|}{32.72/11.08/29.31} &
              0.19/0.16/0.18 \\ \cline{2-12} 
             &
              FedProx &
              \multicolumn{1}{c|}{33.05/11.38/29.63} &
              0.09/0.09/0.08 &
              \multicolumn{1}{c|}{32.78/11.19/29.39} &
              0.13/0.12/0.12 &
              \multicolumn{1}{c|}{32.88/11.26/29.49} &
              0.13/0.09/0.11 &
              \multicolumn{1}{c|}{32.93/11.28/29.52} &
              0.03/0.02/0.04 &
              \multicolumn{1}{c|}{32.76/11.12/29.36} &
              0.26/0.24/0.25 \\ \cline{2-12} 
             &
              Scaffold &
              \multicolumn{1}{c|}{32.83/11.19/29.42} &
              0.08/0.1/0.09 &
              \multicolumn{1}{c|}{32.7/11.06/29.28} &
              0.16/0.18/0.16 &
              \multicolumn{1}{c|}{32.8/11.14/29.38} &
              0.21/0.19/0.21 &
              \multicolumn{1}{c|}{32.67/11.0/29.26} &
              0.28/0.26/0.28 &
              \multicolumn{1}{c|}{32.76/11.14/29.34} &
              0.25/0.25/0.25 \\ \cline{2-12} 
             &
              FedDC &
              \multicolumn{1}{c|}{33.2/11.52/29.77} &
              0.02/0.02/0.02 &
              \multicolumn{1}{c|}{33.02/11.43/29.63} &
              0.17/0.15/0.16 &
              \multicolumn{1}{c|}{32.99/11.37/29.58} &
              0.23/0.18/0.21 &
              \multicolumn{1}{c|}{\textbf{33.15/11.47/29.73}} &
              0.04/0.06/0.04 &
              \multicolumn{1}{c|}{33.1/11.49/29.69} &
              0.08/0.04/0.06 \\ \cline{2-12} 
             &
              FedNova &
              \multicolumn{1}{c|}{32.43/10.86/29.01} &
              0.01/0.02/0.02 &
              \multicolumn{1}{c|}{32.6/10.92/29.18} &
              0.33/0.4/0.34 &
              \multicolumn{1}{c|}{32.48/10.88/29.07} &
              \textbf{0.02/0.01/0.02} &
              \multicolumn{1}{c|}{32.44/10.89/29.03} &
              \textbf{0.02/0.01/0.02} &
              \multicolumn{1}{c|}{32.46/10.91/29.05} &
              \textbf{0.02/0.01/0.02} \\ \cline{2-12} 
             &
              FedProto &
              \multicolumn{1}{c|}{32.41/10.75/28.99} &
              \textbf{0.02/0.02/0.02} &
              \multicolumn{1}{c|}{32.42/10.75/29.0} &
              \textbf{0.04/0.06/0.05} &
              \multicolumn{1}{c|}{32.39/10.73/28.97} &
              0.02/0.02/0.02 &
              \multicolumn{1}{c|}{32.42/10.75/29.0} &
              0.02/0.02/0.03 &
              \multicolumn{1}{c|}{32.43/10.77/29.02} &
              0.06/0.07/0.06 \\ \cline{2-12} 
             &
              \textbf{FedSum} &
              \multicolumn{1}{c|}{\textbf{33.3/11.62/29.86}} &
              0.03/0.03/0.02 &
              \multicolumn{1}{c|}{\textbf{33.32/11.67/29.9}} &
              0.39/0.33/0.36 &
              \multicolumn{1}{c|}{\textbf{33.18/11.53/29.78}} &
              0.39/0.3/0.39 &
              \multicolumn{1}{c|}{33.01/11.38/29.61} &
              0.05/0.07/0.06 &
              \multicolumn{1}{c|}{\textbf{32.76/11.12/29.45}} &
              0.28/0.23/0.33 \\ \hline
            \end{tabular}
            }
            \caption{Experimental results with different heterogeneous settings on PubMed.
            The mean and variance of three optimal results for each method are presented.
            }
            \label{Table: PubMed full result summary}
            \end{table*}

        \begin{table*}[htpb]
            \centering
            \resizebox{1\textwidth}{!}{   
            \begin{tabular}{|c|c|cc|cc|cc|cc|cc|}
            \hline
            \textbf{WikiHow} &
              \multirow{2}{*}{\textbf{Method}} &
              \multicolumn{2}{c|}{\textbf{$Dir=0.1$}} &
              \multicolumn{2}{c|}{\textbf{$Dir=0.5$}} &
              \multicolumn{2}{c|}{\textbf{$Dir=1$}} &
              \multicolumn{2}{c|}{\textbf{$Dir=8$}} &
              \multicolumn{2}{c|}{\textbf{$Dir=+\infty$}} \\ \cline{1-1} \cline{3-12} 
            \textbf{Metric} &
               &
              \multicolumn{1}{c|}{\textbf{Mean}} &
              \textbf{Var.} &
              \multicolumn{1}{c|}{\textbf{Mean}} &
              \textbf{Var.} &
              \multicolumn{1}{c|}{\textbf{Mean}} &
              \textbf{Var.} &
              \multicolumn{1}{c|}{\textbf{Mean}} &
              \textbf{Var.} &
              \multicolumn{1}{c|}{\textbf{Mean}} &
              \textbf{Var.} \\ \hline
            \multirow{8}{*}{\textbf{R-F(1/2/L)}} &
              Separate &
              \multicolumn{1}{c|}{22.41/4.85/20.79} &
              0.06/0.03/0.05 &
              \multicolumn{1}{c|}{22.15/4.74/20.55} &
              0.07/0.02/0.05 &
              \multicolumn{1}{c|}{22.2/4.76/20.61} &
              0.1/0.02/0.1 &
              \multicolumn{1}{c|}{22.39/4.88/20.74} &
              0.25/0.15/0.22 &
              \multicolumn{1}{c|}{22.2/4.75/20.59} &
              0.14/0.05/0.14 \\ \cline{2-12} 
             &
              FedAvg &
              \multicolumn{1}{c|}{22.21/4.77/20.61} &
              0.14/0.05/0.13 &
              \multicolumn{1}{c|}{22.38/4.85/20.76} &
              0.1/0.03/0.08 &
              \multicolumn{1}{c|}{22.24/4.8/20.64} &
              0.02/0.02/0.02 &
              \multicolumn{1}{c|}{22.25/4.79/20.66} &
              0.11/0.03/0.1 &
              \multicolumn{1}{c|}{22.36/4.82/20.75} &
              0.07/0.03/0.06 \\ \cline{2-12} 
             &
              FedProx &
              \multicolumn{1}{c|}{22.29/4.81/20.69} &
              0.17/0.07/0.15 &
              \multicolumn{1}{c|}{22.12/4.75/20.53} &
              0.05/0.03/0.05 &
              \multicolumn{1}{c|}{22.24/4.78/20.64} &
              0.11/0.06/0.1 &
              \multicolumn{1}{c|}{22.21/4.77/20.6} &
              0.16/0.06/0.15 &
              \multicolumn{1}{c|}{22.28/4.8/20.67} &
              0.18/0.07/0.16 \\ \cline{2-12} 
             &
              Scaffold &
              \multicolumn{1}{c|}{22.02/4.7/20.44} &
              0.03/0.01/0.03 &
              \multicolumn{1}{c|}{22.09/4.73/20.49} &
              0.06/0.03/0.06 &
              \multicolumn{1}{c|}{22.17/4.76/20.58} &
              0.06/0.02/0.06 &
              \multicolumn{1}{c|}{22.18/4.77/20.59} &
              0.01/0.0/0.01 &
              \multicolumn{1}{c|}{22.37/4.82/20.76} &
              0.09/0.02/0.08 \\ \cline{2-12} 
             &
              FedDC &
              \multicolumn{1}{c|}{22.14/4.75/20.54} &
              0.13/0.08/0.12 &
              \multicolumn{1}{c|}{22.35/4.84/20.73} &
              0.35/0.16/0.32 &
              \multicolumn{1}{c|}{22.11/4.74/20.52} &
              0.04/0.03/0.03 &
              \multicolumn{1}{c|}{22.31/4.83/20.7} &
              0.1/0.04/0.09 &
              \multicolumn{1}{c|}{22.42/4.83/20.8} &
              0.04/0.01/0.03 \\ \cline{2-12} 
             &
              FedNova &
              \multicolumn{1}{c|}{22.11/4.73/20.55} &
              0.05/0.03/0.05 &
              \multicolumn{1}{c|}{22.09/4.72/20.51} &
              0.05/0.0/0.05 &
              \multicolumn{1}{c|}{22.02/4.71/20.44} &
              0.01/0.0/0.01 &
              \multicolumn{1}{c|}{22.01/4.69/20.42} &
              \textbf{0.01/0.0/0.01} &
              \multicolumn{1}{c|}{22.05/4.71/20.48} &
              \textbf{0.01/0.0/0.01} \\ \cline{2-12} 
             &
              FedProto &
              \multicolumn{1}{c|}{22.16/4.73/20.6} &
              \textbf{0.01/0.0/0.01} &
              \multicolumn{1}{c|}{22.17/4.74/20.61} &
              \textbf{0.0/0.0/0.0} &
              \multicolumn{1}{c|}{22.21/4.75/20.64} &
              \textbf{0.01/0.0/0.01} &
              \multicolumn{1}{c|}{22.21/4.77/20.64} &
              0.03/0.02/0.02 &
              \multicolumn{1}{c|}{22.2/4.75/20.63} &
              0.0/0.01/0.01 \\ \cline{2-12} 
             &
              \textbf{FedSum} &
              \multicolumn{1}{c|}{\textbf{22.32/4.83/20.7}} &
              0.14/0.05/0.12 &
              \multicolumn{1}{c|}{\textbf{22.38/4.85/20.75}} &
              0.2/0.08/0.18 &
              \multicolumn{1}{c|}{\textbf{22.44/4.88/20.82}} &
              0.14/0.06/0.14 &
              \multicolumn{1}{c|}{\textbf{22.45/4.87/20.81}} &
              0.25/0.1/0.21 &
              \multicolumn{1}{c|}{\textbf{22.2/4.78/20.64}} &
              0.06/0.03/0.07 \\ \hline
            \multirow{8}{*}{\textbf{R-R(1/2/L)}} &
              Separate &
              \multicolumn{1}{c|}{33.22/7.97/30.96} &
              0.08/0.05/0.06 &
              \multicolumn{1}{c|}{32.73/7.75/30.53} &
              0.2/0.06/0.16 &
              \multicolumn{1}{c|}{33.06/7.84/30.83} &
              0.01/0.01/0.0 &
              \multicolumn{1}{c|}{33.52/8.13/31.21} &
              0.53/0.3/0.48 &
              \multicolumn{1}{c|}{32.99/7.8/30.75} &
              0.09/0.05/0.09 \\ \cline{2-12} 
             &
              FedAvg &
              \multicolumn{1}{c|}{33.03/7.88/30.79} &
              0.06/0.02/0.05 &
              \multicolumn{1}{c|}{33.23/7.97/30.98} &
              0.22/0.07/0.18 &
              \multicolumn{1}{c|}{32.99/7.9/30.77} &
              0.04/0.03/0.04 &
              \multicolumn{1}{c|}{33.0/7.86/30.79} &
              0.1/0.02/0.07 &
              \multicolumn{1}{c|}{33.07/7.88/30.84} &
              0.04/0.02/0.03 \\ \cline{2-12} 
             &
              FedProx &
              \multicolumn{1}{c|}{33.14/7.95/30.9} &
              0.14/0.07/0.13 &
              \multicolumn{1}{c|}{32.89/7.84/30.67} &
              0.11/0.05/0.09 &
              \multicolumn{1}{c|}{32.98/7.84/30.76} &
              0.12/0.09/0.12 &
              \multicolumn{1}{c|}{32.98/7.86/30.75} &
              0.17/0.06/0.15 &
              \multicolumn{1}{c|}{33.08/7.91/30.85} &
              0.18/0.08/0.15 \\ \cline{2-12} 
             &
              Scaffold &
              \multicolumn{1}{c|}{32.78/7.77/30.58} &
              0.02/0.01/0.02 &
              \multicolumn{1}{c|}{32.9/7.83/30.67} &
              0.06/0.04/0.05 &
              \multicolumn{1}{c|}{32.92/7.84/30.7} &
              0.06/0.01/0.06 &
              \multicolumn{1}{c|}{32.99/7.88/30.77} &
              0.06/0.02/0.05 &
              \multicolumn{1}{c|}{33.1/7.9/30.87} &
              0.1/0.03/0.09 \\ \cline{2-12} 
             &
              FedDC &
              \multicolumn{1}{c|}{32.72/7.76/30.52} &
              0.18/0.1/0.16 &
              \multicolumn{1}{c|}{33.3/8.03/31.03} &
              0.47/0.25/0.42 &
              \multicolumn{1}{c|}{32.98/7.87/30.76} &
              0.04/0.02/0.03 &
              \multicolumn{1}{c|}{33.22/7.99/30.97} &
              0.18/0.07/0.16 &
              \multicolumn{1}{c|}{33.2/7.92/30.94} &
              0.1/0.03/0.08 \\ \cline{2-12} 
             &
              FedNova &
              \multicolumn{1}{c|}{32.58/7.69/30.43} &
              0.01/0.03/0.02 &
              \multicolumn{1}{c|}{32.8/7.76/30.59} &
              0.02/0.02/0.03 &
              \multicolumn{1}{c|}{32.8/7.8/30.57} &
              0.06/0.03/0.06 &
              \multicolumn{1}{c|}{32.58/7.7/30.38} &
              \textbf{0.01/0.01/0.01} &
              \multicolumn{1}{c|}{32.53/7.7/30.36} &
              \textbf{0.0/0.0/0.01} \\ \cline{2-12} 
             &
              FedProto &
              \multicolumn{1}{c|}{32.68/7.71/30.53} &
              \textbf{0.01/0.0/0.01} &
              \multicolumn{1}{c|}{32.69/7.72/30.55} &
              \textbf{0.0/0.0/0.0} &
              \multicolumn{1}{c|}{32.77/7.75/30.61} &
              \textbf{0.04/0.02/0.03} &
              \multicolumn{1}{c|}{32.76/7.75/30.6} &
              0.03/0.02/0.02 &
              \multicolumn{1}{c|}{32.77/7.75/30.61} &
              0.06/0.03/0.04 \\ \cline{2-12} 
             &
              \textbf{FedSum} &
              \multicolumn{1}{c|}{\textbf{33.18/7.97/30.93}} &
              0.2/0.07/0.16 &
              \multicolumn{1}{c|}{\textbf{33.35/8.02/31.06}} &
              0.3/0.13/0.26 &
              \multicolumn{1}{c|}{\textbf{33.37/8.06/31.09}} &
              0.17/0.07/0.16 &
              \multicolumn{1}{c|}{\textbf{33.46/8.07/31.16}} &
              0.34/0.17/0.3 &
              \multicolumn{1}{c|}{\textbf{33.11/7.91/30.86}} &
              0.03/0.02/0.03 \\ \hline
            \end{tabular}
            }
            \caption{Experimental results with different heterogeneous settings on WikiHow.
            The mean and variance of three optimal results for each method are presented.
            }
            \label{Table: WikiHow full result summary}
            \end{table*}

        \begin{table*}[htpb]
            \centering
            \resizebox{1\textwidth}{!}{   
            \begin{tabular}{|c|c|cc|cc|cc|cc|cc|}
            \hline
            \textbf{Reddit} &
              \multirow{2}{*}{\textbf{Method}} &
              \multicolumn{2}{c|}{\textbf{$Dir=0.1$}} &
              \multicolumn{2}{c|}{\textbf{$Dir=0.5$}} &
              \multicolumn{2}{c|}{\textbf{$Dir=1$}} &
              \multicolumn{2}{c|}{\textbf{$Dir=8$}} &
              \multicolumn{2}{c|}{\textbf{$Dir=+\infty$}} \\ \cline{1-1} \cline{3-12} 
            \textbf{Metric} &
               &
              \multicolumn{1}{c|}{\textbf{Mean}} &
              \textbf{Var.} &
              \multicolumn{1}{c|}{\textbf{Mean}} &
              \textbf{Var.} &
              \multicolumn{1}{c|}{\textbf{Mean}} &
              \textbf{Var.} &
              \multicolumn{1}{c|}{\textbf{Mean}} &
              \textbf{Var.} &
              \multicolumn{1}{c|}{\textbf{Mean}} &
              \textbf{Var.} \\ \hline
            \multirow{8}{*}{\textbf{R-F(1/2/L)}} &
              Separate &
              \multicolumn{1}{c|}{17.88/3.17/15.1} &
              0.02/0.01/0.02 &
              \multicolumn{1}{c|}{17.86/3.17/15.1} &
              0.01/0.0/0.01 &
              \multicolumn{1}{c|}{17.87/3.17/15.11} &
              0.03/0.0/0.02 &
              \multicolumn{1}{c|}{17.83/3.14/15.06} &
              0.0/0.01/0.0 &
              \multicolumn{1}{c|}{17.84/3.18/15.09} &
              \textbf{0.0/0.0/0.0} \\ \cline{2-12} 
             &
              FedAvg &
              \multicolumn{1}{c|}{17.87/3.19/15.11} &
              \textbf{0.0/0.0/0.0} &
              \multicolumn{1}{c|}{17.86/3.18/15.1} &
              0.01/0.0/0.0 &
              \multicolumn{1}{c|}{17.92/3.21/15.14} &
              \textbf{0.0/0.0/0.0} &
              \multicolumn{1}{c|}{17.89/3.19/15.14} &
              \textbf{0.0/0.0/0.0} &
              \multicolumn{1}{c|}{17.86/3.18/15.1} &
              0.01/0.0/0.0 \\ \cline{2-12} 
             &
              FedProx &
              \multicolumn{1}{c|}{17.86/3.19/15.08} &
              0.01/0.0/0.0 &
              \multicolumn{1}{c|}{17.86/3.18/15.11} &
              \textbf{0.0/0.0/0.0} &
              \multicolumn{1}{c|}{17.87/3.18/15.11} &
              0.0/0.01/0.01 &
              \multicolumn{1}{c|}{17.89/3.19/15.12} &
              0.01/0.01/0.01 &
              \multicolumn{1}{c|}{17.88/3.19/15.12} &
              0.0/0.01/0.0 \\ \cline{2-12} 
             &
              Scaffold &
              \multicolumn{1}{c|}{17.86/3.17/15.09} &
              \textbf{0.0/0.0/0.0} &
              \multicolumn{1}{c|}{17.84/3.18/15.08} &
              0.01/0.0/0.01 &
              \multicolumn{1}{c|}{17.9/3.2/15.13} &
              0.0/0.0/0.01 &
              \multicolumn{1}{c|}{17.87/3.19/15.11} &
              0.0/0.0/0.01 &
              \multicolumn{1}{c|}{17.86/3.19/15.1} &
              \textbf{0.0/0.0/0.0} \\ \cline{2-12} 
             &
              FedDC &
              \multicolumn{1}{c|}{17.91/3.2/15.15} &
              0.01/0.0/0.03 &
              \multicolumn{1}{c|}{17.89/3.19/15.13} &
              0.02/0.01/0.01 &
              \multicolumn{1}{c|}{17.95/3.21/15.22} &
              0.02/0.01/0.02 &
              \multicolumn{1}{c|}{17.92/3.21/15.18} &
              0.05/0.03/0.08 &
              \multicolumn{1}{c|}{17.91/3.19/15.13} &
              0.02/0.0/0.04 \\ \cline{2-12} 
             &
              FedNova &
              \multicolumn{1}{c|}{17.86/3.17/15.13} &
              0.01/0.01/0.0 &
              \multicolumn{1}{c|}{17.9/3.19/15.17} &
              0.0/0.01/0.01 &
              \multicolumn{1}{c|}{17.89/3.19/15.14} &
              0.02/0.01/0.03 &
              \multicolumn{1}{c|}{17.85/3.17/15.13} &
              0.01/0.0/0.01 &
              \multicolumn{1}{c|}{17.87/3.17/15.14} &
              0.0/0.0/0.01 \\ \cline{2-12} 
             &
              FedProto &
              \multicolumn{1}{c|}{17.84/3.17/15.09} &
              0.01/0.0/0.01 &
              \multicolumn{1}{c|}{17.91/3.2/15.19} &
              \textbf{0.0/0.0/0.0} &
              \multicolumn{1}{c|}{17.91/3.2/15.19} &
              \textbf{0.0/0.0/0.0} &
              \multicolumn{1}{c|}{17.91/3.2/15.17} &
              \textbf{0.0/0.0/0.0} &
              \multicolumn{1}{c|}{17.88/3.19/15.12} &
              0.01/0.01/0.0 \\ \cline{2-12} 
             &
              \textbf{FedSum} &
              \multicolumn{1}{c|}{\textbf{17.95/3.21/15.17}} &
              0.01/0.01/0.02 &
              \multicolumn{1}{c|}{\textbf{17.96/3.21/15.2}} &
              0.05/0.02/0.07 &
              \multicolumn{1}{c|}{\textbf{17.96/3.22/15.22}} &
              0.04/0.01/0.02 &
              \multicolumn{1}{c|}{\textbf{17.91/3.2/15.19}} &
              0.0/0.0/0.02 &
              \multicolumn{1}{c|}{\textbf{17.91/3.19/15.15}} &
              0.02/0.01/0.02 \\ \hline
            \multirow{8}{*}{\textbf{R-R(1/2/L)}} &
              Separate &
              \multicolumn{1}{c|}{41.57/9.02/35.73} &
              0.05/0.03/0.04 &
              \multicolumn{1}{c|}{41.53/9.02/35.7} &
              0.0/0.01/0.0 &
              \multicolumn{1}{c|}{41.59/9.05/35.78} &
              0.06/0.02/0.06 &
              \multicolumn{1}{c|}{41.42/8.97/35.58} &
              0.0/0.01/0.0 &
              \multicolumn{1}{c|}{41.52/9.07/35.76} &
              \textbf{0.0/0.0/0.0} \\ \cline{2-12} 
             &
              FedAvg &
              \multicolumn{1}{c|}{41.66/9.1/35.81} &
              0.05/0.0/0.03 &
              \multicolumn{1}{c|}{41.56/9.06/35.73} &
              0.03/0.01/0.03 &
              \multicolumn{1}{c|}{41.73/9.12/35.89} &
              0.0/0.01/0.02 &
              \multicolumn{1}{c|}{41.63/9.12/35.84} &
              0.02/0.02/0.0 &
              \multicolumn{1}{c|}{41.63/9.1/35.82} &
              0.01/0.02/0.03 \\ \cline{2-12} 
             &
              FedProx &
              \multicolumn{1}{c|}{41.59/9.09/35.71} &
              \textbf{0.0/0.01/0.0} &
              \multicolumn{1}{c|}{41.62/9.06/35.79} &
              \textbf{0.0/0.0/0.0} &
              \multicolumn{1}{c|}{41.62/9.09/35.8} &
              0.03/0.0/0.0 &
              \multicolumn{1}{c|}{41.64/9.1/35.78} &
              0.01/0.0/0.02 &
              \multicolumn{1}{c|}{41.62/9.1/35.81} &
              0.04/0.04/0.03 \\ \cline{2-12} 
             &
              Scaffold &
              \multicolumn{1}{c|}{41.58/9.07/35.74} &
              0.0/0.02/0.01 &
              \multicolumn{1}{c|}{41.6/9.07/35.75} &
              0.02/0.02/0.02 &
              \multicolumn{1}{c|}{41.64/9.12/35.79} &
              0.03/0.02/0.06 &
              \multicolumn{1}{c|}{41.59/9.09/35.78} &
              0.0/0.0/0.05 &
              \multicolumn{1}{c|}{41.63/9.1/35.8} &
              0.01/0.02/0.0 \\ \cline{2-12} 
             &
              FedDC &
              \multicolumn{1}{c|}{41.77/9.13/35.94} &
              0.11/0.04/0.15 &
              \multicolumn{1}{c|}{41.69/9.07/35.85} &
              0.01/0.01/0.0 &
              \multicolumn{1}{c|}{41.76/9.13/36.04} &
              0.08/0.04/0.07 &
              \multicolumn{1}{c|}{41.76/9.16/35.99} &
              0.1/0.08/0.19 &
              \multicolumn{1}{c|}{41.76/9.14/35.94} &
              0.08/0.02/0.12 \\ \cline{2-12} 
             &
              FedNova &
              \multicolumn{1}{c|}{41.56/9.03/35.81} &
              0.01/0.02/0.03 &
              \multicolumn{1}{c|}{41.65/9.07/35.91} &
              0.04/0.04/0.05 &
              \multicolumn{1}{c|}{41.66/9.08/35.86} &
              0.1/0.06/0.12 &
              \multicolumn{1}{c|}{41.49/9.03/35.76} &
              0.04/0.0/0.03 &
              \multicolumn{1}{c|}{41.54/8.99/35.8} &
              0.01/0.02/0.02 \\ \cline{2-12} 
             &
              FedProto &
              \multicolumn{1}{c|}{41.66/9.07/35.83} &
              \textbf{0.0/0.0/0.01} &
              \multicolumn{1}{c|}{41.75/9.13/36.01} &
              \textbf{0.0/0.0/0.0} &
              \multicolumn{1}{c|}{41.75/9.13/36.01} &
              \textbf{0.0/0.0/0.0} &
              \multicolumn{1}{c|}{41.75/9.13/36.01} &
              \textbf{0.0/0.0/0.0} &
              \multicolumn{1}{c|}{41.56/9.09/35.77} &
              0.08/0.01/0.11 \\ \cline{2-12} 
             &
              \textbf{FedSum} &
              \multicolumn{1}{c|}{\textbf{41.78/9.13/35.92}} &
              0.04/0.0/0.02 &
              \multicolumn{1}{c|}{\textbf{41.87/9.18/36.06}} &
              0.03/0.05/0.09 &
              \multicolumn{1}{c|}{\textbf{41.76/9.17/36.04}} &
              0.01/0.03/0.04 &
              \multicolumn{1}{c|}{\textbf{41.76/9.13/35.99}} &
              0.02/0.0/0.03 &
              \multicolumn{1}{c|}{\textbf{41.79/9.12/35.91}} &
              0.04/0.02/0.04 \\ \hline
            \end{tabular}
            }
            \caption{Experimental results with different heterogeneous settings on Reddit.
            The mean and variance of three optimal results for each method are presented.
            }
            \label{Table: Reddit full result summary}
            \end{table*}
    \newpage{}
    \paragraph{Learning rate turning.}
        We trained FedAvg with different training rates in CNNDM and WikiHow datasets.
        Every time the FL server communicates with the clients, we perform the performance test and record it.
        We run every experiment 3 times in order to reduce variance.
        The experimental results are shown in Table \ref{tb:cnndm_learning_rate} and Table \ref{tb:wikihow_learning_rate}. 
        According to the result, FedAvg can achieve better performance under the training rate of 5e-3. Therefore, we set $\eta$ as 5e-3 for all experiments.
        
        \begin{table*}[htpb]
            \centering
            \resizebox{1\textwidth}{!}{   
            \begin{tabular}{|c|c|c|c|c|c|c|}
            
                \hline
                \multicolumn{1}{|l|}{\textbf{FedAvg}} & \textbf{$Dir=0.1$} & \textbf{$T = 1$}  & \textbf{$T = 2$}  & \textbf{$T = 3$}  & \textbf{$T = 4$}  & \textbf{$T = 5$}  \\ \hline
                \multirow{2}{*}{5e-2}          & R-F(1/2/l)     & 28.88/8.35/22.02 &28.88/8.35/22.02  & 28.88/8.35/22.02  &28.88/8.35/22.02  & 28.88/8.35/22.02  \\ \cline{2-7} 
                                                      & R-R(1/2/l)     & 35.11/10.62/26.88 & 35.11/10.62/26.88 & 35.11/10.62/26.88 & 35.11/10.62/26.88 & 35.11/10.62/26.88 \\ \hline
                \multirow{2}{*}{\textbf{5e-3}} &
                  \textbf{R-F(1/2/l)} &
                  \textbf{33.58/12.66/26.81} &
                  \textbf{33.95/12.92/27.17} &
                  \textbf{34.19/13.13/27.39} &
                  \textbf{32.45/11.11/25.30} &
                  \textbf{34.13/12.50/26.89} \\ \cline{2-7} 
                 &
                  \textbf{R-R(1/2/l)} &
                  \textbf{41.92/16.50/33.52} &
                  \textbf{42.53/16.92/34.09} &
                  \textbf{42.86/17.21/34.39} &
                  \textbf{39.93/14.34/31.22} &
                  \textbf{42.69/16.46/33.72} \\ \hline
                \multirow{2}{*}{5e-4}          & R-F(1/2/l)     & 32.83/12.02/25.90 & 32.88/12.06/25.95 & 33.01/12.17/26.09 & 33.09/12.22/26.16 & 33.17/12.31/26.26 \\ \cline{2-7} 
                                                      & R-R(1/2/l)     & 40.41/15.44/31.97 & 40.49/15.49/32.04 & 40.69/15.65/32.23 & 40.81/15.74/32.33 & 40.90/15.86/32.46 \\ \hline
                \multirow{2}{*}{5e-5}          & R-F(1/2/l)     & 32.10/11.41/25.18 & 32.13/11.44/25.23 & 32.16/11.48/25.27 & 32.17/11.49/25.28 & 32.19/11.50/25.29 \\ \cline{2-7} 
                                                      & R-R(1/2/l)     & 39.28/14.55/30.91 & 39.33/14.60/30.98 & 39.39/14.65/31.04 & 39.41/14.67/31.05 & 39.43/14.69/31.07 \\ \hline
                \multirow{2}{*}{5e-6}          & R-F(1/2/l)     & 31.53/10.94/24.62 & 31.62/11.00/24.69 & 31.64/11.02/24.72 & 31.68/11.06/24.76 & 31.71/11.09/24.80 \\ \cline{2-7} 
                                                      & R-R(1/2/l)     & 38.39/13.87/30.06 & 38.53/13.96/30.16 & 38.56/13.99/30.22 & 38.61/14.03/30.27 & 38.66/14.07/30.32 \\ \hline
            \end{tabular}
            }
            \caption{Performance comparison of FedAvg trained with different learning rate $\eta$=\{5e-2, 5e-3, 5e-4, 5e-5, 5e-6\} in CNNDM. R-F and R-R stand for the ROUGE-F1 metric and ROUGE-Recall metric, respectively.
            }
            \label{tb:cnndm_learning_rate}
        \end{table*}

        \begin{table*}[htpb]
            \centering
            \resizebox{1\textwidth}{!}{   
            \begin{tabular}{|c|c|c|c|c|c|c|}
            \hline
            \multicolumn{1}{|l|}{\textbf{FedAvg}} &
              $Dir=0.1$ &
              $T = 1$ &
              $T = 2$ &
              $T = 3$ &
              $T = 4$ &
              $T = 5$ \\ \hline
            \multirow{2}{*}{5e-2} &
              R-F(1/2/l) &
              22.25/4.78/20.64 &
              22.28/4.80/20.68 &
              22.28/4.80/20.68 &
              22.28/4.80/20.68 &
              22.28/4.80/20.68 \\ \cline{2-7} 
             &
              R-R(1/2/l) &
              32.93/7.83/30.69 &
              32.97/7.85/30.73 &
              32.97/7.85/30.73 &
              32.97/7.85/30.73 &
              32.97/7.85/30.73 \\ \hline
            \multirow{2}{*}{\textbf{5e-3}} &
              \textbf{R-F(1/2/l)} &
              \textbf{22.36/4.80/20.75} &
              \textbf{22.29/4.79/20.68} &
              \textbf{22.26/4.78/20.66} &
              \textbf{22.27/4.78/20.68} &
              \textbf{22.25/4.77/20.65} \\ \cline{2-7} 
             &
              \textbf{R-R(1/2/l)} &
              \textbf{33.06/7.86/30.83} &
              \textbf{32.97/7.83/30.74} &
              \textbf{32.88/7.82/30.68} &
              \textbf{32.90/7.82/30.69} &
              \textbf{32.88/7.81/30.67} \\ \hline
            \multirow{2}{*}{5e-4} &
              R-F(1/2/l) &
              21.97/4.68/20.41 &
              21.99/4.68/20.42 &
              21.98/4.68/20.42 &
              21.99/4.68/20.43 &
              21.99/4.69/20.43 \\ \cline{2-7} 
             &
              R-R(1/2/l) &
              32.50/7.66/30.35 &
              32.51/7.66/30.35 &
              32.47/7.65/30.31 &
              32.49/7.66/30.34 &
              32.46/7.65/30.32 \\ \hline
            \multirow{2}{*}{5e-5} &
              R-F(1/2/l) &
              22.18/4.73/20.61 &
              22.15/4.73/20.58 &
              22.09/4.71/20.53 &
              22.03/4.69/20.47 &
              22.02/4.69/20.47 \\ \cline{2-7} 
             &
              R-R(1/2/l) &
              32.64/7.69/30.50 &
              32.60/7.68/30.45 &
              32.57/7.67/30.42 &
              32.51/7.65/30.36 &
              32.54/7.67/30.39 \\ \hline
            \multirow{2}{*}{5e-6} &
              R-F(1/2/l) &
              22.20/4.74/20.64 &
              22.19/4.74/20.63 &
              22.19/4.74/20.62 &
              22.19/4.74/20.62 &
              22.18/4.73/20.62 \\ \cline{2-7} 
             &
              R-R(1/2/l) &
              32.66/7.70/30.53 &
              32.67/7.70/30.52 &
              32.66/7.69/30.51 &
              32.66/7.70/30.51 &
              32.64/7.69/30.49 \\ \hline
          \end{tabular}
          }
          \caption{Performance comparison of FedAvg trained with different learning rate $\eta$=\{5e-2, 5e-3, 5e-4, 5e-5, 5e-6\} in WikiHow. R-F and R-R stand for the ROUGE-F1 metric and ROUGE-Recall metric, respectively.}
          \label{tb:wikihow_learning_rate}   
        \end{table*}


    
    \paragraph{Hyperparameter $\gamma$ turning.}
    $\gamma$ controls the dropout rate of Bernoulli perturbation, which helps to avoid catastrophic forgetting and protect privacy. We adjust $\gamma$ and experiment in heterogeneity scenarios on CNNDM, PubMed, and WikiHow datasets. We observe that $\gamma$ has the most obvious effect on the CNNDM dataset. As showns in Fig.~\ref{fig:gamma_all_dataset}, when $\gamma$ increases from $0.1$ to $0.3$, the perturbations of the uploaded classification on modules become stronger, and ROUGE scores go better. When $\gamma$ over $0.3$, the perturbations are too strong and cause degeneration.

    \begin{figure}[!htb]
            \centering
            \includegraphics[width=0.45\columnwidth]{latex/fig/hyperparameter_γ_all_dataset.png}
            \includegraphics[width=0.45\columnwidth]{latex/fig/hyperparameter_γ_CNNDM.png}

            \caption{ROUGE scores of FedSum with different $\gamma$ in CNNDM, PubMed and WikiHow.}
            \label{fig:gamma_all_dataset}
        \end{figure}

    % \begin{figure}[htpb]
    %         \centering
    %         \includegraphics[width=0.45\columnwidth]{latex/fig/hyperparameter_γ_CNNDM.png}
    %         \caption{Comprehensive ROUGE scores of FedSum in CNNDM dataset.}
    %         \label{fig:gamma_CNNDM}
    %     \end{figure}
    

\newpage

\section{Detail of Data Partition Algorithm}

% Center the algorithm using center environment
\begin{center}
\begin{algorithm}[!htb]
    \caption{Data Partition.}
    \begin{algorithmic}[1]
    \REQUIRE $LD$, $LM$, $m$, $\lambda$, $\bar\epsilon^{(t-1)}$, $\bar Q^{(t-1)}$, $D_{(i,N)}$, $D_{(i,P)}$
    
    \textcolor[rgb]{0.25, 0.5, 0.75}{\textit{\# Sentences' Masked loss matrix}}
    \STATE $MM = LD \odot LM$
    
    \textcolor[rgb]{0.25, 0.5, 0.75}{\textit{\# Local significance: Avg. of masked loss over batch}}
    \STATE $\epsilon_{(i,j)}^{(e,t)} = \sum_b^B\sum_c^m\frac{MM[b][c]}{B}$
    \textcolor[rgb]{0.25, 0.5, 0.75}{\textit{\# Iteratively find the local bias level $Q_{(i,j)}^{(e,t)}$}}
    
    \STATE \textbf{for} $Q_{(i,j)}^{(e,t)} = 1$ \textbf{to} $m$ \textbf{do}
    \textcolor[rgb]{0.25, 0.5, 0.75}{\textit{\quad \# $m$ is the max index of the sentence in $\mathcal B_j$}}
    
    \STATE \quad $\sigma = \sum_{b=1}^{B}\sum_{c=1}^{Q_{(i,j)}^{(e,t)}} \frac{LD[b][c]}{B}$
    
    \STATE \quad \textbf{if} $\sigma \geq \Big\lceil \lambda \cdot \sum_{b=1}^{B}\sum_{c=1}^{m} \frac{LD[b][c]}{B} \Big\rceil$ \textbf{then}
    \STATE \qquad \textbf{break}
    \STATE \quad \textbf{end if}
    \STATE \textbf{end for}
    
    \textcolor[rgb]{0.25, 0.5, 0.75}{\textit{\# Comparison between local and global levels}}
    \STATE \textbf{if} $(Q_{(i,j)}^{(e,t)} < \bar Q^{(t-1)})$ \& $(\epsilon_{(i,j)}^{(e,t)} < \bar\epsilon^{(t-1)})$ \textbf{then}
    \STATE \quad $D_{(i,N)} := D_{(i,N)} \cup \mathcal B_j$ \textcolor[rgb]{0.25, 0.75, 0.25}{\textit{~\# belongs to normal subset}}
    \STATE \textbf{else}
    \STATE \quad $D_{(i,P)} := D_{(i,P)} \cup \mathcal B_j$ \textcolor[rgb]{0.75, 0.25, 0.25}{\textit{~\# belongs to the prime subset}}
    \STATE \textbf{end if}
    \RETURN $Q_{(i,j)}^{(e,t)}$, $\epsilon_{(i,j)}^{(e,t)}$, $D_{(i,N)}$, $D_{(i,P)}$
    \end{algorithmic}
\end{algorithm}
\end{center}




%%%%%%%%%%%%%%%%%%%%%%%%%%%%%%%%%%%%%%%%%%%%%%%%%%%%%%%%%%%%

%\newpage
%\section{Reproducibility Checklist}

This paper:
\begin{itemize}
    \item Includes a conceptual outline and/or pseudocode description of AI methods introduced (yes)
    \item Clearly delineates statements that are opinions, hypothesis, and speculation from objective facts and results (yes)
    \item Provides well marked pedagogical references for less-familiare readers to gain background necessary to replicate the paper (yes)
\end{itemize}

Does this paper make theoretical contributions? (no)

If yes, please complete the list below.

\begin{itemize}
    \item All assumptions and restrictions are stated clearly and formally. (yes/partial/no)
    \item All novel claims are stated formally (e.g., in theorem statements). (yes/partial/no)
    \item Proofs of all novel claims are included. (yes/partial/no)
    \item Proof sketches or intuitions are given for complex and/or novel results. (yes/partial/no)
    \item Appropriate citations to theoretical tools used are given. (yes/partial/no)
    \item All theoretical claims are demonstrated empirically to hold. (yes/partial/no/NA)
    \item All experimental code used to eliminate or disprove claims is included. (yes/no/NA)
\end{itemize}

Does this paper rely on one or more datasets? (yes)

If yes, please complete the list below.

\begin{itemize}
    \item A motivation is given for why the experiments are conducted on the selected datasets (yes)
    \item All novel datasets introduced in this paper are included in a data appendix. (NA)
    \item All novel datasets introduced in this paper will be made publicly available upon publication of the paper with a license that allows free usage for research purposes. (NA)
    \item All datasets drawn from the existing literature (potentially including authors’ own previously published work) are accompanied by appropriate citations. (yes)
    \item All datasets drawn from the existing literature (potentially including authors’ own previously published work) are publicly available. (yes)
    \item All datasets that are not publicly available are described in detail, with explanation why publicly available alternatives are not scientifically satisficing. (NA)
\end{itemize}

Does this paper include computational experiments? (yes)

If yes, please complete the list below.

\begin{itemize}
    \item Any code required for pre-processing data is included in the appendix. (yes).
    \item All source code required for conducting and analyzing the experiments is included in a code appendix. (yes)
    \item All source code required for conducting and analyzing the experiments will be made publicly available upon publication of the paper with a license that allows free usage for research purposes. (yes)
    \item All source code implementing new methods have comments detailing the implementation, with references to the paper where each step comes from (yes)
    \item If an algorithm depends on randomness, then the method used for setting seeds is described in a way sufficient to allow replication of results. (yes)
    \item This paper specifies the computing infrastructure used for running experiments (hardware and software), including GPU/CPU models; amount of memory; operating system; names and versions of relevant software libraries and frameworks. (yes)
    \item This paper formally describes evaluation metrics used and explains the motivation for choosing these metrics. (yes)
    \item This paper states the number of algorithm runs used to compute each reported result. (yes)
    \item Analysis of experiments goes beyond single-dimensional summaries of performance (e.g., average; median) to include measures of variation, confidence, or other distributional information. (yes)
    \item The significance of any improvement or decrease in performance is judged using appropriate statistical tests (e.g., Wilcoxon signed-rank). (yes)
    \item This paper lists all final (hyper-)parameters used for each model/algorithm in the paper’s experiments. (yes)
    \item This paper states the number and range of values tried per (hyper-) parameter during development of the paper, along with the criterion used for selecting the final parameter setting. (yes)
\end{itemize}



\end{document}