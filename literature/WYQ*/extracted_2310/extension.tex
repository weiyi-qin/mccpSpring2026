%\section{Generalizing the \ocs ~problem with the $S$-feasibility assumption}
%\subsection{Adversaries Ensuring Non-negative Regret} \label{improved_rates}
%In Theorem \ref{gen-cvx-bd} and \ref{str-cvx-bd}, we showed that under the assumption of the non-negativity of the worst-case regret, the constraint violation bounds can be improved to $O(\sqrt{{T}})$ and $O(\ln T/\alpha)$ for convex and strongly-convex cost functions, respectively. In this section, we additionally show that the same improved bounds hold in the case of a time-invariant fixed constraint function and a certain class of worst-case adversaries, called \emph{convex adversary}, defined next. COCO with time-invariant constraints has been studied in the literature where the main objective is to design gradient-based first-order policies that avoid the costly projection step on the constraint set \citep{jenatton2016adaptive, yuan2018online}. 

\paragraph{Convex adversary:} An adversary is called \emph{convex} if for any sequence of  action sequence $\{x_t\}_{t=1}^T,$ the adversary chooses the cost function sequence $\{f_t\}_{t=1}^T$ such that for any $T \geq 1,$ we have
\begin{eqnarray} \label{jensenadv}
	\sum_{t=1}^T f_t(x_t) \geq \sum_{t=1}^T f_t(\bar{x}_T),
\end{eqnarray}     
where $\bar{x}_T \equiv \frac{1}{T}\sum_{t=1}^T x_t.$ Hence, by definition, a convex adversary guarantees a non-negative regret with respect to the average action $\bar{x}_T$ for all rounds. In the following, we give two examples of convex adversaries.


 \paragraph{1. Fixed adversary:} An adversary which always selects a fixed convex function $f$ on all rounds is a convex adversary. In this case, Eqn.\ \eqref{jensenadv} holds due to the Jensen's inequality. 

\paragraph{2. Minimax adversary:} Let $\mathcal{F}$ denote an arbitrary non-empty set of convex functions defined on the admissible set $\mathcal{X}$. Consider an adversary $\mathcal{M}$, which, upon seeing the selected action $x_t$, chooses the worst cost function $f_t$ from the set $\mathcal{F}$ on round $t:$ 
\[f_t \in \arg\max_{f\in \mathcal{F}} f(x_t). \]
We now show that $\mathcal{M}$ is a convex adversary. By definition, for any round $\tau \in [T],$ we have 
\[ f_\tau(x_\tau) \geq f_t(x_\tau) \implies f_\tau(x_\tau) \geq \frac{1}{T} \sum_{t=1}^T f_{t}(x_\tau). \]
Summing up the above inequalities for each $\tau \in [T],$ we have 
\begin{eqnarray}\label{conv-adv-def}
 \sum_{\tau=1}^T f_\tau(x_\tau) \geq \sum_{t=1}^T \frac{1}{T}\sum_{\tau=1}^T f_t(x_{\tau}) \stackrel{\textrm{(a)}}{\geq} \sum_{t=1}^T f_t(\bar{x}_T),
 \end{eqnarray}
where inequality (a) follows upon applying Jensen's inequality to each cost function. 
Eqn.\ \eqref{conv-adv-def} shows that $\mathcal{M}$ is a convex adversary. 

P.S. It can be easily seen that Fixed adversary is a special case of Minimax adversary where $\mathcal{F}=\{f\}.$
\iffalse
\subsection{Assumptions}
\begin{assumption}[Time-invariant constraints] \label{constr_assump1}
Assume that the constraint functions on every round are fixed and given by $g_t=g.$ Hence, the feasible set is defined as:
\begin{eqnarray*}
	\mathcal{X}^\star = \{x^\star \in \mathcal{X}: g(x^\star) \leq 0\}.  
\end{eqnarray*}
\end{assumption}
\begin{assumption}[Convex adversary] \label{adversary_assump2}
	The cost functions are chosen by a convex adversary. 
\end{assumption}
\begin{assumption}[Bounded dual optimal variable] \label{sensitivity}
	Let $\bar{f}_T \equiv \frac{1}{T} \sum_{t=1}^T f_t$ be the average of the cost functions. Consider the following optimization problem $P_T:$
	%For any $\beta \geq 0,$ let us define:
	%\begin{eqnarray*}
		%S_T(\beta)= \min_{x \in \mathcal{X}} \{\bar{f}_T (x), ~ \textrm{s.t.} ~ g(x) \leq \beta\}.
		\[ \min_{x\in \mathcal{X}} \bar{f}_T(x)~ \textrm{s.t.}~ g(x) \leq 0. \]
		We assume that for each $T \geq 1,$ strong duality holds for the problem $P_T$ and that the limsup of a sequence of optimal dual variables is strictly bounded above by a finite constant $\lambda $. Note that the value of $\lambda$ need not be known a priori.
%	\end{eqnarray*}
	%It is easy to verify that $S_T (\cdot)$ is a non-increasing, convex function of $\beta$ \citep[Section 5.6.1]{boyd}. We assume that $\limsup_{T} |S_T'(0^+)| \leq \lambda$ for some finite $\lambda.$
\end{assumption}
\textbf{Remark:} In the case of a fixed adversary, we have $\bar{f}_T = f.$ Hence, $\lambda$ can be efficiently determined by examining the dual of a single convex problem. Note that \citep{nedic2009subgradient} made a similar assumption for the standard offline convex optimization problem with a given cost and a constraint function. However, their algorithm must know an upper bound to $\lambda$ \emph{a priori}.  The following proposition gives a sufficient condition for Assumption \ref{sensitivity}. In particular, it shows that if the unclipped constraint function satisfies Slater's condition, then Assumption \ref{sensitivity} holds \footnote{Clearly, the clipped constraint function can not satisfy Slater's condition. Hence, we study the unclipped constraint function in Proposition \ref{slater-bdd}.}. 
 \begin{proposition} \label{slater-bdd}
 	Let $\tilde{g}$ be the constraint function which can be negative and that satisfies Slater's condition, \emph{i.e.,} there exists an admissible $z \in \mathcal{X}$ s.t. $\tilde{g}(z) \leq -\epsilon$ for some $\epsilon >0. $ Then Assumption \ref{sensitivity} holds for some $\lambda = O(\frac{1}{\epsilon})$.
 \end{proposition} 
 \begin{proof}
 	It is easy to verify that any dual optimal solution to the problem with the unclipped constraint function is also a dual optimal solution to $P$ with the clipped constraint function. Let $(z^\star, \lambda)$ be an optimal primal-dual solution pair for the problem with the unclipped constraint function. Since $(z^\star, \lambda)$ is a saddle point for the Lagrangian, we can write 
 	\begin{eqnarray*}
 		\bar{f}_T(z)+\lambda \tilde{g}(z) \geq \bar{f}_T(z^\star)+ \lambda \tilde{g}(z^\star)= \bar{f}_T(z^\star), 
 	\end{eqnarray*}
 	where we have used the complementary slackness property on the RHS. Now since $\tilde{g}(z)\leq -\epsilon,$ from the above, we have 
 	\begin{eqnarray*}
 		\bar{f}_T(z) -\lambda \epsilon \geq \bar{f}_T(z^\star) ~\implies \lambda = O(\frac{1}{\epsilon}),
 	\end{eqnarray*}
 	where in the last step, we have used the fact that the range of any convex function with a bounded subgradient over a bounded domain is bounded.
 \end{proof}
 

\subsection{Cumulative violation bound under Assumptions \ref{constr_assump1}, \ref{adversary_assump2}, and \ref{sensitivity}} 
Taking  \eqref{q-bd-eqn} as our starting point, for any feasible action $x^\star \in \mathcal{X}^\star$, we have:
\begin{eqnarray} \label{master_eq2}
	Q^2(T) + V \textrm{Regret}_T(x^\star) \leq \textrm{Regret}_T'(x^\star). 
\end{eqnarray} 
%We can lower bound the regret term as follows:
In Theorem \ref{gen-cvx-bd} and \ref{str-cvx-bd}, we trivially lower bounded the regret term by a negative linear term, which resulted in a sub-optimal violation bound. Using Assumptions \ref{constr_assump1}, \ref{adversary_assump2}, and \ref{sensitivity}, we next derive a tighter lower bound to the regret by directly relating it to the queue length. We have 
\begin{eqnarray} \label{reg_lb_jensen}
	\textrm{Regret}_T(x^\star) &=& \sum_{t=1}^T f_t(x_t) - \sum_{t=1}^T f_t(x^\star) \nonumber\\
	&\stackrel{(a)}{\geq} & \sum_{t=1}^T f_t(\bar{x}_T) - \sum_{t=1}^T f_t(x^\star) \nonumber\\
	&=& T (\bar{f}_T(\bar{x}_T) - \bar{f}_T(x^\star)). 
\end{eqnarray}
where in step (a), we have used the fact that the adversary is convex. Next, from the CCV bound of COCO Meta-policy, we have 
\begin{eqnarray*}
T g(\bar{x}_T)	\stackrel{(\textrm{Jensen's ineq.})}{\leq} \sum_{t=1}^T g(x_t) \leq Q(T),
\end{eqnarray*}
\emph{i.e.,} 
\begin{eqnarray*}
	g(\bar{x}_T) \leq \frac{Q(T)}{T}.
\end{eqnarray*}
Now let $y^\star \in \mathcal{X}$ be a solution to the following optimization problem:
\begin{eqnarray}\label{opt_prob2}
	y^\star \in \arg \min_{x \in \mathcal{X}} \bar{f}_T(x), ~ \textrm{s.t.}~ g(x) \leq \frac{Q(T)}{T}.
\end{eqnarray}
Since the average action $\bar{x}_T$ is a feasible solution to the above program, we have 
\begin{eqnarray*}
	\bar{f}_T(\bar{x}_T) \geq \bar{f}_T(y^\star).
\end{eqnarray*}
Finally, choose the feasible action $x^\star$ as follows:
\begin{eqnarray}\label{opt_prob3}
	x^\star \in \arg\min_{x \in \mathcal{X}} \bar{f}_T(x), ~ \textrm{s.t.} ~ g(x) \leq 0. 
\end{eqnarray}
Hence, from Eqn.\ \eqref{reg_lb_jensen}, \eqref{opt_prob2}, and \eqref{opt_prob3}, we have 
\begin{eqnarray*}
	\textrm{Regret}_T(x^\star) \geq T(\bar{f}_T(y^\star)- \bar{f}_T(x^\star)).
\end{eqnarray*}
Since $Q(T)=o(T),$ from Theorem \ref{gen-cvx-bd}, we now make the key observation that $x^\star$ and $y^\star$ are the solutions to optimization problems \eqref{opt_prob2} and \eqref{opt_prob3} respectively, where the latter problem has been obtained from the former by perturbing the inequality constraint by a small amount for a large $T$. Hence, the difference in their objective values can be obtained by studying the sensitivity of the convex programs. Hence, using Eqn.\ (5.57) from \citet[Section 5.6.2]{boyd}, we conclude that for a sufficiently large horizon length $T,$ we have:
\begin{eqnarray*}
	\textrm{Regret}_T(x^\star) \geq T \big(-\lambda \frac{Q(T)}{T}\big) = - \lambda Q(T). 
\end{eqnarray*}
Plugging in the above bound in  \eqref{master_eq2}, we have the following relaxation of our key regret decomposition result for large enough $T:$
\begin{eqnarray} \label{reg_decomp_new}
	 Q^2(T) - V\lambda Q(T) \leq \textrm{Regret}_T'(x^\star),
\end{eqnarray}
where, as before, the $\textrm{Regret}_T'$ term on the right-hand side denotes the worst-case regret (over the admissible set $\mathcal{X}$) of the OCO sub-routine. We now have the following theorem
\begin{theorem} \label{improved_violation_bd}
	Under Assumptions \ref{constr_assump1}, \ref{adversary_assump2}, and \ref{sensitivity}, the cumulative constraint violation bounds in Theorem \ref{gen-cvx-bd} and Theorem \ref{str-cvx-bd} can be improved to $Q(T)= \mathbb{V}(T)=O((1+\lambda) \sqrt{T}),$ and $Q(T)=\mathbb{V}(T)=O((1+\lambda)\frac{\log T}{\alpha}),$ respectively. 
\end{theorem} 
\begin{proof}
	The proof proceeds similarly to Theorem \ref{gen-cvx-bd} part 1 and  Theorem \ref{str-cvx-bd} part 1, respectively, where we now take into account the additional linear term. 
	\paragraph{Case I (Convex costs):}
	Similar to Eqn.\ \eqref{main_eq}, plugging in the adaptive regret bound for convex cost functions on the RHS of Eqn.\ \eqref{reg_decomp_new}, we have 
	\begin{eqnarray*}
		Q^2(T)-V\lambda Q(T) \leq 2GDQ(T)\sqrt{T}+ 2GDV\sqrt{T},
	\end{eqnarray*}	
	where, as before, we have used the non-decreasing property of the queue-length sequence. Setting $V=\sqrt{T}$ and completing the square, we obtain the following bound for the queue-length sequence for a sufficiently large horizon length $T:$
	\begin{eqnarray*}
		Q(T) = O(\lambda\sqrt{T})+O(\sqrt{T}).
	\end{eqnarray*} 
		\paragraph{Case II (Strongly-convex costs):}
		Plugging in the adaptive regret bound \eqref{adaptive_str_cvx_bd} for strongly-convex cost functions on the RHS of Eqn.\ \eqref{reg_decomp_new}, we obtain: 
		\begin{eqnarray*}
			Q^2(T) - V\lambda Q(T) \leq \frac{VG^2\ln (Te)}{\alpha} + \frac{G^2 \ln (Te)}{\alpha V} Q^2(T),
		\end{eqnarray*}
		where, as before, we have used the non-decreasing property of the queue-length sequence. Setting $V=\frac{2G^2 \ln(Te)}{\alpha}$ as before, we have
		\begin{eqnarray*}
			Q^2(T) - 2V\lambda Q(T) \leq 2\frac{VG^2}{\alpha}\ln(Te). 
		\end{eqnarray*}
		Completing the square, we conclude that 
		\begin{eqnarray*}
			Q(T) = O(\frac{\lambda \log T}{\alpha})+ O(\frac{\log T}{\alpha}).
		\end{eqnarray*}
\end{proof}
\fi




\subsection{Proof of Theorem \ref{S-benchmark}} \label{S-benchmark-pf}
\label{ext}
\iffalse
In all the previous sections, we assumed the existence of a fixed action $x^\star \in \mathcal{X}$ that satisfies each of the online constraints on \emph{each round}. In particular, we assumed that $\mathcal{X}^\star \neq \emptyset.$
In this section, we revisit the \ocs ~problem by relaxing this assumption and only assuming that there is a feasible action $x^\star$ that satisfies the aggregate of the constraints over any consecutive $S$ rounds\footnote{This extension is meaningful only when the range of the constraint functions includes both positive and negative values. For non-negative constraints, clearly, $\mathcal{X}_S = \mathcal{X}^\star, \forall S\geq 1.$}, where the parameter $S \in [T]$ need not be known to the policy \emph{a priori}. Towards this end, we define the set of all $S$-feasible actions as below: 
\begin{eqnarray} \label{extended-benchmark}
\mathcal{X}_S =\{x^\star: \sum_{\tau \in |\mathcal{I}|} g_{\tau,i}(x^\star) \leq 0, \forall \textrm{sub-intervals}~ \mathcal{I} \subseteq [1,T], ~|\mathcal{I}| = S, \forall i \in [k]\}. 
\end{eqnarray}
We now replace Assumption \ref{feas-constr} with the following
\begin{assumption}[$S$-feasibility] \label{s-feas-assump}
	$\mathcal{X}_S \neq \emptyset.$
\end{assumption}
We now only assume that $\mathcal{X}_S \neq \emptyset$ for some $S: 1\leq S \leq T$. Clearly, Assumption \ref{s-feas-assump} is weaker than Assumption \ref{feas-constr} as $\mathcal{X}^\star \subseteq \mathcal{X}_S, \forall S \geq 1.$ 
This problem was considered earlier by \cite{georgios-cautious} in the COCO setting with a single constraint function per round. However, since the parameter $V$ used in their algorithm is restricted as $S \leq V \leq T,$ their algorithm naturally needs to know the value of $S.$ In reality, the value of the parameter $S$ is generally not available \emph{a priori} as it depends on the online constraint functions. Fortunately, our proposed meta-policy does not need to know the value of $S$ and hence, it can automatically adapt itself to the best feasible $S$.
\fi 

%\edit{argue why our result is better}.

%By extending the drift-plus-penalty methodology of \cite{neely2017online}, they proved a regret bound of $O(T^{1-\frac{\epsilon}{2}})$ and cumulative violation penalty of $O(T^{1-\frac{\epsilon}{4}})$ for $S=T^{1-\epsilon}$ \citep[Theorem 1]{georgios-cautious}. For the \texttt{OCS} problem, we show that our policy improves the latter bound to $O(T^{1-\frac{\epsilon}{2}}).$

\paragraph{Generalized regret decomposition:} Fix any $S$-feasible benchmark $x^\star \in \mathcal{X}_S,$ as given by Eqn.\ \eqref{extended-benchmark}. Then, from Eqn.\ \eqref{drift-bd}, we have 
\begin{eqnarray*}
	\Phi(\tau)- \Phi(\tau-1) &\leq& 2 \sum_{i=1}^k Q_i(\tau)g_{\tau, i}(x_\tau) \\
	&=& 2 \sum_{i=1}^k Q_i(\tau)\big(g_{\tau, i}(x_\tau)-g_{\tau, i}(x^\star)\big) + 2\sum_{i=1}^k Q_i(\tau)g_{\tau, i}(x^\star)\\
	&=& \hat{f}_\tau (x_\tau) - \hat{f}_\tau(x^\star)  +  2\sum_{i=1}^k Q_i(\tau)g_{\tau, i}(x^\star). 
\end{eqnarray*} 
 Summing up the above inequalities from $\tau = 1$ to $\tau=t,$ we have
 \begin{eqnarray} \label{new-reg-decomp}
 	\sum_{i=1}^k Q_i^2(t) =\Phi(t) \leq \textrm{Regret}'_t(x^\star) + 2 \sum_{i=1}^k\sum_{\tau=1}^t Q_i(\tau) g_{\tau, i}(x^\star),
 \end{eqnarray}
 where $\textrm{Regret}'(\cdot)$ refers to the regret of the surrogate costs as before. 
 We now bound the last term by making use of the $S$-feasibility of the action $x^\star$ as given by Eqn.\ \eqref{extended-benchmark}.
 Let us now divide the entire interval $[1,t]$ into disjoint and consecutive sub-intervals $\{\mathcal{I}_j\}_{j=1}^{\lceil t/S \rceil},$ each of length $S$ (except the last interval which could be of a smaller length). %Let $Q^\star_i(j) = \max_{\tau \in \mathcal{I}_j}Q_i(\tau)$ be the maximum queue
 Let $Q^\star_i(j)$ be the value of the variable $Q_i(\cdot)$ at the beginning of the $j$\textsuperscript{th} interval. We have
 \begin{eqnarray} \label{S-bd1}
 	\sum_{\tau=1}^t Q_i(\tau)g_{\tau, i}(x^\star) = \sum_{j=1}^{\lceil t/S \rceil} \sum_{\tau \in \mathcal{I}_j}\big(Q_i(\tau)-Q_i^\star(j)\big)g_{\tau, i}(x^\star) + \sum_{j=1}^{\lceil t/S \rceil}Q_i^\star(j) \sum_{\tau \in \mathcal{I}_j}g_{\tau, i}(x^\star) .   
 \end{eqnarray}
 %Let $g_{t,i}(x) \leq F, \forall x \in \mathcal{X}, t, i.$ 
% From the Lipschitzness assumption, we have $g_{t,i}(x) \leq GD \equiv F ~(\textrm{say}), \forall x \in \mathcal{X}, t, i.$ 
Using the boundedness assumption, let $g_{t,i}(x) \leq F, \forall x \in \mathcal{X}, t, i.$
 Using the Lipschitzness property of the queueing dynamics \eqref{q-ev2} with respect to time, we have 
 \begin{eqnarray*}
 	\max_{\tau \in \mathcal{I}_j} |Q_i(\tau)-Q_i^\star(j)| \leq F(S-1).
 \end{eqnarray*}
 Substituting the above bound into Eqn.\ \eqref{S-bd1}, we obtain 
 \begin{eqnarray} \label{new-Q-S}
 	\sum_{\tau=1}^t Q_i(\tau)g_{\tau, i}(x^\star)  \leq \big(1+\frac{t}{S}\big) F^2S(S-1) + F(S-1)(Q_i(t)+F(S-1)), 
 \end{eqnarray}
 where in the last term, we have used the $S$-feasibility of the action $x^\star$ in all intervals, except possibly the last interval. 
 %Clearly, when $S=1$, the RHS of the above bound becomes zero, and we recover Eqn.\ \eqref{q-regret-reln}. 
 Substituting the bound \eqref{new-Q-S} into Eqn.\ \eqref{new-reg-decomp}, we arrive at the following extended regret decomposition inequality:
 \begin{eqnarray} \label{gen-reg-decomp2}
 	\sum_{i=1}^k Q_i^2(t) &\leq& \textrm{Regret}'_t(x^\star) +  2kF^2S t + 2FS\sum_{i=1}^k Q_i(t) + 4F^2S^2k.
%&\leq & \textrm{Regret}'_t(x^\star) + 6kF^2S t + 2FS \sqrt{k} \sqrt{\sum_{i=1}^k Q_i^2(t)}.
 \end{eqnarray}
Eqn.\ \eqref{gen-reg-decomp2} leads to the following bound on the cumulative constraint violation.
% 
% \begin{theorem} \label{S-benchmark}
%Using the OGD policy with adaptive step-sizes given in part 1 of Theorem \ref{data-dep-regret} as a sub-routine, Algorithm \ref{ocs-policy} achieves the following CCV bound with the $S$-feasibility assumption (Assumption \ref{s-feas-assump}) for convex constraints: 
% 	 \[\max_{i=1}^k\mathbb{V}_i(T)= O(\max(\sqrt{ST},S )).\]
% \end{theorem}
% See below for the proof of the result.
%
%\paragraph{Remarks:} Recall that our proof of the $O(\sqrt{T})$ regret bound for the COCO problem with the $1$-benchmark in Theorem \ref{gen-cvx-bd} crucially uses the non-negativity of the pre-processed constraint functions. However, with $S$-feasible benchmarks, pre-processing by clipping the constraints does not work as then the positive violations can not be cancelled with a strictly feasible violation on a different round. We leave the problem of obtaining an optimal $O(\sqrt{T})$ regret bound for Algorithm \ref{g-oco-policy} for the COCO problem with the $S$-feasibility assumption as an open problem. 
%It is not clear how to  
%which is better than $O(T^{1-\epsilon/4})$ constraint violation bound obtained by \citet{georgios-cautious}.

  

%\textbf{Note:} This extension is meaningful only for the convex case. Any strongly-convex function 
%\vspace{5pt}
%\hrule 
%\textbf{Note:} \footnote{This extension is meaningful only when the range of the constraint functions includes both positive and negative values. For non-negative constraints, clearly $\mathcal{X}_S = \mathcal{X}_1 \forall S\geq 1.$}\\
%\hrule