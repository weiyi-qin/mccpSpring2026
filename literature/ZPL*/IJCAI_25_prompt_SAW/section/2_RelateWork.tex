\vspace{-1.7ex}
\section{Related Work}
\label{sec:RL}
\vspace{-0.7ex}
{\bf Prompt Compression.}
Prompt compression techniques are used to reduce the inference 
cost of  LLMs across a wide range of applications.
Existing work can be categorized into 
soft prompt compression and discrete prompt compression.

Soft prompts were introduced by~\citet{softPrompt}. A soft prompt integrates additional trainable parameters at the model's input stage.
\citet{softpromptcompression} emphasized that soft prompt 
compression effectively retains crucial abstract information 
with a reduced parameter count.
\citet{CompressThenPropmt} emphasized that carefully crafted 
prompts are helpful in augmenting the end-performance of 
compressed LLMs, also the compressed LLMs are helpful in the prompt 
learning phase. 

Compared to soft prompt compression, discrete prompt compression 
seeks to optimize the effectiveness of prompts via token-level search 
strategies. 
\citet{Jung2023DiscretePC} employed policy networks to eliminate
unnecessary tokens for prompt compression.
\citet{selectiveContext} utilized self-information metrics to identify 
and remove superfluous information in prompts. 
Capitalizing on these 
advancements, \citet{llmlingua} and \citet{longllmlingua} \eat{have}
formulated algorithms for dynamically adjusting compression rates 
across different prompt sections, giving precedence to tokens with 
higher perplexity\eat{due to their substantial influence}. 

Despite the significant advancements achieved by these studies, 
their primary focus lies on token-level compression, neglecting the 
comprehensive graph structure information inherent in the prompt. 
\eat{To the best of our knowledge, our study represents the first 
exploration of prompt compression using graph structure.}



\eat{{\bf Chain-of-Thought (CoT).}
CoT is a novel prompting technique aimed at enhancing 
the reasoning abilities of LLMs~\citep{CoT}.
It incorporates intermediate problem-solving steps 
to improve LLMs' capacity to tackle 
complex reasoning tasks significantly.
\citet{ToT} exploited the CoT technique and developed Tree-of-Thought (ToT) in order to augment the reasoning 
capabilities of LLMs further. \cite{GoT} proposed Graph-of-Thought 
that employs graphical reasoning capabilities to construct 
prompts.}


\noindent{\bf Knowledge Graphs (KGs) for LLM.}
KGs organize information as structured units, \emph{i.e.,} relational 
triplets (explained in Appendix~\ref{sec:BCKGD_KG}), that 
encapsulate a wide variety of entities/concepts along with 
underlying relations~\citep{KGSurvey}. 
\citet{KGLLMSurvey} illustrated multiple different scenarios for 
integration of KGs with LLM for knowledge and data-driven 
bi-directional reasoning.
\eat{namely: KG-augmented LLMs, LLMs augmented KGs, 
showcasing advancements through KG-enhanced LLM, KG-augmented 
LLM, and synergistic LLM and KG methodologies.}
\citet{ReasoningOnGraph} combined LLMs with KGs for interpretable
reasoning over KG Question Answering tasks.
\citet{KGGPT} introduced an innovative framework that 
leverages LLM's reasoning capabilities for executing KG-based tasks. 
To the best of our knowledge, \OurMODEL{} is the first to
make an attempt to leverage knowledge graph structure for prompt compression.




\begin{figure*}[h]
    \centering
    \includegraphics[width=0.75\linewidth]{Figure/Pframework4.pdf}
    \vspace{-2.7ex}
    \caption{Workflow and an example illustration of \OurMODEL{}.}
    \vspace{-3.7ex}
    \label{fig:framework}
\end{figure*}

%%%%%%%%%%% EXTRA STUFF %%%%%%%%%%%%%%%
\eat{
\subsection{Retrieval-Augmented Generation}
Retrieval-Augmented Generation (RAG) integrates pre-trained parametric 
and non-parametric memory to improve the language generation quality of 
LLMs (\citet{RAG},\citep{RAG}).
\warn{By dynamically retrieving pertinent information throughout 
the answer generation phase, the RAG model facilitates the production 
of responses derived from a wider and more accurate knowledge base. 
Recent research identifies RAG as a pivotal technology in enhancing 
the quality and accuracy of LLM.}
\eat{ investigates a model architecture known as, which }
\fixme{We don't need related work on RAG I think. I am omitting it.}}





\eat{
Knowledge Graphs organize structured information as relational triplets 
that encapsulate a vast variety of entities/concepts along with 
underlying relations (\citet{KGSurvey}, \citep{KGSurvey}). 
\cite{KGLLMSurvey} delves into the cutting-edge integration of 
knowledge graphs with LLM, showcasing advancements through KG-enhanced 
LLM, KG-augmented LLM, and synergistic LLM and KG methodologies. 
\cite{ReasoningOnGraph} advocates for the improvement of LLM 
performance in Knowledge Graph Question Answering (KGQA) tasks 
through the innovative generation and retrieval of relational 
paths within Knowledge Graphs. 
\cite{KGGPT} introduces an 
innovative framework that leverages LLM's reasoning capabilities 
for executing Knowledge Graph based tasks. To the best of our 
knowledge, this study represents the inaugural effort to leverage 
knowledge graphs for prompt compression.
}

\eat{To re-emphasized the prompts are widely used to get the best 
possible performance of LLMs across a wide range of application 
scenarios, e.g., Incremental Context Learning~\citep{ICL},
Chain-of-Thought~\citep{CoT}, 
and Retrieval-Augmented Generation~\citep{RAG} etc.,
However, these methods often results in 
wordy/verbose prompts, underscoring the need of 
\eat{research in }prompt compression.}

