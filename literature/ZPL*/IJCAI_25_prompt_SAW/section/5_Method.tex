\vspace{-1.7ex}
\section{\OurMODEL{}}
\label{sec:proposed}
\vspace{-0.7ex}
In this section, we provide details of \OurMODEL{}. 
The workflow is shown in Figure~\ref{fig:framework}.
As shown in the figure, \OurMODEL{} takes the original prompt text as input and generates the compressed 
prompt as the output.



In contrast to the existing token-level compression methods, 
in \OurMODEL{} we use a graph structure to effectively 
represent the textual information in the prompt, which is  helpful to 
analyze the key aspects of the prompt. 
Later, we can refine the information in the graph structure to come up with a compressed prompt in a way that:
(i) The semantic consistency of the compressed prompt is preserved; 
(ii) The end performance and/or utility of the prompt is not distorted.
Below, we first introduce the motivation of \OurMODEL{}, 
followed by the prompt compression process.

\subsection{Motivation of \OurMODEL{}} 
\label{method_structure}
\OurMODEL{} is motivated by the observation that the key information 
within the prompt text could be inferred as a set of entities 
and relations, which can also be organized into a graph structure, 
commonly known as a knowledge graph in literature.


Formally, given a prompt text $P$, we claim it encompasses a 
set of entities
$\mathcal{E}\eat{=\{e_1,e_2,\cdots\}}$, \emph{i.e.,} names of 
persons, locations, 
organizations, miscellaneous elements, etc.,~\citep{ali2020fine}. 
These entities 
serve as the key elements of the prompt structure.
In addition to the entities, we can also infer some relations $\mathcal{R}$ in 
$P$ that may be used to describe the connections between the entities.
\OurMODEL{} re-organizes these key elements of the prompt (\emph{i.e.,} 
entities and their relations) in a graph structure, 
{represented by $\mathcal{G}=
\{(e_i, r_i,e_i^{'})\subseteq\mathcal{E}\times\mathcal{R}\times\mathcal{E}\}$. We use $g_i=(e_i,r_i,e_i^{'})$ to represent the $i$-th information 
element of $\mathcal{G}$, \emph{i.e.,} a fact stating that $e_i$
has $r_i$-$th$ relation with $e_i^{'}$.} 

We argue this transformation of text information to graph is a more reasonable and
natural approach as:
(i) {It helps in highlighting the key information elements in 
the prompt.}
(ii) {Later, analyzing these key entities in combination 
with underlying relations helps in filtering/digging out 
the salient content within the prompt to come up with a
compressed prompt.}




\subsection{Workflow of \OurMODEL{}}
The workflow of \OurMODEL{} consists of two parts. First, it 
uses the information in prompt $P$ to construct a graph $\mathcal{G}$.
Then, based on the specific scenario, we proceed as follows:

\noindent{\bf(a) Task-aware scenario.} For this scenario, 
we traverse the graph $(\mathcal{G})$ in a way to preserve 
only the information elements that are relevant to the task as 
task-specific subgraphs, indicative of information useful 
for the compressed prompt.
\eat{\di{Rewrite this sentence, grammar error}}  \\


\noindent{\bf(b) Task-agnostic scenario.} For this scenario, we no 
longer have access to 
task-specific information. Thus, we use similarity scores 
between the information 
elements in $\mathcal{G}$ to identify and remove the 
redundant elements to obtain 
subgraphs that are helpful for compression. 

Further details about the model components of~\OurMODEL{} 
are provided in the following subsections.  

\subsubsection{Graph Construction.} 
For graph construction from the text data, 
we primarily rely traditional knowledge extraction approaches, \emph{i.e.,} OpenIE~\citep{OpenIE}, to 
construct a graph $G$, as follows

\vspace{-15pt}
\begin{equation}
   \mathcal{G} = {IE}({P}), 
   \label{eq:Stage:OIE}
\end{equation}
\vspace{-15pt}

where $P$ is the prompt text, and $IE$ is an 
information extraction module that takes $P$
as input and return graph $G$ as output.
For the cases not addressed by the above equation,
we use an in-context learning 
prompt as our auxiliary method that 
prompts the language model to construct the graph from the original prompt text as follows:
\begin{equation}
   \mathcal{G} = {M}(P_{\text{template}}(\text{P})), 
   \label{eq:graphConstruct}
\end{equation}
where $(P)$ is the prompt text and $P_{\text{template}}$ is the prompt template (explained in Appendix~\ref{Appendix:prompt_Graph}) used 
to guide the LLM $(M)$ to extract the graph $\mathcal{G}$. 
Note, for Equation~\ref{eq:graphConstruct}, we typically prefer a small-scale open-source LLM in order to avoid
higher computational costs incurred by large models.
\eat{, which eventually contributes to the overall computational overhead.}


\subsubsection{(i) Task-aware Prompts.}
Task-aware scenarios refer to the settings when the information within the prompt 
is helpful and/or is related to the end-task, \emph{e.g.,} question answering. 
For such cases,~\OurMODEL{} aims to retain only the task-specific information
in $\mathcal{G}$, while filtering out the redundant/useless information.
For this, it first uses an encoder function to get the embeddings for 
the prompt question, as follows.
\begin{equation}
    Emb_{p^{\mathrm{que}}} = E(p^{\mathrm{que}})
\end{equation}
where $Emb_{p^{\mathrm{que}}}$ is the embedding for prompt question 
$(p^{\mathrm{que}})$, and $E$ is the encoder network. 
Then, it computes the pair-wise similarity between the $Emb_{p^{\mathrm{que}}}$ 
and information elements in $\mathcal{G}$, as shown below.
\begin{equation}
    Sim_{\mathcal{G}} = \{E(g_i)\cdot Emb_{p^{\mathrm{que}}} \; | \forall \; g_i \in \mathcal{G} \}
\end{equation}
where $g_i$ corresponds to the $i$-$th$ information element in $\mathcal{G}$, 
$E(g_i)$ is used to encode the information in $g_{i}$, 
$Sim_{\mathcal{G}}$ is the set of the similarity scores between information element 
in $\mathcal{G}$ and the question embeddings $Emb_{p^{\mathrm{que}}}$.
Later, it ranks the scores in $Sim_{\mathcal{G}}$ in order to retain only the elements 
in $\mathcal{G}$ showing a higher degree of similarity with $p^{\mathrm{que}}$, 
as shown below.
\begin{equation}
   Index_{\text{ranked}} = \text{Rank}(Sim_{\mathcal{G}})
\end{equation}
where $\text{Rank}(\cdot)$ is used to sort the similarity scores in 
$Sim_{\mathcal{G}}$ and return corresponding high-ranked information 
elements as $Index_{\text{ranked}}$. 
We then use the information in $Index_{\text{ranked}}$ 
to iterate over $\mathcal{G}$ to extract the sub-graph 
$\mathcal{G}_{\text{subset}}$ 
not surpassing  the targeted compression 
ratio $\eta^{*}$. Its process-flow is illustrated in Algorithm~\ref{alg:alg1}.

\noindent{\bf Workflow of Algorithm~\ref{alg:alg1}.}
The workflow of Algorithm~\ref{alg:alg1} is explained as follows:
(i) initialize $\mathcal{G}_{subset}$ as an empty set (line-1);
(ii) for each element in $Index_{\text{ranked}}$ repeatedly 
add $g_i$ in $\mathcal{G}_{subset}$ until the compression rate
surpasses the target compression rate $\eat^{*}$ (lines 2-7);
(iii) return final graph $\mathcal{G}_{subset}$ as output (line-10).

Finally, we restore/reconstruct the information elements in 
$\mathcal{G}_{\text{subset}}$ to come up with our compressed prompt $C$, as shown below.
\begin{equation}
    C=e_1\oplus r_1 \oplus e_1^\prime ; \cdots ; e_n\oplus r_n \oplus e_n^\prime
   \label{eq:Stage1}
\end{equation}

\eat{
\begin{equation}
    C=e_1\oplus r_1 \oplus e_1^\prime \oplus \text{[SEP]} \oplus \cdots e_i\oplus r_i \oplus e_i^\prime \oplus \text{[SEP]} \oplus\cdots e_n\oplus r_n \oplus e_n^\prime
   \label{eq:Stage1}
\end{equation}
}
where $\oplus$ is the concatenation operator used to combine the entities and 
relations within the information elements $(g_i)$ {in the extracted subgraph $\mathcal{G}_{\text{subset}}$}, and $({;})$ is the delimiter used to separate different information elements in 
$\mathcal{G}_{\text{subset}}$. 

\begin{algorithm}[t]
    \caption{\scshape Subgraph Extraction }
    \label{alg:alg1}
    \begin{algorithmic}[1]
    \Require
        \item[] { \#$\eta^{*}:$ {Target compression rate}}
        \item[] { \#$\mathcal{G}:$ Graph structure of prompt}
        \item[] { \#$len():$ Length of graph as the sum of tokens.}
    \Ensure subgraph $\mathcal{G}_{subset}$
        \State $\mathcal{G}_{subset}=\{\}$
        \For{$i \in Index_{\text{ranked}}$}
        \State $\mathcal{G}_{subset}.insert(g_i)$
        \State { \#compute compression rate}
        \State $rate= len(\mathcal{G}_{subset})/len(\mathcal{G})$
        \State { \#break if meet the constraint}
        \State $\mathbf{If}$ $rate> \eta^*$ \textbf{then}
        \State $ \qquad Break$
        \State \textbf{end If}
        \EndFor
        \State \Return $\mathcal{G}_{subset}$
    \end{algorithmic}
\end{algorithm}
 

\subsubsection{(ii) Task-agnostic Prompts.}

A task-agnostic scenario implies that it is almost impossible to filter useful 
and/or task-specific information within the original prompt text $(P)$.
{In such cases,~\OurMODEL{} looks for recurring information elements in 
$P$ for probable prompt compression.}
We assume two main sources of recurring elements in $P$, \emph{i.e.,} (i) the 
verbose expression of the prompt itself and (ii) the repeated element 
generated by auxiliary models. 
Note that these assumptions are based on empirical observation illustrating 
that large models' re-reading phenomenon leads to the repeated generation of 
the extracted knowledge \citep{Yan2023UnderstandingIL}. 
\eat{\warn{Some examples in this regard are show in Appendix~\ref{Appendix:repeated_know}}.
\fixchen{Add some examples in the Appendix~\ref{Appendix:repeated_know}.}}


For compression over task-agnostic scenarios, we sequentially traverse 
the information elements in $\mathcal{G}$ and select only the elements
exhibiting a lower similarity with priorly selected information 
elements.
Our underlying intuition is that highly similar information elements 
will carry repeated information. Thus, we could avoid redundant information in $P$ by selecting only dissimilar elements.

For this, we use a threshold $\delta$ as a selection criteria for~\OurMODEL{}.
The value of the $\delta$ is determined using a binary search algorithm 
(shown in Algorithm~\ref{alg:alg2}) that computes an appropriate value 
of threshold $\delta$ required to meet the targeted compression rate $\eta^{*}$.


\noindent{\bf Workflow of Algorithm~\ref{alg:alg2}.} The process-flow of Algorithm~\ref{alg:alg2} is explained as follows:
(i) firstly, we initialize an interval $[l,r]$ for the 
threshold $\delta$ (line-1);
(ii) at each step, we partition the interval into two parts $[l, mid]$ and 
$[mid, r]$ via the midpoint $ mid= (l+r)/2$ 
(line-3);
(iii) we will compute the graph subset, \emph{i.e.,} $\mathcal{G}_{subset}$ via function $Compress()$ (explained below) with the value of $mid$ as threshold, shown in line-4;
(iv) compute the compression rate for the $\mathcal{G}_{subset}$ and 
accordingly update the values of $l$ and $r$ (lines 6-9). Specifically, if the compression rate is smaller than $\eta^{*}$, then the current threshold is too stringent thus we judge $\delta$ is in $[mid,r]$, otherwise it is in $[l,mid]$;
(v) depending upon the interval threshold $\gamma$ (line-2),
we compute the value $(l+r)/2$ as the final similarity threshold $\delta$ 
(line-11); 
(vi) finally, use the value of $\delta$ to return the final 
graph subset $\mathcal{G}_{subset}$ (line-13).


\noindent{\bf Compress Function.} The workflow of 
the $Compress()$ is shown in Algorithm \ref{alg:compress} 
and explained as follows: 
(i) start with an empty graph $(\mathcal{G}^\prime)$ (line-1);
(ii) iterate the information elements in the graph $(g_i \in \mathcal{G})$
to compute the similarity score of $g_i$ with the 
elements in $\mathcal{G}^\prime$ to look for maximal similarity, 
\emph{i.e.,} $sim_{max}$ (lines 2-3);
(iii) compare $sim_{max}$ against the compression threshold 
$\delta$ to insert $g_i$ in $(\mathcal{G}^\prime)$ (lines 4-5);
(iv) finally, return $(\mathcal{G}^\prime)$ as the final subset 
of the graph.
The end-goal of Algorithm~\ref{alg:alg2} is to select highly
dis-similar information elements by neglecting cases with
$sim_{max} > \delta$.
For such cases, we assume that the corresponding information element, 
\emph{i.e.,} $g_i = (e_i,r_i,e_i^{'})$ is redundant 
because there is already an element in  $\mathcal{G}^\prime$ that 
is very similar to $g_i$.

\begin{algorithm}[t]
    \caption{\scshape Binary Search}
    \label{alg:alg2}
    \begin{algorithmic}[1]
    \Require
        \item[] { \#$\eta^{*}:$ {Target compression rate}}
        \item[] { \#$\mathcal{G}:$ Graph structure of prompt}
        \item[] { \#$\gamma:$ interval threshold}
    \Ensure subgraph $\mathcal{G}_{subset}$
        \State double $l=0,r=1$
        \While{$r-l > \gamma$}
        \State double $mid=(l+r)/2$
        \State $\mathcal{G}_{subset} = Compress(\mathcal{G},mid)$
        \State $Rate= len(\mathcal{G}_{subset})/len(\mathcal{G})$
        \State $\mathbf{If}$ $Rate > \eta^{*}$ \textbf{then}
        \State $ \qquad r=mid$
        \State \textbf{Else} 
        \State $ \qquad l=mid$
        \EndWhile
        \State $\delta=(l+r)/2$  \# compression threshold
        \State $\mathcal{G}_{subset} = Compress(\mathcal{G},\delta)$
        \State \Return $\mathcal{G}_{subset}$
    \end{algorithmic}
\end{algorithm}




%%%%%%%%%%%%%% EXTRA STUFF %%%%%%%%%%%%
\eat{and populate it over time with selected information elements following 
a similarity threshold.
In order to check the upcoming information elements $(g_i)$ for 
redundancy, we first compute the similarity score of $g_i$ with the 
elements in $\mathcal{G}^\prime$ to look for maximal similarity.
\begin{equation}
    max_{\text{sim}} = \max_{g\in \mathcal{G}^\prime}(E(g)\cdot E(g_i))
   \label{eq:Stage2}
\end{equation}
We use this function to iterate the information elements 
($g_i \in \mathcal{G}$) in order to filter out redundant information.
where $\mathcal{G}^\prime$ is the set of selected }

